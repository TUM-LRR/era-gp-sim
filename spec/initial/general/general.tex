\chapter{Einführung}

Dieses Dokument stellt einen Statusbericht des ERA-Großpraktikum-Teams sowie
eine initiale, konzeptuelle Ausarbeitung des zu entwickelnden Assembler-
Simulators dar.  Wir beginnen mit einer groben, kurzen Übersicht unserer
Teilgruppen und ihren Aufgabenbereichen.  In den folgenden Kapiteln behandeln
wir dann die einzelnen Module und berichten dabei von den jeweiligen
Erfahrungen, Beobachtungen und weiterführenden Ideen.  Ebenfalls erläutern wir,
wie die einzelnen Module von einander abhängen und miteinander kommunizieren.

\section{Übersicht}

Wir haben es uns zum Ziel gesetzt, einen architekturunabhängigen
Assembler-Simulator in C++ zu bauen. Mit \emph{architekturunabhängig} meinen wir
hierbei, dass der Simulator so flexibel sein soll, dass nicht nur eine bestimmte
Architektur, sondern eine Vielzahl existenter sowie auch zukünftiger
Befehlssätze darin simuliert werden können. Als ersten Befehlssatz wollen wir
dabei \emph{RISC-V} unterstützen, auf welchen wir uns in unserer
Entwicklungsphase zunächst auch fokussieren werden. Im Weiteren sollen aber
beispielsweise auch die Intel x86 oder ARM v7 Architektur simulierbar sein.

Unsere grundsätzliche und hauptsächliche Zielgruppe wird hierbei die Lehre
sein. Wir werden unser Augenmerk also vor Allem darauf legen, die vielen
verschiedenen und teilweise auch komplexen Komponenten eines Befehlssatzes
intuitiv und informativ für den Nutzer zu visualisieren und im Hintergrund auch
entsprechend zu implementieren. So sollen Studenten und andere
Assembler-Enthusiasten die Assembler Sprache praktisch erlernen und erkennen,
wie die einzelnen Zahnräder einer Computer-Architektur zusammenspielen. Wir
wollen unsere potentielle Nutzergruppe jedoch nicht exklusiv auf die Lehre
beschränken. Unser Simulator soll auch außerhalb der Lehre, also in der
Forschung oder im professionellen Bereich, von Nutzen sein.

Neben der Architekturunabhängigkeit war eine unserer fundamentaleren
Entscheidungen jene, Assembler-Programme \emph{addressbasiert} und nicht
\emph{zeilenbasiert} zu simulieren. Damit ist gemeint, dass Sprungadressen im
Programm nicht Zeilen des Text-Editors referenzieren, sondern echte
Speicheradressen. Als Konsequenz davon wird vor der Ausführung eines
Assemblerprogramms das Programm auch assembliert in den Speicher gelegt. Wir
erwarten uns davon zwei Vorteile. Der erste Vorteil ist, dass unerfahrene Nutzer
auch sofort den Zusammenhang zwischen der textuellen Darstellung ihres
Programms, und den finalen, assemblierten Instruktionen im Speicher
lernen. Zweitens können wir durch addressbasierte Simulation auch \emph{echte}
Assemblerprogramme simulieren, welche auch auf realer Hardware ausführbar wären.

Theoretisch wäre durch addressbasierte Referenzierung auch das Laden von
Objektdateien direkt in unseren Speicher, sowie die Manipulation von
Instruktionen im Speicher, möglich. Dabei wäre jedoch auch die Implementierung
eines \emph{Disassemblierers} nötig, was wir vorerst außerhalb unseres Fokus
sehen. Diese beiden Funktionen werden wir somit nicht unterstützen, wobei dies
in Zukunft ohne Weiteres möglich wäre. Die assemblierte Darstellung von
Programmen im Speicher als Folge der adressbasierten Simulation soll also nur
kosmetischen bzw. pädagogischen Nutzen haben.

Als \emph{minimale Funktionalität}, die wir unterstützen wollen, legen wir
jegliche Programme fest, die in der Vorlesung \emph{Einführung in die
  Rechnerarchitektur} sowie allen dazugehörenden Klausuren der letzten Jahre
verwendet wurden. Es soll also jedes x86 Assembler Programm, das in der
ERA-Lehre gezeigt wurde, nach Übersetzung in den RISC-V Befehlssatz in unserem
Simulator ausführbar sein. Als \emph{maximale} Funktionalität geben wir uns die
Ausführung beliebiger \emph{realer} Programme als Ziel. Dies sollte durch die
Address-Basierung näherungsweise auch sehr möglich sein. Diese sind dabei aber
natürlich durch die von uns unterstützten Teile eines Befehlssatzes
beschränkt. Wir planen beispielsweise nicht, \emph{syscalls} zu
implementieren. Somit schließen wir also die Simulation eines Betriebssystems
zum Beispiel (vorerst) aus.

\section{Organisation}

Wir haben uns für eine Gliederung des Projekts in vier Gruppen entschieden.
Diese bezeichnen wir hierbei als \emph{Parser}, \emph{Core}, \emph{Architektur}
und \emph{GUI}. In den folgenden Absätzen beschreiben wir kurz die einzelnen
Gruppen und nennen ihre Teammitglieder. Die Teamleitung wird von Peter
Goldsborough, sowie in Vertretung von Daniel Riedel, übernommen.

\subsection{Architektur}

Jegliche architektur-spezifische Aspekte
unseres Simulators fallen hierbei in den Aufgabenbereich der
\emph{Architektur}-, kurz \emph{Arch}-, Gruppe. Unter
\emph{architektur-spezifisch} verstehen wir hierbei die Implementierung der
Instruktionen eines Befehlssatzes sowie die Bereitstellung aller Informationen
und Daten, die andere Module des Simulators benötigen. Beispiele hierfür wären
die Registerorganisation sowie das Speichermodell, welche die GUI zur
Visualisierung benötigt; die reservierten Schlüsselwörter der Assemblersprache
der Architektur als notwendige Information für den Parser sowie die
unterstützten Datentypen für die Speicherverwaltung des Cores.

Das Architektur-Team besteht aus Yuriy Arabskyy, Erik Pohle, Lukas Karnowski und
Peter Goldsborough.

\subsection{Parser}

Die Assemblersprache stellt ein mächtige Schnittstelle zur Manipulation des
Speichers sowie des Prozessors einer jeden Architektur dar. Jegliche
Instruktionen an unsere simulierte Maschine müssen notwendigerweise in Assembler
formuliert werden. Als roher Textkörper ist ein Assemblerprogramm jedoch keinem
unserer Module nützlich. Der Assemblercode muss somit zuerst in eine
Representation überführt werden, die die Evaluierung einzelner Instruktionen
sowie deren Argumente als Datenobjekte in anderen Modulen des Simulators
ermöglicht. Hierzu widmet sich das \emph{Parser}-Team. Zwei wesentliche Aufgaben
dieser Gruppe sind es, den Assemblertext für Core und Architektur in einen
abstrakten Syntaxbaum zu transformieren sowie bezüglich syntaktischen und
semantischen Aspekten des Quelltextes mit der GUI zu kommunizieren.

Die Mitglieder des Parser-Teams sind Dominik Kreutzer und David Schneller.

\subsection{GUI}

Um unsere simulierte Maschine interaktiv zu visualisieren, planen wir, mit dem
Qt-Framework eine grafische Benutzeroberfläche zu enwickeln. In dieser
Oberfläche sollen eine Eingaberegion für Assemblercode, den Hauptspeicher, die
Register sowie weitere zusätzliche Elemente dargestellt werden. Wir legen
hierbei Wert darauf, den Assemblercode nützlich und intuitiv hervorzuheben
(highlighten) sowie auch in weiteren Teilen der graphischen Oberfläche dem
Nutzer beim Erlernen der Rechnerarchitektur und des Assemblierens zu
helfen. Auch ist es Aufgabe der GUI-Gruppe, Elemente wie simulierte LED-Streifen
oder Matrizen zu kreieren, auf welche man durch Assemblercode zugreifen
kann.

Im GUI-Team sind Sabine Rieder, Felix Opolka, Andreas Zimmerer sowie Jan
Schopohl (auch im Core-Team).

\subsection{Core}

Die Aufgabe des \emph{Core}-Teams ist es, ein Modul zu entwickeln, das die
Koordination aller Teile des Programms leitet. Es trägt somit die Verantwortung,
den Austausch von Daten und Kontrollinformationen zwischen der Architektur, dem
Parser sowie der GUI zu steuern. Weiterhin hat es die Aufgabe, den Speicher und
die Register des simulierten Prozessors zu verwalten. Letztlich kümmert sich
dieses Modul noch um Threadverwaltung.

Das Core-Team setzt sich aus drei Mitgliedern zusammen: Tobias Holl, Daniel
Riedel und Jan Schopohl (auch im GUI-Team).

\section{Zeitplan}

Wir haben das ERA-Großpraktikum offiziell Mitte April begonnen und wollen es in
zwei Semestern, bis Februar 2017, zur Gänze vollenden. Hierbei beschränken wir
uns jedoch auf eine \emph{Entwicklungszeit} von sechs Monaten, zwischen Juni und
November 2016 (beide inklusive). Die danach verbleibende Zeit halten wir für
Dokumentation, Ausarbeitung der finalen Spezifikation sowie sonstige
logistische Tätigkeiten frei.

Für jeden dieser sechs Monate haben wir uns gemeinsam Ziele gesetzt, welche wir
bis zum Ende dieser einzelnen Monate vollendet haben wollen. Diese Ziele lauten,
chronologisch geordnet, wie folgt:

\begin{itemize}
  \item \textbf{Juni}
     \begin{itemize}
     \item \textbf{Arch}: RISC-V Architektur ist als Objekt definiert und für
       andere Module nutzbar. Syntaxbaum ist im Abstrakten implementiert, sodass
       der Parser aus Ausdrücken schon einen Baum erstellen kann. Einfache
       RISC-V Befehle sollen auch schon vollendet sein.
     \item \textbf{Parser}: Grundlegende Befehle, ohne komplexe Operanden,
       können in einen Syntaxbaum verarbeitet werden.
     \item \textbf{Core}: Das Speichermodell, inklusive Hauptspeicher und
       Register, ist zumindest als ``Dummy'' implementiert, sodass andere Module
       darauf operieren können. Somit sind auch schon einfache Datentypen
       erstellt. Die Threading-Infrastruktur steht.
     \item \textbf{GUI}: Die Oberfläche des Mockups, wie in der Spezifikation
       gezeigt, soll implementiert sein (ohne eigentliche Funktionalität).
     \end{itemize}
  \item \textbf{Juli}:
    \begin{itemize}
      \item \textbf{Arch}: Alle Befehle sind implementiert und der Syntaxbaum
        ist als Ganzes fertig.
      \item \textbf{Parser}: Alle Befehle, auch mit komplexen Operanden, werden unterstützt.
      \item \textbf{Core}: Speichermodell ist vollständig. Der \emph{Main}-Loop
        verbindet alle Module. Threading wird vollständig behandelt.
      \item \textbf{GUI}: Kommunikation mit dem Core (Update der View) als
        Prototyp fertig.
    \end{itemize}
  \item \textbf{August + September (Sommerblock)}:
    \begin{itemize}
      \item \textbf{Arch}: Alle Befehle können in ihre Binärdarstellung
        assembliert werden. Implementierung des RISC-V Befehlssatzes, mit
        Multiplikation und Division, ist fertig.
      \item \textbf{Parser}: Direktiven, Labels und Konstanten werden
        erfolgreich verarbeitet bzw. eingesetzt und der Speicher reserviert.
      \item \textbf{Core}: Basis wird erweitert: BigInteger, BigFloat, variable Speicherzellengröße.
      \item \textbf{GUI}: Kommunikation mit dem Core fertig. Zeilenweises
        durchschreiten möglich. Kontextinformationen im Speicher
        implementiert. Fehlermeldungen für Syntaxfehler.
    \end{itemize}
  \item \textbf{Oktober}:
    \begin{itemize}
      \item \textbf{Arch}: Implementierung der Pseudo-Instruktionen\footnote{sieh RISC-V Specification Table 21.1} sowie didaktisch wertvolle Instruktionen ähnlich zu ``jasminsleep`` und ``jasmincrash``. Hilfe bei der Integration der GUI
      \item \textbf{Parser}: Direktiven fertig machen, Makros sollen auch mit Argumenten funktionieren. Wenn fertig Auflösung in andere Teams.
      \item \textbf{Core}: Serialisierung des Speichermodells (Speicher und Register) in eine menschlich lesbare Form (z.B. JSON, XML, etc). Hilfe bei der Integration der GUI
      \item \textbf{GUI}: Integration des bisherigen Fortschritts von Arch, Parser und Core in die GUI bis \textbf{25.10.16}.\\
      Serialisierung des Projekts (mit Core), Ausgabe-Anzeigen (Lichtband, 7-Segment-Anzeige, Display, Konsole; Priorität in dieser Reihenfolge)
    \end{itemize}
  \item \textbf{November}:
  \begin{itemize}
  	\item \textbf{GUI}: Fertigstellung der Ausgabe-Anzeigen. Implementierung eines Eingabe-Moduls, das bei gedrückten Tasten bestimmte Werte an eine festgelegte Speicheradresse schreibt (z.B ASCII-Code des Zeichens)
  	\item \textbf{Arch, Parser, Core}: Fehlersuche, kleine Verbesserungsvorschläge (v.a. GUI betreffend) suchen.
  	Vereinheitlichung des Fehlermeldungsstil (v.a. Parser, Core)
  \end{itemize}
  \item \textbf{Optional}: Floating-Point
  \item \textbf{Dezember + Januar}: Dokumentation, Bericht, Cleanup.
\end{itemize}


Um diese Ziele jeweils zu erreichen, teilen wir unsere Entwicklungszeit weiter
in 2-wöchge \emph{Sprints} auf. Innerhalb eines Sprints bewältigt jedes Team
bzw. jedes Mitglied konkrete, atomare Aufgaben, welche zur Vollendung des
monatlichen Ziels führen sollen.

\section{Logistik}

In dieser Sektion erläutern wir unsere Entscheidungen zu logistischen und
organisatorischen Aspekten der Entwicklung.

\subsection{Taskboard}

In unserem \emph{Taskboard} (agil: \emph{KanBan Board}) werden zu jedem
Zeitpunkt die einzelnen Aufgaben (agil: \emph{Backlog}) des momentanen Sprints
angezeigt und organisiert. Hierfür haben wir die
waffle.io\footnote{\url{http://waffle.io}} Online-Platform in unser GitHub
Projekt integriert. In diesem Taskboard liegen Aufgaben stets in einer von vier
Spalten:

\begin{enumerate}
  \item \emph{Open}: Noch offen, wird gerade von niemandem bearbeitet.
  \item \emph{In Progress}: Wird momentan von jemandem bearbeitet.
  \item \emph{Code Review}: Bearbeitung der Aufgabe ist vorerst fertig und für
    Reviews offen.
  \item \emph{Done}: Die Aufgabe wurde bewältigt.
\end{enumerate}

An einer Aufgabe arbeitet hierbei stets genau ein Mitglied des Teams, was im
Taskboard entsprechend gekennzeichnet ist.

\subsection{Definitions of Done}

Was es für uns bedeutet, dass eine Aufgabe \emph{fertig} (\emph{done}) ist,
legen wir als unsere \emph{Defintions of Done} wie folgt fest:

\begin{enumerate}
  \item Code wurde vorerst geschrieben.
  \item Unit-Tests sind geschrieben und passen.
  \item In-Code Dokumentation ist vollständig.
  \item Task wurde von \emph{In Progress} nach \emph{Code Review} geschoben.
  \item OK von den Code-Reviewern.
  \item Merge des Feature Branch in den Master Branch.
  \item Task wurde von \emph{Code Review} nach \emph{Done} geschoben.
\end{enumerate}

\subsection{Code Reviews}

Um eine hohe Code-Qualität zu garantieren sowie das Einschleichen von Bugs zu
verhindern, finden wir es wertvoll, \emph{Code Reviews} durchzuführen. Hierbei
soll jede Zeile Code, welche als ``stabil'' gelten soll (Feature Branch
$\rightarrow$ Master Branch), zunächst von vier weiteren Augen gesehen worden
sein. Wir benutzten hierbei das bei GitHub integrierte Code-Review Tool. Eine
Code-Review soll es dabei immer genau dann geben, wenn ein Mitglied eines Teams
ein Feature (vorerst) vollendet hat und entsprechende Unit-Tests passen (Schritt
4 in den \emph{Definitions of Done}).

Konkret hat jedes Mitglied zwei \emph{Review-Buddies}. Möchte ein Mitglied der
Buddy-Group ein Feature in den Master Branch mergen, so müssen zuerst seine/ihre
beiden Review-Buddies den Code inspizieren und das OK dafür geben. Die Buddies
sind dabei so gewählt, dass jedes Mitglied einen Buddy aus seinem Team (Arch,
Core, Parser, GUI) und einen aus einem anderen Team hat:

\begin{table}[h!]
  \centering
  \begin{tabular}{lll}
    \textbf{Opfer} & \textbf{1. Reviewer (Intern)} & \textbf{2. Reviewer (Extern)} \\
    \toprule \\
    Erik Pohle & Lukas Karnowski & Daniel Riedel \\
    Lukas Karnowski & Yuriy Arabskyy &  Sabine Rieder \\
    Yuriy Arabskyy & Peter Goldsborough & Felix Opolka \\
    Peter Goldsborough & Erik Pohle & Tobias Holl \\

    Daniel Riedel & Jan Schopohl & David Schneller \\
    Jan Schopohl & Tobias Holl & Andreas Zimmerer \\
    Tobias Holl & Daniel Riedel & Peter Goldsborough \\

    Sabine Rieder & Felix Opolka & Dominik Kreutzer \\
    Felix Opolka & Sabine Rieder & Lukas Karnowski \\
    Andreas Zimmerer & Jan Schopohl & Yuriy Arabskyy \\

    Dominik Kreutzer & David Schneller & Erik Pohle \\
    David Schneller & Dominik Kreutzer &  Andreas Zimmerer \\
    \bottomrule
  \end{tabular}
\end{table}

\subsection{Git Flow}

Wir verwenden \emph{Git} als Versionskontrollsystem. Diese Tatsache alleine,
klärt aber natürlich noch nicht, \emph{wie} wir Git nutzen. Ein Modell der
Git-Nutzung, dass sich in der Software-Industrie hoher Beliebtheit erfreut, ist
\emph{Git-Flow}\footnote{\url{https://www.atlassian.com/git/tutorials/comparing-workflows/}}. Eine
Variation dieses Modells, welche wir nutzen wollen, lässt sich kompakt wie folgt
charakterisieren:

\begin{itemize}
  \item Es gibt einen \emph{Master}-Branch, der immer stabilen Code enthält.
  \item Jede Aufgabe wird isoliert in einem \emph{Feature} Branch bearbeitet.
  \item Ist eine Aufgabe (nach den Definitions of Done) bewältigt, wird der
    Feature Branch in den Master Branch gemerged.
  \item Findet sich im Master-Branch ein Bug, wird ein \emph{Hotfix} Branch
    eröffnet, wo der Bug schnell (und zunächst auch isoliert) behandelt wird,
    bevor er auch wieder in den Master Branch gemerged wird.
\end{itemize}

So erhoffen wir uns eine produktiven Workflow, wo Aufgaben isoliert behandelt
werden. So interferieren Aufgaben nicht miteinander und wir können davon
ausgehen, dass im sich Master Branch immer sauberer, stabiler Code befindet.

\subsection{Style Guide}

Ein Software-Projekt lässt sich sehr leicht mit einem Sandwich vergleichen. Es
ist gut bzw. notwendig, dass das Sandwich gut schmeckt und der Code korrekt
läuft. Jedoch isst jeder von uns doch viel lieber ein Sandwich, dass auch
schmackhaft aussieht. Ebenso sollte Code entsprechenden Standards folgen. Wir
haben uns hierbei entschieden, den Style Guide von Google\footnote{\url{https://google.github.io/styleguide/cppguide.html}} zu
befolgen. Dieser umfasst nicht nur kosmetische Vorschriften, sondern schlägt
auch dort, wo es für das selbe Problem zwei Lösungen gibt, eine konkrete vor.

%%% Local Variables:
%%% mode: latex
%%% TeX-master: "../combined/spec"
%%% End:
