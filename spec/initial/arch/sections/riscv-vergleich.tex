\section{Unterschiede zwischen x86 und RISC-V}

Die folgenden Abschnitte beschreiben die wichtigsten Unterschiede zwischen des x86- und RISC-V-Befehlssatzes.
Dadurch soll u.a. der Umstieg für unser Team von x86 zu RISC-V erleichtert werden.

\subsection{Load-/Store-Paradigma}

Der zentrale Unterschied zwischen RISC-V, als RISC-Befehlssatz, und x86, als
CISC-Befehlssatz, ist die Beschränkung der
Zugriffsmöglichkeiten auf den Speicher. In x86 ist es erlaubt, bei fast
jedem Befehl einen oder keinen Speicherzugriff durchzuführen,
zwei in einer Instruktion sind nicht erlaubt.
Wir können beispielsweise
die folgenden wohlgeformten Befehle ausdrücken:

\begin{lstlisting}[style=x86Assembler]
  ; Schreibe den Wert an der Speicheradresse
  ; 0xDEADBEEF in das Register EAX
  mov eax, [0xDEADBEEF]
  ; Addiere den Wert im Register EAX auf den
  ; Wert an der Speicheadresse 0xDEADAFFE
  add [0xDEADAFFE], eax
\end{lstlisting}

In RISC-V wären diese Befehle nicht erlaubt, da es das Load-/Store-Paradigma
befolgt. Das bedeutet, dass es nur zwei Instruktionen gibt, die auf den Speicher
zugreifen können: \lstinline[style=risc-v_Assembler]!LOAD! (Laden) und \lstinline[style=risc-v_Assembler]!STORE! (Speichern). Der erste
Befehl lädt einen Wert an einer Speicheradresse in ein Register. Im
Gegensatz dazu erlaubt \lstinline[style=risc-v_Assembler]!STORE!, einen Wert aus einem Register an eine
Speicheradresse zu legen.

Im RISC-V gibt es mehrere Varianten dieser beiden Befehlsklassen für
verschieden breite Speicherzugriffe. Für \lstinline[style=risc-v_Assembler]!LOAD! wären das:

\begin{itemize}
  \item \lstinline[style=risc-v_Assembler]!LW dest, base, immediate!: Lädt ein Wort (32-Bit für RV32I) an der
    Speicheradresse, die sich aus den oberen 20 Bits der \lstinline[style=risc-v_Assembler]!base! und der
    12-Bit-Konstante \lstinline[style=risc-v_Assembler]!immediate! zusammensetzt.
  \item\lstinline[style=risc-v_Assembler]!LH/LHU dest, base, immediate!: Lädt ein Halbwort
    (engl. \emph{halfword}; 16-Bit für RV32I) an der Speicheradresse, die sich
    aus den oberen 20 Bits der \lstinline[style=risc-v_Assembler]!base! und der 12-Bit-Konstante
    \lstinline[style=risc-v_Assembler]!immediate! zusammensetzt. Für \lstinline[style=risc-v_Assembler]!LH! wird der geladene 16-Bit-Wert
    mit dem Vorzeichen erweitert\footnote{Das bedeutet, dass negative Werte mit gesetztem höchstewertigem Bit/MSB
      mit weiteren Einsen nach links aufgefüllt werden. Positive Werte, deren
      Vorzeichenbit nicht gesetzt ist, werden mit weiteren Nullen aufgefüllt.}, bevor er im
    Zielregister abgelegt wird. Bei \lstinline[style=risc-v_Assembler]!LHU!\footnote{Es gibt kein \lstinline[style=risc-v_Assembler]!LWU!,
      weil ein Wort schon die vollen 32 Bit belegt und man sich somit nicht um
      Erweiterung auf größere Datentypen kümmern muss.} (\emph{load halfword
      unsigned}) wird der Wert hingegen nur mit Nullen aufgefüllt.
  \item \lstinline[style=risc-v_Assembler]!LB/LBU dest, base, immediate! Lädt wie \lstinline[style=risc-v_Assembler]!LH/LHU! einen Wert von
    der gegebenen Speicheradresse in das Zielregister. Hierbei wird jedoch ein
    einziges \textbf{B}yte, also 8-Bit, geladen. Die übrigen 24 Bit werden wieder
    aufgefüllt, entweder mit dem Vorzeichenbit (\lstinline[style=risc-v_Assembler]!LB!) oder mit Nullen
    (\lstinline[style=risc-v_Assembler]!LBU!).
\end{itemize}

Um den gesamten 32-Bit-Adressraum anzusprechen, können wir wie folgt vorgehen:
Zuerst legt man dafür mit \lstinline[style=risc-v_Assembler]!LUI! (\emph{load upper immediate}) einen 20-Bit-Wert
in ein Register und dieses Register dann als Basisregister
verwenden. Auf diese 20 Bit kann man dann eine beliebige 12-Bit-Konstante
aufaddieren:

\begin{lstlisting}[style=risc-v_Assembler]
  ; Wir wollen die Addresse 0xDEADBEEF nach Register x1 laden
  ; Zuerst geben wir die oberen 20 Bit in ein Register
  lui x2, 0xDEADB
  ; Dann benutzen wir x1 als Basis und addieren die unteren 12 Bit
  lw x1, x2, 0xEEF
\end{lstlisting}

Betrachten wir nun \lstinline[style=risc-v_Assembler]!STORE!:

\begin{itemize}
  \item \lstinline[style=risc-v_Assembler]!SW source, base, immediate!: Speichert den 32-Bit-Wert in \lstinline[style=risc-v_Assembler]!source!
    an die Addresse \lstinline[style=risc-v_Assembler]!base + immediate!. Die Konstante \lstinline[style=risc-v_Assembler]!immediate! ist
    hierbei wieder 12 Bit groß.
  \item \lstinline[style=risc-v_Assembler]!SH source, base, immediate! Speichert die unteren 16 Bit aus \lstinline[style=risc-v_Assembler]!source! an die Addresse
    \lstinline[style=risc-v_Assembler]!base + immediate!. Die Konstante \lstinline[style=risc-v_Assembler]!immediate! ist hierbei wieder
    12 Bit groß.
  \item \lstinline[style=risc-v_Assembler]!SB source, base, immediate! Speichert die unteren 8 Bit aus \lstinline[style=risc-v_Assembler]!source! an die Addresse
    \lstinline[style=risc-v_Assembler]!base + immediate!. Die Konstante \lstinline[style=risc-v_Assembler]!immediate! ist hierbei wieder
    12 Bit groß.
\end{itemize}

Theoretisch lassen die obigen Befehle auch \emph{unausgerichtete}
Zugriffe zu. Die Spezifikation merkt hierbei jedoch an,
dass damit nicht die optimale Performance erreicht werden kann.

\subsection{Reduzierter Befehlssatz}

Da es sich bei RISC-V um einen RISC-Befehlssatz handelt, ist dieser im Vergleich
zu x86 wesentlich reduziert. So müssen bestimmte Befehle durch andere äquivalent
ausgedrückt werden. Wir wollen einige Beispiele hierfür betrachten.

Ein anschaulicher Fall ist die \lstinline[style=x86Assembler]!NOP!-Instruktion, welche keine Funktion
besitzt, außer den Befehlszähler um eine Instruktion fortzuschreiten. In CISC-Befehlssätzen
wie x86 ist diese Instruktion meist vorhanden. In RISC-V muss sie
durch eine Addition nachgebildet werden:

\begin{lstlisting}[style=risc-v_Assembler]
  ; NOP in RISC-V
  add x0, x0, 0
\end{lstlisting}

Da das Register x0 fest auf die 0 verdrahtet ist, kann man auf dieses
jede beliebige Zahl addieren, ohne dass etwas passiert.

Das x0-Register ist allgemein in Situationen wie oben oftmals sehr hilfreich. Ein
weiteres Beispiel hierfür ist ein einfacher Sprung ohne Rücksprungadresse, entsprechend des  \lstinline[style=x86Assembler]!JMP!-Befehls im x86-Assembler. (im Gegensatz zu einem Unterprogrammaufruf mit
\lstinline[style=x86Assembler]!CALL!) In RISC-V gibt es nur \lstinline[style=x86Assembler]!CALL!-ähnliche Befehle. Diese sind:

\begin{itemize}
  \item \lstinline[style=risc-v_Assembler]!JAL destination, immediate!: Springt an die Adresse, die man
    erhält, wenn man die 20-Bit-Konstante \lstinline[style=risc-v_Assembler]!immediate! auf den Befehlszähler
    addiert. \lstinline[style=risc-v_Assembler]!immediate! wird hierbei implizit um einen Bit nach links
    geshiftet, wodurch man also nur Vielfache von 2 Byte ansprechen
    kann. Dadurch ergibt sich ein addressierbarer Bereich von $\pm 1$ MiB
    relativ zum Befehlszähler. Die \emph{Rücksprungsadresse} \lstinline[style=risc-v_Assembler]!pc + 4!
    (\lstinline[style=risc-v_Assembler]!pc! = \emph{program counter}) wird dabei in das Register
    \lstinline[style=risc-v_Assembler]!destination! gelegt.
 \item \lstinline[style=risc-v_Assembler]!JALR destination, source, immediate!: Dieser Befehl erlaubt einen Sprung an eine Adresse, die nicht relativ zum Befehlszähler sein muss (sie
   kann es sein, wenn \lstinline[style=risc-v_Assembler]!register! mit \lstinline[style=risc-v_Assembler]!pc! gewählt wird). Hierbei wird
   die 12-Bit-Konstante \lstinline[style=risc-v_Assembler]!immediate! auf den 32-Bit-Wert addiert, der im Register
   \lstinline[style=risc-v_Assembler]!source! liegt. Dieser Wert ergibt die Zieladresse. Die
   Rücksprungadresse, \lstinline[style=risc-v_Assembler]!pc + 4!, wird in das Register \lstinline[style=risc-v_Assembler]!destination!
   gegeben.
\end{itemize}

Um nun einen einfachen Sprung zu machen, benutzt man \lstinline[style=risc-v_Assembler]!JALR! mit \lstinline[style=risc-v_Assembler]!x0! als
destination Register.

Das letzte Beispiel für den reduzierten Befehlssatz wollen wir durch die
bedingten Sprungbefehle geben. In x86 gibt es ein Vielzahl an Möglichkeiten,
Relationen zwischen zwei Werten nach einem Vergleich als Sprungbedingung zu
nutzen, wie z.B. \lstinline[style=x86Assembler]!je, jne, jg, jl, jle! usw. In
RISC-V gibt es nur vier mögliche Prädikate:

\begin{itemize}
  \item \lstinline[style=risc-v_Assembler]!BEQ r1, r2, immediate!: Springe an die Adresse, die man durch
    Addition der Konstante \lstinline[style=risc-v_Assembler]!immediate! mit dem Befehlszähler erhält, wenn
    die Inhalte der Register \lstinline[style=risc-v_Assembler]!r1! und \lstinline[style=risc-v_Assembler]!r2! gleich sind. Die Konstante
    ist hierbei wieder 12-Bit- und in 2-Byte-Schritten, wodurch man einen
    Spannweite von $\pm 4$ KiB relativ zum Befehlszähler erreichen kann.
  \item \lstinline[style=risc-v_Assembler]!BNE r1, r2, immediate!: Wie \lstinline[style=risc-v_Assembler]!BEQ!, aber springt dann, wenn
    die Registerinhalte ungleich sind.
  \item \lstinline[style=risc-v_Assembler]!BLT/BLTU r1, r2, immediate!: Springt an die wie oben berechnete
    Adresse, wenn der Wert in \lstinline[style=risc-v_Assembler]!r1! kleiner dem in \lstinline[style=risc-v_Assembler]!r2! ist. Hierbei
    kann der Vergleich vorzeichenbehaftet (\lstinline[style=risc-v_Assembler]!BLT!) oder vorzeichenlos
    (\lstinline[style=risc-v_Assembler]!BLTU!) sein.
  \item \lstinline[style=risc-v_Assembler]!BGE/BGEU r1, r2, immediate!: Springt, wenn \lstinline[style=risc-v_Assembler]!r1! größer oder
    gleich r2 ist.
\end{itemize}

Wie man sehen kann, gibt es kein \lstinline[style=risc-v_Assembler]!BLE/BLEU! oder \lstinline[style=risc-v_Assembler]!BGT/BGTU!. Diese beiden
Befehle, welche es in x86 sicher gäbe, müssen also durch Vertauschen
der Operanden im Vergleich ausgedrückt werden.

\subsection{Fehlende Flags}

In x86 besitzt das Statusregister eine große Bedeutung, schließlich ist es ohne dieses unmöglich, bedingte Sprünge auszuführen: Vergleichsbefehle setzen bestimmte Bits in diesem Register, die die Sprungbefehle wie \lstinline[style=x86Assembler]!JGE! anschließend abfragen.
In RISC-V existiert jedoch kein solches Statusregister. Ebenso wenig gibt es keine
Vergleichsoperation wie \lstinline[style=x86Assembler]!CMP!. Bedingungen für Sprünge werden hingegen, wie schon oben vorgeführt, explizit für jeden Sprungbefehl berechnet. Daher wird das folgende Stück x86-Code:

\begin{lstlisting}[style=x86Assembler]
  ; Vergleich das Register EAX mit EBX
  cmp eax, ebx
  ; Springe auf die Marke, wenn EAX gleich EBX war
  je marke
\end{lstlisting}

zu folgendem RISC-V-Code:

\begin{lstlisting}[style=risc-v_Assembler]
  ; Vergleiche das Register x1 mit dem Register x2 und springe
  ; an die Adresse pc + 0xABC, falls die Bedingung galt
  beq x1, x2, 0xABC
\end{lstlisting}

\subsection{Aufrufkonventionen}

Ebenso wie das Statusregister ist der Stapel/Stack ein wichtiger Bestandteil der x86
Architektur. Ebenso wie das Statusregister gibt es aber auch keinen Stack in
RISC-V. Da die Assembler (bzw. C) \emph{Aufrufkonvention}\footnote{Eine
  \emph{Aufrufkonvention} (Englisch: Calling Convention) ist eine Vereinbarung, wie Funktionsaufrufe verlaufen
  sollten. Hierbei interessiert man sich dafür, wie Parameter übergeben werden,
  welche Register durch die Funktion verändert werden dürfen, sowie dafür, wie
  Rückgabewerte zurückgegeben werden.} starken Gebrauch des Stacks macht,
muss die Aufrufkonvention für RISC-V vollkommen überdacht werden.

Die offizielle RISC-V-Spezifikation gibt nur an, in welches Register die
Rücksprungadresse bei einem Funktionsaufruf gegeben werden soll, nämlich
\lstinline[style=risc-v_Assembler]!x1!. Das Register, in welches man die Rücksprungadresse gibt, nennt man
hierbei auch \emph{Link Register}. Nach der offiziellen Calling Convention würde
man also stets nach folgendem Schema Funktionsaufrufe durchführen:

\begin{lstlisting}[style=risc-v_Assembler]
  ; jump and link register, x1 als Link Register
  ; Basis (20-Bit Register) + Offset (12-Bit Konstante) ergibt die Adresse
  jalr x1, <basis>, <offset>
\end{lstlisting}

\subsection{Registerorganisation}

Ein weiterer Unterschied zwischen den x86- und RISC-V-Befehlssätzen ist die
Registerorganisation. In x86 gibt es sechs \emph{Allzweckregister}
\lstinline[style=x86Assembler]!EAX! bis \lstinline[style=x86Assembler]!EDX!, \lstinline[style=x86Assembler]!ESI! und \lstinline[style=x86Assembler]!EDI!. Auch Teile der Register sind dabei adressierbar.
So ist es beispielsweise
möglich, die unteren 8 Bit des Registers \lstinline[style=x86Assembler]!EAX! als \lstinline[style=x86Assembler]!AL!
anzusprechen. Die Architektur hat natürlich auch noch einen Befehlszähler,
welcher auch in einem Register gespeichert ist. Dieses Register ist aber vom
Assemblerprogrammierer nicht ansprechbar.

Die Registerorganisation von RISC-V unterscheidet sich stark von der der x86-Architektur. Hier existieren
32 Allzweckregister, welche je nach Variante von RISC-V jeweils 32 oder
64 Bit breit sind. Wir erkennen drei Unterschiede:

\begin{enumerate}
  \item Das erste Register, \lstinline[style=risc-v_Assembler]!x0!, ist fest zur Null verdrahtet. Es enthält
    also immer den Wert 0. Somit sind also nur 31 der 32 Register
    \emph{wirklich} zu allen Zwecken verwendbar.
  \item Die Register sind atomar, also nicht weiter aufteilbar oder teilweise
    ansprechbar.
  \item Zusätzlich zu den 31 Allzweckregistern ist auch das
    \textbf{Befehlszählerregister} \lstinline[style=risc-v_Assembler]!pc! für den Programmierer sichtbar.
\end{enumerate}

\subsection{Multiplikation und Division}

Zum Schluss wollen wir noch die Unterschiede zu x86 bezüglich der Multiplikation
und der Division erläutern. Hierbei sei angemerkt, dass Multiplikations- und
Divisionsinstruktionen nicht Teil des \emph{Base Integer Instruction Set} RV32I
bzw. RV64I sind, der minimalsten Standardausgabe von RISC-V. Vielmehr sind
diese zwei Operationsfamilien in der \emph{M}-Standarderweiterung des Basis-Befehlssatzes
enthalten.

Im Wesentlichen unterscheiden sich x86 und RISC-V in der Behandlung
des Übertrages bei der Multiplikation und des Restes bei der Division.
x86 speichert bei der Multiplikation der beiden 32-Bit Register \lstinline[style=x86Assembler]!EAX! und
(in diesem Beispiel) \lstinline[style=x86Assembler]!EBX! die oberen 32-Bit des Ergebnisses\footnote{Die
  Multiplikation zweier 32-Bit Werte ergibt einen 64-Bit Wert.} in \lstinline[style=x86Assembler]!EDX!
und die unteren in \lstinline[style=x86Assembler]!EAX!. In RISC-V gibt es für die Multiplikation hingegen
zwei verschiedene Befehle, welche zusammen das Problem des Übertrags behandeln:

\begin{itemize}
  \item \lstinline[style=risc-v_Assembler]!MUL dest, r1, r2!: Multipliziert die Inhalte der Register
    \lstinline[style=risc-v_Assembler]!r1! und \lstinline[style=risc-v_Assembler]!r2! und speichert \emph{die unteren 32 Bit} des
    resultierenden 64-Bit-Produkts im Zielregister \lstinline[style=risc-v_Assembler]!dest!.
  \item \lstinline[style=risc-v_Assembler]!MULH/MULHU/MULHSU dest, r1, r2!: Multipliziert die Inhalte der Register
    \lstinline[style=risc-v_Assembler]!r1! und \lstinline[style=risc-v_Assembler]!r2! und speichert \emph{die oberen 32 Bit} des
    resultierenden Produkts im Zielregister \lstinline[style=risc-v_Assembler]!dest!. Da der Sign-Bit des
    64-Bit-Produkts im letzten Bit der höheren 32 Bit liegt, gibt es für diesen
    Befehl noch drei Optionen, um die Beachtung des Vorzeichens der
    beiden Operanden zu kontrollieren. So führt \lstinline[style=risc-v_Assembler]!MULH! eine
    $\text{Vorzeichen} \cdot \text{Vorzeichen}$ Multiplikation durch, \lstinline[style=risc-v_Assembler]!MULHU!
    eine $\text{Vorzeichenlos} \cdot \text{Vorzeichenlos}$-Operation, wohingegen
    \lstinline[style=risc-v_Assembler]!MULHSU! ein $\text{Vorzeichen} \cdot \text{Vorzeichenlos}$-Produkt bildet.
\end{itemize}

Bei der Division eines 64-Bit-Wertes (\lstinline[style=x86Assembler]!EDX:EAX!) durch einen 32-Bit-Wert speichert x86 den Rest der Division in
\lstinline[style=x86Assembler]!EDX! und den ganzzahligen Quotienten in \lstinline[style=x86Assembler]!EAX!. In der
RISC-V-Befehlssatz existieren hierfür wieder zwei separate Befehle:

\begin{itemize}
\item \lstinline[style=risc-v_Assembler]!DIV/DIVU dest, r1, r2!: Dividiert den Inhalt des Registers \lstinline[style=risc-v_Assembler]!r1!
  durch den Inhalt des Registers \lstinline[style=risc-v_Assembler]!r2! und speichert den ganzzahligen 32 Bit
  \emph{Quotienten} im Zielregister \lstinline[style=risc-v_Assembler]!dest!. \lstinline[style=risc-v_Assembler]!DIV! interpretiert hierbei die Operanden als vorzeichenbehaftet, \lstinline[style=risc-v_Assembler]!DIVU! als vorzeichenlos.
  \item \lstinline[style=risc-v_Assembler]!REM/REMU dest, r1, r2!: Führt die selbe Operation wie \lstinline[style=risc-v_Assembler]!DIV/DIVU!
    durch, speichert jedoch den 32 Bit \emph{Rest} der Division im Zielregister
    \lstinline[style=risc-v_Assembler]!dest!. Wieder existieren zwei Varianten, eine für vorzeichenbehaftete und eine für vorzeichenlose Berechnungen.
\end{itemize}

Für eine Multiplikation oder Division des x86-Befehlssatzes müsste man also zwei Befehls des RISC-V-Befehlssatzes verwenden.
 Die offizielle RISC-V-Spezifikation empfiehlt hierbei, zuerst
\lstinline[style=risc-v_Assembler]!MULH/MULHU/MULHSU! und dann \lstinline[style=risc-v_Assembler]!MUL! zu berechnen, bzw. zuerst \lstinline[style=risc-v_Assembler]!DIV/DIVU!
und dann \lstinline[style=risc-v_Assembler]!REM/REMU!.

\begin{lstlisting}[style=risc-v_Assembler]
  ; Multipliziere register x1 und x2
  ; Intepretiere x1 als vorzeichenbehaftet, x2 nicht
  ; Speichere die oberen 32-Bit des Produkts in x3
  mulhsu x3, x1, x2

  ; Erhalte nun noch die unteren 32-Bit
  mul x4, x1, x2

  ; Dividiere x5 durch x6
  ; Speichere den ganzzahligen 32-Bit Quotienten in x7
  div x7, x5, x6

  ; Berechne noch den Rest und lege ihn in x8 ab
  div x8, x5, x6
\end{lstlisting}
