\section{Unterschiede zwischen x86 und RISC-V}

In den folgenden Absätzen wollen wir die wichtigsten Unterschiede zwischen dem
x86 und RISC-V Befehlssatz erläutern. Wir möchten dadurch den übrigen
Teammitgliedern den Umstieg von x86 erleichtern sowie unsere eigenen
Beobachtungen festhalten.

\subsection{Load/Store Paradigma}

Ein grundlegender Unterschied zwischen RISC-V, als
reduced-instruction-set-comuting (RISC) ISA, und x86, als
complex-instruction-set-computing (CISC) Befehlssatz, ist die Beschränkung der
Zugriffsmöglichkeiten auf den Speicher. In x86 ist es erlaubt, bei so gut wie
jedem Befehl einen Speicherzugriff als Operand anzugeben, wobei aber die
Beschränkung auf nur einen solchen Zugriff gilt. So können wir beispielsweise
die folgenden wohl-geformten Befehle ausdrücken:

\begin{lstlisting}
  ; Schreibe den Wert an der Speicheradresse
  ; 0xDEADBEEF in das Register EAX
  mov eax, [0xDEADBEEF]
  ; Addiere den Wert im Register EAX auf den
  ; Wert an der Speicheadresse 0xDEADAFFE
  add [0xDEADAFFE], eax
\end{lstlisting}

In RISC-V wären diese Befehle nicht erlaubt, da es das LOAD/Store Paradigma
befolgt. Das bedeutet, dass es nur zwei Instruktionen gibt, die auf den Speicher
zugreifen können: \code{LOAD} und \code{STORE} (in Pseudo-Code). Der erste
Befehl lädt ein einen Wert an einer Speicheradresse in ein Register. Im
Gegensatz dazu erlaubt \code{STORE}, einen Wert aus einem Register an eine
Speicheradresse zu legen.

Im RISC-V gibt es mehrere Varianten dieser beiden Befehlsklassen, für
verschieden breite Speicherzugriffe. Für \code{LOAD} wären das:

\begin{itemize}
  \item \code{LW dest, base, immediate}: Lädt ein Wort (32-Bit für RV32I) an der
    Speicheradresse, die sich aus den oberen 20 Bit von \code{base} und der
    11-Bit Konstante \code{immediate} zusammensetzt.
  \item \code{LH[U] dest, base, immediate}: Lädt ein Halbwort
    (engl. \emph{halfword}; 16-Bit für RV32I) an der Speicheradresse, die sich
    aus den oberen 20 Bit von \code{base} und der 11-Bit Konstante
    \code{immediate} zusammensetzt. Für \code{LH} wird der geladene 16-Bit Wert
    sign-extended\footnote{Das bedeutet, dass negative Werte mit gesetztem MSB
      mit weiteren Einsen nach links aufgefüllt werden. Positive Werte, dessen
      Sign-Bit gelöscht ist, werden mit weiteren Nullen gepaddet.}, bevor er im
    Zielregister abgelegt wird. Bei \code{LHU}\footnote{Es gibt kein \code{LWU},
      weil ein Wort schon die vollen 32-Bit belegt und man sich somit nicht um
      Erweiterung auf größere Datentypen kümmern muss.} (\emph{load halfword
      unsigned}) wird der Wert hingegen nur mit Nullen aufgefüllt.
  \item \code{LB[U] dest, base, immediate} Lädt wie \code{LH[U]} einen Wert von
    der gegebenen Speicheradresse in das Zielregister. Hierbei wird jedoch ein
    enziger \textbf{B}yte, also 8-Bit, geladen. Die übrigen 24 Bit werden wieder
    aufgefüllt, entweder mit dem Sign-Bit (\code{LB}) oder mit Nullen
    (\code{LBU}).
\end{itemize}

Wie kann man mit diesen Befehlen nun den gesamten 32-Bit Adressraum ansprechen?
Zuerst würde man dafür mit \code{LUI} (\emph{load upper immediate}) einen 20-Bit
Wert in ein Register legen und dieses Register dann als Basisregister
verwenden. Auf diese 20-Bit kann man dann eine beliebige 12-Bit Konstante
aufaddieren:

\begin{lstlisting}
  ; Wir wollen die Addresse 0xDEADBEEF nach Register x1 laden
  ; Zuerst geben wir die oberen 20 Bit in ein Register
  lui x2, 0xDEADB
  ; Dann benutzen wir x1 als Basis und addieren die unteren 12 Bit
  lw x1, x2, 0xEEF
\end{lstlisting}

Nun für Store:

\begin{itemize}
  \item \code{SW source, base, immediate}: Speichert den 32-Bit Wert in \code{source}
    an die Addresse \code{base + immediate}. Die Konstante \code{immediate} ist
    hierbei wieder 11-Bit.
  \item \code{SH} Speichert die unteren 16-Bit aus \code{source} an die Addresse
    \code{base + immediate}. Die Konstante \code{immediate} ist hierbei wieder
    11-Bit.
  \item \code{SB} Speichert die unteren 8-Bit aus \code{source} an die Addresse
    \code{base + immediate}. Die Konstante \code{immediate} ist hierbei wieder
    11-Bit.
\end{itemize}

Theoretisch erlauben die obigen Befehle auch \emph{unausgerichtete}
Zugriffe. Die RISC-V Architektur erlaubt solche Zugriffe, wobei die
Spezifikation jedoch anmerkt, dass sie für optimale Performance vermieden werden
sollten.

\subsection{Reduzierter Befehlssatz}

Da es sich bei RISC-V um einen RISC Befehlssatz handelt, ist dieser im Vergleich
zu x86 wesentlich reduziert. So müssen bestimmte Befehle durch andere äquivalent
ausgedrückt werden. Wir wollen einige Beispiele hierfür betrachten.

Ein anschaulicher Fall ist die \code{NOP} Instruktion, welche keine Funktion
hat, außer den Befehlszähler um eine Instruktion fortzuschreiten. In CISC
Befehlssätzen wie x86 ist diese Instruktion meist gegeben. In RISC-V, muss sie
durch eine Addition nachgebildet werden:

\begin{lstlisting}
  ; NOP in RISC-V
  add x0, x0, 0
\end{lstlisting}

Das Register x0 ist fest auf die Null verdrahtet. Insofern kann man auf sie
jede beliebige Zahl addieren, ohne dass etwas passiert.

Das x0 ist allgemein oftmals für Späße wie den obigen zu gebrauchen. Ein
weiteres Beispiel hierfür ist ein einfacher Sprung. Mit \emph{einfach} meinen
wir hier einen Sprung wie \code{jmp} in x86, der nicht mit einer
Rücksprungadresse verbunden ist, im Gegensatz zu einem Unterprogrammaufruf mit
\code{call}. In RISC-V gibt es nur \code{call}-ähnliche Befehle. Diese sind:

\begin{itemize}
  \item \code{JAL destination, immediate}: Springt an die Adresse, die man
    erhält, wenn man die 20-Bit Konstante \code{immediate} auf den Befehlszähler
    addiert. \code{immediate} wird hierbei implizit um einen Bit nach links
    geshiftet, wodurch man also nur Vielfache von 2 Byte ansprechen
    kann. Dadurch erhält man einen addressierbaren Bereich von $\pm 1$ MiB
    relativ zum Befehlszähler. Die \emph{Rücksprungsadresse} \code{pc + 4}
    (\code{pc} = \emph{program counter}) wird dabei in das Register
    \code{destination} gelegt.
 \item \code{JALR destination, source, immediate}: Dieser Befehl erlaubt einen
   Sprung an eine Adresse, die nicht relativ zum Befehlszähler sein muss (sie
   kann es sein, wenn \code{register} mit \code{pc} gewählt wird). Hierbei wird
   die 12-Bit \code{immediate} auf den 32-Bit Wert addiert, der im Register
   \code{source} liegt. Dieser Wert ergibt die Zieladresse. Die
   Rücksprungadresse, \code{pc + 4}, wird in das Register \code{destination}
   gegeben.
\end{itemize}

Um nun einen einfachen Sprung zu machen, benutzt man \code{JALR} mit \code{x0} als
destination Register.

Das letzte Beispiel für den reduzierten Befehlssatz wollen wir durch die
bedingten Sprungbefehle geben. In x86 gibt es ein Vielzahl an Möglichkeiten,
Relationen zwischen zwei zwei Werten nach einem Vergleich als Sprungbedingung zu
nutzen. So gab es \code{je, jne, jg, jl, jle} und noch weitere Varianten. In
RISC-V gibt es nur vier mögliche Prädikate:

\begin{itemize}
  \item \code{BEQ r1, r2, immediate}: Springe an die Adresse, die man durch
    Addition der Konstante \code{immediate} mit dem Befehlszähler erhält, wenn
    die Inhalte der Register \code{r1} und \code{r2} gleich sind. Die Konstante
    ist hierbei wieder 12-Bit und in 2-Byte Schritten, wodurch man einen
    Spannweite von $\pm 4$ KiB relativ zum Befehlszähler erreichen kann.
  \item \code{BNE r1, r2, immediate}: Wie \code{BEQ}, aber springt dann, wenn
    die Registerinhalte ungleich sind.
  \item \code{BLT[U] r1, r2, immediate}: Springt an die wie oben berechnete
    Adresse, wenn der Wert in \code{r1} kleiner dem in \code{r2} ist. Hierbei
    kann der Vergleich vorzeichenbehaftet sein (\code{BLT}) oder nicht
    (\code{BLTU}).
  \item \code{BGE[U] r1, r2, immediate}: Springt, wenn \code{r1} größer oder
    gleich r2 ist.
\end{itemize}

Wie man sehen kann, gibt es kein \code{BLE[U]} oder \code{BGT[U]}. Diese beiden
Befehle, welche es in x86 sicher gäbe, muss man also einfach durch Vertauschen
der Operanden im Vergleich ausdrücken.

\subsection{Fehlende Flags}

In x86 ist das Statusregister ein wichtiger Bestandteil einer jeden
Programmieraufgabe. Ohne dieses Register wäre es nicht möglich, bedingte Sprünge
zu machen. Das ist so, weil Vergleiche bestimmte Bits in diesem Register setzen,
welche folglich von Sprungbefehlen wie \code{JGE} abgefragt werden. In RISC-V
gibt es jedoch kein solches Statusregister. Ebenso gibt es keinen
Vergleichsoperator \code{CMP}. Bedingungen für Sprünge werden nämlich, wie schon
weiter oben gezeigt, explizit für jeden Sprungbefehl berechnet. Daher wird der
folgende x86 Code:

\begin{lstlisting}
  ; Vergleich das Register EAX mit EBX
  cmp eax, ebx
  ; Springe auf die Marke, wenn EAX gleich EBX war
  je marke
\end{lstlisting}

zu folgendem RISC-V Code:

\begin{lstlisting}
  ; Vergleiche das Register x1 mit dem Register x2 und springe
  ; an die Adresse pc + 0xABC, falls die Bedingung galt
  beq x1, x2, 0xABC
\end{lstlisting}

\subsection{Calling Conventions}

Ebenso wie das Statusregister ist der Stack ein wichtiger Bestandteil der x86
Architektur. Ebenso wie das Statusregister gibt es aber auch keinen Stack in
RISC-V. Da die Assembler (bzw. C) \emph{Calling Convention}\footnote{Eine
  \emph{Calling Convention} ist eine Abmachung, wie Funktionsaufrufe verlaufen
  sollten. Hierbei interessiert man sich dafür, wie Parameter übergeben werden,
  welche Register durch die Funktion verändert werden dürfen sowie dafür, wie
  Rückgabewerte zurückgegeben werden.} starken Gebrauch des Stacks macht, muss
die Calling Convention für RISC-V vollkommen überdacht werden.

Die offizielle RISC-V Spezifikation gibt nur an, in welches Register die
Rücksprungadresse bei einem Funktionsaufruf gegeben werden soll, nämlich
\code{x1}. Das Register, in welches man die Rücksprungadresse gibt, nennt man
hierbei auch \emph{Link Register}. Nach der offiziellen Calling Convention würde
man also stets nach folgendem Schema Funktionsaufrufe durchführen:

\begin{lstlisting}
  ; jump and link register, x1 als Link Register
  ; Basis (20-Bit Register) + Offset (11-Bit Konstante) ergibt die Adresse
  jalr x1, <basis>, <offset>
\end{lstlisting}

\subsection{Registerorganisation}

Ein weiterer Unterschied zwischen den x86 und RISC-V ISAs ist die
Registerorganisation. In x86 gibt es sechs \emph{general-purpose} Register
\code{EAX} bis \code{EDX}, \code{ESI} und \code{EDI}. Vor Allem interessant ist,
dass diese Register auch teilweise adressierbar sind. So ist es beispielsweise
möglich, die unteren 8-Bit des Registers \code{EAX} als \code{AL}
anzusprechen. Die Architektur hat natürlich auch noch einen Befehlszähler,
welcher auch in einem Register gespeichert ist. Dieses Register ist aber vom
Assemblerprogrammierer nicht ansprechbar.

Die Registerorganisation von RISC-V ist in manchen Belangen sehr anders. Es gibt
32 general-purpose Register, welche je nach Variante von RISC-V jeweils 32 oder
64 Bit breit sind. Wir erkennen drei Unterschiede:

\begin{enumerate}
  \item Das erste Register, \code{x0}, ist fest zur Null verdrahtet. Es enthält
    also immer den Wert Null. Somit sind also nur 31 der 32 Register
    \emph{wirklich} general-purpose.
  \item Die Register sind atomar, also nicht weiter aufteilbar oder teilweise
    ansprechbar.
  \item Zusätzlich zu den 31 general-purpose Registern ist auch das
    \textbf{Befehlszählerregister} \code{pc} für den Programmierer sichtbar.
\end{enumerate}

\subsection{Multiplikation und Division}

Letztlich wollen wir noch die Unterschiede zu x86 bezüglich der Multiplikation
und Division erläutern. Hierbei sei angemerkt, dass Multiplikations- und
Divisionsinstruktionen nicht Teil der \emph{Base Integer Instruction Set} RV32I
bzw. RV64I sind, also der minimalsten Standardausgabe von RISC-V. Vielmehr sind
diese zwei Operationsfamilien in der \emph{M}-Standarderweiterung der Base ISA
enthalten.

Im wesentlichen x86 und RISC-V diesbezüglich anders, wie sie den Übertrag bei
der Multiplikation und den Rest bei der Division behandeln. Bekanntlich
speichert x86 bei der Multiplikation der beiden 32-Bit Register \code{EAX} und
\code{EBX} die oberen 32-Bit des Ergebnisses\footnote{Man erinnere sich: die
  Multiplikation zweier 32-Bit Werte ergibt einen 64-Bit Wert.} in \code{EDX}
und die unteren in \code{EAX}. In RISC-V gibt es für die Multiplikation hingegen
zwei verschiedene Befehle, welche zusammen das Problem des Übertrags behandeln:

\begin{itemize}
  \item \code{MUL dest, r1, r2}: Multipliziert die Inhalte der Register
    \code{r1} und \code{r2} und speichert \emph{die unteren 32-Bit} des
    resultierenden 64-Bit Produkts im Zielregister \code{dest}.
  \item \code{MULH[[S]U] dest, r1, r2}: Multipliziert die Ihhalte der Register
    \code{r1} und \code{r2} und speichert \emph{die oberen 32-Bit} des
    resultierenden Produkts im Zielregister \code{dest}. Da der Sign-Bit des
    64-Bit Produkts im letzten Bit der höheren 32-Bit liegt, gibt es für diesen
    Befehl noch drei Optionen, um die Beachtung der Vorzeichenbehaftung der
    beiden Operanden zu kontrollieren. So führt \code{MULH} eine
    $signed \cdot signed$ Multiplikation durch, \code{MULHU}
    eine $unsigned \cdot unsigned$ Operation wohingegen
    \code{MULHSU} ein $signed \cdot unsigned$ Produkt bildet.
\end{itemize}

Bei der Division zweier 32-Bit Werte speichert x86 den Rest der Division in
\code{EDX} und den ganzzahligen Quotienten im ersten Operandenregister. In der
RISC-V ISA gibt es hierfür wieder zwei separate Befehle:

\begin{itemize}
\item \code{DIV[U] dest, r1, r2}: Dividiert den Inhalt des Registers \code{r1}
  durch den Inhalt des Registers \code{r2} und speichert den ganzzahligen 32-Bit
  \emph{Quotienten} im Zielregister \code{dest}. \code{DIVU} interpretiert die
  Operanden hierbei als nicht vorzeichenbehaftete Werte, \code{DIV} schon.
  \item \code{REM[U] dest, r1, r2}: Führt die selbe Operation wie \code{DIV[U]}
    durch, speichert jedoch den 32-Bit \emph{Rest} der Division im Zielregister
    \code{dest}. Wieder gibt es eine Variante mit Beachtung von Vorzeichen und
    eine ohne.
\end{itemize}

Für sowohl Multiplikation als auch Division müsste man also zwei separate
Befehle ausführen, um das ganze Ergebnis eines Produkt bzw. einer Division zu
erhalten. Die offizielle RISC-V Spezifikation empfiehlt hierbei, zuerst
\code{MULH[[S]U]} und dann \code{MUL} zu berechnen, bzw. zuerst \code{DIV[U]}
und dann \code{REM[U]}.

\begin{lstlisting}
  ; Multipliziere register x1 und x2
  ; Intepretiere x1 als vorzeichenbehaftet, x2 nicht
  ; Speichere die oberen 32-Bit des Produkts in x3
  mulhsu x3, x1, x2

  ; Erhalte nun noch die unteren 32-Bit
  mul x4, x1, x2

  ; Dividiere x5 durch x6
  ; Speichere den ganzzahligen 32-Bit Quotienten in x7
  div x7, x5, x6

  ; Berechne noch den Rest und lege ihn in x8 ab
  div x8, x5, x6
\end{lstlisting}