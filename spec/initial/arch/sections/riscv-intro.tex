\section{Die RISC-V ISA}

RISC-V ist eine neue Open-Source-ISA (instruction set architecture), die
ursprünglich für Lern- und Forschungsziele entwickelt wurde.

Es gibt zwei Versionen mit 32-Bit und 64-Bit Architektur. Wir werden uns
zuerst ausschließlich für die erste Variante interessieren.

Die Architektur ist in Module aufgeteilt, mit Ausnahme des Basis-Integer-Moduls ''I'', %??
das in allen Implementierungen vorhanden sein muss. Es enthält ganzzahlige
Rechen-, Lade-, Store- und Kontrollflussinstruktionen. Es kann durch folgende
Standardmodule erweitert werden:

\begin{itemize}
\item ''M'' ganzzahlige Multiplikation und Division
\item ''A'' atomicity für inter-processor Synchronization
\item ''F'' und ''D'' für floats und doubles entsprechend
\end{itemize}

Die Instruktionen der Standard ISA sind 32-Bit lang und auf die
32-Bit-Grenzen ausgerichtet. Jedoch unterstützt Risc-V auch Befehle variabler Länge, die
aus einer beliebigen Anzahl von 16-Bit-Parcels bestehen können.

Standardmäßig wird Little-Endian als Speicherformat verwendet.

Die RISV-V ISA hat 31 Allzweckregister (x1 - x31). Zusätzlich existiert ein Register x0, der mit der Konstante 0 hardwired ist, und einen Programmzähler. Sie sind alle 32 Bit breit.

\subsection{Integer-Basis-Instruktionen}

Keine Integer Rechenbefehle verursachen arithmetische Ausnahmen.

\subsubsection{Register-Immediate}

\begin{lstlisting}[style=risc-v_Assembler]
ADDI rd, rs1, 0  ; addiert 12-bit immediate mit dem Register r1, Ziel: rd
SLTI rd, rs1, 0  ; set less than immediate, rd = 1, falls rs1 < 0, sonst rd=0
SLTIU rd, rs1, 1 ; dasselbe, aber unsigned
\end{lstlisting}

ANDI, ORI, XORI sind logische Operationen, die AND, OR und XOR entsprechen. Ihr
Befehlsformat entspricht dem obigen.

Es gibt logische und arithmetische Shifts (SLLI, SRLI, SLAI, SRAI). Die Anzahl
der geshifteten Bits ist gleich der Zahl, die aus den ersten fünf Bits des
Immediate entsteht.

In den nächsten beiden Befehlen ist der Immediate 20 Bit lang, anstatt 12.

\begin{lstlisting}[style=risc-v_Assembler]
LUI dest, immediate  ; load upper immediate
AUIPC dest, immediate  ; add upper immediate to pc
\end{lstlisting}

LUI platziert den Immediate in die höheren 20 Bits des Zielregisters, die unteren 12 Bits werden genullt.

AUIPC addiert den Immediate mit den höheren 20 Bits des Instruktionszählers und speichert das Ergebnis in das Zielregister.

\subsubsection{Register-Register}

\begin{lstlisting}[style=risc-v_Assembler]
ADD/SLT/SLTU dest, src1, src2
AND/OR/XOR   dest, src1, src2
SLL/SRL      dest, src1, src2
/SRA      dest, src1, src2

SLTU rd, x0, rs2 ; setzt rd zu 0 nur wenn rs2 = 0
\end{lstlisting}

\textbf{NOP-Instruktion}

\begin{lstlisting}[style=risc-v_Assembler]
ADDI x0, x0, 0
\end{lstlisting}

\subsubsection{Kontrolltransfer-Instruktionen}

\begin{lstlisting}[style=risc-v_Assembler]
JAL rd, immediate  ; rd = (pc + 4), Immediate 20 bit lang, range +-1MiB
\end{lstlisting}

JAL (jump and link) setzt den Befehlszähler auf den Wert (pc + immediate) und
der vorherige Wert des (Befehlszählers + 4) wird in das Register rd gespeichert,
der als eine Rücksprungadresse verwendet werden kann. Nach den Konventionen
benutzt man dafür x1 und, falls JAL ein unbedingter Sprung ist, dann x0.

\begin{lstlisting}[style=risc-v_Assembler]
JALR rd, rs1, immediate  ; rd = (pc + 4), Immediate 12 bit
\end{lstlisting}

JALR setzt den Befehlszähler auf (rs1 + imm). Zusammen mit dem Befehl LUI kann
man auf eine beliebige Adresse im Adressraum springen.

\subsubsection{Bedingte Verzweigungen}

Ein wichtiger Aspekt der Risc-V ISA ist, dass es keine Flags gibt. Stattdessen
werden Branchbefehle verwendet.

\begin{lstlisting}[style=risc-v_Assembler]
BEQ/BNE src1, src2, offset ; branch equal/not equal
BLT[U]/BGE[U]  src1, src2, offset  ; branch less/greater than
\end{lstlisting}

Falls die Bedingung erfüllt ist, wird der Offset zu dem Befehlszähler addiert.

\subsubsection{Load/Store Instruktionen}

\begin{lstlisting}[style=risc-v_Assembler]
LW/LH/LB rd, rs1, offset ; kopiert 32/16/8 bits vom Speicher von der Adresse (rs1+offset)
SW/SH/SB rd, rs1, offset ; kopiert den Wert vom rd in den Speicher
\end{lstlisting}

Es gibt auch LHU und LBU (analog SHU, SBU). Der Unterschied ist, dass der
kopierte 16/8 Bit Wert mit Nullen auf 32 Bits erweitert wird, wohingegen bei normalen LH
und LB mit dem Vorzeichen erweitert wird.
