\section{Schnittstelle}

Das Ziel des Simulators ist es, nicht nur RISC-V, sondern dank eines modularen Aufbaus auch andere Befehlssätze zu unterstützen. Dazu aber ist es wichtig, dass wir zwischen der abstrakten, allgemeinen Repräsentation der Architektur und der konkreten Implementierung dieser unterscheiden. Die Schnittstelle zwischen beiden Schichten ist somit entscheidend.

Die abstrakte Architektur wird dabei durch eine abstrakte Oberklasse realisiert, die konkreten Architekturen erben dann von dieser. Die Instanz von einer konkreten Architektur ist für sich abgeschlossen: Sie hat ihren eigenen Speicher, ihre eigenen Register und Statusflags und bietet über Methoden der Oberklasse Zugriff darauf. Desweiteren legt jede konkrete Architektur die Eigenschaften der Register, des Speichers und der Statusflags fest. Darunter fallen z.B.:

\begin{itemize}
\item Ausrichtung, Größe und Breite des Speichers und der Register
\item Name, Aliase, Beschreibung und Bedeutung der Register und Statusflags
\end{itemize}

Um architekturunabhängig Instruktionen ausführen und validieren zu können, wird eine Factory für Knoten des abstraken Syntaxbaums (Abstract Syntax Tree, kurz AST) angeboten. Diese Funktionalität wird vom Parser benötigt, um einen AST aus dem Assemblertext generieren zu können. Jeder Knotentyp bietet dabei eine Methode zur Ausführung bzw. zur Validierung des jeweiligen Befehls und dessen Kinder an. Diese Methoden werden über Unterklassen der Knotentypen in jeder konkreten Architektur realisiert. Mittels dieser Methode wird also eine zweischrittige Validierung eines Befehls umgesetzt: Der Parser prüft auf Fehler syntaktischer Art, die (konkrete) Architektur auf semantische Korrektheit der Argumente bzw. des Befehlsaufbaus.

Grundsätzlich erwartet jede Architektur den AST in einer standardisierten Form. Dabei ist die Wurzel des AST immer der Befehlsknoten, der den Typ der Instruktion angibt. Die Kinder der Wurzel spezifizieren die Argumente, wobei die Zielregister vor den Quellregistern stehen. Desweiteren arbeitet die Architektur nicht mit Sprungmarken, sondern mit echten Adressen. Deshalb ist es Aufgabe des Parser, etwaige Marken in Konstanten umzuwandeln.

\textbf{TODO: Harvard-Architektur-Konzept beschreiben}

\textbf{TODO: Eventuell könnte man den Befehlsknoten auch weglassen und dann jeweils zwei Arrays mit Zielen und Quellen verwalten, das würde die Unschönheit entfernen, dass man das Ende von Quellen und Zielen feststellen kann und nichts falsch interpretiert...}