\chapter{Einführung}

Dieses Dokument stellt einen Statusbericht des ERA-Großpraktikum-Teams sowie
eine initiale, konzeptuelle Ausarbeitung des zu entwickelnden Assembler-
Simulators dar.
Wir beginnen mit einer groben, kurzen Übersicht
unserer Teilgruppen und ihren Aufgabenbereichen.
In den folgenden Kapiteln
behandeln wir dann die einzelnen Module und berichten dabei von den jeweiligen
Erfahrungen, Beobachtungen und weiterführenden Ideen.
Ebenfalls erläutern wir, wie die einzelnen
Module von einander abhängen und was miteinander kommuniziert werden muss.

\section{Übersicht}

Wir haben uns für eine Gliederung des Projekts in vier Gruppen entschieden.
Diese bezeichnen wir hierbei als \emph{Parser}, \emph{Core},
\emph{Architektur} und \emph{GUI}. In den folgenden Absätzen beschreiben wir kurz die
einzelnen Gruppen und nennen ihre Teammitglieder.

\textbf{Architektur}

Wir haben uns das Ziel eines architekturunabhängigen Assembler-Simulators gesetzt,
da dieser einer weiten Spannbreite an Anwendungen in
der Bildung gerecht werden kann. Als erstes wollen wir den \emph{RISC-V}-Befehlssatz
als unterstützen, jedoch architekturunabhängige Schnittstellen bewahren.
Jegliche architektur-spezifische Aspekte
unseres Simulators fallen hierbei in den Aufgabenbereich der
\emph{Architektur}-, kurz \emph{Arch}-, Gruppe. Unter
\emph{architektur-spezifisch} verstehen wir hierbei die Implementierung der
Instruktionen eines Befehlssatzes sowie die Bereitstellung aller Informationen
und Daten, die andere Module des Simulators benötigen. Beispiele hierfür wären
die Registerorganisation sowie das Speichermodell, welche die GUI zur
Visualisierung benötigt; die reservierten Schlüsselwörter der Assemblersprache
der Architektur als notwendige Information für den Parser sowie die
unterstützten Datentypen für die Speicherverwaltung des Cores. Das
Architektur-Team besteht aus Yuriy Arabskyy, Erik Pohle, Lukas Karnowski und
Peter Goldsborough.

\textbf{Parser}

Die Assemblersprache stellt ein mächtige Schnittstelle zur Manipulation des
Speichers sowie des Prozessors einer jeden Architektur dar. Jegliche
Instruktionen an unsere simulierte Maschine müssen notwendigerweise in Assembler
formuliert werden. Als roher Textkörper ist ein Assemblerprogramm jedoch keinem
unserer Module nützlich. Der Assemblercode muss somit zuerst in eine
Representation überführt werden, die die Evaluierung einzelner Instruktionen
sowie deren Argumente als Datenobjekte in anderen Modulen des Simulators
ermöglicht. Hierzu widmet sich das \emph{Parser}-Team, bestehend aus Dominik
Kreutzer und David Schneller. Zwei wesentliche Aufgaben dieser Gruppe sind es,
den Assemblertext für Core und Architektur in einen abstrakten Syntaxbaum zu
transformieren sowie bezüglich syntaktischen und semantischen Aspekten des Quelltextes
mit der GUI zu kommunizieren.

\textbf{GUI}

Um unsere simulierte Maschine interaktiv zu visualisieren, planen wir, mit dem
Qt-Framework eine grafische Benutzeroberfläche zu enwickeln. In dieser
Oberfläche sollen eine Eingaberegion für Assemblercode, den Hauptspeicher, die
Register sowie weitere zusätzliche Elemente dargestellt werden. Wir legen
hierbei Wert darauf, den Assemblercode nützlich und intuitiv hervorzuheben
(highlighten) sowie auch in weiteren Teilen der graphischen Oberfläche dem
Nutzer beim Erlernen der Rechnerarchitektur und des Assemblierens zu
helfen. Auch ist es Aufgabe der GUI-Gruppe, Elemente wie simulierte LED-Streifen
oder Matrizen zu kreieren, auf welche man durch Assemblercode zugreifen
kann. Die Mitglieder dieses Teams sind Sabine Rieder, Felix Opolka, Andreas
Zimmerer sowie Jan Schopohl (auch im Core-Team).

\textbf{Core}

Das \emph{Core} Team setzt sich aus drei Mitgliedern zusammen: Tobias Holl,
Daniel Riedel und Jan Schopohl (auch im GUI-Team). Die Aufgabe dieses Teams ist es, ein
Modul zu entwickeln, das die Koordination aller Teile des Programms leitet. Es
trägt somit die Verantwortung, den Austausch von Daten und Kontrollinformationen
zwischen der Architektur, dem Parser sowie der GUI zu steuern. Weiterhin hat
es die Aufgabe, den Speicher und die Register des simulierten Prozessors zu
verwalten. Letztlich kümmert sich dieses Modul noch um Threadverwaltung.
