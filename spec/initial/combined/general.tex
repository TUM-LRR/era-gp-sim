\chapter{Einführung}

Dieses Dokument stellt einen Statusbericht des ERA Großpraktikums-Teams sowie
eine initiale, konzeptuelle Ausarbeitung des zu entwickelnden Assembler Software
Simulators dar. In den folgenden Kapiteln stellen wir die einzelnen Untergruppen
der Mannschaft vor, sowie dessen Planung und Ausarbeitung für ihre respektive
Teildomäne des Projekts. Wir beginnen hierbei mit einer groben, kurzen Übersicht
über unsere Untergruppen sowie dessen Modulen. Folglich fokussieren wir uns in
jedem Kapitel auf ein bestimmtes Modul und besprechen dabei unsere Erfahrungen,
Beobachungen und weitere Planung. Ebenfalls erläutern wir, wie die einzelnen
Module von einander abhängen und welche Verantwortungen auf welche Weise geteilt
werden müssen.

\section{Übersicht}

Wir haben uns für eine vierfaltige Gliederung des Projekts entschieden. Die
einzelnen Gruppen bezeichnen wir hierbei als \emph{Parser}, \emph{Core},
\emph{Arch} und \emph{GUI}. In den folgenden Absätzen beschreiben wir diese
einzelnen Gruppen kurz und nennen ihre Teammitglieder. Eine fokussiertere Analyse
der einzelnen Domänen folgt in den übrigen Kapiteln dieser Spezifikation.

\textbf{Architektur}

Wir haben es uns als Ziel genommen, einen Assembler Simulator zu entwickeln, der
architekturunabhängig ist und somit einer weiten Spannbreite an Anwendungen in
der Bildung gerecht werden kann. Ohne Beschränkung dieser Allgemeinheit für die
Zukunft haben wir uns im Weiteren dazu entschlossen, den \emph{RISC-V}
Befehlssatz als erstes zu unterstützen. Jegliche architektur-spezifische Aspekte
unseres Simulators fallen hierbei in den Aufgabenbereich der
\emph{Architektur}-, kurz \emph{Arch}-, Gruppe. Unter
\emph{architektur-spezifisch} verstehen wir hierbei die Implementierung der
Instruktionen eines Befehlssatzes sowie die Bereitstellung aller Informationen
und Daten, die andere Module des Simulators benötigen. Beispiele hierfür wären
die Registerorganisation sowie das Speichermodell, welche die GUI zur
Visualisierung benötigt; die reservierten Schlüsselwörter der Assemblersprache
der Architektur als notwendige Information für den Parser sowie die
unterstützten Datentypen für die Speicherverwaltung des Cores. Das
Architektur-Team besteht aus Yuriy Arabskyy, Erik Pohle, Lukas Karnowski und
Peter Goldsborough.

\textbf{Parser}

Die Assemblersprache stellt ein mächtiges Interface zur Manipulation des
Speichers sowie des Prozessors einer jeden Architektur dar. Jegliche
Instruktionen an unsere simulierte Maschine müssen notwendigerweise in Assembler
formuliert werden. Als roher Textkörper ist ein Assemblerprogramm jedoch keinem
unserer Module nützlich. Der Assemblercode muss somit zuerst in eine
Representation überführt werden, die die Evaluierung einzelner Instruktionen
sowie deren Argumente als Datenobjekte in anderen Modulen des Simulators
ermöglicht. Hierzu widmet sich das \emph{Parser}-Team, bestehend aus Dominik
Kreutzer und David Schneller. Zwei wesentliche Aufgaben dieser Gruppe sind es,
den Assembler Code für Core und Architektur in einen abstrakten Syntaxbaum zu
transformieren sowie bezüglich syntaktischen und semantischen Aspekten des Codes
mit der GUI zu kommunizieren.

\textbf{GUI}

Um unsere simulierte Maschine interaktiv zu visualisieren planen wir, mit dem
Qt-Framework eine graphische Benutzeroberfläche zu enwickeln. In dieser
Oberfläche sollen eine Eingaberegion für Assemblercode, der Hauptspeicher, die
Register sowie weitere zusätzliche Elemente dargestellt werden. Wir legen
hierbei darauf Wert, den Assemblercode nützlich und intuitiv hervorzuheben
(highlighten) sowie auch in weiteren Teilen der graphischen Oberfläche dem
Nutzer beim Erlernen der Rechnerarchitektur sowie des Assemblierens zu
helfen. Auch ist es Aufgabe der GUI-Gruppe, Elemente wie simulierte LED Streifen
oder Matrizen zu kreieren, auf welche man durch Assemblercode zugreifen
kann. Die Mitglieder dieses Teams sind Sabine Rieder, Felix Opolka, Andreas
Zimmerer sowie Jan Schopohl (1. Hälfte).

\textbf{Core}

Das \emph{Core} Team setzt sich aus drei Mitgliedern zusammen: Tobias Holl,
Daniel Riedel und Jan Schopohl (2. Hälfte). Die Aufgabe dieses Teams ist es, ein
Modul zu entwickeln, dass die Koordination aller Teile des Programms leitet. Es
trägt somit die Verantwortung, den Austausch von Daten und Kontrollinformationen
zwischen der Architektur, dem Parser sowie der GUI zu steuern. Des Weiteren hat
es die Aufgabe, den Speicher und die Register des simulierten Prozessors zu
verwalten. Letztlich kümmert sich dieses Modul noch um Threadverwaltung.
