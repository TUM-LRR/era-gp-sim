\chapter{Core}

Das Core-Modul ist das Herzstück des Assemblersimulators. Es regelt die Kommunikation zwischen den einzelnen Modulen, sodass diese leicht ausgetauscht werden können, um unterschiedliche Architekturen, Assemblerdialekte und Benutzeroberflächen zu erlauben. Ebenso steuert es die eigentliche Ausführung des von den Parser- und Architektur-Modulen erzeugten Syntaxbaums. Zusätzlich ist es für die Verwaltung der Threads zuständig, sodass z.B. die GUI nicht blockiert ist, während das Assemblerprogramm ausgeführt wird.

\section{Threading}

Kein Modul sollte blockierend auf die Simulation warten müssen, insbesondere jedoch nicht die GUI, die immer auf Nutzereingaben reagieren können sollte, und das Core-Modul, welches zumindest so weit reagieren können sollte, wie es zur Koordination der einzelnen Module nötig ist.
Daher erhält jeder Programmteil, für den dies nötig ist, einen eigenen Thread (oder notfalls einen Threadpool). Um Implementierungskomplikationen durch \textit{Race Conditions} oder \textit{Deadlocks} zu vermeiden, sollen die Threads hier weitestgehend dem \textit{Active Object}-Entwurfsmuster folgen---auf jedes Objekt wird aus höchstens einem  Thread zugegriffen, Datentransfer und Funktionsaufrufe erfolgen über Queues und Callbacks, sodass Threadsicherheit weitestgehend vom Threading-Framework selbst (hier über das Core-Modul) statt über die einzelnen Datentypen garantiert wird. Natürlich ist es nicht unbedingt erstrebenswert, für jedes Objekt einen einzelnen Thread zu erzeugen, also sollten auch mehrere Objekte in einem Thread verwaltet werden können. Insbesondere soll damit jedem Modul ein Thread (oder bei Bedarf mehrere Threads) zugewiesen werden. 

Das Threading wird mithilfe der \lstinline[style=C++]!std::thread!-Teilbibliothek der C++ Standard Library implementiert werden. Um den Simulator (abgesehen von der GUI-Referenzimplementierung) von Qt unabhängig zu halten, wurde bewusst auf das Verwenden von QThread verzichtet.Dies ist besonders relevant, da es von mehreren Seiten den Wunsch gab, bis zur Fertigstellung der GUI die Möglichkeit zu schaffen, kleinere ``Test-GUIs'' zu schreiben, die das Testen der anderen Module erleichtern.

\section{Projekte}

Um mehrere gleichzeitig geöffnete Projekte mit unterschiedlichen Projekteinstellungen, Speicher- und Registerzuständen zu unterstützen, wird das Core-Modul für jedes offene Projekt verschiedene Architektur- und Parser-Module sowie Speicher- und Registerobjekte verwalten. Bei der Kommunikation mit dem GUI-Modul muss dieses dann das betroffene Projekt eindeutig identifizieren---alle anderen Module (sowie die Speicher- und Registerobjekte) müssen nur jeweils genau ein Projekt verwalten können.

\section{Datentypen}

Um möglichst alle Architekturen abzudecken, wollen wir den Rückgabetyp der Zugriffsfunktionen auf Speicher und Register möglichst allgemein halten. Da die Interpretation der Daten von der Architektur abhängt, wird ein Speicherzugriff zunächst nur die rohen Bits als \lstinline[style=C++]!std::vector<bool>! zurückgegeben. Falls mehr Operationen benötigt werden, könnte dies auch in einem ähnlichen, eigenen Typ geschehen. Das Core-Modul stellt für die Rohdaten Konvertierungsfunktionen bereit, besonders in Ganzzahl- und Gleitkommazahlen variabler Bitgrö{\ss}e (``Bigints'' und ``Bigfloats''). Diese Typen sollen soweit möglich die gleichen Operationen anbieten, die auch von den normalen \lstinline[style=C++]!int!- und \lstinline[style=C++]!float!-Typen bereitgestellt werden. Falls der Bedarf besteht, könnte auch eine Konvertierung in normale C++-Typen angeboten werden, dies sollte aber per se nicht nötig sein.


\section{Hauptspeicher}
Der Hauptspeicher stellt einen ausgedehnten numerisch adressierbaren Speicher dar, welcher direkt von der CPU aus angesprochen werden kann. Wir beschränken uns hier auf homogene Hauptspeicherzellen mit konsistenten Grö{\ss}en. Um den Zugriff auf den simulierten Hauptspeicher, der ja vergleichsweise häufig benötigt wird, möglichst einfach zu gestalten, werden die einzelnen Speicherzellen auf ganze Bytezahlen aufgefüllt; eine Implementierung, in der die Speicherzellen bitgenau aneinandergereiht werden, wäre untunlich.

\section{Register}
Die Register stellen den ALU-nächsten Speicher dar. Gegensätzlich zum Hauptspeicher ist dieser nicht numerisch-bytegenau zu adressieren, sondern individuell, teils sogar als Teilbitabschnitt dieser.  Um Register zeichenkettenförmig zu identifizieren wird eine Map/Dictionary und das Proxy-Entwurfsmuster verwendet.

\section{Interfaces}

\subsection{Architektur}
Die Architektur muss zusammen mit dem Parser-Modul die Interpretation des Assemblertextes vornehmen. Der Core selbst, obgleich er von der genauen Wahl der Architektur und des Parsers unabhängig ist, muss dabei die benötigten Daten zwischen den Modulen übertragen. Insbesondere muss die Architektur---neben der Registrierung der Factorys, die die Syntaxbäume erzeugen---auch zu jedem Zustand dem Core kundtun können, welche Instruktion als nächstes ausgeführt werden soll, schließlich ist dem Core die genaue Bedeutung der einzelnen Register vollkommen unbekannt.

\subsection{GUI}

Core und GUI kommunizieren mit Hilfe von Callbacks, so dass der Core keinen Zugriff auf die Funktionalität der GUI benötigt. Die GUI wird dabei über Änderungen an Registern, am Speicher und über Fehlermeldungen informiert, damit die Darstellung aktualisiert werden kann. Gleichzeitig muss sie den Core über Änderungen am Assemblercode, den Registern oder dem Speicher informieren, die vom Benutzer ausgelöst wurden. Beim Programmstart wird das GUI-Modul vom Core gestartet, da dieser die verschiedenen Programmteile verwaltet und die Kommunikation zwischen den Threads regelt.

\subsection{Parser}

Der Core startet und verwaltet auch den Parser und übergibt diesem die Informationen der Architektur, beispielsweise die Factories für die Syntaxbäume. Au{\ss}erdem erhält der Parser den Assemblercode von der GUI, dieser wird als \lstinline[style=C++]!std::string! weitergegeben. Der Core erhält vom Parser Fehlermeldungen für die GUI sowie die Syntaxbäume, die vom Core ausgeführt werden können. Dabei wird der als nächstes auszuführende Befehl von der Architektur ermittelt.
