\chapter{Graphical User Interface}

{\color{red} Kurze Einführung in das GUI Modul (siehe Kapitel zu Core, RISCV)}

{\color{red} Könnt ihr das da unten vielleicht in eine eigene Sektion zu
  \emph{Implementierung} machen? Vielleicht erwähnen, wie ihr Qt, QML und
  weiteres (CSS/SASS für Themes!) verwenden wollt?}

Die GUI wird mit dem Qt-Framework umgesetzt, wobei grafische Elemente im
Besonderen mit QML implementiert werden. Dies unterstützt die strikte Trennung
von Präsentation, Steuerung und Struktur der zugrundeliegenden Daten.

Die Kommunikation mit dem \textit{Core} erfolgt über eine C++-Schnittstelle auf
Basis des Observer-Patterns.


\section{Fenstermanagement}

Die GUI soll aus verschiedenen Modulen bestehen, die zunächst als Editor-,
Register-, Speicher-, Projekt-, Hilfs- und Output-Modul identifiziert
wurden. Die darüberliegende Toolbar steuert Abläufe und Einstellungen, die über
Modulgrenzen hinweg Relevanz haben.

Die Unterbringung der Module erfolgt in einem tabellarischen Layout bestehend
aus 4 Spalten zu je 1 bis 2 Zeilen. Die untere Zeile einer Spalte kann je nach
Bedarf angezeigt oder ausgeblendet werden.

Die dadurch entstehenden Modulzellen sollen variabel mit den oben beschriebenen
Modulen belegt werden können, was die Möglichkeit mit einschließt, ein Modul
mehrfach in verschiedenen Zellen anzuzeigen. Ein einfaches Anwendungsbeispiel
für dieses Feature ist das parallele Arbeiten auf zwei Bereichen den Speichers.

\begin{figure}
  \includegraphics[width=\textwidth]{../ui/figures/mockup}
  \label{fig:Mockup}
  \caption{Der obige rein visuelle Entwurf soll einen groben Überblick über die
    geplante Umsetzung der Benutzeroberfläche geben. Im Wesentlichen ist die
    Aufteilung des Fensters in Toolbar und die vier Spalten mit ein bis zwei
    Zeilen erkennbar. Zudem wurden die einzelnen Module skizziert.}
\end{figure}

\section{Toolbar}

Über die Toolbar kann die \textbf{Ausführung} des Programms gesteuert werden,
wobei die Optionen \textit{Ausführung des gesamten Programms},
\textit{Ausführung einer einzelnen Instruktion}, \textit{Ausführung bis zum
  nächsten Breakpoint} und \textit{Abbrechen der Programmausführung} zur
Verfügung stehen.

Des Weiteren lässt sich das \textbf{Zahlenformat} modulübergreifend festlegen,
was insebesondere Einfluss auf Register- und Speicherinhalte hat.

\section{Editor}

Der Editor für die Eingabe der Assembler-Instruktion soll
\textbf{Syntax-Highlighting} unterstützen, um Instruktions-Komponenten visuell
zu trennen. Die Umsetzung erfolgt mit dem Qt-eigenen Syntax-Highlighter
(\textit{QHighlighter}), welcher mit Regex-Ausdrücken initialisiert wird. Diese
werden vom Parser in Kombination mit Architektur-eigenen Keywords generiert und
über den Core zur Verfügung gestellt.

\textbf{Fehlermeldungen}, die vom Parser ermittelt wurden, werden innerhalb der
zugehörigen Zeile im Editor angezeigt, was es ermöglicht mehrere Fehlermeldungen
gleichzeitig anzuzeigen.

Ein \textbf{Makroaufruf} wird im Editor als einzelne Zeile dargestellt, die bei
Bedarf vom User aufgeklappt werden kann, wodurch die Makrodefinition in-place
eingeblendet wird. Dies ermöglicht u.a. die schrittweise Ausführung des Codes.

\section{Register}

Die Registerwerte werden übersichtlich mit Hilfe von Byte-Separatoren
dargestellt. Zu den jeweiligen Registern gehörige Unterregister können separat
eingeblendet werden.

Registern wie Unterregistern kann unabhängig von den globalen Einstellungen ein
eigenes Zahlenformat (binär, hexadezimal etc.) zugewiesen werden.

Intern werden die einzelnen Zahlenformate vom \textit{Core} berechnet und als
String an die GUI übergeben.

\section{Speicher}

Der Speicher teilt sich in drei Hauptbereiche: Speicheradressen, Speicherinhalte
und Kontextinformationen.

Der \textbf{Adressraum} erlaubt die Unterteilung in verschiedene Datenformate,
darunter Byte, Halbwort, Wort etc.

Das Zahlenformat der \textbf{Speicherinhalte} ist spaltenweise anpassbar.

\textbf{Kontextinformationen} zu einzelnen Speicherzellen geben Hinweise auf die
Verwendung im Programm, darunter etwa Marken zu Datendefinitionen,
Speicherbereiche für Ausgabegeräte und Speicher-Referenzierungen durch Register.

\section{Projekt}

Das Projekt-Modul dient der \textbf{Datei-Verwaltung}. Gegebenefalls können ein
oder mehrere Dateien zu Projekten zusammengefasst werden, welche dann
zusätzliche Information etwa über die verwendete Architektur sowie Speicher- und
Registerinhalte speichern können.
