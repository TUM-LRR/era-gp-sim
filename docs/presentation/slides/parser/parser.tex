% !TEX root = ../../presentation.tex
% Parser

\textframe{\texttt{.parser}}


\begin{frame}{Aufgabe des Parsers}
\begin{center}
	\begin{tikzpicture}[node distance=1.0cm and 2.5cm]
	\tikzstyle{myarrow} = [->, thick]

	\uncover<2->{\node (intern) [align=center]
	{
		\textbf{Interne}\\
		\textbf{Definitionen}
	};}
	\node (invis1) [below = of intern] {};
	\uncover<3->{\node (code) [below = of invis1]
	{
		\textbf{Programmcode}
	};}
	\node (parser) [rounded corners=1pt, rectangle, draw = black, minimum height = 5cm, right = of invis1]
	{
		\textbf{Parser}
	};
	\node (invis2) [right = of parser] {};
	\uncover<4->{\node (tree) [above = of invis2]
	{
		\textbf{Syntaxbaum}
	};}
	\uncover<5->{\node (errors) [below = of invis2]
	{
		\textbf{Fehlerliste}
	};}
	\uncover<2->{\draw[myarrow] (intern) -- (intern -| parser.west);}
	\uncover<3->{\draw[myarrow] (code) -- (code -| parser.west);}
	\uncover<4->{\draw[myarrow] (parser.east |- tree) -- (tree);}
	\uncover<5->{\draw[myarrow] (parser.east |- errors) -- (errors);}
	\end{tikzpicture}
\end{center}
\end{frame}


\begin{frame}{Parser-Factory}
Verschiedene Parser ermöglicht durch Factory-Pattern:
\uncover<2->{\begin{center}
	\begin{tikzpicture}[node distance=1.0cm and 2.0cm]
	\tikzstyle{myarrow} = [->, thick]

	\node (name) [ellipse, draw = black]
	{
		\texttt{"name"}
	};
	\node (factory) [rounded corners=1pt, rectangle, draw = black, right = of name]
	{
		\textbf{Factory}
	};
	\node (parser) [ellipse, right = of factory, draw = black]
	{
		\textbf{Parser}
	};
	\draw[myarrow] (name) -- (factory);
	\draw[myarrow] (factory) -- (parser);
	\end{tikzpicture}
\end{center}}
\end{frame}


\begin{frame}[fragile]{Prinzip}
\begin{center}
	\begin{tikzpicture}[node distance=1.5cm and 0.3cm]
	\tikzstyle{follows} = [->, thick, double]
	\tikzstyle{treearr} = [->]

	\node (code) [draw = black]
	{
		\begin{lstlisting}[basicstyle=\tiny]{style=risc-v_Assembler}
.equ magic, 3 * 7
addi x1, x0, magic + 2
		\end{lstlisting}
	};
	\node (help) [below = of code] {};
	\uncover<2->{
		\node (inter1) [rounded corners=1pt, rectangle split, rectangle split parts = 2, draw = black, left = of help]
		{
			\textbf{Konstante}
			\nodepart{second}
			{\tiny
			\begin{tabular}{rl}
			name: & \texttt{"magic"} \\
			value: & \texttt{"3 * 7"}
			\end{tabular}}
		};
		\node (inter2) [rounded corners=1pt, rectangle split, rectangle split parts = 2, draw = black, right = of help]
		{
			\textbf{Instruktion}
			\nodepart{second}
			{\tiny
			\begin{tabular}{rl}
			name: & \texttt{\char`\"addi"} \\
			targets: & \texttt{["x1"]} \\
			sources: & \texttt{["x0",} \\
			& \texttt{   "magic + 2" ]}
			\end{tabular}}
		};
	}
	\uncover<2->{\node (intertxt) [left = of inter1] {Zwischendarstellung};}
	\node (codetxt) [above = of intertxt] {Programmcode};
	\uncover<3->{\node (treetxt) [below = of intertxt] {Syntaxbaum};}
	\uncover<3->{
		\node (tree_add) [ellipse, draw = black, below = of help]
		{
			\texttt{addi}
		};
		\node (tree_x0) [ellipse, draw = black, below = of tree_add, yshift=1cm]
		{
			\texttt{x0}
		};
		\node (tree_x1) [ellipse, draw = black, left = of tree_x0]
		{
			\texttt{x1}
		};
		\node (tree_n) [ellipse, draw = black, right = of tree_x0]
		{
			\texttt{23}
		};
	}

	\uncover<2->{\draw [dashed] (inter1) -- (inter2);}

	\uncover<3->{
		\draw [treearr] (tree_add) -- (tree_x1);
		\draw [treearr] (tree_add) -- (tree_x0);
		\draw [treearr] (tree_add) -- (tree_n);
	}

	\uncover<2->{\draw [follows] (code) -- (code |- inter2.north);}
	\uncover<3->{\draw [follows] (tree_add |- inter2.south) -- (tree_add);}

	\end{tikzpicture}
\end{center}
\end{frame}


\begin{frame}{Symboltabelle}
\begin{center}
	\begin{tikzpicture}[node distance=1.5cm and 0.3cm]
	\tikzstyle{follows} = [->, thick, double]
	\tikzstyle{treearr} = [->]
		\uncover<1-4>{
			\node (s_0) [ellipse, draw = black]
			{
				$S_0\rightarrow S_1aS_3$
			};
		}
		\uncover<1-3>{
			\node (s_1) [ellipse, draw = black, below = of s_0]
			{
				$S_1\rightarrow S_2S_3$
			};
		}
		\uncover<5->{
			\node (s_0_) [ellipse, draw = black]
			{
				$S_0\rightarrow bcac$
			};
		}
		\uncover<4->{
			\node (s_1_) [ellipse, draw = black, below = of s_0]
			{
				$S_1\rightarrow bc$
			};
		}
		\uncover<1->{
			\node (s_2) [ellipse, draw = black, left = of s_1]
			{
				$S_2\rightarrow b$
			};
		}
		\uncover<1->{			
			\node (s_3) [ellipse, draw = black, right = of s_1]
			{
				$S_3\rightarrow c$
			};
		}
		\uncover<2-4>{
			\draw [treearr] (s_3) -- (s_0);
		}
		\uncover<2-4>{
			\draw [treearr] (s_1) -- (s_0);
		}
		\uncover<3-3>{
			\draw [treearr] (s_2) -- (s_1);
		}
		\uncover<3-3>{
			\draw [treearr] (s_3) -- (s_1);
		}
		\uncover<1->{\node (graphtxt) [left = of s_2] {Symbolgraph};}
		\uncover<6->{\node (tabletxt) [below = of graphtxt] {Symbolersetzer};}
	\end{tikzpicture}
	\uncover<6->{
		\begin{tabular}{l|l}
			Symbol & Ersetzung \\
			\hline
			$S_0$ & $bcac$ \\
			$S_1$ & $bc$ \\
			$S_2$ & $b$ \\
			$S_3$ & $c$ \\
		\end{tabular}
	}
\end{center}
\end{frame}


\begin{frame}{Arithmetische Ausdrücke}
\begin{center}
	\begin{tikzpicture}
		\begin{tikzpicture}[node distance=1.5cm and 0.3cm]
		\node (outstacktxt)  {Ausgabe-Stack};
		\node (opstacktxt) [above = of outstacktxt] {Operator-Stack};
			\node (expr) [above = of opstacktxt] {				
				$
					{\color<1-2>[rgb]{1,0,0}{\color<3->[rgb]{0.5,0.5,0.5}1}}
					{\color<3-4>[rgb]{1,0,0}{\color<5->[rgb]{0.5,0.5,0.5}+}}
					{\color<5-6>[rgb]{1,0,0}{\color<7->[rgb]{0.5,0.5,0.5}2}}
					{\color<7-8>[rgb]{1,0,0}{\color<9->[rgb]{0.5,0.5,0.5}*}}
					{\color<9-10>[rgb]{1,0,0}{\color<11->[rgb]{0.5,0.5,0.5}3}}
					{\color<11-16>[rgb]{1,0,0}{\color<17->[rgb]{0.5,0.5,0.5}+}}
					{\color<17-18>[rgb]{1,0,0}{\color<19->[rgb]{0.5,0.5,0.5}4}}
				$
			};
			\uncover<4-14> {
				\node (op_add)[rectangle, draw = black, right = of opstacktxt] {
					\color<14>[rgb]{1,0,0}$+$
				};
			}
			\uncover<8-12> {
				\node (op_mul)[rectangle, right = of op_add, draw = black] {
					\color<12>[rgb]{1,0,0}$*$
				};
			}
			\uncover<16-20> {
				\node (op_add2)[rectangle, draw = black, right = of opstacktxt] {
					\color<20>[rgb]{1,0,0}$+$
				};
			}
			\uncover<2-14> {
				\node (num1)[rectangle, draw = black, right = of outstacktxt] {
					\color<14>[rgb]{1,0,0}$1$
				};
			}
			\uncover<6-12> {
				\node (num2)[rectangle, draw = black, right = of num1] {
					\color<12>[rgb]{1,0,0}$2$
				};
			}
			\uncover<10-12> {
				\node (num3)[rectangle, draw = black, right = of num2] {
					\color<12>[rgb]{1,0,0}$3$
				};
			}
			\uncover<15-20> {
				\node (num6)[rectangle, draw = black, right = of outstacktxt] {
					\color<20>[rgb]{1,0,0}$7$
				};
			}
			\uncover<18-20> {
				\node (num4)[rectangle, draw = black, right = of num6] {
					\color<20>[rgb]{1,0,0}$4$
				};
			}
			\uncover<13-14> {
				\node (num5)[rectangle, draw = black, right = of num1] {
					\color<14>[rgb]{1,0,0}$6$
				};
			}
			\uncover<21-> {
				\node (num10)[rectangle, draw = black, right = of outstacktxt] {
					$11$
				};
			}
		\end{tikzpicture}
	\end{tikzpicture}
\end{center}
\end{frame}


\begin{frame}{Makros}
\begin{overprint}
\onslide<1>
Makroaufrufe suchen und ersetzen.
\onslide<2>
Syntaxbaum generieren.
\end{overprint}

\begin{center}
	\begin{tikzpicture}[node distance=1.5cm and 0.3cm]
	\tikzstyle{follows} = [->, thick, double]
	\tikzstyle{treearr} = [->]

	\node (inter1) [rounded corners=1pt, rectangle split, rectangle split parts = 4, draw = black]
	{
		\textbf{Makro}
		\nodepart{second}
		{\tiny
		\begin{tabular}{rl}
		name: & \texttt{"verdopple"} \\
		args: & \texttt{["reg"]}
		\end{tabular}}
		\nodepart{three}
		\textbf{\small{Instruktionen}}
		\nodepart{four}
		{\tiny
		\begin{tabular}{rl}
		name: & \texttt{\char`\"add"} \\
		targets: & \texttt{["\textbackslash{}reg"]} \\
		sources: & \texttt{["\textbackslash{}reg", "\textbackslash{}reg"]}
		\end{tabular}}
	};

	\uncover<2->{
	\node (inter4) [rounded corners=1pt, rectangle split, rectangle split parts = 2, draw = black, right = of inter1]
	{
		\textbf{Makro-Instr.}
		\nodepart{second}
		{\tiny
		\begin{tabular}{rl}
		name: & \texttt{\char`\"add"} \\
		targets: & \texttt{["x1"]} \\
		sources: & \texttt{["x1", "x1"]}
		\end{tabular}}
	};
	\node (inter5) [rounded corners=1pt, rectangle split, rectangle split parts = 2, draw = black, right = of inter4]
	{
		\textbf{Makro-Instr.}
		\nodepart{second}
		{\tiny
		\begin{tabular}{rl}
		name: & \texttt{\char`\"add"} \\
		targets: & \texttt{["x2"]} \\
		sources: & \texttt{["x2", "x2"]}
		\end{tabular}}
	};
	\draw [dashed] (inter1) -- (inter4) -- (inter5);}

	\uncover<1>{
	\node (inter2) [rounded corners=1pt, rectangle split, rectangle split parts = 2, draw = black, at = (inter4)]
	{
		\textbf{Instruktion}
		\nodepart{second}
		{\tiny
		\begin{tabular}{rl}
		name: & \texttt{\char`\"verdopple"} \\
		targets: & \texttt{["x1"]}
		\end{tabular}}
	};
	\node (inter3) [rounded corners=1pt, rectangle split, rectangle split parts = 2, draw = black, at = (inter5)]
	{
		\textbf{Instruktion}
		\nodepart{second}
		{\tiny
		\begin{tabular}{rl}
		name: & \texttt{\char`\"verdopple"} \\
		targets: & \texttt{["x2"]}
		\end{tabular}}
	};
	\draw [dashed] (inter1) -- (inter2) -- (inter3);}

	\uncover<3->{
	\node (tree1) [ellipse, below left = of inter4, draw = black] {add};
	\node (tree12) [ellipse, below = of tree1, draw = black, yshift = 1cm] {x1};
	\node (tree11) [ellipse, left = of tree12, draw = black] {x1};
	\node (tree13) [ellipse, right = of tree12, draw = black] {x1};
	\draw [treearr] (tree1) -- (tree11);
	\draw [treearr] (tree1) -- (tree12);
	\draw [treearr] (tree1) -- (tree13);

	\node (tree2) [ellipse, below right = of inter4, draw = black] {add};
	\node (tree22) [ellipse, below = of tree2, draw = black, yshift = 1cm] {x2};
	\node (tree21) [ellipse, left = of tree22, draw = black] {x2};
	\node (tree23) [ellipse, right = of tree22, draw = black] {x2};
	\draw [treearr] (tree2) -- (tree21);
	\draw [treearr] (tree2) -- (tree22);
	\draw [treearr] (tree2) -- (tree23);

	\draw [dashed] (tree1) -- (tree2);

	\draw [follows] (inter4 |- inter1.south) -- (inter4 |- tree1);}

	\end{tikzpicture}
\end{center}
\end{frame}
