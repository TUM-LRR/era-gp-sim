% !TEX root = ../../presentation.tex
% GUI

\textframe{\texttt{.gui}}


\tikzset{hide on/.code={\only<#1>{\color{white}}}}

\begin{slide}{Aufbau der GUI}
\begin{tikzpicture}
  \tikzset{font=\tiny,
    edge from parent fork down,
    level distance=1.3cm,
    every node/.style=
      {rectangle,rounded corners,
      minimum height=7mm,
      fill=white,
      draw=black!50,
      drop shadow,
      align=center,
      text depth = 0pt
      },
    edge from parent/.style=
      {{Diamond}-,
      draw=black!50,
      thick
      }}

  \Tree [.{ApplicationWindow\\(main.qml)}
            [.MenuBar ]
            [.Toolbar ]
            [.ProjectTabView
                [.Splitview
                     [.InnerSplitviews
                         [.SplitViewItem
                             [.Snapshots ]
                             [.Help ]
                             [.Registers ]
                             [.Memory ]
                             [.Input/Output ] ] ]
                     [.{InnerSplitviews-\\Editor}
                         [.SplitViewItem ]
                         [.Editor ] ] ] ] ]
\end{tikzpicture}
\end{slide}

\begin{slide}{MVP}

\begin{tikzpicture}[node distance=1.0cm and 2.5cm, minimum height=1.5cm, minimum width=2.5cm]

\node (view)  
	[rectangle, draw = black] {VIEW};

\node (presenter)  
	[rectangle, draw = black, below = of view] {PRESENTER};

\node (model)  
	[rectangle, draw = black, right = of presenter] {MODEL};

\draw[->, thick] (model.170) -- (presenter.10) node [midway, above=1pt, fill=white] {Update};
\draw[<-, thick] (model.190) -- (presenter.350);
\draw[->, thick] (presenter.80) -- (view.280);
\draw[<-, thick] (presenter.100) -- (view.260);

\end{tikzpicture}

\end{slide}
