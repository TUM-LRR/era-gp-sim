% !TEX root = ../../presentation.tex
% GUI

\textframe{\texttt{.gui}}


\tikzset{hide on/.code={\only<#1>{\color{white}}}}
\tikzset{font=\tiny,
  edge from parent fork down,
  level distance=1.3cm,
  every node/.style=
    {rectangle,rounded corners,
    minimum height=7mm,
    fill=white,
    draw=black!50,
    align=center,
    text depth = 0pt
    },
  edge from parent/.style=
    {,
    draw=black!50,
    thick
    }}

\begin{slide}{Aufbau der GUI}
\begin{tikzpicture}
  \Tree [.\node[color=white,draw=white]{ApplicationWindow\\(main.qml)};
            \edge[draw=white];[.\node[color=white,draw=white]{Menubar}; ]
            \edge[draw=white];[.\node[color=white,draw=white]{Toolbar}; ]
            \edge[draw=white];[.\node[color=white,draw=white]{ProjectTabView};
                \edge[draw=white];[.\node[color=white,draw=white]{Splitview};
                     \edge[draw=white];[.\node[color=white,draw=white]{InnerSplitviews};
                         \edge[draw=white];[.\node[color=white,draw=white]{SplitViewItem};
                             \edge[draw=white];[.Snapshots ]
                             \edge[draw=white];[.Help ]
                             \edge[draw=white];[.Registers ]
                             \edge[draw=white];[.Memory ]
                             \edge[draw=white];[.Input/Output ] ] ]
                     \edge[draw=white];[.\node[color=white,draw=white]{InnerSplitviews-\\Editor};
                         \edge[draw=white];[.\node[color=white,draw=white]{SplitViewItem}; ]
                         \edge[draw=white];[.\node[color=white,draw=white]{Editor}; ] ] ] ] ]
\end{tikzpicture}
\end{slide}

% Mit ApplicationWindow
\begin{slide}{Aufbau der GUI}
\begin{tikzpicture}
  \Tree [.\node{ApplicationWindow\\(main.qml)};
            \edge[draw=white];[.\node[color=white,draw=white]{Menubar}; ]
            \edge[draw=white];[.\node[color=white,draw=white]{Toolbar}; ]
            \edge[draw=white];[.\node[color=white,draw=white]{ProjectTabView};
                \edge[draw=white];[.\node[color=white,draw=white]{Splitview};
                     \edge[draw=white];[.\node[color=white,draw=white]{InnerSplitviews};
                         \edge[draw=white];[.\node[color=white,draw=white]{SplitViewItem};
                             \edge[draw=white];[.Snapshots ]
                             \edge[draw=white];[.Help ]
                             \edge[draw=white];[.Registers ]
                             \edge[draw=white];[.Memory ]
                             \edge[draw=white];[.Input/Output ] ] ]
                     \edge[draw=white];[.\node[color=white,draw=white]{InnerSplitviews-\\Editor};
                         \edge[draw=white];[.\node[color=white,draw=white]{SplitViewItem}; ]
                         \edge[draw=white];[.\node[color=white,draw=white]{Editor}; ] ] ] ] ]
\end{tikzpicture}
\end{slide}

% Mit MenuBar, Toolbar, ProjectTabView
\begin{slide}{Aufbau der GUI}
\begin{tikzpicture}
  \Tree [.\node{ApplicationWindow\\(main.qml)};
            [.\node{Menubar}; ]
            [.\node{Toolbar}; ]
            [.\node{ProjectTabView};
                \edge[draw=white];[.\node[color=white,draw=white]{Splitview};
                     \edge[draw=white];[.\node[color=white,draw=white]{InnerSplitviews};
                         \edge[draw=white];[.\node[color=white,draw=white]{SplitViewItem};
                             \edge[draw=white];[.Snapshots ]
                             \edge[draw=white];[.Help ]
                             \edge[draw=white];[.Registers ]
                             \edge[draw=white];[.Memory ]
                             \edge[draw=white];[.Input/Output ] ] ]
                     \edge[draw=white];[.\node[color=white,draw=white]{InnerSplitviews-\\Editor};
                         \edge[draw=white];[.\node[color=white,draw=white]{SplitViewItem}; ]
                         \edge[draw=white];[.\node[color=white,draw=white]{Editor}; ] ] ] ] ]
\end{tikzpicture}
\end{slide}

% Mit SplitView
\begin{slide}{Aufbau der GUI}
\begin{tikzpicture}
  \Tree [.\node{ApplicationWindow\\(main.qml)};
            [.\node{Menubar}; ]
            [.\node{Toolbar}; ]
            [.\node{ProjectTabView};
                [.\node{Splitview};
                     \edge[draw=white];[.\node[color=white,draw=white]{InnerSplitviews};
                         \edge[draw=white];[.\node[color=white,draw=white]{SplitViewItem};
                             \edge[draw=white];[.Snapshots ]
                             \edge[draw=white];[.Help ]
                             \edge[draw=white];[.Registers ]
                             \edge[draw=white];[.Memory ]
                             \edge[draw=white];[.Input/Output ] ] ]
                     \edge[draw=white];[.\node[color=white,draw=white]{InnerSplitviews-\\Editor};
                         \edge[draw=white];[.\node[color=white,draw=white]{SplitViewItem}; ]
                         \edge[draw=white];[.\node[color=white,draw=white]{Editor}; ] ] ] ] ]
\end{tikzpicture}
\end{slide}

% Mit SplitViewItems
\begin{slide}{Aufbau der GUI}
\begin{tikzpicture}
  \Tree [.\node{ApplicationWindow\\(main.qml)};
            [.\node{Menubar}; ]
            [.\node{Toolbar}; ]
            [.\node{ProjectTabView};
                [.\node{Splitview};
                     [.\node{InnerSplitviews};
                         \edge[draw=white];[.\node[color=white,draw=white]{SplitViewItem};
                             \edge[draw=white];[.Snapshots ]
                             \edge[draw=white];[.Help ]
                             \edge[draw=white];[.Registers ]
                             \edge[draw=white];[.Memory ]
                             \edge[draw=white];[.Input/Output ] ] ]
                     [.\node{InnerSplitviews-\\Editor};
                         \edge[draw=white];[.\node[color=white,draw=white]{SplitViewItem}; ]
                         \edge[draw=white];[.\node[color=white,draw=white]{Editor}; ] ] ] ] ]
\end{tikzpicture}
\end{slide}

% Mit SplitViewItems, Editor
\begin{slide}{Aufbau der GUI}
\begin{tikzpicture}
  \Tree [.\node{ApplicationWindow\\(main.qml)};
            [.\node{Menubar}; ]
            [.\node{Toolbar}; ]
            [.\node{ProjectTabView};
                [.\node{Splitview};
                     [.\node{InnerSplitviews};
                         [.\node{SplitViewItem};
                             \edge[draw=white];[.Snapshots ]
                             \edge[draw=white];[.Help ]
                             \edge[draw=white];[.Registers ]
                             \edge[draw=white];[.Memory ]
                             \edge[draw=white];[.Input/Output ] ] ]
                     [.\node{InnerSplitviews-\\Editor};
                         [.\node{SplitViewItem}; ]
                         [.\node{Editor}; ] ] ] ] ]
\end{tikzpicture}
\end{slide}

% Mit SplitViewItems, Editor
\begin{slide}{Aufbau der GUI}
\begin{tikzpicture}
  \Tree [.\node{ApplicationWindow\\(main.qml)};
            [.\node{Menubar}; ]
            [.\node{Toolbar}; ]
            [.\node{ProjectTabView};
                [.\node{Splitview};
                     [.\node{InnerSplitviews};
                         [.\node{SplitViewItem};
                             [.Snapshots ]
                             [.Help ]
                             [.Registers ]
                             [.Memory ]
                             [.Input/Output ] ] ]
                     [.\node{InnerSplitviews-\\Editor};
                         [.\node{SplitViewItem}; ]
                         [.\node{Editor}; ] ] ] ] ]
\end{tikzpicture}
\end{slide}

\begin{slide}{MVP}

\begin{tikzpicture}[node distance=2.0cm and 2.5cm]

\node (view)  
	[rectangle, draw = black, minimum height=1.5cm, minimum width=2.5cm] 
	{VIEW};
	\node (qml) [draw = white, node distance=0cm, above = of view] 
		{\textbf{qml + JavaScript}};

\node (presenter)  
	[rectangle, draw = black, minimum height=1.5cm, minimum width=2.5cm, below = of view] 
	{PRESENTER};
	\node (qobject) [draw = white, node distance=0cm, below = of presenter] 
		{\textbf{QObject}\\ \textbf{(QAbstractItemModel)}};

\node (model)  
	[rectangle, draw = black, minimum height=1.5cm, minimum width=2.5cm, right = of presenter] 
	{MODEL};
	\node (core) [draw = white, node distance=0cm, below = of model] 
		{\textbf{Core} \\ \textbf{(Memory, Architecture, ...)}};

\draw[->, thick] (model.170) -- (presenter.10) 
	node [midway, above=1pt, fill=white, draw=white] 
	{\textit{update callback}};
	
\draw[<-, thick] (model.190) -- (presenter.350)
	node [midway, below=1pt, fill=white, draw=white] 
	{\textit{set data}};
	
\draw[->, thick] (presenter.80) -- (view.280)
	node [midway, right=1pt, fill=white, draw=white] 
	{\textit{Qt signal: dataChanged() }};
	
\draw[<-, thick] (presenter.100) -- (view.260)
	node [midway, left=1pt, fill=white, draw=white] 
	{\textit{new data by user}};

\end{tikzpicture}

\end{slide}
