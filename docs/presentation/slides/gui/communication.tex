% !TEX root = ../../presentation.tex
% GUI

\begin{slide}{Kommunikation mit dem Core}

  \begin{tikzpicture}[auto, node distance=0.75cm and 1.5cm, thick]
  % NODE DEFINITIONS
  \tikzstyle{module} = [draw, rectangle,
  text width=4cm]
  \tikzstyle{class} = [rectangle, rounded corners=1pt, draw=black, fill=white, text width=2cm, align=center]
  \tikzstyle{modulearrow} = [->, thick]
  \tikzstyle{extension} = [dashed]

  \node (guiproject) [class] {GUIProject};

  \node (registermodel) [class, left = of guiproject] {RegisterModel};
  \node (memorycomponentpresenter) [class, above = of registermodel] {MemoryComponent-\\Presenter};
  \node (editorcomponent) [class, below = of registermodel] {EditorComponent};
  \node (guiecetera) [below = of editorcomponent, text width=2cm, align=center] {\large\dots};
  \node (guipackage) [above = 0.5cm of memorycomponentpresenter] {\large GUI};
  \node (guiext1) [above left = 0.2cm of memorycomponentpresenter] {};
  \node (guiext2) [above right = 0.2cm of memorycomponentpresenter] {};
  \node (guiext3) [below right = 0.2cm of guiecetera] {};
  \node (guiext4) [below left = 0.2cm of guiecetera] {};
  \draw [extension, rounded corners=0pt] (guiext1) -- (guiext2) -- (guiext3) -- (guiext4) -- (guiext1);

  \node (register) [class, right = of guiproject] {Register};
  \node (memory) [class, above = of register] {Memory};
  \node (parsingandexecutionunit) [class, below = of register] {ParsingAnd-\\ExecutionUnit};
  \node (coreecetera) [below = of parsingandexecutionunit, text width=2cm, align=center] {
    \vspace{-1cm}
    \large\dots
  };
  \node (corepackage) [above = 0.5cm of memory] {\large Core-Instanz};
  \node (coreext1) [above left = 0.2cm of memory] {};
  \node (coreext2) [above right = 0.2cm of memory] {};
  \node (coreext3) [below right = 0.2cm of coreecetera] {};
  \node (coreext4) [below left = 0.2cm of coreecetera] {};
  \draw [extension] (coreext1) -- (coreext2) -- (coreext3) -- (coreext4) -- (coreext1);

  \node (interfaces) [class, below = of guiproject] {Interfaces (Proxies)};

  \draw [->, shorten >= 2pt] (2.25, 0) -- (guiproject)
        node [midway, above] {Callback};

  \draw [->, shorten >= 2pt] (guiproject) -- (-2.3, 0)
        node [midway, above] {Signal};

  \draw [->, shorten >= 1pt] (-2.3, -1.2) -- (interfaces);

  \draw [->] (interfaces) -- (2.3, -1.225);

  \end{tikzpicture}
\end{slide}




\begin{slide}{MVP}

\begin{tikzpicture}[node distance=2.0cm and 2.5cm]

\node (view)
	[rectangle, draw = black, minimum height=1.5cm, minimum width=2.5cm]
	{VIEW};
	\node (qml) [draw = white, node distance=0cm, above = of view, above=1pt]
		{\textbf{qml + JavaScript}};

\node (presenter)
	[rectangle, draw = black, minimum height=1.5cm, minimum width=2.5cm, below = of view]
	{PRESENTER};
	\node (qobject) [draw = white, node distance=0cm, below = of presenter, below=1pt]
		{\textbf{QObject (QAbstractItemModel)}};

\node (model)
	[rectangle, draw = black, minimum height=1.5cm, minimum width=2.5cm, right = of presenter]
	{MODEL};
	\node (core) [draw = white, node distance=0cm, below = of model, below=1pt]
		{\textbf{Core (Memory, Architecture, ...)}};

\draw[->, thick] (model.170) -- (presenter.10)
	node [midway, above=1pt, fill=white, draw=white]
	{\textit{update callback}};

\draw[<-, thick] (model.190) -- (presenter.350)
	node [midway, below=1pt, fill=white, draw=white]
	{\textit{set data}};

\draw[->, thick] (presenter.80) -- (view.280)
	node [midway, right=1pt, fill=white, draw=white]
	{\textit{Qt signal: dataChanged() }};

\draw[<-, thick] (presenter.100) -- (view.260)
	node [midway, left=1pt, fill=white, draw=white]
	{\textit{new data by user}};

\end{tikzpicture}
\end{slide}
