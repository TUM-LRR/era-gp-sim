% !TEX root = dev-manual.tex
% ERA-Großpraktikum: Entwickleranleitung -- Architekturmodul (Architektur Implementierung)

\paragraph{Implementierungdetails der Architekturrepräsentation}
\label{dev:arch-impl}

Für die Implementierung von Architektur und Extensions wird intern das
\emph{Builder} Pattern\ zum sukzessivem Aufbau eines Objekts verwendet. Als
Oberklasse dient dazu das \texttt{BuilderInterface}, von dem alle Komponenten
abgeleitet werden. Da eine Architektur lediglich aus strukturierten
Informationen besteht, existiert eine weitere Oberklasse
\texttt{InformationInterface}. Von ihr sind wiederum die konkreten
Informationsklassen abgeleitet, welche Register, Instruktionen oder Datentypen
beschreiben. Nennenswert dabei ist die \texttt{ExtensionInformation} Klasse, die
in unserer modularen Auffassung eine Erweiterung einer Architektur
widerspiegelt. Eine finale \texttt{Architecture} Klasse besteht schließlich aus
einem \texttt{ExtensionInformation} Objekt, welches mit allen Erweiterungen
vereint wurde (also \texttt{vollständig} ist). \autoref{fig:arch-impl} zeigt die
Vererbungshierarchie für Informationsobjekte in der Architektur.

\begin{figure}[h!]
  \centering
  \begin{tikzpicture}[thick, node distance=1cm and 0.5cm]
    % Classes
    \node [class] (builder) {\texttt{BuilderInterface}};
    \node [class] (info) [below = of builder] {\texttt{InformationInterface}};
    \node [class] (inst) [below = of info] {\texttt{InstructionInformation}};
    \node [class] (reg) [left = of inst] {\texttt{RegisterInformation}};
    \node [class] (ext) [right = of inst] {\texttt{ExtensionInformation}};
    \coordinate (center) at ($ (inst) !.5! (info) $);

    % Edges
    \draw [inheritance-arrow] (info) -- (builder);
    \draw [inheritance-arrow] (inst) -- (info);
    \draw (reg) -- (reg |- center) -- (center);
    \draw (ext) -- (ext |- center) -- (center);

  \end{tikzpicture}
  \caption{Ein Ausschnitt der Vererbungshierarchie sämtlicher Informationsklassen der Architektur. Als Oberklasse dient das \texttt{BuilderInterface} sowie darunter das \texttt{InformationInterface}, von welchem weitere Informationsklassen wie \texttt{RegisterInformation} oder \texttt{Instructioninformation} erben.}
  \label{fig:arch-impl}
  \vspace{-0.5cm}
\end{figure}

Es ist zu beachten, dass das \texttt{Architecture} Objekt lediglich die
Eigenschaften einer Architektur beschreibt, d.h. es stellt Informationen wie die
Byte-Reihenfolge, Wortgröße, Speicherausrichtung oder die Eigenschaften der
Register bereit. Es beschreibt auch, welche Instruktionen zur Verfügung stehen,
definiert aber nicht, wie diese konkret implementiert sind, da dies für eine
Konfigurationsdatei zu komplex wäre. Die Implementierung der einzelnen
Instruktionen wird deshalb in C++ Code übernommen.
