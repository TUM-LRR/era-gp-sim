% !TEX root = dev-manual.tex
% ERA-Großpraktikum: Entwickleranleitung -- Architekturmodul (ISA Dateien)

\paragraph{Beschreibung einer Architektur in YAML}
\label{dev:arch-yaml}

Um die Flexibilität, Wartbarkeit und Sauberkeit unserer
Architekturimplementierung zu gewährleisten, haben wir uns entschieden, die
konkrete Beschreibung einer Architektur (die "Daten") von der abstrakten
Repräsentation im Code zu trennen. Die eben
beschriebenen Informationsklassen im Architekturmodul geben also lediglich die Struktur
eines Befehlssatzes vor. Weitere Klassen wie \texttt{RegisterInformation} oder
\texttt{InstructionInformation} agieren als Platzhalter für konkrete Architekturen.
Die eigentliche Spezifikation von Registernamen und -größen,
Instruktionsformaten und -operanden sowie Wortgröße, Endianness oder
Vorzeichenrepräsentation einer Architektur erfolgen in unserer Implementierung
über externe \emph{ISA}-Dateien. Diese werden aufgrund der intuitiven
und menschenlesbaren Syntax im YAML\footnote{\url{yaml.org}} Format definiert
und von diesem für die eigentliche Verarbeitung des Simulators schlussendlich
nach JSON konvertiert\footnote{Für die YAML $\rightarrow$ JSON Konvertierung kann das \texttt{y2j.py} Skript im \texttt{scripts/} Verzeichnis genutzt werden.}.

Eine Architekturbeschreibung folgt bezüglich seiner Struktur im Dateisystem
einem einfachen Schema: Eine Architektur ist ein Ordner mit einer \texttt{.isa}
Endung, beispielsweise \texttt{riscv.isa}. Dieser Ordner enthält für jede
definierte \emph{Extension}, sei es \texttt{rv32f} oder \texttt{rv64a}, einen
entsprechend benannten Unterordner. Jeder dieser Unterordner besteht dann aus
zwei weiteren Dateien: \texttt{config.yaml} und \texttt{config.json}. Ersteres
ist die Datei, welche Entwickler bearbeiten sollten, um die entsprechende
Extension zu definieren; letzteres sollte lediglich das Resultat einer
maschinellen Konvertierung sein. \autoref{lst:isa} zeigt die Definition zweier
Extensions einer fiktiven Architektur.

\begin{figure}[h!]
  \begin{minipage}{0.5\textwidth}
    \begin{lstlisting}[language=ISA]
      # Base Extension

      name: base
      word-size: 32
      endianness: mixed
      signed-representation: sign-bit

      units:
        - name: cpu
          registers:
            - name: r1
              id: 1
              size: 31
              type: float
    \end{lstlisting}
    \vspace{2.3cm}
  \end{minipage}
  \begin{minipage}{0.5\textwidth}
    \begin{lstlisting}[language=ISA, frame=l]
      # Derived Extension

      name: derived
      extends: base
      alignment-behavior: relaxed

      instructions:
        - mnemonic: add
          length: 32
          format: R
          key:
            opcode: 42
            funct: 6

      builtin-macros:
        - |
          .macro nop
          add x0, x0, 0
          .endm
    \end{lstlisting}
  \end{minipage}
  \begin{lstlisting}[label={lst:isa}, caption={Zwei Beispiele einer ISA-Definition in YAML Syntax. Die linke Definition zeigt eine \emph{unvollständige} Extension, welche lediglich eine Unit, aber keine Instruktionen definiert. Die rechte Definition erbt von dieser und erweitert sie durch einen Instruktionssatz und einem Alignment-Attribut. Die rechte Extension ist somit vollständig.}]
  \end{lstlisting}
  \vspace{-0.8cm}
\end{figure}

Grundsätzlich entsprechen die Namen der Schlüssel in diesen Dateien genau den
Attributen der Informationsklassen der Architektur. Beispielsweise hat jede
Extension einen Namen, welcher dann über die \texttt{name()} Methode einer
\texttt{ExtensionInformation} Instanz abgerufen werden kann. Eine vollständige
Auflistung aller möglichen Schlüsselnamen und erlaubter Werte findet sich in
Tabellen \ref{tbl:isa-top-keys}, \ref{tbl:isa-reg} und \ref{tbl:isa-inst}. Die
folgenden zwei Absätze beleuchten die Schlüssel \texttt{extends} und
\texttt{builtin-macros} näher.

\begin{table}[p]
  \centering
  \fontsize{10}{12}
  \begin{tabular}{>{\ttfamily}l >{\small}p{7.8cm} >{\ttfamily}p{3.6cm}}
    {\normalfont\bfseries Schlüssel} & \textbf{Beschreibung} & \textbf{Erlaubte Werte}\\
    \toprule
    name & Name der Extension & <string> \\

    extends & Nam(en) der Basis-Extension(s) & <string>/list<string>\\

    reset-units & Setzt vererbte Register zurück & true/false\\

    reset-instructions & Setzt vererbte Instruktionen zurück & true/false\\

    word-size & Die Wortgröße & <number>\\

    byte-size & Die Anzahl Bits im Byte & <number>\\

    alignment-behavior & Verhalten bei unausgerichteten Speicherzugriffen & \texttt{relaxed/strict}\\

    signed-representation & Vorzeichendarstellung & \texttt{ones-complement/} \texttt{twos-complement/} sign-bit \\

    endianness & Speicherausrichtung & bi/big/little/mixed \\

    builtin-macros & Liste vordefinierter Makros & list<string> \\

    units & Die Units der Extension & list<unit> \\

    instructions & Der Instruktionssatz der Extension & list<instruction>
  \end{tabular}
  \caption{Die erlaubten Schlüssel und Werte zur Beschreibung einer Extension im YAML Format. Die erste Spalte nennt den jeweiligen Schlüsselnamen, die zweite Spalte beschreibt diesen und die dritte Spalte gibt die erlaubten Werte an. Mit \texttt{<string>} ist eine beliebige Zeichenkette gemeint, mit \texttt{<number>} eine beliebige Ganzzahl und mit \texttt{list<type>} eine Liste mit Elementen vom Typ \texttt{type}.}
  \label{tbl:isa-top-keys}

  \vspace{0.9cm}

  \begin{tabular}{>{\ttfamily}l >{\small}l >{\ttfamily}l}
    {\normalfont\bfseries Schlüssel} & \textbf{Beschreibung} & \textbf{Erlaubte Werte}\\
    \toprule

    name & Name des Registers & <string>\\

    id & Eindeutiger Schlüssel des Registers & <number>\\

    alias & Mögliche Aliase des Registers & <string>/list<string>\\

    size & Größe des Registers & <number> \\

    constant & Fixer Wert des Registers (Optional) & <string> (0x0)
  \end{tabular}
  \caption{Die Schlüssel und erlaubten Werte zur Spezifikation von Registern in ISA-Dateien. Das Format folgt jenem von \autoref{tbl:isa-top-keys}.}
  \label{tbl:isa-reg}

  \vspace{0.9cm}

  \begin{tabular}{>{\ttfamily}l >{\small}l >{\ttfamily}l}
    {\normalfont\bfseries Schlüssel} & \textbf{Beschreibung} & \textbf{Erlaubte Werte}\\
    \toprule
    mnemonic & Kürzel der Instruktion & <string>\\

    length & Länge der Instruktion in Bits & <number>\\

    format & Format der Instruktion & <string> \\

    operand length & Länge der Operanden in Bits & list<number>\\

    key & Instruktionsschlüssel (Generalisierung des Opcodes) & map<key, number>

  \end{tabular}
  \caption{Die Schlüssel und erlaubten Werte zur Spezifikation von Instruktionen in ISA-Dateien. Das Format folgt jenem von \autoref{tbl:isa-top-keys}, bis auf \texttt{key}, welches eine Abbildung von Teilen des Instruktionsschlüssels auf die konkreten Werte dieser Teile ist.}
  \label{tbl:isa-inst}
\end{table}

Das Schlüsselwort \texttt{extends} stellt die mächtigste Funktionalität der
ISA-Spezifikationssprache dar: die Möglichkeit, Extensions voneinander erben zu
lassen. Das Konzept der Attributvererbung ist in vielen Situationen nicht nur
intuitiv, sondern schützt insbesondere vor Duplikation und spart Schreibarbeit.
Als anschauliches Beispiel lässt sich die Beziehung zwischen den \emph{RV32I}
und \emph{RV64I} Extensions des RISC-V Befehlssatzes betrachten. Letztere
Extension basiert laut Spezifikation auf der ersteren. Das bedeutet, dass RV64I
sämtliche Instruktionen und Registersätze besitzt, die in RV32I spezifiziert
sind. Zusätzlich erhöht RV64I jedoch noch die Wortgröße von 32 auf 64 Bit und
fügt weitere Instruktionen hinzu, die auf 32-Bit Werten operieren (da die
"geerbten" Instruktionen in RV64I implizit mit Wortgröße, also 64 Bit,
arbeiten). Diese Beziehung lässt sich in ISA-Dateien exakt spezifizieren. Ein
Ausschnitt der RV64I Spezifikation wird hierzu in \autoref{lst:rv64i} gezeigt.

Neben dieser einfachen Vererbungshierarchie zwischen RV64I und RV32I sind noch
weitere beliebig komplexe Beziehungen möglich, sofern diese sich in einem
\emph{azyklischen} Graphen darstellen lassen. Beim Laden einer Architektur über
eine \texttt{ArchitectureFormula}, also einer Liste der Namen aller Extensions,
wird über einen Graphalgorithmus die transitive Hülle sämtlicher benötigter
Erweiterungen berechnet. Steht diese Menge von Extensions fest, kann sie in eine
einzige vereint werden, wobei einfach die Vereinigung sämtlicher Attribute,
Register- und Instruktionssätze gebildet wird. Sofern im Beziehungsgraphen keine
Zyklen gefunden wurden und die resultierende Extension vollständig ist, entsteht
so eine fertige Architektur.

\begin{table}
\begin{lstlisting}[language=ISA]
                                    # RISC-V: RV64I

                                    name: rv64i
                                    extends: rv32i
                                    word-size: 64

                                    instructions:
                                      - mnemonic: addiw
                                        length: 32
                                        format: I
                                        operand length: [0, 0, 12]
                                        key:
                                          opcode: 27
                                          funct3: 0
\end{lstlisting}
\begin{lstlisting}[caption={Ein Ausschnitt der Definition der RV64I-Extension des RISC-V Befehlssatzes. Zur Vererbung der Attribute von RV32I benötigt es lediglich der Spezifikation der \texttt{extends} Klausel. Zusätzlich wird die Wortgröße sowie eine Hand voll weiterer Befehle neu definiert.}, label={lst:rv64i}]
\end{lstlisting}
\vspace{-0.8cm}
\end{table}

Neben dem \texttt{extends} Schlüssel ist letztlich noch \texttt{builtin-macros}
hervorzuheben. Dieser Schlüssel erwartet eine Liste von Strings, welche jeweils
ein Makro definieren, das für eine Architektur schon vordefiniert ist. Ein
solches vordefiniertes Makro kann hierbei auch \emph{Pseudoinstruktion} genannt
werden, da es dem Benutzer wie eine "eingebaute" Instruktion erscheint, jedoch
nicht standardmäßig zum Befehlssatz gehört und auch nicht in C++ implementiert
wird. Das Format eines solchen eingebauten Makros ist identisch zu einem benutzerdefinierten Makro.
