% ERA-Großpraktikum: Entwickleranleitung -- Allgemeines

\section{Allgemeines}

Diese Sektion ist eine Einführung in die von uns genutzten Entwicklungswerkzeuge,
welche für die Entwicklung des Simulators benötigt werden. Das Ziel dieser Sektion
ist es, dass Leser am Ende in der Lage sind, eigenständig einen Build des Projekts
durchzuführen und somit am Projekt arbeiten können. \\
Des Weiteren wird erläutert, welche Philosophie hinter der Auswahl an Werkzeugen
steckt, von der wir uns erhoffen, dass zukünftige Entwickler diese fortsetzen.

\subsection{Git}
Bei der Entwicklung von Software im Team ist es zwingend Notwendig, viele verschiedene Versionen
der selben Software zu verwalten. Deshalb haben wir uns für
\textit{Git}\footnote{\url{https://git-scm.com/}} als Versionskontrollsystem entschieden. \\
Das allein beschreibt allerdings noch nicht, \textit{wie} wir Git verwenden. Daher haben wir uns für
die folgenden Systeme entschieden:

\subsubsection{Gitflow}

\textit{Gitflow}\footnote{\url{https://www.atlassian.com/git/tutorials/comparing-workflows/}} ist
ein beliebtes Modell der Git-Nutzung, welches vor allem bei der Verwaltung von großen Projekten
zum Einsatz kommt. Die von uns genutzte Variante lässt sich wie folgt beschreiben:

\todo[inline]{Ich bin hier schon mal einen Schritt weiter gegangen und hab noch eine Unterscheidung
zwischen dev und master Branch eingefügt. So wollten wir das ja ursprünglich machen, aber aus
Bequemlichkeit nutzen wir nur den master. Falls nach der Abgabe noch Änderungen vorgenommen werden,
sollten wir das dann nach dem "richtigen" Modell machen.}

\begin{itemize}
	\item Es gibt einen \textit{Master}-Branch, der immer die aktuellste, stabile Version enthält.
	\item Es gibt einen \textit{Development}-Branch, der den aktuellen Stand bei der Entwicklung
	einer neuen Version enthält. Der Code ist ebenfalls stabil, wenn auch manche Funktionen noch
	nicht vollständig sein können.
	\item Die einzelnen Aufgaben werden isoliert in einer \textit{Feature}-Branch bearbeitet.
	\item Ist eine Aufgabe fertig, so wird der entsprechende Feature Branch in den Development
	Branch gemerged.
	\item Ist eine neue Version auf dem Development Branch fertiggestellt, so wird die Development
	Branch in den Master Branch gemerged.
\end{itemize}

\subsubsection{Github}

\todo[inline]{Das hier geht schon in Richtung Projektleiterbericht, ich denke es ist aber trzdm sinnvoll,
hier ein paar Worte zu verlieren, da es schließlich zur Entwicklung dazugehört}

Wir nutzen \textit{Github}\footnote{\url{https://github.com/}} as Filehoster des Simulators. Neben
Gründen wie der großen Bekanntheit von Github und der auch für Einsteiger geeigneten grafischen
Oberfläche, sind vor allem folgenden Funktionen hilfreich bei der Entwicklung:

\begin{itemize}
	\item \textit{Issues}: Die im vorherigen Kapitel genannten Aufgaben werden mit einem Issue
	verwaltet. Über die Nummer eines Issues lässt sich der Status der Entwicklung verfolgen.
	\item \textit{Pull Requests}: Nach der Fertigstellung einer Feature Branch wird ein Pull Request
	erstellt, sodass andere Entwickler ein Code Review anfertigen können.
	\item Integration von Tools wie \textit{Waffle}\footnote{\url{https://waffle.io/}} und
	\textit{Travis CI}\footnote{\url{https://travis-ci.org/}}
\end{itemize}

\subsection{C++}

Der Simulator ist zum Großteil in C++ geschrieben. Unser Augenmerk liegt dabei auf Portabilität
zwischen verschiedenen Compilern, weshalb wir uns auf den \textbf{C++14} Standard geeinigt haben,
sowie möglichst wenige externe Bibliotheken verwenden.

\subsubsection{Compiler}

Wir arbeiten mit folgenden Compilern:

\begin{itemize}
	\item \textit{g++ >= 5}
	\item \textit{clang++ >= 3.7}
	\item \textit{MinGW >= ???}
	\item \textit{MSVC >= ???}
	\todo{stimmt das?}
\end{itemize}

\subsubsection{Bibliotheken}

Die einzige, für das Endprodukt benötigte Abhängigkeit ist \textit{Qt}\footnote{\url{https://www.qt.io/}}
>= 5.6 in Kombination mit QtQuick >= 2.6. Wir verwenden Qt zur Erstellung der grafischen Oberfläche,
haben aber grundsätzlich eine strikte Trennung zwischen Code der GUI und dem Rest des Simulators.
Durch diese Trennung wäre es möglich, den Simulator auch ohne Qt zur Verfügung zu stellen, z.B.
mit einer anderen Grafikbibliothek oder als Kommandozeilenprogramm. \\
Eine Ausnahme zu dieser Trennung bildet die Internationalisierung des Simulators. Der Simulator ist
in der aktuellen Version ausschließlich in Englisch verfügbar, uns war es aber ein großes Anliegen,
auch andere Sprachen unterstützen zu können. Somit verwenden wir die von Qt zur Verfügung gestellten
Mittel, Nutzer sichtbare Nachrichten übersetzbar zu machen, auch in allen anderen Modulen außerhalb
der GUI. So können z.B. Prozessorarchitektur-spezifische Fehlermeldungen übersetzt werden.


\subsection{Build Umgebung}

Möchte man den Simulator selbst kompilieren, so wird eine Build Umgebung vorausgesetzt. 
Wie bereits genannt, liegt unser Augenmerk auf Portabilität zwischen Betriebssystemen,
deshalb wird in dieser Sektion gezeigt, wie man eine solche Umgebung unter den gängigen
Betriebssystemen (Linux, Mac, Windows) einrichten kann.

\subsubsection{Linux}

Da wir eine relativ neu Versionen von Qt verwenden, ist eine reine Installation der
Abhängigkeiten über die Paketverwaltung nur unter solchen Linux Distributionen möglich,
die auch die entsprechenden Versionen in den Paketquellen beinhalten. Während dies unter
Rolling Release Systemen (wie z.B. Arch Linux) ein geringeres Problem ist, haben Anwender von
z.B. Debian basierter Betriebssysteme häufiger mit älterer Software zu kämpfen.
Unter Debian selbst sind die benötigten Versionen erst ab Debian Stretch (Debian 9)
verfügbar, unter Ubuntu erst ab Ubuntu 16.10. \\
Eine Installation unter Debian basierten Betriebssystemen könnte wie folgt ablaufen: \\

\begin{lstlisting}
apt install git build-essential cmake qtdeclarative5-dev
\end{lstlisting}

Ist eine Installation über die Paketquellen nicht möglich, bietet es sich an, Qt manuell
zu installieren\footnote{\url{https://wiki.qt.io/Install_Qt_5_on_Ubuntu}}.

\todo [inline]{Manuelle Installation von Qt und dann Verwendung des QtCreators noch genauer
beschreiben. Ich machs über die Kommandozeile, kenn mich also mit QtCreator etc. nicht so aus}

\subsubsection{Mac}

Wirklich?

\subsubsection{Windows}

Selbst schuld.

\subsection{CMake / Make}

Um auch den Build Prozess unabhängig vom Betriebssystem durchzuführen, haben wir uns für
\textit{CMake}\footnote{\url{https://cmake.org/}} entschieden. CMake generiert die benötigten
\texttt{Makefiles} für jedes Betriebssystem bzw. jeden Compiler neu, welche anschließend
von \textit{Make}\footnote{\url{https://www.gnu.org/software/make/}} ausgeführt werden können.

Um einen Build des Projekts durchzuführen, sollte vorher einen entsprechende Build Umgebung
aufgesetzt werden. Anschließend kann man mit den Folgenden Befehlen einen Clean Build erzeugen:

\begin{lstlisting}
# Projekt herunterladen
git clone https://github.com/TUM-LRR/era-gp-sim

# In das Projektverzeichnis wechseln
cd era-gp-sim

# Submodule initialisieren (Google Test)
git submodule update --init

# Build Verzeichnis erstellen und betreten
mkdir build && cd build

# Mit CMake die Makefiles erzeugen
cmake ..

# Projekt bauen (Die Option -j setzt Anzahl an
# Jobs und beschleunigt so den Build Prozess)
make -j 4

# Programm starten
./bin/era-sim
\end{lstlisting}

\subsection{Tests}

Wir nutzten das \textit{Google Test}\footnote{\url{https://github.com/google/googletest}} Framework
zur Automatisierung unserer Tests. Die Tests befinden sich im \texttt{tests/} Verzeichnis, gruppiert
nach ihren Modulen. Da Tests der grafischen Oberfläche zu komplex wären, haben wir uns auf die übrigen
Module des Simulators beschränkt, deren korrekte Funktionalität in jeweils separaten Unit Tests
überprüft wird.

Um ein einheitliches Ergebnis bei den Tests zu erzielen, nutzen wir das Continuous Integration Tool
\textit{Travis CI}\footnote{\url{https://travis-ci.org/}}. Dabei wird nach jedem hoch geladenen
Commit ein Clean Build angefertigt, der alle verfügbaren Tests durchführt. Das Ergebnis dieses Tests
wird auf Github veröffentlicht und ist bindend. Die Tests werden sowohl mit g++, als auch mit clang
durchgeführt.

Die Tests können auch lokal ausgeführt werden. Dazu wechselt man mit \texttt{cd} in das Build
Verzeichnis und nutzt das von CMake mitgelieferte Programm \texttt{ctest} um die Tests zu starten.
Um die Testgeschwindigkeit zu erhöhen empfiehlt es sich, mit der \texttt{-j} Option die Anzahl
der Jobs zu erhöhen. 


