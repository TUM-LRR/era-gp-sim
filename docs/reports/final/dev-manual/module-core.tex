\subsection{Core}
\label{Dev-Kapitel: Core}

Der Core ist das zentrale Modul des Simulators. Er übernimmt die Verbindung der
GUI mit den anderen Modulen, der Architektur und dem Parser. Dabei ist der Core
aber nicht nur ein Interface, sondern verwaltet auch das interne Speicher- und
Registermodell. Dabei werden Art, Anzahl und Namen der Register direkt aus der
Architekturdefinition geladen, die dem Core über ein Architekturobjekt zur
Verfügung steht. Weiterhin verwaltet der Core auch das Parser-Objekt und erhält
auch eine Liste fertiger Befehlsknoten, wenn der Parser ein Programm übersetzt
hat. Dabei ist der Core dafür zuständig, die Assemblierungsfunktionen der
Befehlsknoten aufzurufen und das Ergebnis in den Speicher zu schreiben. Der Core
steuert auch die Ausführung, indem die Befehlsknoten nacheinander ausgeführt
werden.

\subsubsection{Aufbau}

Der Core ist zwar das zentrale Modul des Simulators, seine Instanzen werden
jedoch von der GUI verwaltet. Dabei erstellt die GUI für jedes neue Projekt, was
in der Benutzeroberfläche einem Tab entspricht, eine neue Instanz des Cores über
das \texttt{ProjectModule}. Dieses verwaltet (indirekt) die Komponenten eines
Projektes, also Speicher, Register, Architekturdefinition und Parser, und
ermöglicht der GUI über Interface-Klassen sicheren Zugriff auf die anderen
Komponenten des Simulators. Eine zentrale Anforderung an den Simulator war, dass
die Ausführung der Befehle nicht die Ansprechbarkeit der GUI beeinflusst. Daher
ist es notwendig den GUI-Thread, in dem ein Event-Loop der Qt-Engine läuft, von
der Ausführung unabhängig zu machen. Daher werden pro Core-Instanz weitere
Threads erstellt.

\begin{figure}[H]
    \begin{center}
        \begin{tikzpicture}[node distance=1.0cm and 2.5cm]

            \node (projectModule) [class] {ProjectModule};

            \node (architectureAccess) [class, right = of projectModule] {ArchitectureAccess};
            \node (commandInterface) [class, below = of architectureAccess] {CommandInterface};
            \node (parserInterface) [class, below = of commandInterface] {ParserInterface};
            \node (memoryManager) [class, above = of architectureAccess] {MemoryManager};
            \node (memoryAccess) [class, above = of memoryManager] {MemoryAccess};
            \node (interface) [below = of parserInterface] {\textbf{Synchronisiertes Interface}};

            \draw[composition-arrow] (projectModule) -- (memoryAccess);
            \draw[composition-arrow] (projectModule) -- (memoryManager);
            \draw[composition-arrow] (projectModule) -- (architectureAccess);
            \draw[composition-arrow] (projectModule) -- (commandInterface);
            \draw[composition-arrow] (projectModule) -- (parserInterface);

            \begin{pgfonlayer}{background}
                \path (memoryAccess.west |- memoryAccess.north)+(-1.3,1.0) node (a1) {};
                \path (parserInterface.east |- parserInterface.south)+(1.3,-2.0) node (a2) {};
                \path[rounded corners, draw=black!50, dashed] (a1) rectangle (a2);
            \end{pgfonlayer}{background}

        \end{tikzpicture}
    \end{center}
    \caption{Core Modul von außen}
    \label{fig:core-structure}
\end{figure}

\subsubsection{Threading}

Bei der Verwendung von mehreren Threads ist erhöhter Programmieraufwand
notwendig, um Probleme wie Race-Conditions zu vermeiden. Daher verwenden wir das
\textit{Active Object Pattern}, da dadurch der Aufwand hauptsächlich auf die
Implementierung dieses Patterns im Core reduziert wird. Das \textit{Active
Object Pattern} sieht vor, dass für ein Objekt auf dem Synchronisierung
notwendig ist, auch \textit{Active Object} oder \textit{Servant} genannt, ein
eigener Thread erstellt wird. Dabei wird sichergestellt, dass die Methoden
dieses Objektes nur in diesem Thread laufen. Dazu wird pro \textit{Servant} ein
\textit{Scheduler} benötigt, der Befehle zur Ausführung einer Methode des
Objektes in einer Warteschlange verwaltet und nacheinander ausführt. Dieser
verwaltet auch den Thread des Servants. Der Zugriff von außen auf dieses
\textit{Active Object} darf nur über sogenannte \textit{Proxy}-Objekte erfolgen,
die Methodenaufrufe in die Warteschlange des Schedulers einreihen. Die Rückgabe
von Werten erfolgt dabei über \texttt{futures} oder Callbacks. Es wurden dabei
auch sichere Callbacks implementiert, die jeweils im aufrufenden Thread
ausgeführt werden. Diese können jedoch nur verwendet werden, wenn der Aufrufer
auch ein Active Object ist. Diese Art der Callbacks fand daher bis jetzt keine
Verwendung. Bei normalen Callbacks muss darauf geachtet werden, dass diese immer
noch im jeweiligen Active-Object-Thread ausgeführt werden. Im Bereich der GUI
werden die Callbacks daher über Qt-Signals in den GUI-Thread übertragen (siehe
\ref{Dev-Kapitel: gui-kommunikation}).

Wenn in einem \textit{Active Object} gerade kein Befehl ausgeführt wird, schläft
der Scheduler (also der Active-Object-Thread) so lange, bis er durch das
Einfügen eines Befehls in seine Warteschlange aufgeweckt wird. Dadurch wird
durch die zusätzlichen Threads im Allgemeinen keine zusätzliche Performance
beansprucht. Trotzdem ist es natürlich nicht sinnvoll, sehr viele \textit{Active
Objects} zu erstellen. Im \texttt{ProjectModule} werden daher nur zwei
\textit{Active Objects} verwaltet: Das \texttt{Project} Objekt verwaltet den
Speicher, die Register und das Architekturobjekt. Dabei wird ein eigenes Active
Object mit einem Thread benötigt, da es beispielsweise möglich sein muss, über
die Benutzeroberfläche während der Ausführung auf den Speicher zuzugreifen. Das
zweite Active Object wird für die Ausführung und Übersetzung der Programme
benötigt. Dies ist in der \texttt{ParsingAndExecutionUnit} umgesetzt. Dadurch
bleibt die GUI auch während der Ausführung ansprechbar. Zusammenfassend gibt es
also pro Projekt (Tab in der GUI) zwei Threads im Core.

Weiterführende Informationen zum Threading-Konzept mit Beispielen können in der Datei \texttt{include/core/proxy.hpp} gefunden werden.

\begin{figure}
    \begin{center}
        \begin{tikzpicture}[node distance=1.0cm and 2.5cm]
        \node (servant) [class] {Servant};
        \node (scheduler) [class, left = of servant] {Scheduler};
        \node (proxy) [class, left = of scheduler] {Proxy};

        \node (activeObject) [below = of {$(servant)!0.5!(scheduler)$}] {\textbf{Active Object Thread}};

        \draw[dashed-arrow] (proxy) to node[yshift=0.3cm]{\small{push(task)}} (scheduler);
        \draw[dashed-arrow] (scheduler) to node [yshift=0.3cm]{\small{execute(task)}} (servant);

        \begin{pgfonlayer}{background}
            \path (scheduler.west |- scheduler.north)+(-0.5, 0.5) node (a1) {};
            \path (servant.east |- servant.south)+(0.5, -1.5) node (a2) {};
            \path[rounded corners, draw=black!50, dashed] (a1) rectangle (a2);
        \end{pgfonlayer}{background}
    \end{tikzpicture}
    \end{center}
    \caption{Aufbau eines Active Objects}
    \label{fig:active-object-pattern}
\end{figure}

\subsubsection{Interface}

\begin{figure}
    \begin{center}
        \begin{tikzpicture}
            \node (memoryAccess) [class] {MemoryAccess};
            \node (memoryManager) [class, below = of memoryAccess] {MemoryManager};
            \node (architectureAccess) [class, below = of memoryManager] {ArchitectureAccess};
            \node (projectLabel) [below = 0.5cm of architectureAccess] {\textbf{Project Proxy}};

            \node (commandInterface) [class, below = 2cm of projectLabel] {CommandInterface};
            \node (parserInterface) [class, below = of commandInterface] {ParserInterface};
            \node (parsingLabel) [below = 0.5cm of parserInterface] {\textbf{ParsingAndExecutionUnit Proxy}};

            \begin{pgfonlayer}{background}
                \path (memoryAccess.west |- memoryAccess.north)+(-1.0, 0.5) node (a1) {};
                \path (architectureAccess.east |- architectureAccess.south)+(1.0, -1.5) node (a2) {};
                \path[rounded corners, draw=black!50, dashed] (a1) rectangle (a2);
            \end{pgfonlayer}{background}


            \begin{pgfonlayer}{background}
                \path (commandInterface.west |- commandInterface.north)+(-2.5, 1.0) node (a1) {};
                \path (parserInterface.east |- parserInterface.south)+(2.5, -1.5) node (a2) {};
                \path[rounded corners, draw=black!50, dashed] (a1) rectangle (a2);
            \end{pgfonlayer}{background}

            \node (project) [class, right = 5cm of memoryManager] {Project};
            \node (parsingUnit) [class, right = 5cm of {$(commandInterface)!0.5!(parserInterface)$}] {ParsingAndExecutionUnit};

            \draw[dashed-arrow] (memoryAccess) -- (project);
            \draw[dashed-arrow] (memoryManager) -- (project);
            \draw[dashed-arrow] (architectureAccess) -- (project);
            \draw[dashed-arrow] (commandInterface) -- (parsingUnit);
            \draw[dashed-arrow] (parserInterface) -- (parsingUnit);

        \end{tikzpicture}

    \end{center}

\end{figure}

Die Kommunikation des Cores mit der GUI teilt sich auf zwei Richtungen auf:

Die GUI muss auf Methoden des Cores zugreifen können, um an den Zustand von
Speicher und Register zu gelangen und diesen zu ändern, oder auch um Befehle
auszuführen. Dabei werden der GUI über das \texttt{ProjectModule} Interface
Klassen zur Verfügung gestellt, die \texttt{Proxy} Klassen der Active Objects
sind. Diese Interface Klassen werden teilweise auch an die anderen Komponenten
übergeben, um dem Parser und den Befehlen beispielsweise Zugriff auf Speicher
und Register zu ermöglichen. Im einzelnen sind die Interface Klassen des
\texttt{Project}s der \texttt{MemoryAccess}, der Zugriff auf Speicher und
Register ermöglicht, der \texttt{MemoryManager}, über den weitere Funktionen von
Speichern und Register verwaltet werden können, und der
\texttt{ArchitectureAccess}, über den auf das Architekturobjekt des Projekts
zugegriffen werden kann. Die Interfaces der \texttt{ParsingAndExecutionUnit}
sind das \texttt{CommandInterface}, das beispielsweise die Übermittlung von
Ausführungsbefehlen erlaubt, und das \texttt{ParserInterface}, über das auf
Informationen des Parsers zugegriffen werden kann.

Des weiteren muss der GUI auch signalisiert werden, dass sich der Zustand des
Cores geändert hat und die Ansichten aktualisiert werden sollen. Dafür werden
Callbacks verwendet, damit der Core keine Referenz auf die GUI braucht und daher
unabhängig von deren Aufbau ist. Diese Callbacks werden über die Interface
Klassen gesetzt und bei entsprechenden Änderungen durch die Core-Komponenten
aufgerufen. Dabei ist zu beachten, dass die Callbacks im jeweiligen Active
Object ausgeführt werden, also auch in dessen Thread. Daher ist es nötig, in den
Callback Funktionen, die in der GUI Implementiert werden, Qt-Signals mit
\texttt{QueuedConnections} zu verwenden, um die entsprechenden die
erforderlichen Aktionen in der GUI im richtigen Thread auszuführen. Da jeweils
nur ein Callback im Core gesetzt werden kann, ist es über diesen Mechanismus
auch möglich, mehrere Komponenten in der GUI zu benachrichtigen.

\subsubsection{Ausführung}

Der Core ist auch die Schnittstelle der GUI zum Parser, daher übernimmt er den
Aufruf des Parsers und das Einfügen der assemblierten Befehlsdarstellung in den
Speicher. Dabei wird dieser Speicherbereich vor der normalen Bearbeitung durch
den Nutzer oder Assemblerbefehle geschützt.

Nach der Übersetzung liefert der Parser dem Core ein Liste von Befehlen, die
ausgeführt werden können, oder eine List an Fehlermeldungen. Die Ausführung
läuft dabei auch in einem Active Object, wodurch die GUI nicht beeinträchtigt
wird. Da deshalb während der laufenden Ausführung alle anderen Befehle für
dieses Active Object warten müssen, kann ein Stop-Signal nicht wie ein normaler
Aufruf realisiert werden. Deshalb wird dazu ein mit dem \texttt{ProjectModule}
geteiltes Flag verwendet. Dieses wird immer geprüft, wenn ein neuer Befehl
ausgeführt wird.

Vor der Ausführung wird jeder Befehl mit seiner \texttt{validateRuntime()}
Methode validiert, um Fehler, die nur während der Laufzeit überprüfbar sind,
abzufangen. Bei einem Fehler wird dabei die Ausführung abgebrochen und die GUI
über einen Callback benachrichtigt. Falls kein Fehler aufgetreten ist, wird der
Befehl mit über seine \texttt{getValue()} Methode ausgeführt, dabei wird als
Rückgabewert der neue Befehlszählerstand geliefert. Dieser Wert wird vom Core in
das Befehlszählerregister geschrieben.

Falls nicht nur einzelne Befehle ausgeführt werden, ist auch zu beachten, dass
durch die Ausführung jedes Befehls einige Callbacks an die GUI geschickt werden,
wodurch in bestimmten Situationen das Message-System der Qt-Engine überlastet
werden kann. Um dies zu verhindern, wird nach jedem Befehl ein
Synchronisations-Callback aufgerufen und mit der Ausführung gewartet, bis dieser
in der GUI ausgeführt wurde. Der Synchronisations-Callback muss dazu die
\texttt{ProjectModule::guiReady()} Methode aufrufen. Dadurch wird
sichergestellt, dass die Message-Queue abgearbeitet wurde und eine Überlastung
verhindert.

\subsubsection{Memory und Register}

Hier passiert viel Magie.

\subsubsection{Weiterführende Dokumentation}

In den vorherigen Sektionen wird ein Überblick über den Core gegeben. Weiterführende Dokumentation, Details und Beispiele können in den entsprechenden Dateien gefunden werden. Diese sind in den Ordnern, die in \autoref{fig:core-further} aufgelistet sind.

\begin{figure}[H]
	\begin{center}
	\begin{tikzpicture}[%
	grow via three points={one child at (0.8,-0.8) and
		two children at (0.8,-0.8) and (0.8,-1.7)},
	edge from parent path={($(\tikzparentnode\tikzparentanchor)+(.2cm,0pt)$) |- (\tikzchildnode\tikzchildanchor)},
	growth parent anchor=west,
	parent anchor=south west]
	\tikzstyle{every node}=[draw=black,anchor=west]
	\node {\erasim}
	child { node {$\{\text{tests/}, \text{include/}, \text{source/}\}$}
		child { node {core/}}
	};
	\end{tikzpicture}
	\end{center}

	\caption{Relevante Dateien des Coremoduls}
	\label{fig:core-further}
\end{figure}
