\subsection{Core}
\label{Dev-Kapitel: Core}

\subsubsection{Aufbau}

Der Core ist zwar das zentrale Modul des Simulators, seine Instanzen werden
jedoch von der GUI verwaltet. Dabei erstellt die GUI für jedes neue Projekt, was
in der Benutzeroberfläche einem Tab entspricht, eine neue Instanz des Cores über
das \texttt{ProjectModule}. Dieses verwaltet (indirekt) die Komponenten eines
Projektes, also Speicher, Register, Architekturdefinition und Parser, und
ermöglicht der GUI über Interface-Klassen sicheren Zugriff auf die anderen
Komponenten des Simulators. Um den GUI-Thread, in dem ein Event-Loop der
Qt-Engine läuft, von der Übersetzung und Ausführung unabhängig zu machen, werden
dabei pro Core-Instanz mehrere Threads benötigt.

\subsubsection{Threading}

Zur Vermeidung von Race Conditions und anderen Problemen wird das Konzept des
\textit{Active Object Pattern} verwendet: Dabei wird für ein Objekt ein Thread
erstellt und sichergestellt, dass die Methoden dieses Objektes nur in diesem
Thread laufen. Dazu wird pro \textit{Active Object} ein \textit{Scheduler}
benötigt, der Befehle zur Ausführung einer Methode des Objektes in einer
Warteschlange verwaltet und nacheinander ausführt. Der Zugriff von außen auf das
\textit{Active Object} darf daher nur über sogenannte \textit{Proxy} Objekte
erfolgen, die Methodenaufrufe in die Warteschlange des Schedulers einreihen. Die
Rückgabe von Werten erfolgt dabei über \texttt{futures} oder callbacks, wobei
letzteres nur zu anderen Active Objects möglich ist und bisher nicht verwendet
wird. Dieses Konzept wurde gewählt, um die Risiken von Threading-bezogenen
Fehlern zu minimieren und den Aufwand zur Synchronisierung, bis auf die
Implementierung des Active Object Patterns, gering zu halten.

Im \texttt{ProjectModule} werden zwei \textit{Active Objects} verwaltet: Das
\texttt{Project} Objekt verwaltet den Memory und die Register. Dabei wird ein
eigenes Active Object mit einem Thread benötigt, da es beispielsweise möglich
sein muss, über die Benutzeroberfläche während der Ausführung auf den Speicher
zuzugreifen. Das zweite Active Object wird für die Ausführung und Übersetzung
der Programme benötigt. Dies ist in der \texttt{ParsingAndExecutionUnit}
umgesetzt. Dadurch bleibt die GUI auch während der Ausführung ansprechbar.

\subsubsection{Interface}

Die Kommunikation des Cores mit der GUI teilt sich auf zwei Richtungen auf:

Die GUI muss auf Methoden des Cores zugreifen können, um an den Zustand von
Speicher und Register zu gelangen und diesen zu ändern, oder auch um Befehle
auszuführen. Dabei werden der GUI über das \texttt{ProjectModule} Interface
Klassen zur Verfügung gestehlt, die \texttt{Proxy} Klassen der Active Objects
sind. Diese Interface Klassen werden teilweise auch an die anderen Komponenten
übergeben, um dem Parser und den Befehlen beispielsweise Zugriff auf Speicher
und Register zu ermöglichen. Im einzelnen sind die Interface Klassen des
\texttt{Project}s der \texttt{MemoryAccess}, der Zugriff auf Speicher und
Register ermöglicht, der \texttt{MemoryManager}, über den weitere Funktionen von
Speichern und Register verwaltet werden können, und der
\texttt{ArchitectureAccess}, über den auf das Architekturobjekt des Projekts
zugegriffen werden kann. Die Interfaces der \texttt{ParsingAndExecutionUnit}
sind das \texttt{CommandInterface}, das beispielsweise die Übermittlung von
Ausführungsbefehlen erlaubt, und das \texttt{ParserInterface}, über das auf
Informationen des Parsers zugegriffen werden kann.

Desweiteren muss der GUI auch signalisiert werden, dass sich der Zustand des
Cores geändert hat und die Ansichten aktualisiert werden sollen. Dafür werden
Callbacks verwendet, damit der Core keine Referenz auf die GUI braucht und daher
unabhängig von deren Aufbau ist. Diese Callbacks werden über die Interface
Klassen gesetzt und bei entsprechenden Änderungen durch die Core-Komponenten
aufgerufen. Dabei ist zu beachten, dass die Callbacks im jeweiligen Active
Object ausgeführt werden, also auch in dessen Thread. Daher ist es nötig, in den
Callback Funktionen, die in der GUI Implementiert werden, Qt-Signals mit
\texttt{QueuedConnections} zu verwenden, um die entsprechenden die
erforderlichen Aktionen in der GUI im richtigen Thread auszuführen. Da jeweils
nur ein Callback im Core gesetzt werden kann, ist es über diesen Mechanismus
auch möglich, mehrere Komponenten in der GUI zu benachrichtigen.

\subsubsection{Ausführung}

Der Core ist auch die Schnittstelle der GUI zum Parser, daher übernimmt er den
Aufruf des Parsers und das Einfügen der assemblierten Befehlsdarstellung in den
Speicher. Dabei wird dieser Speicherbereich vor der normalen Bearbeitung durch
den Nutzer oder Assemblerbefehle geschützt.

Nach der Übersetzung liefert der Parser dem Core ein Liste von Befehlen, die
ausgeführt werden können, oder eine List an Fehlermeldungen. Die Ausführung
läuft dabei auch in einem Active Object, wodurch die GUI nicht beeinträchigt
wird. Da deshalb während der laufenden Ausführung alle anderen Befehle für
dieses Active Object warten müssen, kann ein Stop-Signal nicht wie ein normaler
Aufruf realisiert werden. Deshalb wird dazu ein mit dem \texttt{ProjectModule}
geteiltes Flag verwendet. Dieses wird immer geprüft, wenn ein neuer Befehl
ausgeführt wird.

Falls nicht nur einzelne Befehle ausgeführt werden, ist auch zu beachten, dass
durch die Ausführung jedes Befehls einige Callbacks an die GUI geschickt werden,
wodurch in bestimmten Situationen das Message-System der Qt-Engine überlastet
werden kann. Um dies zu verhindern, wird nach jedem Befehl ein
Synchronisations-Callback aufgerufen und mit der Ausführung gewartet, bis dieser
in der GUI ausgeführt wurde. Der Synchronisations-Callback muss dazu die
\texttt{ProjectModule::guiReady()} Methode aufrufen. Dadurch wird
sichergestellt, dass die Message-Queue abgearbeitet wurde und eine Überlastung
verhindert.

\subsubsection{Memory und Register}

Hier passiert viel Magie.
