\subsection{Core}
\label{Dev-Kapitel: Core}

Der Core ist das zentrale Modul des Simulators. Er übernimmt die Verbindung der
GUI mit den anderen Modulen, der Architektur und dem Parser. Dabei ist der Core
aber nicht nur ein Interface, sondern verwaltet auch das interne Speicher- und
Registermodell. Dabei werden Art, Anzahl und Namen der Register direkt aus der
Architekturdefinition geladen, die dem Core über ein Architekturobjekt zur
Verfügung steht. Weiterhin verwaltet der Core auch das Parser-Objekt und erhält
auch eine Liste fertiger Befehlsknoten, wenn der Parser ein Programm übersetzt
hat. Dabei ist der Core dafür zuständig, die Assemblierungsfunktionen der
Befehlsknoten aufzurufen und das Ergebnis in den Speicher zu schreiben. Der Core
steuert auch die Ausführung, indem die Befehlsknoten nacheinander ausgeführt
werden.

\begin{figure}[H]
    \begin{center}
        \begin{tikzpicture}[thick]
            \tikzset{block/.style={%
            draw,%
            rectangle,%
            rounded corners,%
            text width=2cm,%
            text height=0.45cm}%
            };
            \tikzset{smallblock/.style={block, text width=1.5cm, text height=0.3cm}};

            % GUI
            \path (0, 0) coordinate [block] (gui) node {GUI};

            % Core
            \node (core) [below = 1.3 of gui] {\textbf{Core}};
            \draw [<->] (gui) -- ($(core.north)+(0.0, 0.5)$);
            % core outline
            \path (core.west |- core.north)+(-1.0,0.5) node (a1) {};
            \path (core.east |- core.south)+(1.0,-3.8) node (a2) {};
            \path[rounded corners, draw=black] (a1) rectangle (a2);

            % Memory
            \node (memory) [block, below = 0.5 of core, align=center] {Speicher};

            % Register
            \node (register) [block, below = of memory, align=center] {Register};

            % arch
            \path ($(core.west)+(-3.5, -1.5)$) coordinate [block] (arch) node {Architektur};
            \node (leftConnection) at (-1.4, -3.5) {};
            \draw [<->] (arch) -- ($(core.west)+(-1.0, -1.5)$);

            % parser
            \path ($(core.east)+(3.5, -1.5)$) coordinate [block] (parser) node {Parser};
            \draw [<->] (parser) -- ($(core.east)+(1.0, -1.5)$);
        \end{tikzpicture}
    \end{center}
    \caption{Grundlegende Funktionen des Cores.}
    \label{fig:core-overview}
\end{figure}

\subsubsection{Aufbau}

Der Core ist zwar das zentrale Modul des Simulators, seine Instanzen werden
jedoch von der GUI verwaltet. Dabei erstellt die GUI für jedes neue Projekt, was
in der Benutzeroberfläche einem Tab entspricht, eine neue Instanz des Cores über
das \texttt{ProjectModule}. Dieses verwaltet (indirekt) die Komponenten eines
Projektes, also Speicher, Register, Architekturdefinition und Parser, und
ermöglicht der GUI über Interface-Klassen, die in \autoref{fig:core-interface}
dargestellt sind, sicheren Zugriff auf die anderen Komponenten des Simulators.
Eine zentrale Anforderung an den Simulator war, dass die Ausführung der Befehle
nicht die Ansprechbarkeit der GUI beeinflusst. Daher ist es notwendig den
GUI-Thread, in dem ein Event-Loop der Qt-Engine läuft, von der Ausführung
unabhängig zu machen. Daher werden pro Core-Instanz weitere Threads erstellt.

\begin{figure}[H]
    \begin{center}
        \begin{tikzpicture}[node distance=1.0cm and 2.5cm]

            \node (projectModule) [class] {ProjectModule};

            \node (architectureAccess) [class, right = of projectModule] {ArchitectureAccess};
            \node (commandInterface) [class, below = of architectureAccess] {CommandInterface};
            \node (parserInterface) [class, below = of commandInterface] {ParserInterface};
            \node (memoryManager) [class, above = of architectureAccess] {MemoryManager};
            \node (memoryAccess) [class, above = of memoryManager] {MemoryAccess};
            \node (interface) [below = of parserInterface] {\textbf{Synchronisiertes Interface}};

            \draw[composition-arrow] (projectModule) -- (memoryAccess);
            \draw[composition-arrow] (projectModule) -- (memoryManager);
            \draw[composition-arrow] (projectModule) -- (architectureAccess);
            \draw[composition-arrow] (projectModule) -- (commandInterface);
            \draw[composition-arrow] (projectModule) -- (parserInterface);

            \begin{pgfonlayer}{background}
                \path (memoryAccess.west |- memoryAccess.north)+(-1.3,1.0) node (a1) {};
                \path (parserInterface.east |- parserInterface.south)+(1.3,-2.0) node (a2) {};
                \path[rounded corners, draw=black!50, dashed] (a1) rectangle (a2);
            \end{pgfonlayer}{background}

        \end{tikzpicture}
    \end{center}
    \caption{Core Modul von außen}
    \label{fig:core-interface}
\end{figure}

\subsubsection{Threading}

Bei der Verwendung mehrerer Threads ist erhöhter Programmieraufwand
notwendig, um Probleme wie Race-Conditions zu vermeiden. Daher verwenden wir das
\emph{Active Object Pattern}, da dadurch der Aufwand hauptsächlich auf die
Implementierung dieses Patterns im Core reduziert wird. Das \emph{Active
Object Pattern} sieht vor, dass für ein Objekt auf dem Synchronisierung
notwendig ist, auch \emph{Active Object} oder \emph{Servant} genannt, ein
eigener Thread erstellt wird. Dabei wird sichergestellt, dass die Methoden
dieses Objektes nur in diesem Thread laufen. Dazu wird pro \emph{Servant} ein
\emph{Scheduler} benötigt, der Befehle zur Ausführung einer Methode des
Objektes in einer Warteschlange verwaltet und nacheinander ausführt. Dieser
verwaltet auch den Thread des Servants. Der Zugriff von außen auf dieses
\emph{Active Object} darf nur über sogenannte \emph{Proxy}-Objekte erfolgen,
die Methodenaufrufe in die Warteschlange des Schedulers einreihen. Dieses
Konzept wird in \autoref{fig:active-object-pattern} dargestellt. Die Rückgabe
von Werten erfolgt dabei über \texttt{futures} oder Callbacks. Es wurden dabei
auch sichere Callbacks implementiert, die jeweils im aufrufenden Thread
ausgeführt werden. Diese können jedoch nur verwendet werden, wenn der Aufrufer
auch ein Active Object ist. Diese Art der Callbacks fand daher bis jetzt keine
Verwendung. Bei normalen Callbacks muss darauf geachtet werden, dass diese immer
noch im jeweiligen Active-Object-Thread ausgeführt werden.

Wenn in einem \emph{Active Object} gerade kein Befehl ausgeführt wird, schläft
der Scheduler (also der Active-Object-Thread) so lange, bis er durch das
Einfügen eines Befehls in seine Warteschlange aufgeweckt wird. Dadurch wird
durch die zusätzlichen Threads im Allgemeinen keine zusätzliche Performance
beansprucht. Trotzdem ist es natürlich nicht sinnvoll, sehr viele \emph{Active
Objects} zu erstellen. Im \texttt{ProjectModule} werden daher nur zwei
\emph{Active Objects} verwaltet: Das \texttt{Project} Objekt verwaltet den
Speicher, die Register und das Architekturobjekt. Dabei wird ein eigenes Active
Object mit einem Thread benötigt, da es beispielsweise möglich sein muss, über
die Benutzeroberfläche während der Ausführung auf den Speicher zuzugreifen. Das
zweite Active Object wird für die Ausführung und Übersetzung der Programme
benötigt. Dies ist in der \texttt{ParsingAndExecutionUnit} umgesetzt. Dadurch
bleibt die GUI auch während der Ausführung ansprechbar. Zusammenfassend gibt es
also pro Projekt (Tab in der GUI) zwei Threads im Core.

\begin{figure}
    \begin{center}
        \begin{tikzpicture}[node distance=1.0cm and 2.5cm]
        \node (servant) [class] {Servant};
        \node (scheduler) [class, left = of servant] {Scheduler};
        \node (proxy) [class, left = of scheduler] {Proxy};

        \node (activeObject) [below = of {$(servant)!0.5!(scheduler)$}] {\textbf{Active Object Thread}};

        \draw[dashed-arrow] (proxy) to node[yshift=0.3cm]{\small{push(task)}} (scheduler);
        \draw[dashed-arrow] (scheduler) to node [yshift=0.3cm]{\small{execute(task)}} (servant);

        \begin{pgfonlayer}{background}
            \path (scheduler.west |- scheduler.north)+(-0.5, 0.5) node (a1) {};
            \path (servant.east |- servant.south)+(0.5, -1.5) node (a2) {};
            \path[rounded corners, draw=black!50, dashed] (a1) rectangle (a2);
        \end{pgfonlayer}{background}
    \end{tikzpicture}
    \end{center}
    \caption{Aufbau eines Active Objects}
    \label{fig:active-object-pattern}
\end{figure}

\begin{figure}[H]
    \begin{center}
        \begin{tikzpicture}
            \node (memoryAccess) [class] {MemoryAccess};
            \node (memoryManager) [class, below = of memoryAccess] {MemoryManager};
            \node (architectureAccess) [class, below = of memoryManager] {ArchitectureAccess};
            \node (projectLabel) [below = 0.5cm of architectureAccess] {\textbf{Project Proxy}};

            \node (commandInterface) [class, below = 2cm of projectLabel] {CommandInterface};
            \node (parserInterface) [class, below = of commandInterface] {ParserInterface};
            \node (parsingLabel) [below = 0.5cm of parserInterface] {\textbf{ParsingAndExecutionUnit Proxy}};

            % Interfaces
            \begin{pgfonlayer}{background}
                \path (memoryAccess.west |- memoryAccess.north)+(-1.0, 0.5) node (a1) {};
                \path (architectureAccess.east |- architectureAccess.south)+(1.0, -1.5) node (a2) {};
                \path[rounded corners, draw=black!50, dashed] (a1) rectangle (a2);
            \end{pgfonlayer}{background}

            \begin{pgfonlayer}{background}
                \path (commandInterface.west |- commandInterface.north)+(-2.0, 1.0) node (a1) {};
                \path (parserInterface.east |- parserInterface.south)+(2.0, -1.5) node (a2) {};
                \path[rounded corners, draw=black!50, dashed] (a1) rectangle (a2);
            \end{pgfonlayer}{background}

            % Project
            \node (project) [right = 6.5cm of $(memoryManager)!0.5!(memoryAccess)$] {\textbf{Project}};
            \node (memory) [class, below = 0.2 of project] {Memory};
            \node (register) [class, below = 0.2 of memory] {RegisterSet};
            \node (arch) [class, below = 0.2 of register] {Architecture};

            \begin{pgfonlayer}{background}
                \path (project.west |- project.north)+(-1.5, 0.5) node (a1) {};
                \path (project.east |- project.south)+(1.5, -3.0) node (a2) {};
                \path[rounded corners, draw=black, fill=white, drop shadow] (a1) rectangle (a2);
            \end{pgfonlayer}{background}

            % ParsingAndExecutionUnit
            \node (parsingUnit) [right = 4cm of commandInterface] {\textbf{ParsingAndExecutionUnit}};
            \node (Parser) [class, below = 0.5 of parsingUnit] {Parser};

            \begin{pgfonlayer}{background}
                \path (parsingUnit.west |- parsingUnit.north)+(-0.5, 0.5) node (a1) {};
                \path (parsingUnit.east |- parsingUnit.south)+(0.5, -1.7) node (a2) {};
                \path[rounded corners, draw=black, fill=white, drop shadow] (a1) rectangle (a2);
            \end{pgfonlayer}{background}

            % arrows
            \draw[dashed-arrow] (memoryAccess) -- ($(project.west)+(-1.5, -0.4)$);
            \draw[dashed-arrow] (memoryManager) -- ($(project.west)+(-1.5, -1.2)$);
            \draw[dashed-arrow] (architectureAccess) -- ($(project.west)+(-1.5, -2.0)$);
            \draw[dashed-arrow] (commandInterface) -- ($(parsingUnit.west)+(-0.5, -0.3)$);
            \draw[dashed-arrow] (parserInterface) -- ($(parsingUnit.west)+(-0.5, -1.0)$);

        \end{tikzpicture}

    \end{center}
    \caption{Aufteilung des Interfaces auf verschiedene Core-Komponenten}
    \label{fig:core-proxies}
\end{figure}

\subsubsection{Interface}

Die Kommunikation des Cores mit der GUI teilt sich auf zwei Richtungen auf:

Die GUI muss auf Methoden des Cores zugreifen können, um an den Zustand von
Speicher und Register zu gelangen und diesen zu ändern, oder auch um Befehle
auszuführen. Dabei werden der GUI über das \texttt{ProjectModule} Interface
Klassen zur Verfügung gestellt, die \texttt{Proxy} Klassen der Active Objects
sind. Diese Interface Klassen werden teilweise auch an die anderen Komponenten
übergeben, um dem Parser und den Befehlen beispielsweise Zugriff auf Speicher
und Register zu ermöglichen. Im einzelnen sind die Interface Klassen des
\texttt{Project}s der \texttt{MemoryAccess}, der Zugriff auf Speicher und
Register ermöglicht, der \texttt{MemoryManager}, über den weitere Funktionen von
Speichern und Register verwaltet werden können, und der
\texttt{ArchitectureAccess}, über den auf das Architekturobjekt des Projekts
zugegriffen werden kann. Die Interfaces der \texttt{ParsingAndExecutionUnit}
sind das \texttt{CommandInterface}, das beispielsweise die Übermittlung von
Ausführungsbefehlen erlaubt, und das \texttt{ParserInterface}, über das auf
Informationen des Parsers zugegriffen werden kann. In \autoref{fig:core-proxies}
wird die Aufteilung der Interface Klassen auf die verschiedenen Core Module
dargestellt.

Des Weiteren muss der GUI auch signalisiert werden, dass sich der Zustand des
Cores geändert hat und die Ansichten aktualisiert werden sollen. Dafür werden
Callbacks verwendet, damit der Core keine Referenz auf die GUI benötigt und so
unabhängig von deren Aufbau ist. Diese Callbacks werden über die Interface
Klassen gesetzt und bei entsprechenden Änderungen durch die Core-Komponenten
aufgerufen. Dabei ist zu beachten, dass die Callbacks im jeweiligen Active
Object ausgeführt werden, also auch in dessen Thread. Daher ist es nötig, in den
Callback Funktionen, die in der GUI Implementiert werden, Qt-Signals mit
\texttt{QueuedConnections} zu verwenden, um die entsprechenden die
erforderlichen Aktionen in der GUI im richtigen Thread auszuführen. Da jeweils
nur ein Callback im Core gesetzt werden kann, ist es über diesen Mechanismus
auch möglich, mehrere Komponenten in der GUI zu benachrichtigen.

\subsubsection{Ausführung}

Der Core ist auch die Schnittstelle der GUI zum Parser, daher übernimmt er den
Aufruf des Parsers und das Einfügen der assemblierten Befehlsdarstellung in den
Speicher. Dabei wird dieser Speicherbereich vor der normalen Bearbeitung durch
den Nutzer oder Assemblerbefehle geschützt.

Nach der Übersetzung liefert der Parser dem Core ein Liste von Befehlen, die
ausgeführt werden können, oder eine List an Fehlermeldungen. Die Ausführung
läuft dabei auch in einem Active Object, wodurch die GUI nicht beeinträchtigt
wird. Da deshalb während der laufenden Ausführung alle anderen Befehle für
dieses Active Object warten müssen, kann ein Stop-Signal nicht wie ein normaler
Aufruf realisiert werden. Deshalb wird dazu ein mit dem \texttt{ProjectModule}
geteiltes Flag verwendet. Dieses wird immer geprüft, wenn ein neuer Befehl
ausgeführt wird.

\pagebreak
Vor der Ausführung wird jeder Befehl mit seiner \texttt{validateRuntime()}
Methode validiert, um Fehler, die nur während der Laufzeit überprüfbar sind,
abzufangen. Bei einem Fehler wird dabei die Ausführung abgebrochen und die GUI
über einen Callback benachrichtigt. Falls kein Fehler aufgetreten ist, wird der
Befehl über seine \texttt{getValue()} Methode ausgeführt, dabei wird als
Rückgabewert der neue Befehlszählerstand geliefert. Dieser Wert wird vom Core in
das Befehlszählerregister geschrieben.

Falls nicht nur einzelne Befehle ausgeführt werden, ist auch zu beachten, dass
durch die Ausführung jedes Befehls einige Callbacks an die GUI geschickt werden,
wodurch in bestimmten Situationen das Message-System der Qt-Engine überlastet
werden kann. Um dies zu verhindern, wird nach jedem Befehl ein
Synchronisations-Callback aufgerufen und mit der Ausführung gewartet, bis dieser
in der GUI ausgeführt wurde. Der Synchronisations-Callback muss dazu die
\texttt{ProjectModule::guiReady()} Methode aufrufen. Dadurch wird
sichergestellt, dass die Message-Queue abgearbeitet wurde und eine Überlastung
verhindert.

\subsubsection{Memory und Register}

Ein zentraler Aspekt des Cores ist die Haltung des simulierten Speichers. Dabei ist es, um Generalität zu wahren, vonnöten binäre Daten beliebiger Länge speichern und austauschen zu können. Des weiteren muss dieser Datentyp, um manche Algorithmen effizienter und einfacher zu implementieren, diverse Schreib- und Lese-Zugriffsweisen unterstützen. Die zuvorderst betrachtete und wohl zentralste Fähigkeit die ein solcher Datentyp, um nur im entferntesten die Aufgabe, die dieser zu erfüllen hat, befriedigend nachzukommen, ist jene einzelne Bits an beliebiger Position sowohl zu lesen als auch zu schreiben. Von den zunächst in Betracht gezogenen Datentypen, die zum Zwecke Bitstrings beliebiger Länge zu verwalten verwendet werden können, unterstützt nur \texttt{std::vector<uint8\_t>} keinen solchen Zugriff fakultativ(\autoref{fig:memory-value}). Die nächste unerlässliche Zugriffsweise ist die Möglichkeit Blöcke fester Länge, in den meisten Fällen byteweise, also zu je 8 Bit, effizient wieder lesen und schreiben zu können. In dieser Hinsicht fallen beinahe alle betrachteten Datentypen flach. Nicht nur sind \texttt{std::bitset}, \texttt{boost::dynamic\_bitset} und \texttt{std::vector<bool>} nicht ohne Modifizierung in der Lage auf irgendwelche Datenteile außer einzelne Bits zuzugreifen, sondern eine solche wäre auch nur mit unverhältnismäßigem Aufwand implementationsabhängig oder in mindestens linearer Komplexität in der Länge der Blöcke machbar(\autoref{fig:memory-value}). Überdies ist es von einem Bitstring beliebiger Länge wünschenswert nicht nur Blöcke fester Länge sondern auch Teil-Bitstrings kleinergleicher Länge seinerselbst sowohl auslesen als auch hineinschreiben zu können. Dies wird unter den betrachteten Datentypen nur von den Standarddatentypen \texttt{std::vector<uint8\_t>} und \texttt{std::vector<bool>} unterstützt. \texttt{std::vector<uint8\_t>} kann dieses jedoch nur blockweise, das heißt \texttt{uint8\_t } weise(\autoref{fig:memory-value}). Letztens, neben allgemeinen Gesichtspunkten wie geringem Speicherverbrauch, schnellen Zugriffsmethoden und unkomplizierter Handlung, ist es Teil der Voraussetzung beliebiger Länge, dass diese nicht zu Compilezeit festgelegt zu werden hat. Die Länge von \texttt{std::bitset} wird ist bereits im Gegensatz zu anderen betrachteten Datentypen zu Compilezeit darzulegen(sh. Abb. XYZ).

\begin{figure}[H]
  \begin{center}
    \begin{tabular}{>{\small}l|cccc}
      & \begin{tabular}{>{\small}c}Zugriff auf \\ einzelne Bits\end{tabular}
      & \begin{tabular}{>{\small}c}Zugriff auf \\ Blocks\end{tabular}
      & \begin{tabular}{>{\small}c}Zugriff auf \\beliebige \\ Teilbitstrings\end{tabular}
      & \begin{tabular}{>{\small}c}Festlegung der \\Länge zu \\Erzeugungszeit\end{tabular} \\\\
      \texttt{std::bitset} & $\checkmark$ & $\times$ & $\times$ & $\times$ \\\\
      \texttt{boost::dynamic\_bitset} & $\checkmark$ & $\times$ & $\times$ & $\checkmark$ \\\\
      \texttt{std::vector\textless uint8\textgreater} & $\times$ & $\checkmark$ & $\checkmark$ & $\checkmark$ \\\\
      \texttt{std::vector}\textless bool\textgreater & $\checkmark$ & $\times$ & $\checkmark$ & $\checkmark$ \\\\
      \texttt{MemoryValue} & $\checkmark$ & $\checkmark$ & $\checkmark$ & $\checkmark$ \\\\
    \end{tabular}
  \end{center}
\caption{Vergleich der Typen die zum halten und austauschen binärer Daten in Frage kommen mit den Vorraussetzungen, die diese erfüllen müssen um dies bestmöglich zu tun}
\label{fig:memory-value}
\end{figure}

Insgesamt ergibt sich kein Datentyp der alle Voraussetzungen erfüllt. Sowohl
\texttt{std::vector<bool>} als auch \texttt{std::vector<uint8\_t>} sind jedoch
mit jeweils einem unerfülltem Punkt am besten geeignet als Basis eines
selbsterstellten Datentypen zu dienen. Im direkten Vergleich ist es mit wenig
Aufwand zu erkennen, dass die Implementation eines Einzelbit-Zugriffs für
\texttt{std::vector<uint8\_t>} in konstanter Komplexität einfach zu realisieren,
während der Zugriff auf irgendwelche Blöcke in einem \texttt{std::vector<bool>}
äußerst umständlich oder ineffizient wäre. Der Datentyp, welcher im folgenden
\texttt{MemoryValue}, da dieser Werte(engl. Value) aus dem Speicher enthält,
genannt wird, ist daher über einem \texttt{std::vector<std::uint8\_t>}
aufgebaut.

Ein allgemeiner Datentyp zum Halten und Austauschen binärer Daten ist an sich
gesehen zwar bereits nützlich, jedoch repräsentieren diese in vielerlei Fällen
überdies Zahlwerte. Um diese handhaben zu können muss entweder der
\texttt{MemoryValue} arithmetisch-logische Funktionen unterstützen oder es eine
konvertierung zwischen \texttt{MemoryValue} und Datentypen die zu
arithmetisch-logischen Funktionen fähig sind. Das Implementieren
arithmetisch-logischer Funktionen im \texttt{MemoryValue} wäre jedoch aufwändig,
fehleranfällig und insbesondere inflexibel, da unterschiedliche Systeme
unterschiedliche Arithmetiken, insbesondere die Darstellung vorzeichenbehafteter
Zahlen betreffend, verwenden. Daher ist es sinnvoller diese in
arithmetisch-logische Datentypen, wie den eingebauten integralen
Standard-Zahltypen, umzuwandeln. Durch die Verwendung von C++-Templates ist es
möglich mit nur einer Implementation beliebige Datentypen, gegeben diese
unterstützen gängige arithmetische und logische Operationen, als Ausgang und
Ziel anzugeben. Desweiteren kann man die Konvertierung selbst in verschiedene
Teile aufteilen. Zum einen muss der \texttt{MemoryValue} von einer beliebigen
Byte-Reihenfolge in eine bestimmte umgewandelt werden, bzw. zurück. Hierbei
wurde sich für Little Endian entschieden. Als nächstes muss der
\texttt{MemoryValue}, welcher sich in Little Endian befindet, zu einem Zahltypen
umgewandelt werden, bzw. umgekehrt. All diese Konvertierungen sind als
Funktionsobjekte gestaltet, so dass ohne viel Aufwand eine neue Umwandlung
implementiert werden könnte.

Zu der Möglichkeit binäre Daten beliebiger Länge in Form des
\texttt{MemoryValues} halten, übertragen und in einen zu arithmetisch-logischen
Operationen fähigen Datentypen umzuwandeln wird überdies noch die assoziierte
Institution zum Verwalten dieser benötigt, namentlich der Hauptspeicher und der
Registersatz.

Der Hauptspeicher ist hierbei spektakulär einfach implementiert, bestehend aus
einem großen \texttt{MemoryValue}, d.h. ein Hauptspeicher mit n Zellen á m Bit
besteht zum größten Teil aus einem \texttt{MemoryValue} der Länge n*m. Um nun
auf diese zugreifen zu können ist nur noch die Startadresse zu berechnen und im
Falle einer Schreiboperation die Gui über die Änderungen zu informieren.

Die Register sind etwas komplizierter implementiert da diese erstens nicht über
eine Hauptspeicheradresse sondern einem Registerbezeichner, welcher hier als
\texttt{std::string} implementiert ist, adressiert werden und zweitens
hierarchische Beziehungen zwischen Registern bestehen
können(\autoref{fig:register-set}). Dies ist implementiert durch eine
\texttt{std::unordered\_map} die von  Registerbezeichnern auf
Registeridentifikatoren abbildet. Diese Registeridentifikatoren bestehen zum
ersten aus einem Verweis auf dieses Registers
Elterregister(\autoref{fig:register-set}) und zum zweiten Beginn- und und
End-Adresse des Teilbitstrings den dieses Register im Elterregister ausmacht.
Des weiteren hat jedes Elterregister eine Menge an Kindregistern in welchem die
Registerbezeichner aller Register enthalten sind über deren Änderung die GUI zu
benachrichtigen ist falls sich in ebendiesem Elterregister oder dieses
Kindregister eine Schreiboperation ausgeführt wurde.

\begin{figure}[H]
  \centering
  \scalebox{1.3}{
  \begin{tikzpicture}[thick]
    % pseudo element for centering
    \node (center) {};

    \node (eax)
      [font=\scriptsize, draw=black, rounded corners, align=center, minimum height=0.68cm, minimum width=3cm, above left=0.5cm and 3cm of center]
      {\texttt{EAX}};
    \node (ebx)
      [font=\scriptsize, draw=black, rounded corners, align=center, minimum height=0.68cm, minimum width=3cm, above left=-1.5cm and 3cm of center]
      {\texttt{EBX}};
    \node (mapEax)
      [font=\scriptsize, draw=black,  align=center, minimum height=0.5cm, minimum width=2cm, above left=1.35cm and 7cm of center]
      {\texttt{eax} $\rightarrow$ \texttt{[0,31]}}
        edge [->, out=0, in=180] (eax);
    \node (mapAx)
      [font=\scriptsize, draw=black,  align=center, minimum height=0.5cm, minimum width=2cm, above left=0.85cm and 7cm of center]
      {\texttt{ax} $\rightarrow$ \texttt{[0,15]}}
        edge [->, out=0, in=180] (eax);
    \node (mapAl)
      [font=\scriptsize, draw=black,  align=center, minimum height=0.5cm, minimum width=2cm, above left=0.35cm and 7cm of center]
      {\texttt{al} $\rightarrow$ \texttt{[0,7]}}
        edge [->, out=0, in=180] (eax);
    \node (mapAh)
      [font=\scriptsize, draw=black,  align=center, minimum height=0.5cm, minimum width=2cm, above left=-0.15cm and 7cm of center]
      {\texttt{ah} $\rightarrow$ \texttt{[8,15]}}
        edge [->, out=0, in=180] (eax);
    \node (mapEbx)
      [font=\scriptsize, draw=black,  align=center, minimum height=0.5cm, minimum width=2cm, above left=-0.65cm and 7cm of center]
      {\texttt{ebx} $\rightarrow$ \texttt{[0,31]}}
        edge [->, out=0, in=180] (ebx);
    \node (mapBx)
      [font=\scriptsize, draw=black,  align=center, minimum height=0.5cm, minimum width=2cm, above left=-1.15cm and 7cm of center]
      {\texttt{bx} $\rightarrow$ \texttt{[0,15]}}
        edge [->, out=0, in=180] (ebx);
    \node (mapBl)
      [font=\scriptsize, draw=black,  align=center, minimum height=0.5cm, minimum width=2cm, above left=-1.65cm and 7cm of center]
      {\texttt{bl} $\rightarrow$ \texttt{[0,7]}}
        edge [->, out=0, in=180] (ebx);
    \node (mapBh)
      [font=\scriptsize, draw=black,  align=center, minimum height=0.5cm, minimum width=2cm, above left=-2.15cm and 7cm of center]
      {\texttt{bh} $\rightarrow$ \texttt{[8,15]}}
        edge [->, out=0, in=180] (ebx);

    \node (eaxUpdate)
      [font=\scriptsize, draw=black, align=center, minimum height=1.70cm, minimum width=0.6cm, above left=0cm and -1cm of center, rounded corners]
      {\begin{tabular}{c}
        \texttt{eax} \\
        \texttt{ax} \\
        \texttt{al} \\
        \texttt{ah} \\
      \end{tabular}};
    \node (ebxUpdate)
      [font=\scriptsize, draw=black, align=center, minimum height=1.70cm, minimum width=0.6cm, above left=-2cm and -1cm of center, rounded corners]
      {\begin{tabular}{c}
        \texttt{ebx} \\
        \texttt{bx} \\
        \texttt{bl} \\
        \texttt{bh} \\
      \end{tabular}};

    \draw [->] (eax) -- (eaxUpdate);
    \draw [->] (ebx) -- (ebxUpdate);

  \end{tikzpicture}
  }
  \vspace{0.2cm}

  \caption{Darstellung einer beispielhaften Implementierung eines Registersatzes. Links wird mit einer Map gearbeitet, die Registernamen auf Registeridentifikatoren abbildet. In der Mitte stehen die Wurzelregister, rechts die Namen jener Register, die bei einer Veränderung benachrichtigt werden müssen.}
  \label{fig:register-set}
\end{figure}
