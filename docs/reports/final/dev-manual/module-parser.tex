\subsection{Parser}

Das \textit{Parser}-Modul übernimmt die Übersetzung des eingegebenen Textes in die für das \textit{Architectur}-Modul lesbaren Syntaxbäume. Damit entspricht dieses Modul größtenteils dem eigentlichen Assemblierer.

\subsubsection{Submodule}

Bei Erscheinen der Version 1.0 besteht das Parser-Modul aus vier verschiedenen Untermodulen:

\begin{itemize}

\item Das \emph{Common-Submodul} stellt Klassen bereit, die zur öffentlichen
Schnittstelle des Parsers zu anderen Modulen dienen. Dieses Submodul ist frei
von Abhängigkeiten zu jeglichen konkreten Parser-Implementierungen.

\item Mit dem \emph{Factory-Submodul} können neue, spezifische Parser erzeugt
werden.

\item Das \emph{RISC-V-Submodul} stellt eine konkrete Implementierung eines
Assemblierers für die RISC-V-Architektur zur Verfügung.

\item Im \emph{Independent-Submodul} sind viele Hilfsklassen (zum Beispiel
Symboltabellen, Compiler für arithmetische Ausdrücke) bereitgestellt, welche von
dem RISC-V-Parser verwendet, genauso gut aber auch gerne von zukünftigen Parser
eingebunden werden können.

\end{itemize}

Im Folgenden sind diese Submodule nochmal genauer ausgeführt:

\paragraph{Common-Submodul}

Das Kernstück des gesamten Moduls bildet die (abstrakte) Klasse \texttt{Parser}.
Diese bietet hauptsächlich zwei Funktionen: Anbieten von Syntax-Informationen
mittels der Method \texttt{getSyntaxInformation} (für das Syntax-Highlighting in
der Benutzeroberfläche), sowie dem Assemblieren eines gegebenen
Assemblerprogrammes mithilfe der Methode \texttt{parse}, die als
Eingabeparameter einen C++-Standard-String erwartet.

\subparagraph{Syntax-Highlighting}
\todo[inline]{TODO}
\subparagraph{Das Assemblieren}
Beim Aufruf der Methode \texttt{parse} soll das Assembler-Programm dabei
kompiliert und in einer \texttt{FinalRepresentation}-Datenstruktur zurückgegeben
werden. Diese enthält notwendige Informationen für die Ausführung des
Assemblerprogrammes sowie dessen Darstellung in der Benutzeroberfläche. Die
\texttt{FinalRepresentation} besteht dabei aus folgenden Einzelheiten:
\begin{itemize}
\item \texttt{CommandList}: Die fertig assemblierten Assemblerbefehle,
aneinandergereiht. Jeder der Befehle (des Typs \texttt{FinalCommand}) enthält
einen fertig assemblierten \texttt{InstructionNode}, das
\texttt{CodePositionInterval}, an welchem der Befehl im Text auftritt, sowie der
Speicheradresse, an der der Befehl assembliert werden soll.
\item \texttt{MacroInformationList}: Beinhaltet alle Makros, welche im Code
vorkommen, mit eingesetzten Parametern mit Position des Auftretens.
\item \texttt{CompileErrorList}: Eine Liste von allen Fehler, Warnungen und
Hinweisen, die während des Assembliervorgangs aufgetreten sind. Wenn diese keine
Fehler enthält (sehrwohl aber eventuell Warnungen oder Hinweise), so ist das
Assemblieren erfolgreich gewesen und das Assemblerprogramm kann ausgeführt
werden.
\end{itemize}

Gehen wir noch auf ein paar Feinheiten ein:

Koordinaten im Assemblertext werden in der Datenstruktur \texttt{CodePosition}
als zweidimensionaler Punkt gespeichert, ein Intervall davon entsprechend in der
\texttt{CodePositionInterval}-Klasse. Ein Intervall ist genau dann leer, wenn
sein Startpunkt vor seinem Endpunkt liegt. In diesem Fall also, wenn die
Y-Koordinate des Endpunktes strikt kleiner als die des Startpunktes oder die
Y-Koordinaten identisch aber die X-Koordinate des Endpunktes strikt kleiner ist.
Ein CodePositionInterval wird beidseitig inklusiv gesehen (d.h. beide Randpunkte
liegen noch im Intervall).

Ein \texttt{CompileError} kapselt eine Fehlermeldung (bzw. eine Warnung oder
einen Hinweis). Dabei wird die Position und die Schwere der Meldung (Fehler,
Warnung, Hinweis, vgl. \texttt{CompileErrorSeverity}) festgehalten. Der Begriff
„Error“ ist deswegen etwas überladen. Die Meldung selber wird als
\texttt{Translateable} gespeichert, sodass diese später in verschiedene Sprachen
übersetzt werden können soll. Dabei werden Argumente separat vom eigentlichen
Text kodiert. Eine \texttt{CompileErrorList} kapselt dabei die
\texttt{CompileError}s und stellt Möglichkeiten zur Erweiterung der Liste
bereit. Dies erfolgt über den Aufruf von Makros. Der Grund hierfür ist, dass so
die Fehlermeldungen automatisch von einem Qt-Programm gefunden und für das
Übersetzen markiert werden können. Die Meldungen müssen dabei ein C-String sein
(\texttt{const char*}). Ebenso lassen sich über die \texttt{CompileErrorList}
einfache Abfragen stellen, ob jeweils Fehler, Warnungen oder Hinweise vorhanden
sind und wenn ja, wie viele.

\paragraph{Factory-Submodul}

Kommen wir zum Factory-Submodul: Dieses besteht lediglich aus einer einzigen
Klasse, der \texttt{ParserFactory}. Bei jener werden alle
Parserimplementierungen unter einem Namen zur Auswahl gestellt, sodass sie mit
Architektur und Speicherzugriff kombiniert einen Parser erzeugen können. Diese
Abhängigkeit zu den einzelnen Implementierungsmodulen ist auch der Grund, wieso
das Factory-Submodul aus dem Common-Submodul herausgenommen wurde.

Mit der Methode \texttt{ParserFactory::createParser} kann dabei ein Parser mit
den angegebenen Voraussetzungen generiert werden. Die Map
\texttt{ParserFactory::mapping} enthält alle registrierten Parser.

\paragraph{Independent-Submodul}

Kommen wir nun zum wohl größten Submodul des Parsers, dem
\emph{Independent}-Modul, welches eine Sammlung von Hilfsklassen darstellt,
welche von verschiedenen Assemblern verwendet werden können sollen. Dieses Modul
kann und soll gerne erweitert, dabei aber unabhängig von jeglichem spezifischen
Assemblierer gehalten werden.

\subparagraph{Intermediate-Darstellung für Befehle}
\todo[inline]{TODO}

\subparagraph{Assemblierung der Intermediate-Darstellung}
\todo[inline]{TODO}

\subparagraph{Makros}
\todo[inline]{TODO}

\subparagraph{Symbolgraph und Symbol-Replacer}
\todo[inline]{TODO}

\subparagraph{Compiler für arithmetische Ausdrücke}
\todo[inline]{TODO}

\paragraph{RISC-V-Submodul}

Für RISC-V existiert ein korrespondierendes Parser-Submodul, welches in der
\texttt{ParserFactory} über den Namen \texttt{riscv} aufgerufen werden kann. Es
verwendet dabei größtenteils die Klassen, die im Independent-Submodul definiert
sind.

\todo[inline]{TODO: Zum Text-Parsen hier was sagen, RISC-V-Regex erklären}

\subsubsection{Verwenden eines Parsers}

Von außen kann das Parser-Modul über die \texttt{ParserFactory} angesprochen
werden. Hier erzeugt man mit Name, Speicherzugriff und gegebener Architektur
einen gewünschten Parser und erhält einen Unique-Pointer darauf. Anschließend
kann man bereits die \texttt{parse}- und \texttt{getSyntaxHighliting}-Methoden
des Parsers selber aufrufen und Text assemblieren lassen. Das Deinitialisieren
erfolgt ebenfalls über den eingebauten Destruktor automatisch.

\subsubsection{Einschränkungen}

Trotz der aktuellen Fähigkeiten des Parser-Moduls existieren noch folgende
Einschränkungen:

\todo[inline]{TODO}
\begin{itemize}
\item \emph{Segment-Darstellung}: \todo[inline]{TODO}
\item \emph{Compiler für arithmetische Ausdrücke}: \todo[inline]{TODO}
\item \emph{Symbol-Tabelle}: \todo[inline]{TODO}
\item \emph{Gleitkommazahl-Unterstützung}: \todo[inline]{TODO}
\end{itemize}
