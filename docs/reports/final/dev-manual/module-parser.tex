\subsection{Parser}

Das \textit{Parser}-Modul übernimmt die Übersetzung des eingegebenen Textes in die für das \textit{Architectur}-Modul lesbaren Syntaxbäume. Damit entspricht dieses Modul größtenteils dem eigentlichen Assemblierer.

\subsubsection{Submodule}

Bei Erscheinen der Version 1.0 besteht das Parser-Modul aus vier verschiedenen Untermodulen:
\begin{itemize}
\item Das \emph{Common-Submodul} stellt Klassen bereit, die zur öffentlichen Schnittstelle des Parsers zu anderen Modulen dienen. Dieses Submodul ist frei von Abhängigkeiten zu jeglichen konkreten Parser-Implementierungen.
\item Mit dem \emph{Factory-Submodul} können neue, spezifische Parser erzeugt werden.
\item Das \emph{RISC-V-Submodul} stellt eine konkrete Implementierung eines Assemblierers für die RISC-V-Architektur zur Verfügung.
\item Im \emph{Independent-Submodul} sind viele Hilfsklassen (zum Beispiel Symboltabellen, Compiler für arithmetische Ausdrücke) bereitgestellt, welche von dem RISC-V-Parser verwendet, genauso gut aber auch gerne von zukünftigen Parser eingebunden werden können.
\end{itemize}
Im Folgenden sind diese Submodule nochmal genauer ausgeführt:
\todo[inline]{...was noch zu erledigen ist.}

\paragraph{Common-Submodul}
\todo[inline]{TODO}

\paragraph{Factory-Submodul}
\todo[inline]{TODO}

\paragraph{Independent-Submodul}
\todo[inline]{TODO}

\paragraph{RISC-V-Submodul}
\todo[inline]{TODO}

\subsubsection{Verwendung eines Parsers}
\todo[inline]{TODO}
