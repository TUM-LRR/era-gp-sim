% !TEX root = dev-manual.tex
% ERA-Großpraktikum: Entwickleranleitung -- Einleitung

\section{Einleitung}

Diese Entwickleranleitung beschreibt das Entwicklungs-Ökosystem von \erasim{},
um zukünftigen Entwicklern bei Erweiterungen oder notwendigen Änderungen des
Simulators einen umfassenden Anhaltspunkt zu bieten. Der Bericht ist in drei
Teile gegliedert, die schrittweise in die Thematik einführen.
\autoref{dev:general} gibt einen Überblick über die verwendeten
Entwicklungswerkzeuge. Anschließend wird in \autoref{dev:modules} das
Zusammenspiel der Module beschrieben, um ein umfangreiches Verständnis über die
Funktionsweise des Simulators zu vermitteln. \autoref{dev:extension} beschreibt
letztlich, wie der Simulator dank seines modularen Aufbaus erweitert werden
kann.

Dieses Dokument dient auch dazu, die getroffenen Design-Entscheidungen zu
begründen. Es ist aber keineswegs in der Lage, den gesamten Quelltext von
\erasim{} abzudecken. Weiterführende, für Entwickler relevante Dokumentation
findet sich in Form von umfangreichen Kommentaren innerhalb des Quelltextes, auf
die entsprechend verwiesen wird.
