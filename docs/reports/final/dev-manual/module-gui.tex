\subsection{GUI}

\subsubsection{Aufbau}

Die grafische Benutzeroberfläche des Simulators besteht aus einer Reihe von
Komponenten, die sich innerhalb des gegebenen Rasters nahezu beliebig anordnen
lassen. Diese Aufteilung in Komponenten spiegelt sich auch in der
Implementierung der GUI wieder, da jede der wählbaren Komponente einer
QML-Komponente entspricht. Dazu gehören Snapshots, Hilfe, Register, Speicher,
Input und Output (siehe auch Abbildung \ref{fig:gui-composition}).

Ihnen übergeordnet ist eine Organisationsstruktur mit dem eigentlichen Fenster
(\texttt{ApplicationWindow}) an der Spitze, welches unterteilt wird in die
Menubar mit Editor- und Projekt-Untermenü, die Toolbar mit den Ausführungsbutton
und schließlich den Projekt-Tabview, welcher die geöffneten Projekte enthält.
Bei letzterem gilt zu beachten, dass der Tabview sowohl die Tabbar als auch die
Tab-Inhalte, also die eigentlichen Projekte, repräsentiert.

Jedes Projekt weist, wie bereits erwähnt, eine gerasterte Struktur auf, die
durch die Splitview-Komponente mit vier nebeneinander angeordneten Teilbereichen
voller Höhe realisiert ist. Jeder dieser Teilbereiche ist wiederum in zwei
übereinander angeordnete Bereiche unterteilt, verwirklicht durch die
InnerSplitviews bzw. die InnerSplitviewsEditor. Letztere QML-Komponente ist eine
spezielle Ausprägung, die im oberen Teilbereich den Editor anzeigt, der,
anders als alle anderen Komponenten, nicht variabel positionierbar ist.

Die inneren Splitviews bestehen schließlich aus einem SplitViewItem, welches
neben der frei wählbaren Komponente (Snapshots, Hilfe etc.) auch den
Komponenten-Header enthält, über den zum Einen die angezeigte Komponente gewählt
wird und zum Anderen die Komponenteneinstellungen aufgerufen werden können.

\begin{figure}[H]
	\begin{center}
    \begin{tikzpicture}
      \tikzset{font=\small,
        edge from parent fork down,
        level distance=1.75cm,
        every node/.style=
          {rectangle,rounded corners,
          minimum height=8mm,
          fill=white,
          draw=black!50,
          drop shadow,
          align=center,
          text depth = 0pt
          },
        edge from parent/.style=
          {{Diamond}-,
          draw=black!50,
          thick
          }}
\Tree [.{ApplicationWindow\\(main.qml)}
        [.MenuBar ]
        [.Toolbar ]
        [.ProjectTabView
            [.Splitview
                 [.InnerSplitviews
                     [.SplitViewItem
                         [.Snapshots ]
                         [.Help ]
                         [.Registers ]
                         [.Memory ]
                         [.Inputs ]
                         [.Outputs ] ] ]
                 [.{InnerSplitviews-\\Editor}
                     [.SplitViewItem ]
                     [.Editor ] ] ] ] ]
\end{tikzpicture}
	\end{center}
	\caption{Grundlegende Zusammensetzung der GUI-Komponenten. Die Knotentitel im
	Diagramm entsprechen nicht notwendigerweise den Namen der assoziierten
	Komponente.}
	\label{fig:gui-composition}
\end{figure}

\subsubsection{C++-Komponenten}

QML-Komponenten, die Zugriff auf das vom Core zur Verfügung gestellte Modell
benötigen, werden zusätzlich mit einer C++-Klasse assoziiert. Diese Klassen
halten Instanzen der Interface-Klassen \texttt{MemoryAccess},
\texttt{MemoryManager}, \texttt{ArchitectureAccess}, \texttt{ParserInterface}
oder \texttt{CommandInterface}, über die der Zugriff auf den Core unter
Verwendung des Schedulers erfolgt.

Da das Qt-Framework einen in sich stark abgeschlossenen Aufbau besitzt, werden
für QML-Komponenten mit komplexen Modellen, wie etwa den Speicher oder die
Register, von Qt-Klassen abgeleitete Modelle benötigt. Da keine Abhängigkeiten
vom Qt-Framework innerhalb der GUI-fernen Module Core, Parser und Architektur
entstehen sollen, kann diese Funktion nicht von Klassen des Cores übernommen
werden. Aus diesem Grund übernehmen einige der zu den QML-Komponenten gehörigen
C++-Klassen der GUI die Aufgabe des Modells und werden folglich von
Qt-Modell-Klassen wie etwa \texttt{QAbstractItemModel} abgeleitet. Mit Ausnahme
von Cache-Zwecken halten diese Modelle selbst keine veränderbaren Daten
(Registerwerte, Speicherwerte etc.), sondern holen diese über das zugehörige
Interface-Objekt, sobald Daten seitens der QML-Komponente angefordert werden.

\subsubsection{Kommunikation}
\label{gui-kommunikation}

Der Simulator unterstützt das gleichzeitige Laden mehrerer unabhängiger
Projekte. Diese werden jeweils durch eine \texttt{GUIProject}-Instanz
repräsentiert, welche Komponenten-über\-grei\-fen\-de Funktionalitäten für ein
Projekt zur Verfügung stellt.

Im \texttt{GUIProject} werden mitunter die C++-Klassen der QML-Komponente
gehalten, initialisiert und deren Kommunikation mit dem Core koordiniert.
Dieses übernimmt also die Rolle des Mittelsmann zwischen dem Core, der keine
Qt-Mechanismen verwendet, und der GUI.

Bei der Initialisierung der C++-Klassen übergibt das \texttt{GUIProject} diesen
die benötigten Instanzen der Interface-Klassen (\texttt{MemoryAccess} und
\texttt{ArchitectureAccess} etc.).

Datenaustausch ausgehend vom Core hin zu den QML-Komponenten in der GUI hat
nicht den Scheduler zwischengeschaltet, sondern verläuft über das
\texttt{GUIProject} in zwei Schritten (siehe auch Abbildung
\ref{fig:dev-manual-gui-communication}). Im ersten Schritt ruft der Core einen
Callback auf, der im \texttt{ProjectModule} gesetzt wird (näher Informationen im
Abschnitt \ref{Dev-Kapitel: Core}). Das \texttt{GUIProject} leitet im zweiten
Schritt den im \texttt{GUIProject} eingehenden Callback mit Hilfe des Qt-eigenen
Signal-Slot-Mechanismus an die Instanzen der C++-Klassen der QML-Komponenten
weiter. Diese Umleitung ist auch deshalb notwendig, um die eingehende Nachricht,
die im Thread des Cores gesendet wird, in den Main-Thread zu übertragen, der von
der GUI verwendet wird.

\begin{figure}
	\centering
	\begin{tikzpicture}[auto, node distance=2cm]
	% NODE DEFINITIONS
	\tikzstyle{module} = [draw, rectangle,
	text width=4cm]
	\tikzstyle{class} = [rectangle, rounded corners, draw=black, fill=white, text width=4cm, align=center]
	\tikzstyle{modulearrow} = [->, thick]
	\tikzstyle{extension} = [dashed]

	\node (guiproject) [class] {GUIProject};

	\node (callbackarrow) [draw=black, single arrow,minimum height=1.6cm, minimum width=2.0cm, single arrow, shape border rotate=180] at (3.15,0) {};
	\node[above = 0.2cm of callbackarrow, xshift=-0.1cm] {Callback};

	\node (signalarrow) [draw=black, single arrow,minimum height=1.6cm, minimum width=2.0cm, single arrow, shape border rotate=180] at (-2.85,0) {};
	\node[above = 0.2cm of signalarrow, xshift=0.1cm] {Signal};

	\node (interfacearrow1) [draw=black, single arrow,minimum height=1.6cm, minimum width=2.0cm, single arrow] at (2.85,-2.67) {};

	\node (interfacearrow2) [draw=black, single arrow,minimum height=1.6cm, minimum width=2.0cm, single arrow] at (-3.15,-2.67) {};

	\node (registermodel) [class, left = of guiproject] {RegisterModel};
	\node (memorycomponentpresenter) [class, above = of registermodel] {MemoryComponent-\\Presenter};
	\node (editorcomponent) [class, below = of registermodel] {EditorComponent};
	\node (guiecetera) [below = of editorcomponent, text width=4cm, align=center] {...};
	\node (guipackage) [above = 0.5cm of memorycomponentpresenter] {GUI};
	\node (guiext1) [above left = 0.2cm of memorycomponentpresenter] {};
	\node (guiext2) [above right = 0.2cm of memorycomponentpresenter] {};
	\node (guiext3) [below right = 0.2cm of guiecetera] {};
	\node (guiext4) [below left = 0.2cm of guiecetera] {};
	\draw [extension] (guiext1) -- (guiext2) -- (guiext3) -- (guiext4) -- (guiext1);

	\node (register) [class, right = of guiproject] {Register};
	\node (memory) [class, above = of register] {Memory};
	\node (parsingandexecutionunit) [class, below = of register] {ParsingAnd-\\ExecutionUnit};
	\node (coreecetera) [below = of parsingandexecutionunit, text width=4cm, align=center] {...};
	\node (corepackage) [above = 0.5cm of memory] {Core-Instanz};
	\node (coreext1) [above left = 0.2cm of memory] {};
	\node (coreext2) [above right = 0.2cm of memory] {};
	\node (coreext3) [below right = 0.2cm of coreecetera] {};
	\node (coreext4) [below left = 0.2cm of coreecetera] {};
	\draw [extension] (coreext1) -- (coreext2) -- (coreext3) -- (coreext4) -- (coreext1);

	\node (interfaces) [class, below = of guiproject] {Interfaces (Proxies)};

	\end{tikzpicture}
	\caption{Kommunikation der GUI mit dem Core}
	\label{fig:dev-manual-gui-communication}
\end{figure}

Nachrichten, die von einer der C++-Klassen der Komponenten an den Core gesendet
werden sollen, können mit Hilfe der Interface-Klassen in den Scheduler eingefügt
werden, der diese an das korrespondierende Core-Objekt weiterleitet.

Damit die QML-Komponenten auf Methoden der zugehörigen C++-Klasse zugreifen
können, werden deren Instanzen im \texttt{GUIProject} zum projektspezifischen
\texttt{QQMLContext} hinzugefügt, beispielsweise die Klasse
\texttt{RegisterModel} als Context-Property \texttt{registerModel}. Auf die
Properties dieses Kontexts kann über die zugehörige Bezeichnung (im vorigen Beispiel:
\texttt{registerModel}) global in jeder QML-Datei zugegriffen werden. Die
Trennung der Daten verschiedener Projekte wird gewährleistet, indem jedes
\texttt{GUIProject} seinen eigenen Kontext erhält.

Neben den projektspezifischen Kontexten existiert zusätzlich ein
programmübergreifender Kontext, der die Context-Property \texttt{ui} enthält,
die mit der Instanz der Klasse \texttt{Ui} assoziiert ist. Diese wird von
QML-Komponenten genutzt, die keinem einzelnen Projekt zugeordnet sind, um auf
Methoden der C++-Komponenten zuzugreifen. Die Toolbar beispielsweise nutzt die
\texttt{ui} Context-Property, um die Methode \texttt{run} aufzurufen, wenn der
Ausführungsbutton gedrückt wurde. Zielen diese Methoden auf einzelne Projekte
ab, wie es bei \texttt{run} der Fall ist, so muss zudem der Index des aktiven Projekts
mit angegeben werden, damit in der \texttt{Ui}-Klasse das richtige Projekt
gewählt werden kann.
