\subsubsection{Aufbau}

Die grafische Benutzeroberfläche des Simulators besteht aus einer Reihe von
Komponenten, die sich innerhalb des gegebenen Rasters nahezu beliebig anordnen
lassen. Diese Aufteilung in Komponenten spiegelt sich auch in der
Implementierung der GUI wieder, indem jede Komponente einer QML-Komponente
entspricht, die gegebenenfalls andere Teilkomponenten, nicht zuletzt elementare
Qt-Quick-Komponenten, benutzen. Die Komponenten umfassen den Editor, die sechs
frei wählbaren Komponenten Snapshots, Output, Input, Register, Speicher und
Hilfe, sowie übergeordnete Elemente wie den Split-View, die Projekt-Tabbar, die
Toolbar, die Menubar und den Einstellungsdialog.

QML-Komponenten, die Zugriff auf das vom Core zur Verfügung gestellte Modell
benötigen, werden zusätzlich mit einer C++-Klasse assoziiert. Diese Klassen
halten in der Regel Instanzen der Interface-Klassen \texttt{MemoryAccess},
\texttt{MemoryManager}, \texttt{ArchitectureAccess}, \texttt{ParserInterface}
oder \texttt{CommandInterface}  über die der Zugriff auf den Core unter
Verwendung des Schedulers erfolgt.

Da das Qt-Framework einen in sich stark abgeschlossenen Aufbau besitzt, werden
für QML-Komponenten mit komplexen Modellen, wie etwa der Speicher oder die
Register, von Qt-Klassen abgeleitete Modelle benötigt. Da keine Abhängigkeiten
vom Qt-Framework innerhalb der GUI-fernen Module Core, Parser, Arch entstehen
sollen, kann diese Funktion nicht von Klassen des Cores übernommen werden. Aus
diesem Grund übernehmen einige der zu den QML-Komponenten gehörigen C++-Klassen
der GUI die Aufgabe des Modells und werden folglich von Qt-Modell-Klassen wie
etwa \texttt{QAbstractItemModel} abgeleitet. Ausgenommen von Cache-Zwecken
halten diese Modelle selbst keine Daten, sondern holen diese über das zugehörige
Interface-Objekt, sobald Daten seitens der QML-Komponente angefordert werden.

\subsubsection{Kommunikation}
\label{gui-kommunikation}

Der Simulator unterstützt das gleichzeitige Laden mehrerer unabhängiger
Projekte. Diese werden jeweils durch eine \texttt{GUIProject}-Instanz
repräsentiert, welches Komponenten-übergreifende Funktionalität für ein Projekt
zur Verfügung stellt.

Im \texttt{GUIProject} werden mitunter die C++-Klassen der QML-Komponente
gehalten, initialisiert und deren Kommunikation mit dem Core koordiniert. Dieses
übernimmt also die Rolle des Mittelsmann zwischen dem Core, der keine
Qt-Mechanismen verwendet, und der GUI.

Bei der Initialisierung der C++-Klassen übergibt das \texttt{GUIProject}
Instanzen der Interface-Klassen (bspw. \texttt{MemoryAccess} und
\texttt{ArchitectureAccess}).

Datenaustausch ausgehend vom Core hin zu den QML-Komponenten in der GUI hat
nicht den Scheduler zwischengeschaltet, sondern verläuft über das
\texttt{GUIProject} in zwei Schritten. Dabei ruft der Core im Schritt einen
Callback auf, der im \texttt{ProjectModule} gesetzt werden (näher Informationen
im Abschnitt /Core/). Das \texttt{GUIProject} leitet im zweiten Schritt den im
\texttt{GUIProject} eingehenden Callback mit Hilfe des Qt-eigenen
Signal-Slot-Mechanismus an die Instanzen der C++-Klassen der QML-Komponenten
weiter. Diese Umleitung ist mitunter deshalb notwendig, um die eingehende
Nachricht, die im Thread des Cores gesendet wird, in den Main-Thread zu
übertragen, der von der GUI verwendet wird.

Nachrichten, die von einer der C++-Klassen der Komponenten an den Core gesendet
werden sollen, können mit Hilfe der Interface-Klassen in den Scheduler eingefügt
werden, der diese an das korrespondierende Core-Objekt weiterleitet.

Damit die QML-Komponenten auf Methoden der zugehörigen C++-Klasse zugreifen
können, werden deren Instanzen im \texttt{GUIProject} zum projektspezifischen
\texttt{QQMLContext} hinzugefügt, beispielsweise die Klasse
\texttt{RegisterModel} als Context-Property \texttt{registerModel}. Auf die
Properties dieses Kontexts kann über die zugehörige Bezeichnung (bspw.
\texttt{registerModel}) global in jeder QML-Datei zugegriffen werden. Die
Trennung der Daten verschiedener Projekte wird gewährleistet, indem jedes
\texttt{GUIProject} seinen eigenen Kontext erhält.

Eine Ausnahme von der regulären Kommunikation bildet die Toolbar, die anders als
die übrigen QML-Komponenten, die als Kommunikationspartner auftreten, nicht für
jedes Projekt einzeln existiert, sondern übergeordnet ist. Der Datenaustausch
erfolgt deshalb nicht über das \texttt{GUIProject}, sondern über die
\texttt{Ui}-Klasse, die den QML-Dateien über die Context-Property \texttt{ui}
zur Verfügung gestellt wird. Da die \texttt{Ui}-Instanz Kenntnis über das aktive
Projekt hat, können auch projektspezifische Informationen, etwa über den
aktuellen Ausführungszustand, ausgetauscht werden.

\subsubsection{Initialisierung}

\todo{Ggf. entfernen, da nicht besonders nützlich} Nachdem das Programm in der
main-Methode der Klasse \texttt{main.cpp} gestartet ist, wird eine Instanz der
Klasse \texttt{Ui} erstellt. Durch den Aufruf von \texttt{runUi} wird die
graphische Oberfläche erstellt und gestartet. Das QML-Wurzelelement des
Programms ist ein \texttt{ApplicationWindow}, welches in der Datei
\texttt{main.qml} zu finden ist. Da anfänglich noch keine Projekte geöffnet
sind, zeigt die GUI ein Fenster an, in dem ein neues Projekt angelegt werden
kann. Wurde ein neues Projekt erstellt, so wird die Methode \texttt{addProject}
der \texttt{Ui}-Klasse aufgerufen, in welcher eine \texttt{GUIProject}-Instanz
erstellt und mit gegebener Konfiguration initialisiert wird.
