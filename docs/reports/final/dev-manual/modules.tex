% ERA-Großpraktikum: Entwickleranleitung -- Module

\section{Module}
\label{dev:modules}

Diese Sektion führt in den Aufbau des Simulators ein und beschreibt detailliert
die einzelnen Funktionen der Module. Das Ziel dieses Kapitels ist es, neue
Entwickler mit dem Aufbau unseres Codes vertraut zu machen, sodass diese in der
Lage sind, selbstständig Änderungen am Code vorzunehmen. Zunächst wird ein
Überblick über den Aufbau des Programms gegeben und anschließend ausführlich auf
die einzelnen Module eingegangen.

\subsection{Überblick}

Wir haben den Simulator in vier Module aufgeteilt, die jeweils unterschiedliche
Aufgaben bei der Simulation eines Befehlssatzes übernehmen. Die
\emph{Architektur} ist für die Ausführung eines simulierten Assembler Programms
zuständig und implementiert die verfügbaren Befehlssätze (zum Stand dieses
Berichts \emph{RISCV}). Des Weiteren stellt das Architekturmodul die
Befehlssatz-spezifischen Fehler- und Hilfetexte zur Verfügung, die dem Anwender
im Endprodukt präsentiert werden. Von der Idee geleitet, dass verschiedene
Assembler Dialekte für den gleichen Befehlssatz existieren, ist der
\emph{Parser} ein eigenes Modul, der für die Übersetzung des Assembler
Quelltextes zuständig ist. Der Parser übersetzt die möglicherweise verschiedenen
Assembler Dialekte in ein allgemeines Format (wir sprechen dabei von einem
\emph{Syntaxbaum}), das anschließend von der Architektur verarbeitet werden
kann. Die einzelnen Module werden vom \emph{Core} verbunden. Der Core ist das
Herzstück des Simulators und regelt die Kommunikation zwischen allen anderen
Modulen. Insbesondere stellt er der Architektur eine Umgebung zur Verfügung, in
der diese arbeiten kann (Speicher, Register). Des Weiteren ist er für die
Verwaltung der verschiedenen Threads zuständig. Die \emph{GUI} (Graphical User
Interface) implementiert die grafische Oberfläche des Simulators. Sie ist
bewusst abgekoppelt vom Rest der Module, um auch durch andere Oberflächen (wie
z.B. eine Kommandozeile) ausgetauscht werden zu können. Die grafische Oberfläche
leitet die Benutzereingaben an den Core weiter, welcher dann wiederum die
übrigen Module anspricht.

\subsection{Architektur}

\todo[inline]{
	* Architecture Description Language
}

Die Architektur (kurz: \textit{Arch}) kümmert sich um die Ausführung der
Assembler Programme und wurde vor dem Hintergrund entworfen, einen möglichst
Architektur-unabhängigen Simulator zu entwickeln. So stellt sie Klassen zur
Verfügung, mit der eine konkrete Architektur (wie z.B. RISC-V) umgesetzt
werden kann. Eine Architektur wird in \texttt{YAML} Konfigurationsdateien
beschrieben, und anschließend in das Programm geladen. \\
Da wir uns gegen das Disassemblieren von Maschinencode zur Ausführung eines
Programms entschieden haben, bietet die Architektur eine abstraktere Darstellung
von Assembler Programmen in Form eines Syntax Baums. Des Weiteren ist die
Architektur für die Validierung einer Instruktion zuständig, und stellt bei
Fehlern entsprechende Nachrichten zur Verfügung, die dem Nutzer helfen sollen,
das Problem zu beheben.

Eine Übersicht über die Schnittstellen des Architekturmoduls bietet
\autoref{fig:arch-overview} (Die verwendeten Klassennahmen stimmen nicht mit denen
im Quellcode überein, es geht hier nur um das Prinzip).

\begin{figure}[H]
	\begin{center}
		\begin{tikzpicture}[node distance=3.0cm]
		\tikzstyle{class} = [rectangle, rounded corners, draw=black, drop shadow, fill=white]
		\tikzstyle{myarrow} = [->, thick]

		\node (architecture) [class, rectangle split, rectangle split parts=2]
		{
			\textbf{Architecture}
			\nodepart{second}
			\begin{tabular}{c}
				getName() \\
				getEndianness() \\
				getWordSize() \\
				\ldots
			\end{tabular}
		};
		\node (factory) [class, right = of architecture]
		{
			\textbf{NodeFactory}
		};
		\node (syntaxtree) [class, rectangle split, rectangle split parts=2, right = of factory]
		{
			\textbf{SyntaxTreeNode}
			\nodepart{second}
			\begin{tabular}{c}
				validate() \\
				getValue() \\
				\ldots
			\end{tabular}
		};
		\draw[myarrow] (architecture) edge node [yshift=2mm] {getFactory()} (factory);
		\draw[myarrow] (factory) edge node [yshift=2mm] {create()} (syntaxtree);
		\end{tikzpicture}
	\end{center}
	\caption{Übersicht der Architekturschnittstelle}
	\label{fig:arch-overview}
\end{figure}

Die Architektur ist in einen allgemeinen Teil (im Folgenden \textbf{common})
genannt) und in einen Architektur-spezifischen Teil (benannt nach der
entsprechenden Architektur, z.B. \textbf{riscv} aufgeteilt. Die Schnittstelle
ist so konzeptioniert, dass andere Module nichts von der konkreten Architektur
mitbekommen, und lässt sich folgendermaßen charakterisieren: \\ Das
\textit{Architecture} Objekt stellt allgemeine Informationen über die geladene
Architektur zur Verfügung (wie z.B. der Name, die Byte-Reihenfolge oder die
Wortgröße) und bietet zusätzlich Zugang zur \textit{NodeFactory}. Mit der
NodeFactory werden die Syntax Bäume in Form von \textit{SyntaxTreeNode} Objekten
erzeugt, welche anschließend validiert und ausgeführt werden können. \\ Im
Folgenden wird auf diese Klassen genauer eingegangen.

\subsubsection{Architecture Description Language}

Eine Architektur kann in \texttt{YAML} Dateien beschrieben werden.

\todo[inline]{@Peter}

\subsubsection{Architecture}

Ein \texttt{Architecture} Objekt repräsentiert eine geladene Architektur im
Simulator. Der Grundgedanke bei der Entwicklung war das modulare Design von
RISC-V zu unterstützen, dieses Konzept lässt sich aber leicht auf andere
Prozessorarchitekturen übertragen. So wird angenommen, dass eine Architektur
immer aus einem Basismodul besteht (bei RISC-V: \textit{RV32I}), welches wieder
durch andere Module erweitert werden kann (bei RISC-V z.B.
Multiplikation/Division, Floating Point, etc.). \\ Um also ein Architecture
Objekt zu erzeugen, wird im ersten Schritt eine \textit{ArchitectureFormula}
erstellt, die das Basismodul und alle Erweiterungen spezifiziert. Ein Aufruf von
\textit{Brew()} löst die Abhängigkeiten auf und liest anschließend die
Informationen aus den Konfigurationsdateien ein.

Intern wird das \textit{Builder Pattern} zum sukzessivem Aufbau des Objekts
verwendet. Als Oberklasse dient dazu das \texttt{BuilderInterface}, von dem alle
Komponenten abgeleitet werden. Da eine Architektur lediglich aus strukturierten
Informationen besteht, existiert eine weitere Oberklasse
\texttt{InformationInterface}. Von ihr sind wiederum die konkreten
Informationsklassen abgeleitet. Nennenswert dabei ist die
\texttt{ExtensionInformation} Klasse, die ein Modul in unserer modularen
Auffassung einer Architektur widerspiegelt. Ein finales \texttt{Architecture}
Modul besteht schließlich aus einem ExtensionInformation Objekt, welches mit
allen Erweiterungen vereint wurde.

Das Architecture Objekt beschreibt lediglich die Eigenschaften einer
Architektur, d.h. es stellt Informationen wie die Byte-Reihenfolge, Wortgröße,
Speicherausrichtung oder die Eigenschaften der Register bereit. Es beschreibt
auch, welche Instruktionen zur Verfügung stehen, definiert aber nicht, wie diese
konkret implementiert sind, da dies für eine Konfigurationsdatei zu komplex
wäre. Die Implementation der einzelnen Instruktionen wird deshalb in C++ Code
übernommen. Wie die Verbindung zwischen der Beschreibung der Architektur und der
Implementation abläuft, wird im Folgenden erläutert.

\subsubsection{Syntax Tree}

Der Syntax Baum mit durch Vererbung implementiert und in \textit{common} und
\textit{riscv} (oder jede beliebige andere Architektur) gegliedert. Eine
Übersicht bietet \autoref{fig:arch-syntax-tree}

\begin{figure}[H]
	\begin{center}
		\begin{tikzpicture}[node distance=0.5cm and 1.5cm]

		\node (super) [class] {AbstractSyntaxTreeNode};
		\node (invisible) [right = of super] {};
		\node (imm) [class, below = of invisible] {ImmediateNode};
		\node (bin) [class, below = of imm] {BinaryDataNode};
		\node (reg) [class, below = of bin] {AbstractRegisterNode};
		\node (instr) [class, below = of reg] {AbstractInstructionNode};
		\node (common) [below = of instr] {\textbf{common}};
		\node (reg-riscv) [class, right = of reg] {riscv::RegisterNode};
		\node (instr-riscv) [class, right = of instr] {riscv::AbstractInstructionNode};
		\node (riscv) [below = of instr-riscv] {\textbf{riscv}};

		\draw[inheritance-arrow] (imm) -- (imm -| super) -- (super);
		\draw[inheritance-arrow] (bin) -- (bin -| super) -- (super);
		\draw[inheritance-arrow] (instr) -- (instr -| super) -- (super);
		\draw[inheritance-arrow] (reg) -- (reg -| super) -- (super);

		\draw[inheritance-arrow] (instr-riscv) -- (instr);
		\draw[inheritance-arrow] (reg-riscv) -- (reg);

		\begin{pgfonlayer}{background}
		\path (super.west |- super.north)+(-0.5,0.5) node (a1) {};
		\path (instr.east |- instr.south)+(+0.5,-1.5) node (a2) {};
		\path[rounded corners, draw=black!50, dashed] (a1) rectangle (a2);
		\end{pgfonlayer}

		\begin{pgfonlayer}{background}
		\path (reg-riscv.west |- reg-riscv.north)+(-0.5,0.5) node (a1) {};
		\path (instr-riscv.east |- instr-riscv.south)+(+0.5,-1.5) node (a2) {};
		\path[rounded corners, draw=black!50, dashed] (a1) rectangle (a2);
		\end{pgfonlayer}
		\end{tikzpicture}
	\end{center}
	\caption{Klassendiagramm Syntax Baum}
	\label{fig:arch-syntax-tree}
\end{figure}

\label{module-arch-ast-node-types}
Die aufgeführten Klassen haben folgende Funktion:
\begin{itemize}

  \item \textbf{AbstractSyntaxTreeNode} ist die Oberklasse jedes Syntax Knotens
  und definiert, welche Methoden die Unterklassen implementieren müssen. Des
  Weiteren enthält sie eine Liste an etwaige Kindknoten.

  \item \textbf{ImmediateNode} repräsentiert einen \textit{Immediate}-Wert, also
  einen Wert, der direkt im Assembler Quelltext angegeben ist. Architekturen
  stellen im Allgemeinen keine Spezialisierung eines Immediate Wertes zur
  Verfügung, weshalb diese Klasse vollständig im common Teil implementiert
  werden kann.

  \item \textbf{BinaryDataNode} enthält binäre Daten, wie z.B. Text Nachrichten.
  Konkret wird er für die Implementierung der Crash Instruktion verwendet.

	\item \textbf{AbstractRegisterNode} repräsentiert ein Register in der
	Instruktion. In RISC-V muss die Assemblierung speziell behandelt werden,
	weshalb es einen speziellen RegisterNode für RISC-V gibt.

	\item \textbf{AbstractInstructionNode} ist die oberste Ebene eines jeden
	Syntax Baums und repräsentiert die auszuführende Instruktion.

\end{itemize}

Mit diesen Knotentypen lassen sich alle RISC-V Instruktionen modellieren. Nicht
RISC Architekturen stellen aber häufig die Möglichkeit bereit, einen
Speicherzugriff während einer anderen Instruktion durchzuführen. Ein Beispiel in
x86:

\begin{x86}
add eax, [ebx*2+2]
\end{x86}

Um diese Instruktionen modellieren zu können, wurden folgende Knotentypen
konzeptioniert, die jedoch nicht im Quellcode definiert sind:
\begin{itemize}
	\item \textbf{MemoryAccessNode} repräsentiert einen Inline Speicherzugriff,
	im oben aufgeführten Beispiel also der Inhalt der eckigen Klammern.
	\item \textbf{ArithmeticNode} stellt eine arithmetische Operation dar, im
	obigen Beispiel also sowohl die Multiplikation, als auch die Addition.
\end{itemize}

Die Oberklasse jedes Knotens ist die Klasse \texttt{AbstractSyntaxTreeNode}.
Folgende Methoden werden von ihr vorgegeben:

\begin{itemize}
	\item \textbf{\texttt{getValue(MemoryAccess\&)}}: Diese Methode führt den
	darunter liegenden Syntax Baum aus, und ruft ggf. rekursiv dieselbe Methode
	bei den Kindknoten auf. Je nach Knotentyp variiert der Rückgabewert: So gibt
	beispielsweise ein Instruktionsknoten die Adresse der nächsten Instruktion
	zurück, ein Register seinen aktuellen Wert und ein Immediateknoten die
	abgespeicherte Konstante. Als Parameter wird eine Zugriffsmöglichkeit auf den
	Speicher übergeben, mit dem die Instruktion z.B. das Resultat der Operation
	abspeichern kann. \\
	Mit Diesem Konzept werden Codeduplikate verhindert, da z.B. eine arithmetische
	Operation, die sowohl mit Registern, als auch mit Immediate Werten arbeiten
	kann, nur einmal implementiert werden muss. In RISC-V kann man dieses Konzept
	beispielsweise auf \textit{add} und \textit{addi} anwenden.

	\item \textbf{\texttt{validate(MemoryAccess\&)}}: Während der Parser lediglich
	eine syntaktische Überprüfung des Assembler Quelltext vornimmt, validiert
	diese Methode die semantische Korrektheit einer Instruktion. Es wird zum
	Beispiel überprüft, ob der richtige Typ und die korrekte Anzahl an Operanden
	übergeben wurde, oder ob der übergebene Immediate Werte in die vorgegebenen
	Anzahl an Bits passen. War die Validierung nicht erfolgreich, so wird eine
	übersetzbare Fehlermeldung zurückgegeben.

	\item \textbf{\texttt{validateRuntime(MemoryAccess\&)}}: Validiert, ob eine
	Instruktion zur Laufzeit ausgeführt werden kann. Die Methode wird vor allem
	für Sprunginstruktionen benötigt, sodass geprüft werden kann, ob das Ziel
	des Sprungs innerhalb des zur Verfügung stehenden Programms liegt. Des
	Weiteren lässt es sich so verhindert, dass geschützte Speicherbereiche
	von Store Instruktionen beschrieben werden.

	\item \textbf{\texttt{assemble()}}: Wandelt einen Syntax Baum in die
	Binärdarstellung der Architektur um. Diese Darstellung ist lediglich zur
	Visualisierung für den Benutzer vorgesehen, die eigentliche Simulation der
	Instruktionen wird über den Syntax Baum vorgenommen.

	\item \textbf{\texttt{getIdentifier()}}: Gibt den Typ eines Knotens als
	Zeichenkette zurück. Beispielsweise geben Instruktionen ihren entsprechenden
	Mnemonic (z.B. \textit{addi}) und Register ihren Namen (z.B. \textit{x1})
	zurück. Letzteres wird verwendet, um Schreibzugriffe auf ein Register
	in einem Instruktionsknoten durchzuführen.
\end{itemize}

Die Aufgabe einer konkreten Architektur besteht nun darin, die eben beschriebenen
Unterklassen inklusive ihrer Methoden entsprechend zu implementieren. In
\autoref{sec:extension} wird darauf genauer eingegangen. \\
Um nun Objekte des Syntax Baums zu erzeugen, wird die Node Factory benötigt, auf
die im Folgenden eingegangen wird.

\subsubsection{Node Factory}
\label{module-arch-node-factory}

Mit der \textit{Node Factory} wird der bereits beschriebene Syntax Baum erzeugt.
Um nach außen ein Architektur-unabhängiges Interface zu bieten, basiert die
Node Factory dem \textit{Abstract Factory Pattern} und wird als Diagramm in
\autoref{fig:arch-node-factory} dargestellt.

\begin{figure}[H]
	\begin{center}
		\begin{tikzpicture}[node distance=0.5cm and 3cm]
		\pgfdeclarelayer{background}
		\pgfdeclarelayer{foreground}
		\pgfsetlayers{background,main,foreground}
		\tikzstyle{class} = [rectangle, rounded corners, draw=black, fill=white, drop shadow]
		\tikzstyle{inheritance-arrow} = [->, thick,>=open triangle 90]

		\node (instr-abstr) [class, anchor=west] {AbstractInstructionNodeFactory};
		\node (imm-abstr) [class, below = of instr-abstr] {AbstractImmediateNodeFactory};
		\node (reg-abstr) [class, below = of imm-abstr]	{AbstractRegisterNodeFactory};
		\node (data-abstr) [class, below = of reg-abstr] {AbstractDataNodeFactory};
		\node (mem-abstr) [class, below = of data-abstr] {AbstractMemoryAccessNodeFactory};
		\node (arithmetic-abstr) [class, below = of mem-abstr] {AbstractArithmeticNodeFactory};
		\node (common) [below = of arithmetic-abstr] {\textbf{common}};

		\node (instr-riscv) [class, right = of instr-abstr] {riscv::InstructionNodeFactory};
		\node (imm-riscv) [class, below = of instr-riscv] {riscv::ImmediateNodeFactory};
		\node (reg-riscv) [class, below = of imm-riscv] {riscv::RegisterNodeFactory};
		\node (data-riscv) [class, below = of reg-riscv] {riscv::DataNodeFactory};
		\node (riscv) [below = of data-riscv] {\textbf{riscv}};

		\draw[inheritance-arrow] (instr-riscv) edge (instr-abstr);
		\draw[inheritance-arrow] (imm-riscv) edge (imm-abstr);
		\draw[inheritance-arrow] (reg-riscv) edge (reg-abstr);
		\draw[inheritance-arrow] (data-riscv) edge (data-abstr);

		\begin{pgfonlayer}{background}
		\path (instr-abstr.west |- instr-abstr.north)+(-1,0.5) node (a1) {};
		\path (arithmetic-abstr.east |- arithmetic-abstr.south)+(+1,-1.5) node (a2) {};
		\path[rounded corners, draw=black!50, dashed] (a1) rectangle (a2);
		\end{pgfonlayer}

		\begin{pgfonlayer}{background}
		\path (instr-riscv.west |- instr-riscv.north)+(-0.5,0.5) node (a1) {};
		\path (data-riscv.east |- data-riscv.south)+(+1,-1.5) node (a2) {};
		\path[rounded corners, draw=black!50, dashed] (a1) rectangle (a2);
		\end{pgfonlayer}
		\end{tikzpicture}
	\end{center}
	\caption{Klassendiagramm Node Factory}
	\label{fig:arch-node-factory}
\end{figure}

Wie man erkennen kann, existiert für jeden Knotentyp eine eigene Factory. Dies
dient der Übersicht, da vor allem die Implementation der
\texttt{InstructionNodeFactory} bei Architekturen mit vielen Instruktionen
schnell unübersichtlich werden kann. Um zu verhindern, dass andere Module
mehrere Node Factory Objekte verwalten müssen, wird eine
\texttt{NodeFactoryCollection} über das Architecture Objekt zur Verfügung
gestellt, welches die einzelnen Factory Objekte kapselt und die
\texttt{create()} Aufrufe weiterleitet. \\
Die Factory Methoden geben einen \texttt{std::shared\_ptr} auf den erzeugten
Knoten zurück.

Des Weiteren fällt auf, dass RISC-V die beiden letzten Node Factories nicht
implementiert. Der Grund dafür wurde in der vorherigen Sektion beschrieben:
RISC-V unterstützt keine Inline Speicherzugriffe. Architekturen können durch das
Fehlen einer Factory signalisieren, dass sie einen Knotentyp nicht unterstützen.

\subsubsection{RISC-V}

Bisher wurde fast ausschließlich die Schnittstelle beschrieben. Diese Sektion
soll einen Einblick in unsere Gedanken bei der RISC-V Implementation geben.

Zunächst etwas zur Implementation der Instruktionen. RISC-V definiert wiederum
eine eigene abstrakte Oberklasse \texttt{riscv::InstructionNode}, von der alle
weiteren Instruktionsknoten abgeleitet sind. Dies dient der Vermeidung von
Redundanz, da Methoden wie \texttt{assemble()} und
\texttt{getInstructionDocumentation()} für alle Instruktionen angewandt werden
können. Des Weiteren definiert die Klasse hilfreiche Methoden, die in den
Unterklassen verwendet werden, um zum Beispiel die Validierung der Operanden
einer Instruktion zu vereinfachen.

Die Knoten, die Instruktionen implementieren, sind ebenfalls darauf ausgelegt,
Redundanz zu vermeiden. Beispielhaft seien hier die \textit{Integer
Computational Instructions} (so der Name in der RISC-V Spezifikation)
herangezogen (das sind Instruktionen wie \textit{add}, \textit{addi} oder
\textit{and}). Da sich der Aufbau der Instruktionen lediglich in der
auszuführenden Operation unterscheidet, existiert eine weitere abstrakte
Oberklasse \texttt{riscv::AbstractIntegerInstructionNode}, die all jene
Instruktionen abdeckt. In dieser Oberklasse wird die Validierung der
Instruktionen vollständig behandelt und der Aufruf von \texttt{getValue()} so
weit abstrahiert, dass die konkreten Anwendung der Operation effektiv in einen
einzeiligen Lamda Ausdruck reduziert werden kann. Die Implementierung befindet
sich in der Datei \texttt{integer-instructions.hpp}.

Ein weiteres Konzept der RISC-V Instruktionen basiert auf der Unterstützung
unterschiedlicher Wortgrößen. Derzeit bietet RISC-V ausgiebige Unterstützung für
32 und 64 Bit, in Zukunft soll 128 Bit folgen. Um zu verhindern, dass
Instruktionen für jede Wortgröße neu geschrieben werden müssen, nutzen wir C++
Templates, um die Wortgröße einer Instruktion zu spezifizieren. So wird in der
Node Factory von RISC-V eine Fallunterscheidung nach der verwendeten Wortgröße
gemacht, und dann der entsprechende Zahlentyp als Template Parameter gesetzt
(für 32 Bit z.B. \texttt{std::uint32\_t} und für 64 Bit
\texttt{std::uint64\_t}). Sollte die Entwicklung der 128 Bit Version von RISC-V
voranschreiten, so könnte man das mit dem aktuellen C++ Standard nicht abdecken,
da kein 128 Bit Zahlentyp definiert ist. Man könnte dann aber eine vereinfachte
Implementation eines \texttt{uint128\_t} schreiben, indem man z.B. zwei
\texttt{uint64\_t} in einer Klasse kapselt.

\subsubsection{Weiterführende Dokumentation}

Diese vorherigen Sektionen geben einen Überblick über die Architektur.
Weiterführende Dokumentation findet sich in denen für die Architektur relevanten
Dateien, welche in \autoref{fig:arch-further} aufgelistet sind.

\begin{figure}[H]
	\begin{center}
	\begin{tikzpicture}[%
	grow via three points={one child at (0.8,-0.8) and
		two children at (0.8,-0.8) and (0.8,-1.7)},
	edge from parent path={($(\tikzparentnode\tikzparentanchor)+(.2cm,0pt)$) |- (\tikzchildnode\tikzchildanchor)},
	growth parent anchor=west,
	parent anchor=south west]
	\tikzstyle{every node}=[draw=black,anchor=west]
	\node {\erasim}
	child { node {isa/}
		child { node {riscv.isa/} }
	}
	child [missing] {}
	child { node {$\{\text{tests/}, \text{include/}, \text{source/}\}$}
		child { node {arch/}
			child { node {common/} }
			child { node {riscv/} }
		}
	}
	child [missing] {}
	child [missing] {}
	child [missing] {};
	\end{tikzpicture}
	\end{center}

	\caption{Relevante Dateien des Architekturmoduls}
	\label{fig:arch-further}
\end{figure}
% \subsection{Parser}

% Das \emph{Parser}-Modul übernimmt die Übersetzung des eingegebenen Textes in
% die für das \emph{Architektur}-Modul lesbaren Syntaxbäume. Damit entspricht
% dieses Modul größtenteils dem eigentlichen Assemblierer.
%
% \subsubsection{Submodule}
%
% Bei Erscheinen der Version 1.0 besteht das Parser-Modul aus vier verschiedenen
% Untermodulen:
%
% \begin{itemize}
%
% \item Das \emph{Common-Submodul} stellt Klassen bereit, die zur öffentlichen
% Schnittstelle des Parsers zu anderen Modulen dienen. Dieses Submodul ist frei
% von Abhängigkeiten zu jeglichen konkreten Parser-Implementierungen.
%
% \item Mit dem \emph{Factory-Submodul} können neue, spezifische Parser erzeugt
% werden.
%
% \item Das \emph{RISC-V-Submodul} stellt eine konkrete Implementierung eines
% Assemblierers für die RISC-V-Architektur zur Verfügung.
%
% \item Im \emph{Independent-Submodul} sind viele Hilfsklassen (zum Beispiel
% Symboltabellen, Compiler für arithmetische Ausdrücke) bereitgestellt, welche von
% dem RISC-V-Parser verwendet, genauso gut aber auch gerne von zukünftigen Parser
% eingebunden werden können.
%
% \end{itemize}
%
% Im Folgenden werden diese Submodule genauer erläutert:
%
% \paragraph{Common-Submodul}
%
% Das Kernstück des gesamten Moduls bildet die (abstrakte) Klasse \texttt{Parser}.
% Diese bietet hauptsächlich zwei Funktionen: Anbieten von Syntax-Informationen
% mittels der Methode \texttt{getSyntaxInformation} (für das Syntax-Highlighting
% in der Benutzeroberfläche), sowie dem Assemblieren eines gegebenen
% Assemblerprogrammes mithilfe der Methode \texttt{parse}, die als
% Eingabeparameter einen C++-Standard-String erwartet.
%
% \subparagraph{Syntax-Highlighting}
%
% Beim Aufruf der Methode \texttt{getSyntaxInformation} soll ein Objekt der Klasse
% \texttt{SyntaxInformation} erstellt und zurückgegeben werden. Dieses Objekt
% enthält Informationen zum Syntax-Highlighting.
%
% Zur Bestimmung der hervorzuhebenden Teile des Textes werden reguläre Ausdrücke
% verwendet. Um unterschiedliche Formatierungen zu ermöglichen, wird jeder
% Ausdruck einem Token (siehe \texttt{SyntaxInformation::Token}) zugeordnet.
%
% Um einem \texttt{SyntaxInformation}-Objekt einen regulären Ausdruck
% hinzuzufügen, kann die Methode \texttt{addSyntaxRegex} verwendet werden.
%
% \subparagraph{Das Assemblieren}
%
% Beim Aufruf der Methode \texttt{parse} soll das Assembler-Programm kompiliert
% und in einer \texttt{FinalRepresentation}-Datenstruktur zurückgegeben werden.
% Diese enthält notwendige Informationen für die Ausführung des
% Assemblerprogrammes sowie dessen Darstellung in der Benutzeroberfläche. Die
% \texttt{FinalRepresentation} besteht dabei aus folgenden Einzelheiten:
%
% \begin{itemize} \item \texttt{CommandList}: Die fertig assemblierten
% Assemblerbefehle, aneinandergereiht. Jeder der Befehle (des Typs
% \texttt{FinalCommand}) enthält einen fertig assemblierten
% \texttt{InstructionNode}, das \texttt{CodePositionInterval}, an welchem der
% Befehl im Text auftritt, sowie die Speicheradresse, an der der Befehl
% assembliert werden soll. \item \texttt{MacroInformationList}: Beinhaltet alle
% Makros, welche im Code vorkommen, mit eingesetzten Parametern und Position des
% Auftretens. \item \texttt{CompileErrorList}: Eine Liste von allen Fehlern,
% Warnungen und Hinweisen, die während des Assembliervorgangs aufgetreten sind.
% Wenn diese keine Fehler enthält (sehrwohl aber eventuell Warnungen oder
% Hinweise), so ist das Assemblieren erfolgreich gewesen und das Assemblerprogramm
% kann ausgeführt werden. \end{itemize}
%
% Gehen wir noch auf ein paar Feinheiten ein:
%
% Koordinaten im Assemblertext werden in der Datenstruktur \texttt{CodePosition}
% als zweidimensionaler Punkt gespeichert, ein Intervall davon entsprechend in der
% \texttt{CodePositionInterval}-Klasse. Ein Intervall ist genau dann leer, wenn
% sein Startpunkt vor seinem Endpunkt liegt. In diesem Fall also, wenn die
% Y-Koordinate des Endpunktes strikt kleiner als die des Startpunktes oder die
% Y-Koordinaten identisch aber die X-Koordinate des Endpunktes strikt kleiner ist.
% Ein \texttt{CodePositionInterval} wird beidseitig inklusiv gesehen (d.h. beide
% Randpunkte liegen noch im Intervall).
%
% Ein \texttt{CompileError} kapselt eine Fehlermeldung (bzw. eine Warnung oder
% einen Hinweis). Dabei wird die Position und die Schwere der Meldung (Fehler,
% Warnung, Hinweis, vgl. \texttt{Compile\-Error\-Severity}) festgehalten. Der
% Begriff „Error“ ist deswegen etwas überladen. Die Meldung selbst wird als
% \texttt{Translateable} gespeichert, sodass diese später in verschiedene Sprachen
% übersetzt werden könnte. Dabei werden Argumente separat vom eigentlichen Text
% kodiert.
%
% Eine \texttt{CompileErrorList} kapselt die \texttt{CompileError}s und stellt
% Möglichkeiten zur Erweiterung der Liste bereit. Dies erfolgt über den Aufruf von
% Makros. Der Grund hierfür ist, dass so die Fehlermeldungen automatisch von einem
% Qt-Programm gefunden und für das Übersetzen markiert werden können. Die
% Meldungen müssen dabei ein C-String sein (\texttt{const char*}). Ebenso lassen
% sich über die \texttt{CompileErrorList} einfache Abfragen stellen, ob jeweils
% Fehler, Warnungen oder Hinweise vorhanden sind und wenn ja, wie viele.
%
% \textbf{Factory-Submodul}
%
% Kommen wir zum Factory-Submodul: Dieses besteht lediglich aus einer einzigen
% Klasse, der \texttt{ParserFactory}. Bei jener werden alle
% Parserimplementierungen unter einem Namen zur Auswahl gestellt, sodass sie mit
% Architektur und Speicherzugriff kombiniert einen Parser erzeugen können. Diese
% Abhängigkeit zu den einzelnen Implementierungsmodulen ist auch der Grund, wieso
% das Factory-Submodul aus dem Common-Submodul herausgenommen wurde.
%
% Mit der Methode \texttt{ParserFactory::createParser} kann dabei ein Parser mit
% den angegebenen Voraussetzungen generiert werden. Die Map
% \texttt{ParserFactory::mapping} enthält alle registrierten Parser.
%
% \paragraph{Independent-Submodul}
%
% Kommen wir nun zum wohl größten Submodul des Parsers, dem
% \emph{Independent}-Modul, welches eine Sammlung von Hilfsklassen darstellt,
% die von verschiedenen Assemblern verwendet werden können sollen. Dieses Modul
% kann und soll gerne erweitert, dabei aber unabhängig von jeglichem spezifischen
% Assemblierer gehalten werden.
%
% \subparagraph{Intermediate-Darstellung für Befehle}
%
% Das Independent-Submodul stellt Klassen zur Verfügung, die als
% Übergangsdarstellung zwischen Text und assemblierten Syntaxbaum eines
% Assembler-Programms dienen. Eine Übersicht der wichtigsten Klassen bietet
% \autoref{fig:parser-intermediate}.
%
% \begin{figure}[h!]
% 	\begin{center}
% 		\begin{tikzpicture}[node distance=1.0cm and 0.7cm]
%
% 		\node (super) [class] {IntermediateOperation}; \node (instr) [class,
% 		below = of super] {IntermediateInstruction}; \node (macro) [class, right
% 		= of instr] {IntermediateMacroInstruction}; \node (dir) [class, left =
% 		of instr] {IntermediateDirective}; \node (sub) [below = of dir, yshift =
% 		3mm, xshift = 1cm] {\emph{Diverse implementierte Direktiven}};
%
% 		\draw[inheritance-arrow] (dir.north) -- ++(0,0.5cm) -| (super);
% 		\draw[inheritance-arrow] (macro.north) -- ++(0,0.5cm) -| (super);
% 		\draw[inheritance-arrow] (instr.north) -- (super);
% 		\draw[inheritance-arrow] (sub.north) ++(-1.0cm,0) -- (dir);
% 		\end{tikzpicture}
% 	\end{center} \caption{Klassendiagramm Intermediate-Darstellung}
% 	\label{fig:parser-intermediate}
% \end{figure}
%
% Diese Klassen haben folgende Funktionen:
%
% \begin{itemize}
% 	\item \texttt{IntermediateOperation} ist die abstrakte Oberklasse, die das
% 	Interface definiert. \item \texttt{IntermediateInstruction} steht für eine
% 	Maschineninstruktion, die vom Architekturmodul ausgewertet werden muss.
% 	\item \texttt{IntermediateMacroInstruction} steht für eine Instruktion, bei
% 	der es sich um einen Makroaufruf handelt. Diese Klasse muss nicht von einem
% 	Parser instanziiert werden (näheres im Abschnitt Makros). \item
% 	\texttt{IntermediateDirective} ist eine Oberklasse für Implementierungen von
% 	Parser-Direkti\-ven. Alle unterstützten Direktiven erben von dieser Klasse.
% \end{itemize}
%
% Die Klasse \texttt{IntermediateRepresentator} enthält eine Liste mehrerer dieser
% Operationen und stellt eine Methode bereit, um diese Operationen in eine
% \texttt{FinalRepresentation} umzuwandeln.
%
% \subparagraph{Assemblierung der Intermediate-Darstellung}
% \label{dev:dev_parser_assem_inter} Um die Operationen aus der
% Intermediate-Darstellung in ein Objekt der Klasse \texttt{FinalRepresentation}
% umzuwandeln, kann die Funktion
% \texttt{IntermediateRepresentator::\allowbreak{}transform} aufgerufen werden.
% Diese Methode führt nacheinander folgende Schritte aus:
%
% \begin{enumerate}
% 	\item Aufrufen der \texttt{precompile} Funktion für alle Operationen. \item
% 	Ersetzen aller Instruktionen, bei denen es sich um Makroaufrufe handelt, mit
% 	Makro-Instruktionen. \item Reservieren von Speicher für alle Befehle mit
% 	Hilfe der \texttt{allocateMemory} Funktion. \item Einlesen aller
% 	Labels/Konstanten und deren Werte mit der \texttt{enhanceSymbolTable}
% 	Funktion. \item "`Ausführen"' aller Operationen mit der
% 	\texttt{execute}-Funktion, d.h. meistens Erzeugen eines Syntaxknotens und
% 	Einfügen dieses Knotens in die \texttt{FinalRepresentation}.
% \end{enumerate}
%
% \subparagraph{Makros} Für das Parsen von Makros existieren zwei Direktiven:
% \texttt{Makro\-Directive} und \texttt{Makro\-End\-Directive}. Alle Befehle
% zwischen diesen beiden Direktiven werden in eine interne Liste des
% \texttt{Makro\-Directive}-Objekts eingefügt, anstatt wie gewöhnlich in den
% \texttt{Intermediate\-Representator}.
%
% Im \texttt{precompile}-Schritt trägt sich die Makro-Direktive mit ihrem Namen in
% eine Tabelle (\texttt{Macro\-Directive\-Table}) ein. Daraufhin wird für jede
% \texttt{Intermediate\-Instruction} überprüft, ob der Name ihres Befehls als
% Makro in der Tabelle eingetragen ist. Falls ja, wird die Instruktion mit einer
% \texttt{Intermediate\-Macro\-Instruction} ersetzt und alle Befehle aus der
% Makro-Direktive nach Einsetzen von eventuellen Parametern in die neue
% Makro-Instruktion kopiert.
%
% Im \texttt{allocateMemory}- und \texttt{execute}-Schritt reicht die Makro-
% Instruktion die Funktionsaufrufe einfach an die enthaltenen Instruktionen
% weiter. Falls Fehler entstehen, werden diese an die Position des Makro-Aufrufs
% verschoben.
%
% \subparagraph{Symbolgraph und Symbol-Replacer} Für das Vermerken und Einsetzen
% von Symbolen wie Labels oder Konstanten existiert mit den Klassen
% \texttt{SymbolGraph} und \texttt{SymbolReplacer} ein eigenes System. Dieses kann
% ausschließlich nichtparametrisierte Symbole verarbeiten (d.h. Funktionen sind
% nicht möglich). Ein Symbol ist in unserem Fall ein Name, der für einen anderen
% Namen oder Wert steht.
%
% Zum Generieren eines \texttt{SymbolReplacer}s zum Ersetzen von Symbolen
% konstruiert man einen \texttt{SymbolGraph} mit allen einzufügenden Symbolen
% (Klasse \texttt{Symbol}). Der \texttt{SymbolGraph} selbst besteht dann aus den
% einzelnen Symbole als Knoten und Abhängigkeiten zwischen ihnen als Kanten. Ein
% Symbol A ist dabei abhängig von einem Symbol B, wenn B in dem Ersetzungstext von
% A vorkommt. Im Symbolgraph würde hier dann eine Kante von B nach A geleitet
% werden. Anschließend können wir unseren \texttt{SymbolGraph} evaluieren. Dabei
% werden folgende Dinge überprüft: \begin {itemize} \item Sind die Namen
% (orientiert an C, d.h. Buchstabe oder \_ als erstes Zeichen, danach beliebig
% viele davon incl. von Zahlen) gültig? \item Existiert einer/mehrere Namen
% doppelt? \item Existiert eine zyklische Abhängigkeit unter den Symbolen (bzw. im
% Symbolgraph)? \end {itemize}
%
% Während erste und zweite Eigenschaft leicht per Iterieren über die Symbole bzw.
% Einteilen in Äquivalenzklassen dieser (z.B. mittels einer Map) gelöst werden
% können, ist es für die dritte Eigenschaft nötig, den Symbolgraphen zu betrachten
% (bzw. sie ist überhaupt der Grund, dass wir uns so einen Graphen generiert
% haben): Wir überprüfen hierzu, ob unser nicht zwingend zusammenhängende Digraph
% (gerichteter Graph) kreisfrei ist. Dazu lassen wir über jede
% Zusammenhangskomponente eine Tiefensuche laufen und merken uns den Pfad zu
% unserem Startknoten. Wenn wir nun auf einen Knoten stoßen, der in unserem Pfad
% bereits vorhanden ist, so haben wir einen Kreis entdeckt und geben das so weiter. Kommen wir ohne
% Kreise durch, so geben wir noch die topologische Sortierung unserer Knoten
% zurück (diese entspricht der umgedrehten Reihenfolge, in der wir die Knoten
% abgeschlossen haben). Die topologische Sortierung ist in unserem Fall genau eine
% Anordnung der Symbole, sodass das Symbol an Position $i$ nur von allen
% Symbolen im Bereich $\{0,...,i-1\}$ abhängt. Damit können wir nun unsere
% Symbole insgesamt vereinfachen: Wir iterieren vorwärts über alle Symbole und
% setzen in Symbol $s_i$ alle Symbole ein, von denen $s_i$ abhängt. Wir wissen
% dabei, wegen der topologischen Sortierung, dass $$s_i$$ nur von
% $s_0,...,s_{i-1}$ abhängt und diese Symbole haben wir schon so weit wie möglich
% ersetzt. Damit erhalten wir am Ende eine Darstellung, in der die Symbole
% weitestmöglich ersetzt sind. Diese können wir dann weitergeben an das
% Ersetzungssystem. Wir unterscheiden dabei noch zwischen statischen und
% dynamischen Symbolen: Statische Symbole haben dabei einen unveränderlichen Wert
% und werden daher wie geplant ersetzt. Für dynamische Symbole ist der Wert jedoch
% erst beim Ersetzen des Symbols im Programm bekannt, nicht, wenn der
% Symbolgraph ausgewertet wird. Daher werden diese Symbole \emph{nicht ersetzt}
% (sehr wohl aber auf wohlgeformte Namen und Duplikate geprüft), sondern verbleiben
% mit ihrem Namen in den weitestmöglich ersetzten Symbolen. Ein Einsatzzweck für
% solche dynamischen Symbole sind zum Beispiel relative Adressen: Da beim
% Generieren des Graphen nicht bekannt ist, bei welchen Befehlen das Symbol zum
% Einsatz kommt, kann es nicht ersetzt werden. All dies erfolgt, wenn man
% \texttt{evaluate} beim Symbolgraphen aufruft. Würden wir nun parametrisierte
% Symbole zulassen, so hätten wir das Problem, dass wir hier selbst bei statischen
% Symbolen nicht feststellen können, ob ein Ersetzungsvorgang irgendwann beendet
% ist (vgl. Halteproblem, das ist sogar \emph{beweisbar} nicht möglich), also
% müssten wir ab irgendeiner Ersetzungstiefe abbrechen. Dennoch stünde es
% zukünftigen Projekten frei, dies zu implementieren.
%
% Laufzeiten (Worst-Case, sei $n$ die Anzahl an Symbolen): \begin{itemize} \item
% Namen überprüfen: $\mathcal{O}(n)$ \item Duplikate finden:
% $\mathcal{O}(n*f(n))$ (mit $f(m)$ als maximale Zeit, um in die Map mit $m$
% Elementen einzufügen) \item Zyklenfreiheit überprüfen und topologische
% Sortierung ermitteln: $\mathcal{O}(n^2)$ (dennoch, sehr unwahrscheinlich im
% normalen Gebrauch; Worst-Case wäre die transitive Hülle eines zusammenhängenden
% Symbolgraphen) \end{itemize}
%
% Die Evaluierung (Klasse \texttt{SymbolGraphEvaluation}) wird anschließend an den
% SymbolReplacer weitergegeben. Dieser ersetzt textuell bei einem gegebenen
% Eingabestring alle Symbolvorkommen über mehrere Iterationen hinweg, denn es ist
% ja auch möglich, dass ein dynamisches Symbol zu einem weiteren dynamischen
% Symbol expandiert (wenn auch noch kein solcher Fall bekannt ist). In der Tat ist
% es wegen der Natur der dynamischen Symbole ebenfalls nicht möglich, zu sehen, ob
% eine Ersetzung mit Sicherheit terminiert (vgl. oben). Also brechen wir nach
% einer bestimmten Anzahl an Ersetzungsiterationen (standardmäßig auf 4 gesetzt)
% ab und geben einen Fehler zurück. Anmerkung: Die textuelle Ersetzungs ist noch
% alles andere als ideal, besser wäre, wenn die Ersetzung ausschließlich auf einem
% Stream von Tokens (siehe nächsten Abschnitt) arbeiten würde (und auch einen
% solchen zurückgibt).
%
% \subparagraph{Compiler für arithmetische Ausdrücke} Neben Symbolersetzung
% enthält die \texttt{independent}-Bibliothek auch ein System, um arithmetische
% Ausdrücke zu verarbeiten. Es handelt sich dabei um die Klasse
% \texttt{ExpressonCompiler} bzw. ihre Instanziierungen.
%
% Zum Tokenizen (in \texttt{ExpressionTokenizer}) verwenden wir einen Regulären
% Ausdruck für alle möglichen Tokens (Operatoren, Literale usw.), somit entspricht
% unser Vorgehen dem Einsatz eines NFA bzw. DFA (nichtdeterministischer endlicher
% Automat bzw. deterministischer endlicher Automat). Immer wenn wir einen
% akzeptierenden Zustand erreichen, sehen wir ein Token als abgeschlossen an und
% starten den Automaten neu. Landen wir in einem Fehlerzustand, so geben wir einen
% \texttt{CompileError} zurück. Dabei ist es natürlich der Fall, dass bei
% Sequenzen $s_0$ und $s_0s_1$ stets $s_0s_1$ präferiert wird (d.h. im
% regulären Ausdruck wird $s_0s_1$ zuerst erwähnt).
%
% Das eigenliche Parsen des Token-Streams erfolgt mithilfe des
% Shunting-Yard-Algorithmus \footnote{\url{http://wcipeg.com/wiki/Shunting_yard_algorithm}} in
% der Klasse \texttt{ExpressionParser}, welcher in linearer Zeit über der Eingabe
% arbeitet (und linear viel Speicher verbraucht). Dieser löst das Problem der
% Auswertung von Infix-Operatoren (vgl. Punkt vor Strich: Wenn wir von links nach
% rechts durch unseren Ausdruck gehen und ein Plus und danach eine Zahl einlesen,
% können wir nicht sofort auswerten, denn das nächste Zeichen könnte auch der
% Multiplikationspunkt sein), indem die Anwendung dieser weitestmöglich
% hinausgezögert wird – bis eben klar ist, ob noch höherwertige Operatoren kommen.
%
% Dafür halten wir uns einen Stack mit allen momentan vorhandenen Operatoren und
% öffnenden Klammern, sowie einen Stack für die Ausgabe. Lesen wir eine Zahl, so
% kommt diese auf den Ausgabe-Stack, gelesene öffnende Klammern kommen auf den
% Operator-Stack. Kommen wir zu einem Operator, so betrachten wir unseren
% Operator-Stack und ermitteln, ob dieser Operator eine (echt niedriger für
% rechtsassoziative Operatoren, damit zögern wir die Auswertung desselben
% hintereinander geschriebenen Operators hinaus) niedrigere Präzedenz als der
% aktuell betrachtete hat. Wenn ja, so wenden wir diesen auf die obersten Zahlen
% auf dem Ausgabe-Stack an (auf zwei Zahlen für binäre, auf eine für unäre
% Operatoren). Danach überprüfen wir erneut, ob es Operatoren zum Abarbeiten gibt.
% Lesen wir irgendwann im Text eine schließende Klammer, so wird der
% Operator-Stack bis zur letzten öffnenden Klammer abgearbeitet. So gehen wir
% durch unseren String durch – am Ende unseres Ausdrucks schließlich behandeln wir
% den Rest auf dem Operator-Stack (z.B. indem ein schwächerer Operator als auch
% Klammern, wenn wir die als solche betrachten, anwenden). Auf dem Ausgabe-Stack
% liegt jetzt das Ergebnis des Ausdrucks (im Idealfall, sollten wir nicht ein
% Element auf dem Stack zu diesem Zeitpunkt haben, so ist ein Fehler aufgetreten).
%
% Es ist dabei möglich, einen \texttt{ExpressionCompiler} mit eigenen Operatoren
% zu definieren, mittels der Datenstruktur \texttt{ExpressionCompilerDefinition}.
% Folgende Punkte werden dabei spezifiziert: \begin{itemize} \item Binäre
% Operatoren mit Regex und einer Auswertungsfunktion, sowie Präzedenz \item Unäre
% Operatoren, ebenfalls mit Regex und Auswertungsfunktion; die Präzedenz muss
% nicht angegeben werden, da unäre Operatoren (wg. der Natur des
% Shunting-Yard-Algorithmus) immer vor allen binären Operatoren von links nach
% rechts abgearbeitet werden (d.h. unäre Operatoren sind nur als Präfix erlaubt).
% \item Literaldecoder mit Erkennungsregex und Dekodierfunktion. Unter Literale
% fällt in diesem Fall alles, was kein Symbol ist; also z.B. Zahlen, Strings oder
% Register (vgl. z.B. X86-Speicherzugriffe). \item Weitere Hilfsregexes, wie
% Klammern (öffnend, schließend) und das Format für Symbole, welche separat von
% Literalen behandelt ersetzt werden. \end{itemize} Ein Beispiel, wie dies
% definiert werden kann, findet sich bei den \texttt{CLikeExpressionCompiler}s,
% welche der aktuelle RISC-V-Parser benutzt (mit $$int64_t$$ als Typ). Diese
% führen C-artige Operatoren (alle, die direkt und unverändernd auf Zahlen
% operieren) auf den Standard-Zahlentypen ein.
%
% Statt Zahlen ist es auch möglich – obwohl dies noch nicht entwickelt ist – einen
% Syntaxbaum auf dem Ausgabe-Stack aufzubauen, ohne Änderungen an der
% Funktionsweise des Algorithmus durchführen zu müssen. Hierzu müsste ein neuer,
% Parser-eigener Syntaxbaum entwickelt werden (da die Architektur bereits
% weitestmöglich ausgewertete Bäume erwartet, z.B. erlaubt RISC-V keine
% Operator-Knoten), welcher mit Assoziativität und Kommutativität (vielleicht auch
% Distributivität o.ä.) vereinfachen kann. Das Prinzip wäre wie folgt möglich:
% Literale erhalten einen eigenen Knoten – eventuelle Register würde man hier auch
% als Literal sehen. Wenn ein Operator auf $$n\in\{1,2\}$$ Syntaxbäume angewandt
% wird, so wird ein neuer Syntaxbaum erzeugt, welcher als Wurzel den Operator und
% als Kinder die $$n$$ eingegebenen Knoten besitzt. Würde man dies für den
% \texttt{CLikeExpressionCompiler} machen wollen, so müsste entsprechende Semantik
% für Infix-Operatoren und Implizier Konvertierung von Zahltypen implementiert
% werden – der Typ des Compilers selber kann mit Templates flexibel ausgewechselt
% werden. Sobald dann ein Syntaxbaum für den Ausdruck generiert wurde (d.h. der
% Algorithmus ist durchgelaufen), kann der Syntaxbaum weitestgehend ausgewertet
% werden, indem beispielsweise Operatoren auf Konstanten angewandt werden.
% Abschließend muss dann der Parser-Syntaxbaum noch in einen für die Architektur
% verständlichen Syntaxbaum übersetzt werden, dies kann knotenweise geschehen.
%
% Es sei angemerkt, dass der Shunting-Yard-Algorithmus nur für einfache (hier:
% Klammern, binäre und unäre Operatoren, sowie Literale) arithmetische Ausdrücke
% verwendet werden kann. So ist z.B. der X86-Ausdruck $$[eax+4*ebx+16]$$ nur
% auswertbar (unter Annahme, dass das eigentliche Auswerten mit Syntaxbäumen
% geschieht; sonst käme das Problem auf, wie die Register „geparst“ werden
% sollen), wenn die äußeren Klammern erkannt, dann der innere Ausdruck ausgewertet
% und schließlich mit einem Speicherzugriffsknoten umschlossen wird. Alternativ
% wäre es auch denkbar, verschiedene Klammertypen mit vielleicht auch
% unterschiedlichen Präzedenzen einzuführen, welche wie ein unärer Operator
% fungieren. Die bisherigen „normalen“ Klammern würden dann der
% Identitätsoperation entsprechen (im Idealfall, wenn nicht umdefiniert), die
% neuen, eckigen Klammern hingegen könnten dann einen Parser-eigenen
% Speicherzugriffsknoten über den Syntaxbaum positionieren. Aber generell ist
% dieses Problem eher gering, da die Speicherzugriffsklammer immer auf oberster
% Ebene ist und sich somit als einfachen Spezifallfall handhaben lässt. Wenn wir
% zum Vergleich die AT&T-Syntax betrachten, so wäre auch die oben beschriebene
% Variante mit verschiedenen unterstützten Klammern nicht mehr möglich, dies
% müsste dann auf oberster Ebene auch wieder manuell gelesen werden.
%
% \subparagraph{Segmentverwaltung}
%
% Das Parser-Modul unterstützt Allokation von Speicher in verschiedenen Segmenten.
% Implementiert ist momentan eine rudimentäre Unterscheidung zwischen Code- und
% Datensegment (\texttt{text} und \texttt{data}). Code kann dabei nur in ersterem
% Segment gespeichert werden, Daten hingegen in beiden Segmenten, allerdings
% generieren sie beim Schreiben in das Code-Segment eine Warnung. Wenn auf solche
% Daten getroffen wir, selbst wenn die kodierten Instruktionen valide sind, so
% bricht die Ausführung des Programms ab, da an dieser Stelle keine Instruktion
% kompiliert wurde und so auch kein Syntaxbaum zum Ausführen existiert.
%
% Die zentrale Klasse, um all diese Informationen zu sammeln, ist der
% \texttt{MemoryAllocator}, welcher die Segmente verwaltet und hier fortlaufend
% allokiert. Wichtig zu beachten ist, dass Segmente dicht gepackt hintereinander
% in den Speicher geschrieben werden. Wenn also z.B. beim RISC-V-Parser mehr Daten
% allokiert werden, so verschiebt sich das Programm im Speicher nach hinten. Diese
% Mechanik hat zwar den Vorteil, dass das Programm und Daten nicht weit
% voneinander entfernt im Speicher stehen, was das ganze einfacher zum Betrachten
% macht, gleichzeitig wird dadurch auch aber ein weiterer Assemblerpass benötigt
% (Pass zwei von vier).
%
% \paragraph{RISC-V-Submodul}
%
% Für RISC-V existiert ein korrespondierendes Parser-Submodul, welches in der
% \texttt{ParserFactory} über den Namen \texttt{riscv} aufgerufen werden kann. Es
% verwendet dabei größtenteils die Klassen, die im Independent-Submodul definiert
% sind.
%
% Der RISC-V-Parser liest zuerst zeilenweise den Programm-Code ein und speichert
% ihn in einem \texttt{Intermediate\-Representator}. Dann wird die
% \texttt{transform}-Funktion aufgerufen und die entstehende
% \texttt{Final\-Representation} zurückgegeben.
%
% Zum Parsen einer einzelnen Zeile wird die Hilfsklasse
% \texttt{RiscvParser::\allowbreak{}RiscvRegex} verwendet. Diese Klasse nutzte
% während der Entwicklung einen regulären Ausdruck, um die Zeile in verschiedene
% Elemente aufzuteilen. Um bestimmte Einschränkungen dieses Vorgangs zu umgehen,
% wurde die Klasse neu geschrieben, der Name aber beibehalten.
%
% Nach aktuellem Stand wird über jedes Zeichen der zu parsenden Zeile iteriert und
% bei gewissen Zeichen der letzte Teil der Zeile als Label, Instruktion oder
% Parameter gespeichert. Hierbei werden beliebig viele Parameter unterstützt.
%
% \subsubsection{Verwenden eines Parsers}
%
% Von außen kann das Parser-Modul über die \texttt{ParserFactory} angesprochen
% werden. Hier erzeugt man mit Name, Speicherzugriff und gegebener Architektur
% einen gewünschten Parser und erhält einen Unique-Pointer darauf. Anschließend
% kann man bereits die \texttt{parse}- und \texttt{getSyntaxHighliting}-Methoden
% des Parsers selber aufrufen und Text assemblieren lassen. Das Deinitialisieren
% erfolgt ebenfalls über den eingebauten Destruktor automatisch.
%
% Eine Übersicht über diesen Prozess bietet \autoref{fig:parser-overview}.
%
% \begin{figure}[h!]
% 	\begin{center}
% 		\begin{tikzpicture}[node distance=1.0cm and 3.0cm] \tikzstyle{myarrow} =
% 		[->, thick]
%
% 		\node (factory) [class]
% 		{
% 			\textbf{ParserFactory}
% 		};
% 		\node (invis1) [empty, right = of factory] {}; \node (parser) [class,
% 		rectangle split, rectangle split parts=2, below = of invis1]
% 		{
% 			\textbf{Parser} \nodepart{second} \begin{tabular}{c}
% 				getSyntaxInformation()
% 			\end{tabular}
% 		};
% 		\node (final) [class, rectangle split, rectangle split parts=2, right =
% 		of parser]
% 		{
% 			\textbf{FinalRepresentation} \nodepart{second} \begin{tabular}{c}
% 				commandList() \\ errorList() \\ macroList() \\ \ldots
% 			\end{tabular}
% 		};
% 		\draw[myarrow] (factory) edge node [xshift=1.1cm, yshift=2mm]
% 		{createParser()} (parser); \draw[myarrow] (parser) edge node
% 		[yshift=2mm] {parse()} (final); \end{tikzpicture}
% 	\end{center} \caption{Übersicht der Parserschnittstelle}
% 	\label{fig:parser-overview}
% \end{figure}
%
% \subsubsection{Einschränkungen}
%
% Trotz der aktuellen Fähigkeiten des Parser-Moduls existieren noch folgende
% Einschränkungen:
%
% \begin{itemize} \item \emph{Compiler für arithmetische Ausdrücke}: Momentan
% unterstützt der Parser bereits, wie erklärt, das Compilen von arithmetischen
% Ausdrücken – aber auch nicht mehr. Eine Vision wäre, z.B. einen SLR-Parser zu
% implementieren, welcher langfristig komplette Operanden parsen können soll. Dies
% könnte zum Beispiel realisiert werden, indem ein Python-Skript (fremde Module
% wie Bison sind leider nicht erlaubt!) aus einer gegebenen Grammatik (z.B. mit
% Regexes für die einzelnen Terminalsymbole, sodass Literale usw. erkannt werden
% können) eine Zustandsgraphen-Tabelle erzeugt, welche dann vom Programm verwendet
% wird. Zum Beispiel könnte dann pro Anwendung einer Regel aus der Grammatik eine
% Transformation des Syntaxbaumes erfolgen. \item \emph{Symbol-Tabelle}: Wie
% bereits erwähnt unterstützt die Symboltabelle momentan keine parametrisierten
% Einträge. Eventuell wäre es aber zukünftig wünschenswert, solche zu
% unterstützen, da so einige Ausdrücke (z.B. RISC-V: Pseudo-Instruktion für Laden
% eines 32-Bit-Immediates) schöner geschrieben werden können. \item
% \emph{Segment-Darstellung}: Zwar werden momentan Daten- und Codesegment
% unterschieden – aber eben nicht mehr. Möglich wäre noch, beispielsweise
% \texttt{bss}- und \textt{stack}-Segment (momentan muss der Stack „reingehackt“
% werden, indem einfach auf eine hohe Speicheradresse zugegriffen wird), wenn
% nicht mehr, einzuführen. Die neuen Segmente sollten einfach bei der
% \texttt{MemorySectionDefinition} jeder Architektur angegeben werden, wo sie sich
% relativ gesehen befinden. Viel weiter gedacht, könnte man auch Semantik von
% einfachen Linker-Skripten implementieren, wobei hier ebenfalls ein SLR-Parser
% nützlich sein könnte. \item \emph{Gleitkommazahl-Unterstützung}: Unterstützung
% für Gleitkommazahlen wurde während der Entwicklung aufgegeben, kann aber bei
% Bedarf in den Parser integriert werden. Die wahrscheinlich größte Schwierigkeit
% würde das Auswerten der arithmetischen Ausdrücke mit Gleitkomma- und Ganzzahlen
% darstellen. \end{itemize}
%
% In den Anfangsphasen des Projekts kam kurzzeitig in Betracht, einen möglichst
% unabhängigen Parser zu schreiben, sodass jeder denkbare Parser das gleiche
% Backend verwendet, welches die bereits unterteilten und gelesenen Operanden und
% Ausdrücke in Syntaxbäume umformt. Da dies einiges mehr an Arbeit erfordert
% hätte, wurde die Idee gestrichen. Stattdessen entschieden wir uns, einen
% monolithischen Parser pro Dialekt anzubieten und außerdem häufig verwendete
% Funktionen in eine Bibliothek auszulagern.
%
% \subsubsection{Weiterführende Dokumentation}
%
% Weiterführende Dokumentation findet sich in den für den Parser relevanten
% Dateien, welche in \autoref{fig:parser-further} aufgelistet sind.
%
% \begin{figure}[h!]
% 	\begin{center} \begin{tikzpicture}[% grow via three points={one child at
% 	(0.8,-0.8) and
% 		two children at (0.8,-0.8) and (0.8,-1.7)},
% 	edge from parent path={($(\tikzparentnode\tikzparentanchor)+(.2cm,0pt)$) |-
% 	(\tikzchildnode\tikzchildanchor)}, growth parent anchor=west, parent
% 	anchor=south west] \tikzstyle{every node}=[draw=black,anchor=west] \node
% 	{\erasim{}} child { node {$\{\text{tests/}, \text{include/},
% 	\text{source/}\}$}
% 		child { node {parser/}
% 			child { node {common/} } child { node {factory/} } child { node
% 			{independent/} } child { node {riscv/} }
% 		}
% 	}
% 	child [missing] {} child [missing] {} child [missing] {}; \end{tikzpicture}
% 	\end{center}
%
% 	\caption{Relevante Dateien des Parsermoduls} \label{fig:parser-further}
% \end{figure}

% \subsection{Core}
\label{Kapitel: Core}

\subsubsection{Aufbau}

Der Core ist zwar das zentrale Modul des Simulators, seine Instanzen werden
jedoch von der GUI verwaltet. Dabei erstellt die GUI für jedes neue Projekt, was
in der Benutzeroberfläche einem Tab entspricht, eine neue Instanz des Cores über
das \texttt{ProjectModule}. Dieses verwaltet (indirekt) die Komponenten eines
Projektes, also Speicher, Register, Architekturdefinition und Parser, und
ermöglicht der GUI über Interface-Klassen sicheren Zugriff auf die anderen
Komponenten des Simulators. Um den GUI-Thread, in dem ein Event-Loop der
Qt-Engine läuft, von der Übersetzung und Ausführung unabhängig zu machen, werden
dabei pro Core-Instanz mehrere Threads benötigt.

\subsubsection{Threading}

Zur Vermeidung von Race Conditions und anderen Problemen wird das Konzept des
\textit{Active Object Pattern} verwendet: Dabei wird für ein Objekt ein Thread
erstellt und sichergestellt, dass die Methoden dieses Objektes nur in diesem
Thread laufen. Dazu wird pro \textit{Active Object} ein \textit{Scheduler}
benötigt, der Befehle zur Ausführung einer Methode des Objektes in einer
Warteschlange verwaltet und nacheinander ausführt. Der Zugriff von außen auf das
\textit{Active Object} darf daher nur über sogenannte \textit{Proxy} Objekte
erfolgen, die Methodenaufrufe in die Warteschlange des Schedulers einreihen. Die
Rückgabe von Werten erfolgt dabei über \texttt{futures} oder callbacks, wobei
letzteres nur zu anderen Active Objects möglich ist und bisher nicht verwendet
wird. Dieses Konzept wurde gewählt, um die Risiken von Threading-bezogenen
Fehlern zu minimieren und den Aufwand zur Synchronisierung, bis auf die
Implementierung des Active Object Patterns, gering zu halten.

Im \texttt{ProjectModule} werden zwei \textit{Active Objects} verwaltet: Das
\texttt{Project} Objekt verwaltet den Memory und die Register. Dabei wird ein
eigenes Active Object mit einem Thread benötigt, da es beispielsweise möglich
sein muss, über die Benutzeroberfläche während der Ausführung auf den Speicher
zuzugreifen. Das zweite Active Object wird für die Ausführung und Übersetzung
der Programme benötigt. Dies ist in der \texttt{ParsingAndExecutionUnit}
umgesetzt. Dadurch bleibt die GUI auch während der Ausführung ansprechbar.

\subsubsection{Interface}

Die Kommunikation des Cores mit der GUI teilt sich auf zwei Richtungen auf:

Die GUI muss auf Methoden des Cores zugreifen können, um an den Zustand von
Speicher und Register zu gelangen und diesen zu ändern, oder auch um Befehle
auszuführen. Dabei werden der GUI über das \texttt{ProjectModule} Interface
Klassen zur Verfügung gestehlt, die \texttt{Proxy} Klassen der Active Objects
sind. Diese Interface Klassen werden teilweise auch an die anderen Komponenten
übergeben, um dem Parser und den Befehlen beispielsweise Zugriff auf Speicher
und Register zu ermöglichen. Im einzelnen sind die Interface Klassen des
\texttt{Project}s der \texttt{MemoryAccess}, der Zugriff auf Speicher und
Register ermöglicht, der \texttt{MemoryManager}, über den weitere Funktionen von
Speichern und Register verwaltet werden können, und der
\texttt{ArchitectureAccess}, über den auf das Architekturobjekt des Projekts
zugegriffen werden kann. Die Interfaces der \texttt{ParsingAndExecutionUnit}
sind das \texttt{CommandInterface}, das beispielsweise die Übermittlung von
Ausführungsbefehlen erlaubt, und das \texttt{ParserInterface}, über das auf
Informationen des Parsers zugegriffen werden kann.

Desweiteren muss der GUI auch signalisiert werden, dass sich der Zustand des
Cores geändert hat und die Ansichten aktualisiert werden sollen. Dafür werden
Callbacks verwendet, damit der Core keine Referenz auf die GUI braucht und daher
unabhängig von deren Aufbau ist. Diese Callbacks werden über die Interface
Klassen gesetzt und bei entsprechenden Änderungen durch die Core-Komponenten
aufgerufen. Dabei ist zu beachten, dass die Callbacks im jeweiligen Active
Object ausgeführt werden, also auch in dessen Thread. Daher ist es nötig, in den
Callback Funktionen, die in der GUI Implementiert werden, Qt-Signals mit
\texttt{QueuedConnections} zu verwenden, um die entsprechenden die
erforderlichen Aktionen in der GUI im richtigen Thread auszuführen. Da jeweils
nur ein Callback im Core gesetzt werden kann, ist es über diesen Mechanismus
auch möglich, mehrere Komponenten in der GUI zu benachrichtigen.

\subsubsection{Ausführung}

Der Core ist auch die Schnittstelle der GUI zum Parser, daher übernimmt er den
Aufruf des Parsers und das Einfügen der assemblierten Befehlsdarstellung in den
Speicher. Dabei wird dieser Speicherbereich vor der normalen Bearbeitung durch
den Nutzer oder Assemblerbefehle geschützt.

Nach der Übersetzung liefert der Parser dem Core ein Liste von Befehlen, die
ausgeführt werden können, oder eine List an Fehlermeldungen. Die Ausführung
läuft dabei auch in einem Active Object, wodurch die GUI nicht beeinträchigt
wird. Da deshalb während der laufenden Ausführung alle anderen Befehle für
dieses Active Object warten müssen, kann ein Stop-Signal nicht wie ein normaler
Aufruf realisiert werden. Deshalb wird dazu ein mit dem \texttt{ProjectModule}
geteiltes Flag verwendet. Dieses wird immer geprüft, wenn ein neuer Befehl
ausgeführt wird.

Falls nicht nur einzelne Befehle ausgeführt werden, ist auch zu beachten, dass
durch die Ausführung jedes Befehls einige Callbacks an die GUI geschickt werden,
wodurch in bestimmten Situationen das Message-System der Qt-Engine überlastet
werden kann. Um dies zu verhindern, wird nach jedem Befehl ein
Synchronisations-Callback aufgerufen und mit der Ausführung gewartet, bis dieser
in der GUI ausgeführt wurde. Der Synchronisations-Callback muss dazu die
\texttt{ProjectModule::guiReady()} Methode aufrufen. Dadurch wird
sichergestellt, dass die Message-Queue abgearbeitet wurde und eine Überlastung
verhindert.

\subsubsection{Memory und Register}

Hier passiert viel Magie.

% \subsection{GUI}

\subsubsection{Aufbau}

Die grafische Benutzeroberfläche des Simulators besteht aus einer Reihe von
Komponenten, die sich innerhalb des gegebenen Rasters nahezu beliebig anordnen
lassen. Diese Aufteilung in Komponenten spiegelt sich auch in der
Implementierung der GUI wieder, da jede der wählbaren Komponente einer
QML-Komponente entspricht. Dazu gehören Snapshots, Hilfe, Register, Speicher,
Input und Output (siehe auch Abbildung \ref{fig:gui-composition}).

Ihnen übergeordnet ist eine Organisationsstruktur mit dem eigentlichen Fenster
(\texttt{ApplicationWindow}) an der Spitze, welches unterteilt wird in die
Menubar mit Editor- und Projekt-Untermenü, die Toolbar mit den Ausführungsbutton
und schließlich den Projekt-Tabview, welcher die geöffneten Projekte enthält.
Bei letzterem gilt zu beachten, dass der Tabview sowohl die Tabbar als auch die
Tab-Inhalte, also die eigentlichen Projekte, repräsentiert.

Jedes Projekt weist, wie bereits erwähnt, eine gerasterte Struktur auf, die
durch die Splitview-Komponente mit vier nebeneinander angeordneten Teilbereichen
voller Höhe realisiert ist. Jeder dieser Teilbereiche ist wiederum in zwei
übereinander angeordnete Bereiche unterteilt, verwirklicht durch die
InnerSplitviews bzw. die InnerSplitviewsEditor. Letztere QML-Komponente ist eine
spezielle Ausprägung, die im oberen Teilbereich den Editor anzeigt, der,
anders als alle anderen Komponenten, nicht variabel positionierbar ist.

Die inneren Splitviews bestehen schließlich aus einem SplitViewItem, welches
neben der frei wählbaren Komponente (Snapshots, Hilfe etc.) auch den
Komponenten-Header enthält, über den zum Einen die angezeigte Komponente gewählt
wird und zum Anderen die Komponenteneinstellungen aufgerufen werden können.

\begin{figure}[H]
	\begin{center}
    \begin{tikzpicture}
      \tikzset{font=\small,
        edge from parent fork down,
        level distance=1.75cm,
        every node/.style=
          {rectangle,rounded corners,
          minimum height=8mm,
          fill=white,
          draw=black!50,
          drop shadow,
          align=center,
          text depth = 0pt
          },
        edge from parent/.style=
          {{Diamond}-,
          draw=black!50,
          thick
          }}
\Tree [.{ApplicationWindow\\(main.qml)}
        [.MenuBar ]
        [.Toolbar ]
        [.ProjectTabView
            [.Splitview
                 [.InnerSplitviews
                     [.SplitViewItem
                         [.Snapshots ]
                         [.Help ]
                         [.Registers ]
                         [.Memory ]
                         [.Inputs ]
                         [.Outputs ] ] ]
                 [.{InnerSplitviews-\\Editor}
                     [.SplitViewItem ]
                     [.Editor ] ] ] ] ]
\end{tikzpicture}
	\end{center}
	\caption{Grundlegende Zusammensetzung der GUI-Komponenten. Die Knotentitel im
	Diagramm entsprechen nicht notwendigerweise den Namen der assoziierten
	Komponente.}
	\label{fig:gui-composition}
\end{figure}

\subsubsection{C++-Komponenten}

QML-Komponenten, die Zugriff auf das vom Core zur Verfügung gestellte Modell
benötigen, werden zusätzlich mit einer C++-Klasse assoziiert. Diese Klassen
halten Instanzen der Interface-Klassen \texttt{MemoryAccess},
\texttt{MemoryManager}, \texttt{ArchitectureAccess}, \texttt{ParserInterface}
oder \texttt{CommandInterface}, über die der Zugriff auf den Core unter
Verwendung des Schedulers erfolgt.

Da das Qt-Framework einen in sich stark abgeschlossenen Aufbau besitzt, werden
für QML-Komponenten mit komplexen Modellen, wie etwa den Speicher oder die
Register, von Qt-Klassen abgeleitete Modelle benötigt. Da keine Abhängigkeiten
vom Qt-Framework innerhalb der GUI-fernen Module Core, Parser und Architektur
entstehen sollen, kann diese Funktion nicht von Klassen des Cores übernommen
werden. Aus diesem Grund übernehmen einige der zu den QML-Komponenten gehörigen
C++-Klassen der GUI die Aufgabe des Modells und werden folglich von
Qt-Modell-Klassen wie etwa \texttt{QAbstractItemModel} abgeleitet. Mit Ausnahme
von Cache-Zwecken halten diese Modelle selbst keine veränderbaren Daten
(Registerwerte, Speicherwerte etc.), sondern holen diese über das zugehörige
Interface-Objekt, sobald Daten seitens der QML-Komponente angefordert werden.

\subsubsection{Kommunikation}
\label{gui-kommunikation}

Der Simulator unterstützt das gleichzeitige Laden mehrerer unabhängiger
Projekte. Diese werden jeweils durch eine \texttt{GUIProject}-Instanz
repräsentiert, welche Komponenten-über\-grei\-fen\-de Funktionalitäten für ein
Projekt zur Verfügung stellt.

Im \texttt{GUIProject} werden mitunter die C++-Klassen der QML-Komponente
gehalten, initialisiert und deren Kommunikation mit dem Core koordiniert.
Dieses übernimmt also die Rolle des Mittelsmann zwischen dem Core, der keine
Qt-Mechanismen verwendet, und der GUI.

Bei der Initialisierung der C++-Klassen übergibt das \texttt{GUIProject} diesen
die benötigten Instanzen der Interface-Klassen (\texttt{MemoryAccess} und
\texttt{ArchitectureAccess} etc.).

Datenaustausch ausgehend vom Core hin zu den QML-Komponenten in der GUI hat
nicht den Scheduler zwischengeschaltet, sondern verläuft über das
\texttt{GUIProject} in zwei Schritten (siehe auch Abbildung
\ref{fig:dev-manual-gui-communication}). Im ersten Schritt ruft der Core einen
Callback auf, der im \texttt{ProjectModule} gesetzt wird (näher Informationen im
Abschnitt \ref{Dev-Kapitel: Core}). Das \texttt{GUIProject} leitet im zweiten
Schritt den im \texttt{GUIProject} eingehenden Callback mit Hilfe des Qt-eigenen
Signal-Slot-Mechanismus an die Instanzen der C++-Klassen der QML-Komponenten
weiter. Diese Umleitung ist auch deshalb notwendig, um die eingehende Nachricht,
die im Thread des Cores gesendet wird, in den Main-Thread zu übertragen, der von
der GUI verwendet wird.

\begin{figure}
	\centering
	\begin{tikzpicture}[auto, node distance=2cm]
	% NODE DEFINITIONS
	\tikzstyle{module} = [draw, rectangle,
	text width=4cm]
	\tikzstyle{class} = [rectangle, rounded corners, draw=black, fill=white, drop shadow, text width=4cm, align=center]
	\tikzstyle{modulearrow} = [->, thick]
	\tikzstyle{extension} = [dashed]


	\node (guiproject) [class] {GUIProject};

	\node (registermodel) [class, left = of guiproject] {RegisterModel};
	\node (memorycomponentpresenter) [class, above = of registermodel] {MemoryComponent-\\Presenter};
	\node (editorcomponent) [class, below = of registermodel] {EditorComponent};
	\node (guiecetera) [below = of editorcomponent, text width=4cm, align=center] {...};
	\node (guipackage) [above = 0.5cm of memorycomponentpresenter] {GUI};
	\node (guiext1) [above left = 0.2cm of memorycomponentpresenter] {};
	\node (guiext2) [above right = 0.2cm of memorycomponentpresenter] {};
	\node (guiext3) [below right = 0.2cm of guiecetera] {};
	\node (guiext4) [below left = 0.2cm of guiecetera] {};
	\draw [extension] (guiext1) -- (guiext2) -- (guiext3) -- (guiext4) -- (guiext1);
	\coordinate[right = 0.2cm of registermodel] (guiextension);

	\node (projectmodule) [class, right = of guiproject] {ProjectModule};
	\node (coreecetera) [below = of projectmodule, text width=4cm, align=center] {...};
	\node (corepackage) [above = 0.5cm of projectmodule] {Core};
	\node (coreext1) [above left = 0.2cm of projectmodule] {};
	\node (coreext2) [above right = 0.2cm of projectmodule] {};
	\node (coreext3) [below right = 0.2cm of coreecetera] {};
	\node (coreext4) [below left = 0.2cm of coreecetera] {};
	\draw [extension] (coreext1) -- (coreext2) -- (coreext3) -- (coreext4) -- (coreext1);
	\coordinate[left = 0.2cm of projectmodule] (coreextension);

	\node (scheduler) [class, below = of guiproject] {Scheduler};

	\draw[modulearrow] (guiproject) to node [text width=2cm, xshift=0.5cm, yshift=0.8cm] {Signal} ($ (guiextension) + (0.1cm, 0cm) $);
	\draw[modulearrow] ($ (coreextension) + (-0.1cm, 0cm) $) to node [text width=2cm, xshift=0.3cm, yshift=0.8cm] {Callback} (guiproject);

	\draw[modulearrow] ($ (guiextension) + (0.1cm, 0cm) $) to[bend right] node [text width=4cm, yshift=-0.2cm] {Interfaces (Proxies)} (scheduler);
	\draw[modulearrow] (scheduler) to[bend right] ($ (coreextension) + (-0.1cm, 0cm) $);
	
	\end{tikzpicture}
	\caption{Kommunikation der GUI mit dem Core}
	\label{fig:dev-manual-gui-communication}
\end{figure}

Nachrichten, die von einer der C++-Klassen der Komponenten an den Core gesendet
werden sollen, können mit Hilfe der Interface-Klassen in den Scheduler eingefügt
werden, der diese an das korrespondierende Core-Objekt weiterleitet.

Damit die QML-Komponenten auf Methoden der zugehörigen C++-Klasse zugreifen
können, werden deren Instanzen im \texttt{GUIProject} zum projektspezifischen
\texttt{QQMLContext} hinzugefügt, beispielsweise die Klasse
\texttt{RegisterModel} als Context-Property \texttt{registerModel}. Auf die
Properties dieses Kontexts kann über die zugehörige Bezeichnung (im vorigen Beispiel:
\texttt{registerModel}) global in jeder QML-Datei zugegriffen werden. Die
Trennung der Daten verschiedener Projekte wird gewährleistet, indem jedes
\texttt{GUIProject} seinen eigenen Kontext erhält.

Neben den projektspezifischen Kontexten existiert zusätzlich ein
programmübergreifender Kontext, der die Context-Property \texttt{ui} enthält,
die mit der Instanz der Klasse \texttt{Ui} assoziiert ist. Diese wird von
QML-Komponenten genutzt, die keinem einzelnen Projekt zugeordnet sind, um auf
Methoden der C++-Komponenten zuzugreifen. Die Toolbar beispielsweise nutzt die
\texttt{ui} Context-Property, um die Methode \texttt{run} aufzurufen, wenn der
Ausführungsbutton gedrückt wurde. Zielen diese Methoden auf einzelne Projekte
ab, wie es bei \texttt{run} der Fall ist, so muss zudem der Index des aktiven Projekts
mit angegeben werden, damit in der \texttt{Ui}-Klasse das richtige Projekt
gewählt werden kann.

