% ERA-Großpraktikum: Entwickleranleitung -- Module

\section{Module}

Diese Sektion führt in den Aufbau des Simulators ein und beschreibt detailliert
die einzelnen Funktionen der Module. Das Ziel dieser Sektion ist es, neue
Entwickler mit dem Aufbau unseres Codes vertraut zu machen, sodass diese in der
Lage sind, selbstständig Änderungen am Code vorzunehmen. \\
Zunächst wird ein Überblick über den Aufbau des Programms gegeben und
anschließend ausführlich auf die einzelnen Module eingegangen.

\subsection{Überblick}

Wir haben den Simulator in vier Module aufgeteilt, die jeweils unterschiedliche
Aufgaben bei der Simulation eines Befehlssatzes übernehmen.

Die \textbf{Architektur} ist für die Ausführung eines simulierten Assembler
Programms zuständig und implementiert die verfügbaren Befehlssätze (zum Stand
dieses Berichts \textit{RISCV}). Des Weiteren stellt das Architekturmodul
die Befehlssatz-spezifischen Fehler- und Hilfetexte zur Verfügung, die dem
Anwender im Endprodukt präsentiert werden. \\
Von der Idee geleitet, dass verschiedene Assembler Dialekte für den gleichen
Befehlssatz existieren, ist der \textbf{Parser} ein eigenes Modul, der für die
Übersetzung des Assembler Quelltextes zuständig ist. Der Parser übersetzt die
möglicherweise verschiedenen Assembler Dialekte in ein allgemeines Format (wir
sprechen dabei von einem \textit{Syntax Baum}), das anschließend von der Architektur
verarbeitet werden kann. \\
Die einzelnen Module werden vom \textbf{Core} verbunden. Der Core ist das
Herzstück des Simulators und regelt die Kommunikation zwischen allen anderen
Modulen. Insbesondere stellt er der Architektur eine Umgebung zur Verfügung, in
der diese arbeiten kann (Speicher, Register). Des Weiteren ist er für die
Verwaltung der verschiedenen Threads zuständig. \\
Die \textbf{GUI} (Graphical User Interface) implementiert die grafische
Oberfläche des Simulators. Sie ist bewusst abgekoppelt vom Rest der Module,
um auch durch andere Oberflächen (wie z.B. einer Kommandozeile) ausgetauscht
werden zu können. Die grafische Oberfläche leitet die Benutzereingaben an den
Core weiter, welcher dann wiederum die übrigen Module anspricht.

\subsection{Architektur}

\todo[inline]{
	* Syntax Tree \\
	* Factory \\
	* Architecture \\
	* Teilung common/riscv \\
	* ValidationResult \\
	* InstructionNode \\
}

Die Architektur (kurz: \textit{Arch}) kümmert sich um die Ausführung der
Assembler Programme und wurde vor dem Hintergrund entworfen, einen möglichst
Architektur-unabhängigen Simulator zu entwickeln. So stellt sie Klassen zur
Verfügung, mit der eine konkrete Architektur (wie z.B. RISC-V) umgesetzt
werden kann. Eine Architektur wird in \texttt{YAML} Konfigurationsdateien
beschrieben, und anschließend in das Programm geladen. \\
Da wir uns gegen die Ausführung von Maschinencode entschieden haben, bietet
die Architektur eine abstraktere Darstellung von Assembler Programmen in Form
eines Syntax Baums. Des Weiteren ist die Architektur für die Validierung einer
Instruktion zuständig, und stellt bei Fehlern entsprechende Nachrichten zur
Verfügung, die dem Nutzer helfen sollen, das Problem zu beheben.

Eine Übersicht über die Schnittstellen des Architekturmoduls bietet Abbildung
\ref{fig:arch-overview}

\begin{figure}[H]
	\begin{center}
		\begin{tikzpicture}[node distance=3.0cm]
		\tikzstyle{class} = [rectangle, rounded corners, draw=black, drop shadow, fill=white]
		\tikzstyle{myarrow} = [->, thick]

		\node (architecture) [class, rectangle split, rectangle split parts=2]
		{
			\textbf{Architecture}
			\nodepart{second}
			\begin{tabular}{c}
				getName() \\
				getEndianness() \\
				getWordSize() \\
				\ldots
			\end{tabular}
		};
		\node (factory) [class, right = of architecture]
		{
			\textbf{NodeFactory}
		};
		\node (syntaxtree) [class, rectangle split, rectangle split parts=2, right = of factory]
		{
			\textbf{SyntaxTreeNode}
			\nodepart{second}
			\begin{tabular}{c}
				validate() \\
				getValue() \\
				\ldots
			\end{tabular}
		};
		\draw[myarrow] (architecture) edge node [yshift=2mm] {getFactory()} (factory);
		\draw[myarrow] (factory) edge node [yshift=2mm] {create()} (syntaxtree);
		\end{tikzpicture}
	\end{center}
	\caption{Übersicht der Architekturschnittstelle}
	\label{fig:arch-overview}
\end{figure}

Das \textit{Architecture} Objekt stellt allgemeine Informationen über die
geladene Architektur zur Verfügung (wie z.B. der Name, die Byte-Reihenfolge oder
die Wortgröße) und bietet zusätzlich Zugang zur \textit{NodeFactory}.
Mit der NodeFactory werden die Syntax Bäume in Form von \textit{SyntaxTreeNode}
Objekten erzeugt, welche anschließend validiert und ausgeführt werden können. \\
Im Folgenden wird auf diese Klassen genauer eingegangen.

\subsubsection{Architecture}

\todo[inline]{
	* Builder Pattern \\
	* Formula/Brewery \\
	* Information Interfaces
}

Ein \textit{Architecture} Objekt repräsentiert eine geladene Architektur im
Simulator. Der Grundgedanke bei der Entwicklung war das modulare Design von RISC-V
zu unterstützen, dieses Konzept lässt sich aber leicht auf andere Prozessorarchitekturen
übertragen. So wird angenommen, dass eine Architektur immer aus einem Basismodul
besteht (bei RISC-V: \textit{RV32I}), welches wieder durch andere Module erweitert
werden kann (bei RISC-V z.B. Multiplikation/Division, Floating Point, etc.). \\
Um also ein Architecture Objekt zu erzeugen, wird im ersten Schritt eine
\textit{ArchitectureFormula} erstellt, die das Basismodul und alle Erweiterungen
spezifiziert. Ein Aufruf von \textit{Brew()} löst die Abhängigkeiten auf und liest
anschließend die Informationen aus den Konfigurationsdateien ein.

Das Architecture Objekt beschreibt lediglich die Eigenschaften einer Architektur,
d.h. es stellt Informationen wie die Byte-Reihenfolge, Wortgröße, Speicherausrichtung
oder die Eigenschaften der Register bereit. Es beschreibt auch, welche Instruktionen
zur Verfügung stehen, definiert aber nicht, wie diese konkret implementiert sind,
da dies für eine Konfigurationsdatei zu komplex wäre. Die Implementation der
einzelnen Instruktionen wird deshalb in C++ Code übernommen. Wie die Verbindung
zwischen der Beschreibung der Architektur und der Implementation abläuft, wird
im Folgenden erläutert.

\subsubsection{Syntax Tree}

\todo[inline]{
	* Immediate \\
	* Instruction \\
	* MemoryAccess \\
	* RegisterAccess \\
	* Arithmetic \\
	* DataNode \\
	* getValue() assemble() etc.
}

\subsubsection{Node Factory}

\todo[inline]{
	* NodeFactoryCollection \\
	* NodeFactoryCollectionMaker \\
	* FactoryTypes
}

Die Node Factory basiert auf dem Abstract Factory Pattern und wird als Diagramm
in Abbildung \ref{fig:arch-node-factory} dargestellt.

\begin{figure}[H]
	\begin{center}
		\begin{tikzpicture}[node distance=0.5cm and 3cm]
		\pgfdeclarelayer{background}
		\pgfdeclarelayer{foreground}
		\pgfsetlayers{background,main,foreground}
		\tikzstyle{class} = [rectangle, rounded corners, draw=black, fill=white, drop shadow]
		\tikzstyle{inheritance-arrow} = [->, thick,>=open triangle 90]
	
		\node (instr-abstr) [class, anchor=west] {AbstractInstructionNodeFactory};
		\node (imm-abstr) [class, below = of instr-abstr] {AbstractImmediateNodeFactory};
		\node (reg-abstr) [class, below = of imm-abstr]	{AbstractRegisterNodeFactory};
		\node (data-abstr) [class, below = of reg-abstr] {AbstractDataNodeFactory};
		\node (mem-abstr) [class, below = of data-abstr] {AbstractMemoryAccessNodeFactory};
		\node (arithmetic-abstr) [class, below = of mem-abstr] {AbstractArithmeticNodeFactory};
		\node (abstr) [below = of arithmetic-abstr] {common};

		\node (instr-riscv) [class, right = of instr-abstr] {riscv::InstructionNodeFactory};
		\node (imm-riscv) [class, below = of instr-riscv] {riscv::ImmediateNodeFactory};
		\node (reg-riscv) [class, below = of imm-riscv] {riscv::RegisterNodeFactory};
		\node (data-riscv) [class, below = of reg-riscv] {riscv::DataNodeFactory};
		\node (riscv) [below = of data-riscv] {riscv};

		\draw[inheritance-arrow] (instr-riscv) edge (instr-abstr);
		\draw[inheritance-arrow] (imm-riscv) edge (imm-abstr);
		\draw[inheritance-arrow] (reg-riscv) edge (reg-abstr);
		\draw[inheritance-arrow] (data-riscv) edge (data-abstr);

		\begin{pgfonlayer}{background}
		\path (instr-abstr.west |- instr-abstr.north)+(-1,0.5) node (a1) {};
		\path (arithmetic-abstr.east |- arithmetic-abstr.south)+(+1,-1.5) node (a2) {};
		\path[rounded corners, draw=black!50, dashed] (a1) rectangle (a2);
		\end{pgfonlayer}

		\begin{pgfonlayer}{background}
		\path (instr-riscv.west |- instr-riscv.north)+(-0.5,0.5) node (a1) {};
		\path (data-riscv.east |- data-riscv.south)+(+1,-1.5) node (a2) {};
		\path[rounded corners, draw=black!50, dashed] (a1) rectangle (a2);
		\end{pgfonlayer}
		\end{tikzpicture}
	\end{center}
	\caption{Klassendiagramm Node Factory}
	\label{fig:arch-node-factory}
\end{figure}

\subsubsection{Architecture Description Language}

Eine Architektur kann in \texttt{YAML} Dateien beschrieben werden.

\todo[inline]{@Peter}

\subsection{Parser}

\subsection{Core}

Beschreibung des Moduls + Schnittstellen zu anderen Modulen

\subsection{GUI}
