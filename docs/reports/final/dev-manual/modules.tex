% ERA-Großpraktikum: Entwickleranleitung -- Module

\section{Module}

Diese Sektion führt in den Aufbau des Simulators ein und beschreibt detailliert
die einzelnen Funktionen der Module. Das Ziel dieser Sektion ist es, neue
Entwickler mit dem Aufbau unseres Codes vertraut zu machen, sodass diese in der
Lage sind, selbstständig Änderungen am Code vorzunehmen. \\
Zunächst wird ein Überblick über den Aufbau des Programms gegeben und
anschließend ausführlich auf die einzelnen Module eingegangen.

\subsection{Überblick}

Wir haben den Simulator in vier Module aufgeteilt, die jeweils unterschiedliche
Aufgaben bei der Simulation eines Befehlssatzes übernehmen.

Die \textbf{Architektur} ist für die Ausführung eines simulierten Assembler
Programms zuständig und implementiert die verfügbaren Befehlssätze (zum Stand
dieses Berichts \textit{RISCV}). Des Weiteren stellt das Architekturmodul
die Befehlssatz-spezifischen Fehler- und Hilfetexte zur Verfügung, die dem
Anwender im Endprodukt präsentiert werden. \\
Von der Idee geleitet, dass verschiedene Assembler Dialekte für den gleichen
Befehlssatz existieren, ist der \textbf{Parser} ein eigenes Modul, der für die
Übersetzung des Assembler Quelltextes zuständig ist. Der Parser übersetzt die
möglicherweise verschiedenen Assembler Dialekte in ein allgemeines Format (wir
sprechen dabei von einem \textit{Syntax Baum}), das anschließend von der Architektur
verarbeitet werden kann. \\
Die einzelnen Module werden vom \textbf{Core} verbunden. Der Core ist das
Herzstück des Simulators und regelt die Kommunikation zwischen allen anderen
Modulen. Insbesondere stellt er der Architektur eine Umgebung zur Verfügung, in
der diese arbeiten kann (Speicher, Register). Des Weiteren ist er für die
Verwaltung der verschiedenen Threads zuständig. \\
Die \textbf{GUI} (Graphical User Interface) implementiert die grafische
Oberfläche des Simulators. Sie ist bewusst abgekoppelt vom Rest der Module,
um auch durch andere Oberflächen (wie z.B. einer Kommandozeile) ausgetauscht
werden zu können. Die grafische Oberfläche leitet die Benutzereingaben an den
Core weiter, welcher dann wiederum die übrigen Module anspricht.

\subsection{Architektur}

\todo[inline]{
	* Syntax Tree \\
	* Factory \\
	* Architecture \\
	* Teilung common/riscv \\
	* ValidationResult \\
	* InstructionNode \\
}

Die Architektur (kurz: \textit{Arch}) kümmert sich um die Ausführung der
Assembler Programme und wurde vor dem Hintergrund entworfen, einen möglichst
Architektur-unabhängigen Simulator zu entwickeln. So stellt sie Klassen zur
Verfügung, mit der eine konkrete Architektur (wie z.B. RISC-V) umgesetzt
werden kann. Eine Architektur wird in \texttt{YAML} Konfigurationsdateien
beschrieben, und anschließend in das Programm geladen. \\
Da wir uns gegen die Ausführung von Maschinencode entschieden haben, bietet
die Architektur eine abstraktere Darstellung von Assembler Programmen in Form
eines Syntax Baums. Des Weiteren ist die Architektur für die Validierung einer
Instruktion zuständig, und stellt bei Fehlern entsprechende Nachrichten zur
Verfügung, die dem Nutzer helfen sollen, das Problem zu beheben.

Eine Übersicht über die Schnittstellen des Architekturmoduls bietet Abbildung
\ref{fig:arch-overview}

\begin{figure}[H]
	\begin{center}
		\begin{tikzpicture}[node distance=3.0cm]
		\tikzstyle{class} = [rectangle, rounded corners, draw=black, drop shadow, fill=white]
		\tikzstyle{myarrow} = [->, thick]

		\node (architecture) [class, rectangle split, rectangle split parts=2]
		{
			\textbf{Architecture}
			\nodepart{second}
			\begin{tabular}{c}
				getName() \\
				getEndianness() \\
				getWordSize() \\
				\ldots
			\end{tabular}
		};
		\node (factory) [class, right = of architecture]
		{
			\textbf{NodeFactory}
		};
		\node (syntaxtree) [class, rectangle split, rectangle split parts=2, right = of factory]
		{
			\textbf{SyntaxTreeNode}
			\nodepart{second}
			\begin{tabular}{c}
				validate() \\
				getValue() \\
				\ldots
			\end{tabular}
		};
		\draw[myarrow] (architecture) edge node [yshift=2mm] {getFactory()} (factory);
		\draw[myarrow] (factory) edge node [yshift=2mm] {create()} (syntaxtree);
		\end{tikzpicture}
	\end{center}
	\caption{Übersicht der Architekturschnittstelle}
	\label{fig:arch-overview}
\end{figure}

Das \textit{Architecture} Objekt stellt allgemeine Informationen über die
geladene Architektur zur Verfügung (wie z.B. der Name, die Byte-Reihenfolge oder
die Wortgröße) und bietet zusätzlich Zugang zur \textit{NodeFactory}.
Mit der NodeFactory werden die Syntax Bäume in Form von \textit{SyntaxTreeNode}
Objekten erzeugt, welche anschließend validiert und ausgeführt werden können. \\
Im Folgenden wird auf diese Klassen genauer eingegangen.

\subsubsection{Architecture}

\todo[inline]{
	* Builder Pattern \\
	* Formula/Brewery \\
	* Information Interfaces
}

Ein \textit{Architecture} Objekt repräsentiert eine geladene Architektur im
Simulator. Der Grundgedanke bei der Entwicklung war das modulare Design von RISC-V
zu unterstützen, dieses Konzept lässt sich aber leicht auf andere Prozessorarchitekturen
übertragen. So wird angenommen, dass eine Architektur immer aus einem Basismodul
besteht (bei RISC-V: \textit{RV32I}), welches wieder durch andere Module erweitert
werden kann (bei RISC-V z.B. Multiplikation/Division, Floating Point, etc.). \\
Um also ein Architecture Objekt zu erzeugen, wird im ersten Schritt eine
\textit{ArchitectureFormula} erstellt, die das Basismodul und alle Erweiterungen
spezifiziert. Ein Aufruf von \textit{Brew()} löst die Abhängigkeiten auf und liest
anschließend die Informationen aus den Konfigurationsdateien ein.

Das Architecture Objekt beschreibt lediglich die Eigenschaften einer Architektur,
d.h. es stellt Informationen wie die Byte-Reihenfolge, Wortgröße, Speicherausrichtung
oder die Eigenschaften der Register bereit. Es beschreibt auch, welche Instruktionen
zur Verfügung stehen, definiert aber nicht, wie diese konkret implementiert sind,
da dies für eine Konfigurationsdatei zu komplex wäre. Die Implementation der
einzelnen Instruktionen wird deshalb in C++ Code übernommen. Wie die Verbindung
zwischen der Beschreibung der Architektur und der Implementation abläuft, wird
im Folgenden erläutert.

\subsubsection{Syntax Tree}

\todo[inline]{
	* Immediate \\
	* Instruction \\
	* MemoryAccess \\
	* RegisterAccess \\
	* Arithmetic \\
	* DataNode \\
	* getValue() assemble() etc.
}

\subsubsection{Node Factory}

\todo[inline]{
	* NodeFactoryCollection \\
	* NodeFactoryCollectionMaker \\
	* FactoryTypes
}

Die Node Factory basiert auf dem Abstract Factory Pattern und wird als Diagramm
in Abbildung \ref{fig:arch-node-factory} dargestellt.

\begin{figure}[H]
	\begin{center}
		\begin{tikzpicture}[node distance=0.5cm and 3cm]
		\pgfdeclarelayer{background}
		\pgfdeclarelayer{foreground}
		\pgfsetlayers{background,main,foreground}
		\tikzstyle{class} = [rectangle, rounded corners, draw=black, fill=white, drop shadow]
		\tikzstyle{inheritance-arrow} = [->, thick,>=open triangle 90]

		\node (instr-abstr) [class, anchor=west] {AbstractInstructionNodeFactory};
		\node (imm-abstr) [class, below = of instr-abstr] {AbstractImmediateNodeFactory};
		\node (reg-abstr) [class, below = of imm-abstr]	{AbstractRegisterNodeFactory};
		\node (data-abstr) [class, below = of reg-abstr] {AbstractDataNodeFactory};
		\node (mem-abstr) [class, below = of data-abstr] {AbstractMemoryAccessNodeFactory};
		\node (arithmetic-abstr) [class, below = of mem-abstr] {AbstractArithmeticNodeFactory};
		\node (abstr) [below = of arithmetic-abstr] {common};

		\node (instr-riscv) [class, right = of instr-abstr] {riscv::InstructionNodeFactory};
		\node (imm-riscv) [class, below = of instr-riscv] {riscv::ImmediateNodeFactory};
		\node (reg-riscv) [class, below = of imm-riscv] {riscv::RegisterNodeFactory};
		\node (data-riscv) [class, below = of reg-riscv] {riscv::DataNodeFactory};
		\node (riscv) [below = of data-riscv] {riscv};

		\draw[inheritance-arrow] (instr-riscv) edge (instr-abstr);
		\draw[inheritance-arrow] (imm-riscv) edge (imm-abstr);
		\draw[inheritance-arrow] (reg-riscv) edge (reg-abstr);
		\draw[inheritance-arrow] (data-riscv) edge (data-abstr);

		\begin{pgfonlayer}{background}
		\path (instr-abstr.west |- instr-abstr.north)+(-1,0.5) node (a1) {};
		\path (arithmetic-abstr.east |- arithmetic-abstr.south)+(+1,-1.5) node (a2) {};
		\path[rounded corners, draw=black!50, dashed] (a1) rectangle (a2);
		\end{pgfonlayer}

		\begin{pgfonlayer}{background}
		\path (instr-riscv.west |- instr-riscv.north)+(-0.5,0.5) node (a1) {};
		\path (data-riscv.east |- data-riscv.south)+(+1,-1.5) node (a2) {};
		\path[rounded corners, draw=black!50, dashed] (a1) rectangle (a2);
		\end{pgfonlayer}
		\end{tikzpicture}
	\end{center}
	\caption{Klassendiagramm Node Factory}
	\label{fig:arch-node-factory}
\end{figure}

\subsubsection{Architecture Description Language}

Eine Architektur kann in \texttt{YAML} Dateien beschrieben werden.

\todo[inline]{@Peter}

\subsection{Architektur}

\todo[inline]{
	* Architecture Description Language
}

Die Architektur (kurz: \textit{Arch}) kümmert sich um die Ausführung der
Assembler Programme und wurde vor dem Hintergrund entworfen, einen möglichst
Architektur-unabhängigen Simulator zu entwickeln. So stellt sie Klassen zur
Verfügung, mit der eine konkrete Architektur (wie z.B. RISC-V) umgesetzt
werden kann. Eine Architektur wird in \texttt{YAML} Konfigurationsdateien
beschrieben, und anschließend in das Programm geladen. \\
Da wir uns gegen das Disassemblieren von Maschinencode zur Ausführung eines
Programms entschieden haben, bietet die Architektur eine abstraktere Darstellung
von Assembler Programmen in Form eines Syntax Baums. Des Weiteren ist die
Architektur für die Validierung einer Instruktion zuständig, und stellt bei
Fehlern entsprechende Nachrichten zur Verfügung, die dem Nutzer helfen sollen,
das Problem zu beheben.

Eine Übersicht über die Schnittstellen des Architekturmoduls bietet
\autoref{fig:arch-overview} (Die verwendeten Klassennahmen stimmen nicht mit denen
im Quellcode überein, es geht hier nur um das Prinzip).

\begin{figure}[H]
	\begin{center}
		\begin{tikzpicture}[node distance=3.0cm]
		\tikzstyle{class} = [rectangle, rounded corners, draw=black, drop shadow, fill=white]
		\tikzstyle{myarrow} = [->, thick]

		\node (architecture) [class, rectangle split, rectangle split parts=2]
		{
			\textbf{Architecture}
			\nodepart{second}
			\begin{tabular}{c}
				getName() \\
				getEndianness() \\
				getWordSize() \\
				\ldots
			\end{tabular}
		};
		\node (factory) [class, right = of architecture]
		{
			\textbf{NodeFactory}
		};
		\node (syntaxtree) [class, rectangle split, rectangle split parts=2, right = of factory]
		{
			\textbf{SyntaxTreeNode}
			\nodepart{second}
			\begin{tabular}{c}
				validate() \\
				getValue() \\
				\ldots
			\end{tabular}
		};
		\draw[myarrow] (architecture) edge node [yshift=2mm] {getFactory()} (factory);
		\draw[myarrow] (factory) edge node [yshift=2mm] {create()} (syntaxtree);
		\end{tikzpicture}
	\end{center}
	\caption{Übersicht der Architekturschnittstelle}
	\label{fig:arch-overview}
\end{figure}

Die Architektur ist in einen allgemeinen Teil (im Folgenden \textbf{common})
genannt) und in einen Architektur-spezifischen Teil (benannt nach der
entsprechenden Architektur, z.B. \textbf{riscv} aufgeteilt. Die Schnittstelle
ist so konzeptioniert, dass andere Module nichts von der konkreten Architektur
mitbekommen, und lässt sich folgendermaßen charakterisieren: \\ Das
\textit{Architecture} Objekt stellt allgemeine Informationen über die geladene
Architektur zur Verfügung (wie z.B. der Name, die Byte-Reihenfolge oder die
Wortgröße) und bietet zusätzlich Zugang zur \textit{NodeFactory}. Mit der
NodeFactory werden die Syntax Bäume in Form von \textit{SyntaxTreeNode} Objekten
erzeugt, welche anschließend validiert und ausgeführt werden können. \\ Im
Folgenden wird auf diese Klassen genauer eingegangen.

\subsubsection{Architecture Description Language}

Eine Architektur kann in \texttt{YAML} Dateien beschrieben werden.

\todo[inline]{@Peter}

\subsubsection{Architecture}

Ein \texttt{Architecture} Objekt repräsentiert eine geladene Architektur im
Simulator. Der Grundgedanke bei der Entwicklung war das modulare Design von
RISC-V zu unterstützen, dieses Konzept lässt sich aber leicht auf andere
Prozessorarchitekturen übertragen. So wird angenommen, dass eine Architektur
immer aus einem Basismodul besteht (bei RISC-V: \textit{RV32I}), welches wieder
durch andere Module erweitert werden kann (bei RISC-V z.B.
Multiplikation/Division, Floating Point, etc.). \\ Um also ein Architecture
Objekt zu erzeugen, wird im ersten Schritt eine \textit{ArchitectureFormula}
erstellt, die das Basismodul und alle Erweiterungen spezifiziert. Ein Aufruf von
\textit{Brew()} löst die Abhängigkeiten auf und liest anschließend die
Informationen aus den Konfigurationsdateien ein.

Intern wird das \textit{Builder Pattern} zum sukzessivem Aufbau des Objekts
verwendet. Als Oberklasse dient dazu das \texttt{BuilderInterface}, von dem alle
Komponenten abgeleitet werden. Da eine Architektur lediglich aus strukturierten
Informationen besteht, existiert eine weitere Oberklasse
\texttt{InformationInterface}. Von ihr sind wiederum die konkreten
Informationsklassen abgeleitet. Nennenswert dabei ist die
\texttt{ExtensionInformation} Klasse, die ein Modul in unserer modularen
Auffassung einer Architektur widerspiegelt. Ein finales \texttt{Architecture}
Modul besteht schließlich aus einem ExtensionInformation Objekt, welches mit
allen Erweiterungen vereint wurde.

Das Architecture Objekt beschreibt lediglich die Eigenschaften einer
Architektur, d.h. es stellt Informationen wie die Byte-Reihenfolge, Wortgröße,
Speicherausrichtung oder die Eigenschaften der Register bereit. Es beschreibt
auch, welche Instruktionen zur Verfügung stehen, definiert aber nicht, wie diese
konkret implementiert sind, da dies für eine Konfigurationsdatei zu komplex
wäre. Die Implementation der einzelnen Instruktionen wird deshalb in C++ Code
übernommen. Wie die Verbindung zwischen der Beschreibung der Architektur und der
Implementation abläuft, wird im Folgenden erläutert.

\subsubsection{Syntax Tree}

Der Syntax Baum mit durch Vererbung implementiert und in \textit{common} und
\textit{riscv} (oder jede beliebige andere Architektur) gegliedert. Eine
Übersicht bietet \autoref{fig:arch-syntax-tree}

\begin{figure}[H]
	\begin{center}
		\begin{tikzpicture}[node distance=0.5cm and 1.5cm]

		\node (super) [class] {AbstractSyntaxTreeNode};
		\node (invisible) [right = of super] {};
		\node (imm) [class, below = of invisible] {ImmediateNode};
		\node (bin) [class, below = of imm] {BinaryDataNode};
		\node (reg) [class, below = of bin] {AbstractRegisterNode};
		\node (instr) [class, below = of reg] {AbstractInstructionNode};
		\node (common) [below = of instr] {\textbf{common}};
		\node (reg-riscv) [class, right = of reg] {riscv::RegisterNode};
		\node (instr-riscv) [class, right = of instr] {riscv::AbstractInstructionNode};
		\node (riscv) [below = of instr-riscv] {\textbf{riscv}};

		\draw[inheritance-arrow] (imm) -- (imm -| super) -- (super);
		\draw[inheritance-arrow] (bin) -- (bin -| super) -- (super);
		\draw[inheritance-arrow] (instr) -- (instr -| super) -- (super);
		\draw[inheritance-arrow] (reg) -- (reg -| super) -- (super);

		\draw[inheritance-arrow] (instr-riscv) -- (instr);
		\draw[inheritance-arrow] (reg-riscv) -- (reg);

		\begin{pgfonlayer}{background}
		\path (super.west |- super.north)+(-0.5,0.5) node (a1) {};
		\path (instr.east |- instr.south)+(+0.5,-1.5) node (a2) {};
		\path[rounded corners, draw=black!50, dashed] (a1) rectangle (a2);
		\end{pgfonlayer}

		\begin{pgfonlayer}{background}
		\path (reg-riscv.west |- reg-riscv.north)+(-0.5,0.5) node (a1) {};
		\path (instr-riscv.east |- instr-riscv.south)+(+0.5,-1.5) node (a2) {};
		\path[rounded corners, draw=black!50, dashed] (a1) rectangle (a2);
		\end{pgfonlayer}
		\end{tikzpicture}
	\end{center}
	\caption{Klassendiagramm Syntax Baum}
	\label{fig:arch-syntax-tree}
\end{figure}

\label{module-arch-ast-node-types}
Die aufgeführten Klassen haben folgende Funktion:
\begin{itemize}

  \item \textbf{AbstractSyntaxTreeNode} ist die Oberklasse jedes Syntax Knotens
  und definiert, welche Methoden die Unterklassen implementieren müssen. Des
  Weiteren enthält sie eine Liste an etwaige Kindknoten.

  \item \textbf{ImmediateNode} repräsentiert einen \textit{Immediate}-Wert, also
  einen Wert, der direkt im Assembler Quelltext angegeben ist. Architekturen
  stellen im Allgemeinen keine Spezialisierung eines Immediate Wertes zur
  Verfügung, weshalb diese Klasse vollständig im common Teil implementiert
  werden kann.

  \item \textbf{BinaryDataNode} enthält binäre Daten, wie z.B. Text Nachrichten.
  Konkret wird er für die Implementierung der Crash Instruktion verwendet.

	\item \textbf{AbstractRegisterNode} repräsentiert ein Register in der
	Instruktion. In RISC-V muss die Assemblierung speziell behandelt werden,
	weshalb es einen speziellen RegisterNode für RISC-V gibt.

	\item \textbf{AbstractInstructionNode} ist die oberste Ebene eines jeden
	Syntax Baums und repräsentiert die auszuführende Instruktion.

\end{itemize}

Mit diesen Knotentypen lassen sich alle RISC-V Instruktionen modellieren. Nicht
RISC Architekturen stellen aber häufig die Möglichkeit bereit, einen
Speicherzugriff während einer anderen Instruktion durchzuführen. Ein Beispiel in
x86:

\begin{x86}
add eax, [ebx*2+2]
\end{x86}

Um diese Instruktionen modellieren zu können, wurden folgende Knotentypen
konzeptioniert, die jedoch nicht im Quellcode definiert sind:
\begin{itemize}
	\item \textbf{MemoryAccessNode} repräsentiert einen Inline Speicherzugriff,
	im oben aufgeführten Beispiel also der Inhalt der eckigen Klammern.
	\item \textbf{ArithmeticNode} stellt eine arithmetische Operation dar, im
	obigen Beispiel also sowohl die Multiplikation, als auch die Addition.
\end{itemize}

Die Oberklasse jedes Knotens ist die Klasse \texttt{AbstractSyntaxTreeNode}.
Folgende Methoden werden von ihr vorgegeben:

\begin{itemize}
	\item \textbf{\texttt{getValue(MemoryAccess\&)}}: Diese Methode führt den
	darunter liegenden Syntax Baum aus, und ruft ggf. rekursiv dieselbe Methode
	bei den Kindknoten auf. Je nach Knotentyp variiert der Rückgabewert: So gibt
	beispielsweise ein Instruktionsknoten die Adresse der nächsten Instruktion
	zurück, ein Register seinen aktuellen Wert und ein Immediateknoten die
	abgespeicherte Konstante. Als Parameter wird eine Zugriffsmöglichkeit auf den
	Speicher übergeben, mit dem die Instruktion z.B. das Resultat der Operation
	abspeichern kann. \\
	Mit Diesem Konzept werden Codeduplikate verhindert, da z.B. eine arithmetische
	Operation, die sowohl mit Registern, als auch mit Immediate Werten arbeiten
	kann, nur einmal implementiert werden muss. In RISC-V kann man dieses Konzept
	beispielsweise auf \textit{add} und \textit{addi} anwenden.

	\item \textbf{\texttt{validate(MemoryAccess\&)}}: Während der Parser lediglich
	eine syntaktische Überprüfung des Assembler Quelltext vornimmt, validiert
	diese Methode die semantische Korrektheit einer Instruktion. Es wird zum
	Beispiel überprüft, ob der richtige Typ und die korrekte Anzahl an Operanden
	übergeben wurde, oder ob der übergebene Immediate Werte in die vorgegebenen
	Anzahl an Bits passen. War die Validierung nicht erfolgreich, so wird eine
	übersetzbare Fehlermeldung zurückgegeben.

	\item \textbf{\texttt{validateRuntime(MemoryAccess\&)}}: Validiert, ob eine
	Instruktion zur Laufzeit ausgeführt werden kann. Die Methode wird vor allem
	für Sprunginstruktionen benötigt, sodass geprüft werden kann, ob das Ziel
	des Sprungs innerhalb des zur Verfügung stehenden Programms liegt. Des
	Weiteren lässt es sich so verhindert, dass geschützte Speicherbereiche
	von Store Instruktionen beschrieben werden.

	\item \textbf{\texttt{assemble()}}: Wandelt einen Syntax Baum in die
	Binärdarstellung der Architektur um. Diese Darstellung ist lediglich zur
	Visualisierung für den Benutzer vorgesehen, die eigentliche Simulation der
	Instruktionen wird über den Syntax Baum vorgenommen.

	\item \textbf{\texttt{getIdentifier()}}: Gibt den Typ eines Knotens als
	Zeichenkette zurück. Beispielsweise geben Instruktionen ihren entsprechenden
	Mnemonic (z.B. \textit{addi}) und Register ihren Namen (z.B. \textit{x1})
	zurück. Letzteres wird verwendet, um Schreibzugriffe auf ein Register
	in einem Instruktionsknoten durchzuführen.
\end{itemize}

Die Aufgabe einer konkreten Architektur besteht nun darin, die eben beschriebenen
Unterklassen inklusive ihrer Methoden entsprechend zu implementieren. In
\autoref{sec:extension} wird darauf genauer eingegangen. \\
Um nun Objekte des Syntax Baums zu erzeugen, wird die Node Factory benötigt, auf
die im Folgenden eingegangen wird.

\subsubsection{Node Factory}
\label{module-arch-node-factory}

Mit der \textit{Node Factory} wird der bereits beschriebene Syntax Baum erzeugt.
Um nach außen ein Architektur-unabhängiges Interface zu bieten, basiert die
Node Factory dem \textit{Abstract Factory Pattern} und wird als Diagramm in
\autoref{fig:arch-node-factory} dargestellt.

\begin{figure}[H]
	\begin{center}
		\begin{tikzpicture}[node distance=0.5cm and 3cm]
		\pgfdeclarelayer{background}
		\pgfdeclarelayer{foreground}
		\pgfsetlayers{background,main,foreground}
		\tikzstyle{class} = [rectangle, rounded corners, draw=black, fill=white, drop shadow]
		\tikzstyle{inheritance-arrow} = [->, thick,>=open triangle 90]

		\node (instr-abstr) [class, anchor=west] {AbstractInstructionNodeFactory};
		\node (imm-abstr) [class, below = of instr-abstr] {AbstractImmediateNodeFactory};
		\node (reg-abstr) [class, below = of imm-abstr]	{AbstractRegisterNodeFactory};
		\node (data-abstr) [class, below = of reg-abstr] {AbstractDataNodeFactory};
		\node (mem-abstr) [class, below = of data-abstr] {AbstractMemoryAccessNodeFactory};
		\node (arithmetic-abstr) [class, below = of mem-abstr] {AbstractArithmeticNodeFactory};
		\node (common) [below = of arithmetic-abstr] {\textbf{common}};

		\node (instr-riscv) [class, right = of instr-abstr] {riscv::InstructionNodeFactory};
		\node (imm-riscv) [class, below = of instr-riscv] {riscv::ImmediateNodeFactory};
		\node (reg-riscv) [class, below = of imm-riscv] {riscv::RegisterNodeFactory};
		\node (data-riscv) [class, below = of reg-riscv] {riscv::DataNodeFactory};
		\node (riscv) [below = of data-riscv] {\textbf{riscv}};

		\draw[inheritance-arrow] (instr-riscv) edge (instr-abstr);
		\draw[inheritance-arrow] (imm-riscv) edge (imm-abstr);
		\draw[inheritance-arrow] (reg-riscv) edge (reg-abstr);
		\draw[inheritance-arrow] (data-riscv) edge (data-abstr);

		\begin{pgfonlayer}{background}
		\path (instr-abstr.west |- instr-abstr.north)+(-1,0.5) node (a1) {};
		\path (arithmetic-abstr.east |- arithmetic-abstr.south)+(+1,-1.5) node (a2) {};
		\path[rounded corners, draw=black!50, dashed] (a1) rectangle (a2);
		\end{pgfonlayer}

		\begin{pgfonlayer}{background}
		\path (instr-riscv.west |- instr-riscv.north)+(-0.5,0.5) node (a1) {};
		\path (data-riscv.east |- data-riscv.south)+(+1,-1.5) node (a2) {};
		\path[rounded corners, draw=black!50, dashed] (a1) rectangle (a2);
		\end{pgfonlayer}
		\end{tikzpicture}
	\end{center}
	\caption{Klassendiagramm Node Factory}
	\label{fig:arch-node-factory}
\end{figure}

Wie man erkennen kann, existiert für jeden Knotentyp eine eigene Factory. Dies
dient der Übersicht, da vor allem die Implementation der
\texttt{InstructionNodeFactory} bei Architekturen mit vielen Instruktionen
schnell unübersichtlich werden kann. Um zu verhindern, dass andere Module
mehrere Node Factory Objekte verwalten müssen, wird eine
\texttt{NodeFactoryCollection} über das Architecture Objekt zur Verfügung
gestellt, welches die einzelnen Factory Objekte kapselt und die
\texttt{create()} Aufrufe weiterleitet. \\
Die Factory Methoden geben einen \texttt{std::shared\_ptr} auf den erzeugten
Knoten zurück.

Des Weiteren fällt auf, dass RISC-V die beiden letzten Node Factories nicht
implementiert. Der Grund dafür wurde in der vorherigen Sektion beschrieben:
RISC-V unterstützt keine Inline Speicherzugriffe. Architekturen können durch das
Fehlen einer Factory signalisieren, dass sie einen Knotentyp nicht unterstützen.

\subsubsection{RISC-V}

Bisher wurde fast ausschließlich die Schnittstelle beschrieben. Diese Sektion
soll einen Einblick in unsere Gedanken bei der RISC-V Implementation geben.

Zunächst etwas zur Implementation der Instruktionen. RISC-V definiert wiederum
eine eigene abstrakte Oberklasse \texttt{riscv::InstructionNode}, von der alle
weiteren Instruktionsknoten abgeleitet sind. Dies dient der Vermeidung von
Redundanz, da Methoden wie \texttt{assemble()} und
\texttt{getInstructionDocumentation()} für alle Instruktionen angewandt werden
können. Des Weiteren definiert die Klasse hilfreiche Methoden, die in den
Unterklassen verwendet werden, um zum Beispiel die Validierung der Operanden
einer Instruktion zu vereinfachen.

Die Knoten, die Instruktionen implementieren, sind ebenfalls darauf ausgelegt,
Redundanz zu vermeiden. Beispielhaft seien hier die \textit{Integer
Computational Instructions} (so der Name in der RISC-V Spezifikation)
herangezogen (das sind Instruktionen wie \textit{add}, \textit{addi} oder
\textit{and}). Da sich der Aufbau der Instruktionen lediglich in der
auszuführenden Operation unterscheidet, existiert eine weitere abstrakte
Oberklasse \texttt{riscv::AbstractIntegerInstructionNode}, die all jene
Instruktionen abdeckt. In dieser Oberklasse wird die Validierung der
Instruktionen vollständig behandelt und der Aufruf von \texttt{getValue()} so
weit abstrahiert, dass die konkreten Anwendung der Operation effektiv in einen
einzeiligen Lamda Ausdruck reduziert werden kann. Die Implementierung befindet
sich in der Datei \texttt{integer-instructions.hpp}.

Ein weiteres Konzept der RISC-V Instruktionen basiert auf der Unterstützung
unterschiedlicher Wortgrößen. Derzeit bietet RISC-V ausgiebige Unterstützung für
32 und 64 Bit, in Zukunft soll 128 Bit folgen. Um zu verhindern, dass
Instruktionen für jede Wortgröße neu geschrieben werden müssen, nutzen wir C++
Templates, um die Wortgröße einer Instruktion zu spezifizieren. So wird in der
Node Factory von RISC-V eine Fallunterscheidung nach der verwendeten Wortgröße
gemacht, und dann der entsprechende Zahlentyp als Template Parameter gesetzt
(für 32 Bit z.B. \texttt{std::uint32\_t} und für 64 Bit
\texttt{std::uint64\_t}). Sollte die Entwicklung der 128 Bit Version von RISC-V
voranschreiten, so könnte man das mit dem aktuellen C++ Standard nicht abdecken,
da kein 128 Bit Zahlentyp definiert ist. Man könnte dann aber eine vereinfachte
Implementation eines \texttt{uint128\_t} schreiben, indem man z.B. zwei
\texttt{uint64\_t} in einer Klasse kapselt.

\subsubsection{Weiterführende Dokumentation}

Diese vorherigen Sektionen geben einen Überblick über die Architektur.
Weiterführende Dokumentation findet sich in denen für die Architektur relevanten
Dateien, welche in \autoref{fig:arch-further} aufgelistet sind.

\begin{figure}[H]
	\begin{center}
	\begin{tikzpicture}[%
	grow via three points={one child at (0.8,-0.8) and
		two children at (0.8,-0.8) and (0.8,-1.7)},
	edge from parent path={($(\tikzparentnode\tikzparentanchor)+(.2cm,0pt)$) |- (\tikzchildnode\tikzchildanchor)},
	growth parent anchor=west,
	parent anchor=south west]
	\tikzstyle{every node}=[draw=black,anchor=west]
	\node {\erasim}
	child { node {isa/}
		child { node {riscv.isa/} }
	}
	child [missing] {}
	child { node {$\{\text{tests/}, \text{include/}, \text{source/}\}$}
		child { node {arch/}
			child { node {common/} }
			child { node {riscv/} }
		}
	}
	child [missing] {}
	child [missing] {}
	child [missing] {};
	\end{tikzpicture}
	\end{center}

	\caption{Relevante Dateien des Architekturmoduls}
	\label{fig:arch-further}
\end{figure}

\subsection{Parser}

\subsection{Core}

Beschreibung des Moduls + Schnittstellen zu anderen Modulen

\subsection{GUI}


\subsubsection{Aufbau}

Die grafische Benutzeroberfläche des Simulators besteht aus einer Reihe von Komponenten, die sich innerhalb des gegebenen Rasters nahezu beliebig anordnen lassen. Diese Aufteilung in Komponenten spiegelt sich auch in der Implementierung der GUI wieder, indem jede Komponente einer QML-Komponente entspricht, die gegebenenfalls andere Teilkomponenten, nicht zuletzt elementare Qt-Quick-Komponenten, benutzen. Die Komponenten umfassen den Editor, die sechs frei wählbaren Komponenten Snapshots, Output, Input, Register, Speicher und Hilfe, sowie übergeordnete Elemente wie den Split-View, die Projekt-Tabbar, die Toolbar, die Menubar und den Einstellungsdialog.

QML-Komponenten, die Zugriff auf das vom Core zur Verfügung gestellte Modell benötigen, werden zusätzlich mit einer C++-Klasse assoziiert. Diese Klassen halten in der Regel Instanzen der Interface-Klassen \texttt{MemoryAccess}, \texttt{MemoryManager}, \texttt{ArchitectureAccess}, \texttt{ParserInterface} oder \texttt{CommandInterface}  über die der Zugriff auf den Core unter Verwendung des Schedulers erfolgt.

Da das Qt-Framework einen in sich stark abgeschlossenen Aufbau besitzt, werden für QML-Komponenten mit komplexen Modellen, wie etwa der Speicher oder die Register, von Qt-Klassen abgeleitete Modelle benötigt. Da keine Abhängigkeiten vom Qt-Framework innerhalb der GUI-fernen Module Core, Parser, Arch entstehen sollen, kann diese Funktion nicht von Klassen des Cores übernommen werden. Aus diesem Grund übernehmen einige der zu den QML-Komponenten gehörigen C++-Klassen der GUI die Aufgabe des Modells und werden folglich von Qt-Modell-Klassen wie etwa \texttt{QAbstractItemModel} abgeleitet. Ausgenommen von Cache-Zwecken halten diese Modelle selbst keine Daten, sondern holen diese über das zugehörige Interface-Objekt, sobald Daten seitens der QML-Komponente angefordert werden.

\subsubsection{Kommunikation}

Der Simulator unterstützt das gleichzeitige Laden mehrerer unabhängiger Projekte. Diese werden jeweils durch eine \texttt{GUIProject}-Instanz repräsentiert, welches Komponenten-übergreifende Funktionalität für ein Projekt zur Verfügung stellt.

Im \texttt{GUIProject} werden mitunter die C++-Klassen der QML-Komponente gehalten, initialisiert und deren Kommunikation mit dem Core koordiniert.
Dieses übernimmt also die Rolle des Mittelsmann zwischen dem Core, der keine Qt-Mechanismen verwendet, und der GUI.

Bei der Initialisierung der C++-Klassen übergibt das \texttt{GUIProject} Instanzen der Interface-Klassen (bspw. \texttt{MemoryAccess} und \texttt{ArchitectureAccess}).

Datenaustausch ausgehend vom Core hin zu den QML-Komponenten in der GUI hat nicht den Scheduler zwischengeschaltet, sondern verläuft über das \texttt{GUIProject} in zwei Schritten. Dabei ruft der Core im Schritt einen Callback auf, der im \texttt{ProjectModule} gesetzt werden (näher Informationen im Abschnitt /Core/). Das \texttt{GUIProject} leitet im zweiten Schritt den im \texttt{GUIProject} eingehenden Callback mit Hilfe des Qt-eigenen Signal-Slot-Mechanismus an die Instanzen der C++-Klassen der QML-Komponenten weiter. Diese Umleitung ist mitunter deshalb notwendig, um die eingehende Nachricht, die im Thread des Cores gesendet wird, in den Main-Thread zu übertragen, der von der GUI verwendet wird.

Nachrichten, die von einer der C++-Klassen der Komponenten an den Core gesendet werden sollen, können mit Hilfe der Interface-Klassen in den Scheduler eingefügt werden, der diese an das korrespondierende Core-Objekt weiterleitet.

Damit die QML-Komponenten auf Methoden der zugehörigen C++-Klasse zugreifen können, werden deren Instanzen im \texttt{GUIProject} zum projektspezifischen \texttt{QQMLContext} hinzugefügt, beispielsweise die Klasse \texttt{RegisterModel} als Context-Property \texttt{registerModel}. Auf die Properties dieses Kontexts kann über die zugehörige Bezeichnung (bspw. \texttt{registerModel}) global in jeder QML-Datei zugegriffen werden. Die Trennung der Daten verschiedener Projekte wird gewährleistet, indem jedes \texttt{GUIProject} seinen eigenen Kontext erhält.

Eine Ausnahme von der regulären Kommunikation bildet die Toolbar, die anders als die übrigen QML-Komponenten, die als Kommunikationspartner auftreten, nicht für jedes Projekt einzeln existiert, sondern übergeordnet ist. Der Datenaustausch erfolgt deshalb nicht über das \texttt{GUIProject}, sondern über die \texttt{Ui}-Klasse, die den QML-Dateien über die Context-Property \texttt{ui} zur Verfügung gestellt wird. Da die \texttt{Ui}-Instanz Kenntnis über das aktive Projekt hat, können auch projektspezifische Informationen, etwa über den aktuellen Ausführungszustand, ausgetauscht werden.


\subsubsection{Initialisierung}

Nachdem das Programm in der main-Methode der Klasse \texttt{main.cpp} gestartet ist, wird eine Instanz der graphischen Benutzeroberfläche \texttt{Ui} erstellt.
