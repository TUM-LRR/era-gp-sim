% !TEX root = dev-manual.tex
% ERA-Großpraktikum: Entwickleranleitung -- Architekturmodul

\subsection{Architektur}

Das Architekturmodul (kurz: \emph{Arch}) kümmert sich um die Ausführung der
Assembler Programme und wurde vor dem Hintergrund entworfen, einen möglichst
Architektur-unabhängigen Simulator zu entwickeln. So stellt sie Klassen zur
Verfügung, mit der eine konkrete Architektur (z.B. RISC-V) umgesetzt werden
kann. Eine Architektur wird in \emph{YAML} Konfigurationsdateien beschrieben,
und anschließend in das Programm geladen. Da wir uns gegen das Disassemblieren
von Maschinencode zur Ausführung eines Programms entschieden haben, bietet die
Architektur eine abstraktere Darstellung von Assembler Programmen in Form eines
Syntaxbaums. Des Weiteren ist die Architektur für die Validierung einer
Instruktion zuständig, und stellt bei Fehlern entsprechende Nachrichten zur
Verfügung, die dem Nutzer helfen sollen, das Problem zu beheben.

Eine Übersicht über die Schnittstellen des Architekturmoduls bietet
\autoref{fig:arch-overview} (Die verwendeten Klassennahmen stimmen nicht mit
denen im Quellcode überein, es geht hier nur um das Prinzip).

\begin{figure}[H]
	\begin{center}
		\begin{tikzpicture}[node distance=3.0cm]
		\tikzstyle{class} = [rectangle, rounded corners, draw=black, drop shadow, fill=white]
		\tikzstyle{myarrow} = [->, thick]

		\node (architecture) [class, rectangle split, rectangle split parts=2]
		{
			\textbf{Architecture}
			\nodepart{second}
			\begin{tabular}{c}
				getName()\\
				getEndianness()\\
				getWordSize()\\
				\ldots
			\end{tabular}
		};
		\node (factory) [class, right = of architecture]
		{
			\textbf{NodeFactory}
		};
		\node (syntaxtree) [class, rectangle split, rectangle split parts=2, right = of factory]
		{
			\textbf{SyntaxTreeNode}
			\nodepart{second}
			\begin{tabular}{c}
				validate()\\
				getValue()
				\ldots
			\end{tabular}
		};
		\draw[myarrow] (architecture) edge node [yshift=2mm] {getFactory()} (factory);
		\draw[myarrow] (factory) edge node [yshift=2mm] {create()} (syntaxtree);
		\end{tikzpicture}
	\end{center}
	\caption{Übersicht der Architekturschnittstelle}
	\label{fig:arch-overview}
\end{figure}

Die Architektur ist in einen allgemeinen Teil (im Folgenden \texttt{common})
genannt) und in einen Architektur-spezifischen Teil (benannt nach der
entsprechenden Architektur, z.B. \texttt{riscv} aufgeteilt. Die Schnittstelle
ist so konzipiert, dass andere Module nichts von der konkreten Architektur
mitbekommen, und lässt sich folgendermaßen charakterisieren: Das
\texttt{Architecture} Objekt stellt allgemeine Informationen über die geladene
Architektur zur Verfügung (wie z.B. der Name, die Byte-Reihenfolge oder die
Wortgröße) und bietet zusätzlich Zugang zur \texttt{NodeFactory}. Mit der
\texttt{NodeFactory} werden die Syntaxbäume in Form von \texttt{SyntaxTreeNode}
Objekten erzeugt, welche anschließend validiert und ausgeführt werden können. Im
Folgenden wird auf diese Klassen genauer eingegangen.

% !TEX root = dev-manual.tex
% ERA-Großpraktikum: Entwickleranleitung -- Architekturmodul (Architektur Design)

\subsubsection{\texttt{Architecture}}

Die \texttt{Architecture} Klasse repräsentiert eine vollständig beschriebene
Architektur, welche anderen Modulen im Simulator zur Verfügung steht und dessen
Attribute die abstrakt gehaltenen Klassen im Simulator zur Laufzeit
konkretisieren. Beispielsweise ist eine Architektur unter anderem durch ihre
Wortgröße definiert, welche die Größe der Register im Speichermodell des Cores
bestimmt. Ebenso enthält eine Architektur Informationen zu sämtlichen
Instruktionen des Befehlssatzes, welche dem Parser Bescheid geben, welche
Instruktionen existieren und welche Operanden für diese erlaubt sind.

Zur Modellierung sämtlicher bestehender Befehlssätze wie x86, ARM oder RISC-V,
aber auch beliebiger, zukünftiger und noch unbekannter Architekturen benötigten
wir gewiss ein flexibles Design. Unser hauptsächlicher Leitfaden bei der
Konzeption eines abstrakten Architekturinterfaces war das modulare Design von
RISC-V. Dieses besticht durch eine enorme Flexibilität, da es nicht nur einen,
monolithischen und rundherum vollständigen Befehlssatz definiert, sondern viele
kleine, spezifische \emph{Extensions}. Eine solche Extension kann hierbei neue
Instruktionen oder neue Registersätze definieren, ebenso aber auch einfach eine
Veränderung der Wortgröße, der Repräsentation eines Vorzeichens oder der
Endianness mit sich bringen. Nach diesem Prinzip ist auch in unserem Design eine
Architektur die Vereinigung mehrerer Extensions.

Neben der Gliederung einer Architektur in viele Extensions haben wir auch die
weitere Definition und Implementierung einer Extension offen und modular
gehalten. Genauer besteht eine Extension neben Attributen wie Endianness oder
Wortgröße noch aus zwei weiteren, möglicherweise leeren, Mengen: einer Menge von
\texttt{Unit}s und einer Menge von \texttt{Instruction}s. Eine \texttt{Unit}
repräsentiert hierbei einen Registersatz mit Namen, besteht also selbst im
Weiteren aus einer Menge von Registern, welche ebenso Attribute, Namen und
andere Eigenschaften besitzen können. In Praxis wären für eine Unit eine CPU,
mit Ganzzahl-Registern, oder eine FPU, mit Gleitkommazahl-Registern, als
Beispiele zu nennen. Instruktionen werden hingegen in einer Extension in einem
\texttt{InstructionSet} gesammelt, wobei Instruktionen selbst ein Format, einen
Mnemonic und einen Opcode spezifizieren. \autoref{fig:arch-design} fasst unser
Design einer Architektur zusammen.

Es sei noch angemerkt, dass wie bei RISC-V bestimmte Extensions einen speziellen
Sonderstatus, wenn sie \emph{vollständig} sind (im RISC-V Jargon dann als
\emph{Base-Extension} bezeichnet). Ein Modul ist dann vollständig, wenn es alle
nötigen Attribute und Eigenschaften besitzt, um eine eigenständige und
funktionsfähige Architektur zu bilden. Genauer definieren wir drei Anforderungen:
\begin{enumerate}
  \item Die Extension hat zumindest eine Instruktion,
  \item Die Extension hat zumindest eine nicht leere Unit (also ein Register),
  \item Die Extension definiert sämtliche Attribute wie Endianness oder Wortgröße.
\end{enumerate}

Ist eine Erweiterung selbst noch nicht vollständig, so muss sie mit einer oder
vielen weiteren Extensions vereinigt werden, was durch die \texttt{merge}
Methode der \texttt{ExtensionInformation} Klasse leicht zu realisieren ist.
Schlussendlich ist eine \texttt{Architecture} Instanz nur dann gültig, wenn sie
einer vollständigen Extension entspricht.

\autoref{dev:arch-yaml} geht nun genauer auf die Beschreibung einer Architektur
in YAML ein, während \autoref{dev:arch-impl} kurz die Implementierungsdetails
des \texttt{Architecture} und \texttt{Extension} Codes bespricht.

\begin{figure}[h!]
  \centering
  \scalebox{1.3}{
  \begin{tikzpicture}[thick]
    \tikzset{block/.style={%
      draw,%
      rectangle,%
      rounded corners,%
      text width=2cm,%
      text height=0.45cm}%
    };
    \tikzset{smallblock/.style={block, text width=1.5cm, text height=0.3cm}};

    %%%%%%%%%%%%%%%
    % Architektur %
    %%%%%%%%%%%%%%%
    \node [class] (arch) at (0, 0.2) {\texttt{Architektur}};

    %%%%%%%%%%%%%%
    % Extensions %
    %%%%%%%%%%%%%%
    \node [class] (ext) at (0, -1.2) {\texttt{Extension}};

    % Edge
    \draw [->] (arch) -- (ext)
          node [midway, right] {\footnotesize\texttt{1..N}};

    %%%%%%%%%%%%%%%%%%%%%%%%
    % Units (e.g. CPU/FPU) %
    %%%%%%%%%%%%%%%%%%%%%%%%
    \node [class] (units) at (-1.5, -3) {\texttt{Unit}};
    \path (ext)
          edge [->, bend right]
          node [pos=0.7, right] {\footnotesize\texttt{1..N}}
          (units);

    % Register
    \node [class] (reg) at (-1.5, -4.5) {\texttt{Register}};
    \draw [->] (units) -- (reg)
          node [midway, left] {\footnotesize\texttt{1..N}};

    \node (rname) at (-2.3, -5.5) {\scriptsize Name};
    \node (rtype) at (-1.5, -5.55) {\scriptsize Typ};
    \node (rwidth) at (-0.8, -5.5) {\scriptsize Breite};

    \draw [->, semithick] (reg) -- (rname);
    \draw [->, semithick] (reg) -- (rtype);
    \draw [->, semithick] (reg) -- (rwidth);

    %%%%%%%%%%%%%%%%%%%
    % Instruction Set %
    %%%%%%%%%%%%%%%%%%%
    \node [class] (is) at (1.5, -3) {\texttt{Instruktionssatz}};
    \path (ext)
          edge [->, bend left]
          node [pos=0.7, left] {\footnotesize\texttt{1}}
          (is);

    % Instruktionen
    \node [class] (inst) at (1.5, -4.5) {\texttt{Instruktion}};
    \draw [->] (is) -- (inst) node [midway, right] {\footnotesize\texttt{1..N}};

    \node (iname) at (0.8, -5.5) {\scriptsize Name};
    \node (ikey) at (1.5, -5.55) {\scriptsize Key};
    \node (iformat) at (2.3, -5.5) {\scriptsize Format};

    \draw [->, semithick] (inst) -- (iname);
    \draw [->, semithick] (inst) -- (ikey);
    \draw [->, semithick] (inst) -- (iformat);

    %%%%%%%%%%%%%%
    % Attributes %
    %%%%%%%%%%%%%%
    \draw [->, semithick] (ext) -- (2, -0.8)
          node [right] {\footnotesize Wortgröße};

    \draw [->, semithick] (ext) -- (2, -1.6)
          node [right]{\footnotesize Endianness};

    \draw [->, semithick] (ext) -- (-2, -0.8)
          node [left] {\footnotesize Datentypen};

    \draw [->, semithick] (ext) -- (-2, -1.6)
          node [left] {\footnotesize Signed Rep.}; {\texttt{1}};;
  \end{tikzpicture}
  }
  \caption{Der modulare Aufbau einer Architektur in \erasim{}. Eine Architektur besteht in unserem Modell aus \emph{Extensions}. Eine Extension besteht dann im Weiteren aus \emph{Units} und einem \emph{InstructionSet}. Ersteres repräsentiert einen Registersatz, letzteres sammelt die Instruktionen eines Befehlssatzes. Sowohl eine Extension als auch weitere Akteure in unserem Design besitzen schließlich noch Attribute wie eine Wortgröße, Registergröße oder einen menschenlesbaren Namen.}
  \label{fig:arch-design}
  \vspace{-0.2cm}
\end{figure}

% !TEX root = dev-manual.tex
% ERA-Großpraktikum: Entwickleranleitung -- Architekturmodul (ISA Dateien)

\paragraph{Beschreibung einer Architektur in YAML}
\label{dev:arch-yaml}

Um die Flexibilität, die Wartbarkeit und die Sauberkeit unserer
Architekturimplementierung zu gewährleisten, haben wir uns entschieden, die
konkrete Beschreibung einer Architektur (die "Daten") von der abstrakten
Repräsentation im Code zu trennen. Genauer bedeutet das, dass die eben
beschriebenen Informationsklassen im Architekturmodul lediglich die Struktur
eines Befehlssatzes vorgeben und Klassen wie \texttt{RegisterInformation} oder
\texttt{InstructionInformation} Platzhalter für konkrete Architekturen
darstellen. Die eigentliche Spezifikation von Registernamen und -größen,
Instruktionsformaten und -operanden sowie Wortgröße, Endianness oder
Vorzeichenrepräsentation einer Architektur erfolgen in unserer Implementierung
über externe \emph{ISA}-Dateien. ISA-Dateien werden aufgrund seiner intuitiven
und menschenlesbaren Syntax im YAML\footnote{\url{yaml.org}} Format definiert
und von diesem für die eigentliche Verarbeitung des Simulators schlussendlich
nach JSON konvertiert\footnote{Für die YAML $\rightarrow$ JSON Konvertierung kann das \texttt{y2j.py} Skript im \texttt{scripts/} Verzeichnis genutzt werden.}.

Eine Architekturbeschreibung folgt bezüglich seiner Struktur im Dateisystem
einem einfachen Schema: Eine Architektur ist ein Ordner mit einer \texttt{.isa}
Endung, beispielsweise \texttt{riscv.isa}. Dieser Ordner enthält für jede
definierte \emph{Extension}, sei es \texttt{rv32f} oder \texttt{rv64a}, einen
entsprechend benannten Unterordner. Jeder dieser Unterordner besteht dann aus
zwei weiteren Dateien: \texttt{config.yaml} und \texttt{config.json}. Ersteres
ist die Datei, welche Entwickler bearbeiten sollten, um die entsprechende
Extension zu definieren; letzteres sollte lediglich das Resultat einer
maschinellen Konvertierung sein. \autoref{lst:isa} zeigt die Definition zweier
Extensions einer fiktiven Architektur.

\begin{figure}[h!]
  \begin{minipage}{0.5\textwidth}
    \begin{lstlisting}[language=ISA]
      # Base Extension

      name: base
      word-size: 32
      endianness: mixed
      signed-representation: sign-bit

      units:
        - name: cpu
          registers:
            - name: r1
              id: 1
              size: 31
              type: float
    \end{lstlisting}
    \vspace{2.3cm}
  \end{minipage}
  \begin{minipage}{0.5\textwidth}
    \begin{lstlisting}[language=ISA, frame=l]
      # Derived Extension

      name: derived
      extends: base
      alignment-behavior: relaxed

      instructions:
        - mnemonic: add
          length: 32
          format: R
          key:
            opcode: 42
            funct: 6

      builtin-macros:
        - |
          .macro nop
          add x0, x0, 0
          .endm
    \end{lstlisting}
  \end{minipage}
  \begin{lstlisting}[label={lst:isa}, caption={Zwei Beispiele einer ISA-Definition in YAML Syntax. Die linke Definition zeit eine \emph{unvollständige} Extension, welche lediglich eine Unit, aber keine Instruktionen definiert. Die rechte Definition erbt die linke und erweitert sie durch einen Instruktionssatz und einem Alignment-Attribut. Die rechte Extension ist somit vollständig.}]
  \end{lstlisting}
  \vspace{-1cm}
\end{figure}

Grundsätzlich entsprechen die Namen der Schlüssel in diesen Dateien genau den
Attributen der Informationsklassen der Architektur. Beispielsweise hat jede
Extension einen Namen, welcher dann über die \texttt{name()} Methode einer
\texttt{ExtensionInformation} Instanz abgerufen werden kann. Eine vollständige
Auflistung aller möglichen Schlüsselnamen und erlaubter Werte findet sich in
Tabellen \autoref{tbl:isa-top-keys}, \autoref{tbl:isa-reg} und
\autoref{tbl:isa-inst}. Die folgenden zwei Absätze beleuchten die Schlüssel
\texttt{extends} und \texttt{builtin-macros} näher.

\begin{table}[p]
  \centering
  \small
  \begin{tabular}{>{\ttfamily}l p{8.3cm} >{\ttfamily}p{3.5cm}}
    {\normalfont\bfseries Schlüssel} & \textbf{Beschreibung} & \textbf{Erlaubte Werte}\\
    \toprule
    name & Name der Extension & <string> \\

    extends & Nam(en) der Basis-Extension(s) & <string>/list<string>\\

    reset-units & Kontrolliert Registerübernahme bei Erweiterung & true/false\\

    reset-instructions & Kontrolliert Instruktionsübernahme bei Erweiterung & true/false\\

    word-size & Die Wortgröße & <number>\\

    byte-size & Die Bytegröße & <number>\\

    alignment-behavior & Verhalten bei unausgerichteten Speicherzugriffen & \texttt{relaxed/strict}\\

    signed-representation & Vorzeichendarstellung & \texttt{ones-complement/} \texttt{twos-complement/} sign-bit \\

    endianness & Speicherausrichtung & bi/big/little/mixed \\

    builtin-macros & Liste vordefinierter Makros & list<string> \\

    units & Die Units der Extension & list<unit> \\

    instructions & Der Instruktionssatz der Extension & list<instruction>
  \end{tabular}
  \caption{Die erlaubten Schlüssel und Werte zur Beschreibung einer Extension im YAML Format. Die erste Spalte nennt den jeweiligen Schlüsselnamen, die zweite Spalte beschreibt diesen und die dritte Spalte gibt die erlaubten Werte an. Mit \texttt{<string>} ist eine beliebige Zeichenkette gemeint, mit \texttt{<number>} eine beliebige Ganzzahl und mit \texttt{list<type>} eine Liste mit Elementen vom Typ \texttt{<type>}.}
  \label{tbl:isa-top-keys}

  \vspace{0.7cm}

  \begin{tabular}{>{\ttfamily}l p{8.5cm} >{\ttfamily}p{3.5cm}}
    {\normalfont\bfseries Schlüssel} & \textbf{Beschreibung} & \textbf{Erlaubte Werte}\\
    \toprule

    name & Name des Registers & <string>\\

    id & Eindeutiger ID des Registers & <number>\\

    alias & Mögliche Aliase des Registers & <string>/list<string>\\

    size & Größe des Registers & <number> \\

    constant & Fixer Wert des Registers (Optional) & <string> (0x0)
  \end{tabular}
  \caption{Die Schlüssel und erlaubten Werte zur Spezifikation von Registern in ISA-Dateien. Das Format folgt jenem von \autoref{tbl:isa-top-keys}.}
  \label{tbl:isa-reg}

  \vspace{0.7cm}

  \begin{tabular}{>{\ttfamily}l p{8.5cm} >{\ttfamily}p{3.5cm}}
    {\normalfont\bfseries Schlüssel} & \textbf{Beschreibung} & \textbf{Erlaubte Werte}\\
    \toprule
    mnemonic & Kürzel der Instruktion & <string>\\

    length & Länge der Instruktion in Bits & <number>\\

    format & Format der Instruktion & <string> \\

    operand length & Länge der Operanden in Bits & list<number>\\

    key & Instruktionsschlüssel (Generalisierung des Opcodes) & map<key, number>

  \end{tabular}
  \caption{Die Schlüssel und erlaubten Werte zur Spezifikation von Instruktionen in ISA-Dateien. Das Format folgt jenem von \autoref{tbl:isa-top-keys}, bis auf \texttt{key}, welches eine Abbildung von Teilen des Instruktionsschlüssels auf die konkreten Werte dieser Teile ist.}
  \label{tbl:isa-inst}
\end{table}

Das Schlüsselwort \texttt{extends} stellt die mächtigste Funktionalität der
ISA-Spezifikationssprache dar: die Möglichkeit, Extensions von anderen erben zu
lassen. Das Konzept der Attributvererbung ist in vielen Situationen nicht nur
intuitiv, sondern schützt insbesondere vor Duplikation und spart Schreibarbeit.
Als anschauliches Beispiel lässt sich die Beziehung zwischen den \emph{RV32I}
und \emph{RV64I} Extensions des RISC-V Befehlssatzes betrachten. Letztere
Extension basiert laut Spezifikation auf der ersteren. Das bedeutet, dass RV64I
sämtliche Instruktionen und Registersätze besitzt, die in RV32I spezifiziert
sind. Zusätzlich erhöht RV64I jedoch noch die Wortgröße von 32 auf 64 Bit und
fügt weitere Instruktionen hinzu, die auf 32-Bit Werten operieren (da die
"geerbten" Instruktionen in RV64I implizit mit Wortgröße, also 64 Bit,
arbeiten). Diese Beziehung lässt sich in ISA-Dateien exakt spezifizieren. Ein
Ausschnitt der RV64I Spezifikation wird hierzu in \autoref{lst:rv64i} gezeigt.

Neben dieser einfachen Vererbungshierarchie zwischen RV64I und RV32I sind noch
weitere beliebig komplexe Beziehungen möglich, sofern diese sich in einem
\emph{azyklischen} Graphen darstellen lassen. Beim Laden einer Architektur über
eine \texttt{ArchitectureFormula}, also einer Liste der Namen aller Extensions,
wird über einen Graphalgorithmus die transitive Hülle sämtlicher benötigter
Erweiterungen berechnet. Steht diese Menge von Extensions fest, kann sie in eine
einzige vereint werden, wobei einfach die Vereinigung sämtlicher Attribute,
Register- und Instruktionssätze gebildet wird. Sofern im Beziehungsgraphen keine
Zyklen gefunden wurden und die resultierende Extension vollständig ist, entsteht
so eine fertige Architektur.

\begin{table}
\begin{lstlisting}[language=ISA]
                                    # RISC-V: RV64I

                                    name: rv64i
                                    extends: rv32i
                                    word-size: 64

                                    instructions:
                                      - mnemonic: addiw
                                        length: 32
                                        format: I
                                        operand length: [0, 0, 12]
                                        key:
                                          opcode: 27
                                          funct3: 0
\end{lstlisting}
\begin{lstlisting}[caption={Ein Ausschnitt der Definition der RV64I-Extension des RISC-V Befehlssatzes. Zur Vererbung der Attribute von RV32I benötigt es lediglich der Spezifikation der \texttt{extends} Klausel. Zusätzlich wird die Wortgröße sowie eine Hand voll weiterer Befehle neu definiert.}, label={lst:rv64i}]
\end{lstlisting}
\end{table}

Neben dem \texttt{extends} Schlüssel ist letztlich noch \texttt{builtin-macros}
hervorzuheben. Dieser Schlüssel erwartet eine Liste von Strings, welche jeweils
ein Makro definieren, das für eine Architektur schon vordefiniert ist. Ein
solches vordefiniertes Makro kann hierbei auch \emph{Pseudoinstruktion} genannt
werden, da es dem Benutzer wie eine "eingebaute" Instruktion erscheint, jedoch
nicht standardmäßig zum Befehlssatz gehört und auch nicht in C++ implementiert
wird. Das Format eines solchen eingebauten Makros ist ident zu einem benutzerdefinierten Makro.

% !TEX root = dev-manual.tex
% ERA-Großpraktikum: Entwickleranleitung -- Architekturmodul (Architektur Implementierung)

\paragraph{Implementierungdetails der Architekturrepräsentation}
\label{dev:arch-impl}

Für die Implementierung von Architektur und Extensions wird intern das
\emph{Builder} Pattern\ zum sukzessivem Aufbau eines Objekts verwendet. Als
Oberklasse dient dazu das \texttt{BuilderInterface}, von dem alle Komponenten
abgeleitet werden. Da eine Architektur lediglich aus strukturierten
Informationen besteht, existiert eine weitere Oberklasse
\texttt{InformationInterface}. Von ihr sind wiederum die konkreten
Informationsklassen abgeleitet, welche Register, Instruktionen oder Datentypen
beschreiben. Nennenswert dabei ist die \texttt{ExtensionInformation} Klasse, die
in unserer modularen Auffassung eine Erweiterung einer Architektur
widerspiegelt. Eine finale \texttt{Architecture} Klasse besteht schließlich aus
einem \texttt{ExtensionInformation} Objekt, welches mit allen Erweiterungen
vereint wurde (also \texttt{vollständig} ist). \autoref{fig:arch-impl} zeigt die
Vererbungshierarchie für Informationsobjekte in der Architektur.

\begin{figure}[h!]
  \centering
  \begin{tikzpicture}[thick, node distance=1cm and 0.5cm]
    % Classes
    \node [class] (builder) {\texttt{BuilderInterface}};
    \node [class] (info) [below = of builder] {\texttt{InformationInterface}};
    \node [class] (inst) [below = of info] {\texttt{InstructionInformation}};
    \node [class] (reg) [left = of inst] {\texttt{RegisterInformation}};
    \node [class] (ext) [right = of inst] {\texttt{ExtensionInformation}};
    \coordinate (center) at ($ (inst) !.5! (info) $);

    % Edges
    \draw [inheritance-arrow] (info) -- (builder);
    \draw [inheritance-arrow] (inst) -- (info);
    \draw (reg) -- (reg |- center) -- (center);
    \draw (ext) -- (ext |- center) -- (center);

  \end{tikzpicture}
  \caption{Ein Ausschnitt der Vererbungshierarchie sämtlicher Informationsklassen der Architektur. Als Oberklasse dient das \texttt{BuilderInterface} sowie darunter das \texttt{InformationInterface}, von welchem weitere Informationsklassen wie \texttt{RegisterInformation} oder \texttt{Instructioninformation} erben.}
  \label{fig:arch-impl}
  \vspace{-0.5cm}
\end{figure}

Es ist zu beachten, dass das \texttt{Architecture} Objekt lediglich die
Eigenschaften einer Architektur beschreibt, d.h. es stellt Informationen wie die
Byte-Reihenfolge, Wortgröße, Speicherausrichtung oder die Eigenschaften der
Register bereit. Es beschreibt auch, welche Instruktionen zur Verfügung stehen,
definiert aber nicht, wie diese konkret implementiert sind, da dies für eine
Konfigurationsdatei zu komplex wäre. Die Implementierung der einzelnen
Instruktionen wird deshalb in C++ Code übernommen.


\subsubsection{Der Syntaxbaum}

Der Syntaxbaum mit durch Vererbung implementiert und in \emph{common} und
\emph{riscv} (oder jede beliebige andere Architektur) gegliedert. Eine
Übersicht bietet \autoref{fig:arch-syntax-tree}

\begin{figure}[H]
	\begin{center}
		\begin{tikzpicture}[node distance=0.5cm and 1.5cm]

		\node (super) [class] {AbstractSyntaxTreeNode};
		\node (invisible) [right = of super] {};
		\node (imm) [class, below = of invisible] {ImmediateNode};
		\node (bin) [class, below = of imm] {BinaryDataNode};
		\node (reg) [class, below = of bin] {AbstractRegisterNode};
		\node (instr) [class, below = of reg] {AbstractInstructionNode};
		\node (common) [below = of instr] {\textbf{common}};
		\node (reg-riscv) [class, right = of reg] {riscv::RegisterNode};
		\node (instr-riscv) [class, right = of instr] {riscv::AbstractInstructionNode};
		\node (riscv) [below = of instr-riscv] {\textbf{riscv}};

		\draw[inheritance-arrow] (imm) -- (imm -| super) -- (super);
		\draw[inheritance-arrow] (bin) -- (bin -| super) -- (super);
		\draw[inheritance-arrow] (instr) -- (instr -| super) -- (super);
		\draw[inheritance-arrow] (reg) -- (reg -| super) -- (super);

		\draw[inheritance-arrow] (instr-riscv) -- (instr);
		\draw[inheritance-arrow] (reg-riscv) -- (reg);

		\begin{pgfonlayer}{background}
		\path (super.west |- super.north)+(-0.5,0.5) node (a1) {};
		\path (instr.east |- instr.south)+(+0.5,-1.5) node (a2) {};
		\path[rounded corners, draw=black!50, dashed] (a1) rectangle (a2);
		\end{pgfonlayer}

		\begin{pgfonlayer}{background}
		\path (reg-riscv.west |- reg-riscv.north)+(-0.5,0.5) node (a1) {};
		\path (instr-riscv.east |- instr-riscv.south)+(+0.5,-1.5) node (a2) {};
		\path[rounded corners, draw=black!50, dashed] (a1) rectangle (a2);
		\end{pgfonlayer}
		\end{tikzpicture}
	\end{center}
	\caption{Klassendiagramm Syntaxbaum}
	\label{fig:arch-syntax-tree}
\end{figure}

\label{module-arch-ast-node-types}
Die aufgeführten Klassen haben folgende Funktion:
\begin{itemize}

  \item \textbf{AbstractSyntaxTreeNode} ist die Oberklasse jedes Syntax Knotens
  und definiert, welche Methoden die Unterklassen implementieren müssen. Des
  Weiteren enthält sie eine Liste an etwaige Kindknoten.

  \item \textbf{ImmediateNode} repräsentiert einen \emph{Immediate}-Wert, also
  einen Wert, der direkt im Assembler Quelltext angegeben ist. Architekturen
  stellen im Allgemeinen keine Spezialisierung eines Immediate Wertes zur
  Verfügung, weshalb diese Klasse vollständig im common Teil implementiert
  werden kann.

  \item \textbf{BinaryDataNode} enthält binäre Daten, wie z.B. Text Nachrichten.
  Konkret wird er für die Implementierung der Crash Instruktion verwendet.

	\item \textbf{AbstractRegisterNode} repräsentiert ein Register in der
	Instruktion. In RISC-V muss die Assemblierung speziell behandelt werden,
	weshalb es einen speziellen RegisterNode für RISC-V gibt.

	\item \textbf{AbstractInstructionNode} ist die oberste Ebene eines jeden
	Syntaxbaums und repräsentiert die auszuführende Instruktion.

\end{itemize}

Mit diesen Knotentypen lassen sich alle RISC-V Instruktionen modellieren. Nicht
RISC Architekturen stellen aber häufig die Möglichkeit bereit, einen
Speicherzugriff während einer anderen Instruktion durchzuführen. Ein Beispiel in
x86:

\begin{x86}
add eax, [ebx*2+2]
\end{x86}

Um diese Instruktionen modellieren zu können, wurden folgende Knotentypen
konzeptioniert, die jedoch nicht im Quellcode definiert sind:
\begin{itemize}
	\item \textbf{MemoryAccessNode} repräsentiert einen Inline Speicherzugriff,
	im oben aufgeführten Beispiel also der Inhalt der eckigen Klammern.
	\item \textbf{ArithmeticNode} stellt eine arithmetische Operation dar, im
	obigen Beispiel also sowohl die Multiplikation, als auch die Addition.
\end{itemize}

Die Oberklasse jedes Knotens ist die Klasse \texttt{AbstractSyntaxTreeNode}.
Folgende Methoden werden von ihr vorgegeben:

\begin{itemize}
	\label{dev-manual-arch-node-functions}
	\item \textbf{\texttt{getValue(MemoryAccess\&)}}: Diese Methode führt den
	darunter liegenden Syntaxbaum aus, und ruft ggf. rekursiv dieselbe Methode
	bei den Kindknoten auf. Je nach Knotentyp variiert der Rückgabewert: So gibt
	beispielsweise ein Instruktionsknoten die Adresse der nächsten Instruktion,
	ein Register seinen aktuellen Wert und ein Immediateknoten die
	abgespeicherte Konstante zurück. Als Parameter wird eine Zugriffsmöglichkeit auf den
	Speicher übergeben, mit dem die Instruktion z.B. das Resultat der Operation
	abspeichern kann.
	Mit Diesem Konzept werden Codeduplikate verhindert, da z.B. eine arithmetische
	Operation, die sowohl mit Registern, als auch mit Immediate Werten arbeiten
	kann, nur einmal implementiert werden muss. In RISC-V kann man dieses Konzept
	beispielsweise auf \emph{add} und \emph{addi} anwenden.

	\item \textbf{\texttt{validate(MemoryAccess\&)}}: Während der Parser lediglich
	eine syntaktische Überprüfung des Assembler Quelltext vornimmt, validiert
	diese Methode die semantische Korrektheit einer Instruktion. Es wird zum
	Beispiel überprüft, ob der richtige Typ und die korrekte Anzahl an Operanden
	übergeben wurde, oder ob der übergebene Immediate Werte in die vorgegebenen
	Anzahl an Bits passen. War die Validierung nicht erfolgreich, so wird eine
	übersetzbare Fehlermeldung zurückgegeben.

	\item \textbf{\texttt{validateRuntime(MemoryAccess\&)}}: Validiert, ob eine
	Instruktion zur Laufzeit ausgeführt werden kann. Die Methode wird vor allem
	für Sprunginstruktionen benötigt, sodass geprüft werden kann, ob das Ziel
	des Sprungs innerhalb des zur Verfügung stehenden Programms liegt. Des
	Weiteren lässt es sich so verhindert, dass geschützte Speicherbereiche
	von Store Instruktionen beschrieben werden.

	\item \textbf{\texttt{assemble()}}: Wandelt einen Syntaxbaum in die
	Binärdarstellung der Architektur um. Diese Darstellung ist lediglich zur
	Visualisierung für den Benutzer vorgesehen, die eigentliche Simulation der
	Instruktionen wird über den Syntaxbaum vorgenommen. Ein Registerknoten x5
	gibt z.B. \texttt{00101} (also 5 als Binärzahl der Länge n), ein Konstantenknoten gibt seinen Wert
	zurück. Der Instruktionsknoten (die Wurzel) gibt anschließend die
	zusammengefügte, komplette assemblierte Instruktion zurück.

	\item \textbf{\texttt{getIdentifier()}}: Gibt den Typ eines Knotens als
	Zeichenkette zurück. Beispielsweise geben Instruktionen ihren entsprechenden
	Mnemonic (z.B. \emph{addi}) und Register ihren Namen (z.B. \emph{x1})
	zurück. Letzteres wird verwendet, um Schreibzugriffe auf ein Register
	in einem Instruktionsknoten durchzuführen.
\end{itemize}

Die Aufgabe einer konkreten Architektur besteht nun darin, die eben beschriebenen
Unterklassen inklusive ihrer Methoden entsprechend zu implementieren. In
\autoref{dev:extension} wird darauf genauer eingegangen.
Um nun Objekte des Syntaxbaums zu erzeugen, wird die Node Factory benötigt, auf
die im Folgenden eingegangen wird.

\subsubsection{Node Factory}
\label{module-arch-node-factory}

Mit der \emph{Node Factory} wird der bereits beschriebene Syntaxbaum erzeugt.
Um nach außen ein Architektur-unabhängiges Interface zu bieten, basiert die
Node Factory dem \emph{Abstract Factory Pattern} und wird als Diagramm in
\autoref{fig:arch-node-factory} dargestellt.

\begin{figure}[H]
	\begin{center}
		\begin{tikzpicture}[node distance=0.5cm and 3cm]
		\pgfdeclarelayer{background}
		\pgfdeclarelayer{foreground}
		\pgfsetlayers{background,main,foreground}
		\tikzstyle{class} = [rectangle, rounded corners, draw=black, fill=white, drop shadow]
		\tikzstyle{inheritance-arrow} = [->, thick,>=open triangle 90]

		\node (instr-abstr) [class, anchor=west] {AbstractInstructionNodeFactory};
		\node (imm-abstr) [class, below = of instr-abstr] {AbstractImmediateNodeFactory};
		\node (reg-abstr) [class, below = of imm-abstr]	{AbstractRegisterNodeFactory};
		\node (data-abstr) [class, below = of reg-abstr] {AbstractDataNodeFactory};
		\node (mem-abstr) [class, below = of data-abstr] {AbstractMemoryAccessNodeFactory};
		\node (arithmetic-abstr) [class, below = of mem-abstr] {AbstractArithmeticNodeFactory};
		\node (common) [below = of arithmetic-abstr] {\textbf{common}};

		\node (instr-riscv) [class, right = of instr-abstr] {riscv::InstructionNodeFactory};
		\node (imm-riscv) [class, below = of instr-riscv] {riscv::ImmediateNodeFactory};
		\node (reg-riscv) [class, below = of imm-riscv] {riscv::RegisterNodeFactory};
		\node (data-riscv) [class, below = of reg-riscv] {riscv::DataNodeFactory};
		\node (riscv) [below = of data-riscv] {\textbf{riscv}};

		\draw[inheritance-arrow] (instr-riscv) edge (instr-abstr);
		\draw[inheritance-arrow] (imm-riscv) edge (imm-abstr);
		\draw[inheritance-arrow] (reg-riscv) edge (reg-abstr);
		\draw[inheritance-arrow] (data-riscv) edge (data-abstr);

		\begin{pgfonlayer}{background}
		\path (instr-abstr.west |- instr-abstr.north)+(-1,0.5) node (a1) {};
		\path (arithmetic-abstr.east |- arithmetic-abstr.south)+(+1,-1.5) node (a2) {};
		\path[rounded corners, draw=black!50, dashed] (a1) rectangle (a2);
		\end{pgfonlayer}

		\begin{pgfonlayer}{background}
		\path (instr-riscv.west |- instr-riscv.north)+(-0.5,0.5) node (a1) {};
		\path (data-riscv.east |- data-riscv.south)+(+1,-1.5) node (a2) {};
		\path[rounded corners, draw=black!50, dashed] (a1) rectangle (a2);
		\end{pgfonlayer}
		\end{tikzpicture}
	\end{center}
	\caption{Klassendiagramm Node Factory}
	\label{fig:arch-node-factory}
\end{figure}

Wie man erkennen kann, existiert für jeden Knotentyp eine eigene Factory. Dies
dient der Übersicht, da vor allem die Implementation der
\texttt{InstructionNodeFactory} bei Architekturen mit vielen Instruktionen
schnell unübersichtlich werden kann. Um zu verhindern, dass andere Module
mehrere Node Factory Objekte verwalten müssen, wird eine
\texttt{NodeFactoryCollection} über das Architecture Objekt zur Verfügung
gestellt, welches die einzelnen Factory Objekte kapselt und die
\texttt{create()} Aufrufe weiterleitet.
Die Factory Methoden geben einen \texttt{std::shared\_ptr} auf den erzeugten
Knoten zurück.

Des Weiteren fällt auf, dass RISC-V die beiden letzten Node Factories nicht
implementiert. Der Grund dafür wurde im vorherigen Abschnitt beschrieben:
RISC-V unterstützt keine Inline Speicherzugriffe. Architekturen können durch das
Fehlen einer Factory signalisieren, dass sie einen Knotentyp nicht unterstützen.

\subsubsection{RISC-V}

Bisher wurde fast ausschließlich die Schnittstelle beschrieben. Diese Sektion
soll einen Einblick in unsere Gedanken bei der RISC-V Implementation geben.

Zunächst etwas zur Implementation der Instruktionen. RISC-V definiert wiederum
eine eigene abstrakte Oberklasse \texttt{riscv::InstructionNode}, von der alle
weiteren Instruktionsknoten abgeleitet sind. Dies dient der Vermeidung von
Redundanz, da Methoden wie \texttt{assemble()} und
\texttt{getInstructionDocumentation()} für alle Instruktionen angewandt werden
können. Des Weiteren definiert die Klasse hilfreiche Methoden, die in den
Unterklassen verwendet werden, um zum Beispiel die Validierung der Operanden
einer Instruktion zu vereinfachen.

Die Knoten, die Instruktionen implementieren, sind ebenfalls darauf ausgelegt,
Redundanz zu vermeiden. Beispielhaft seien hier die \emph{Integer
Computational Instructions} (so der Name in der RISC-V Spezifikation)
herangezogen (das sind Instruktionen wie \emph{add}, \emph{addi} oder
\emph{and}). Da sich der Aufbau der Instruktionen lediglich in der
auszuführenden Operation unterscheidet, existiert eine weitere abstrakte
Oberklasse \texttt{riscv::AbstractIntegerInstructionNode}, die all jene
Instruktionen abdeckt. In dieser Oberklasse wird die Validierung der
Instruktionen vollständig behandelt und der Aufruf von \texttt{getValue()} so
weit abstrahiert, dass die konkreten Anwendung der Operation effektiv auf einen
einzeiligen Lamda Ausdruck reduziert werden kann. Die Implementierung befindet
sich in der Datei \texttt{integer-instructions.hpp}.

Ein weiteres Konzept der RISC-V Instruktionen basiert auf der Unterstützung
unterschiedlicher Wortgrößen. Derzeit bietet RISC-V ausgiebige Unterstützung für
32 und 64 Bit, in Zukunft soll 128 Bit folgen. Um zu verhindern, dass
Instruktionen für jede Wortgröße neu geschrieben werden müssen, nutzen wir C++
Templates, um die Wortgröße einer Instruktion zu spezifizieren. So wird in der
Node Factory von RISC-V eine Fallunterscheidung nach der verwendeten Wortgröße
gemacht, und dann der entsprechende Zahlentyp als Template Parameter gesetzt
(für 32 Bit z.B. \texttt{std::uint32\_t} und für 64 Bit
\texttt{std::uint64\_t}). Sollte die Entwicklung der 128 Bit Version von RISC-V
voranschreiten, so könnte man das mit dem aktuellen C++ Standard nicht abdecken,
da kein 128 Bit Zahlentyp definiert ist. Man könnte dann aber eine vereinfachte
Implementation eines \texttt{uint128\_t} schreiben, indem man z.B. zwei
\texttt{uint64\_t} in einer Klasse kapselt.

\subsubsection{Weiterführende Dokumentation}

Die vorherigen Abschnitte geben einen Überblick über die Architektur.
Weiterführende Dokumentation findet sich in denen für die Architektur relevanten
Dateien, welche in \autoref{fig:arch-further} aufgelistet sind.

\begin{figure}[H]
	\begin{center}
	\begin{tikzpicture}[%
	grow via three points={one child at (0.8,-0.8) and
		two children at (0.8,-0.8) and (0.8,-1.7)},
	edge from parent path={($(\tikzparentnode\tikzparentanchor)+(.2cm,0pt)$) |- (\tikzchildnode\tikzchildanchor)},
	growth parent anchor=west,
	parent anchor=south west]
	\tikzstyle{every node}=[draw=black,anchor=west]
	\node {\erasim{}}
	child { node {isa/}
		child { node {riscv.isa/} }
	}
	child [missing] {}
	child { node {$\{\text{tests/}, \text{include/}, \text{source/}\}$}
		child { node {arch/}
			child { node {common/} }
			child { node {riscv/} }
		}
	}
	child [missing] {}
	child [missing] {}
	child [missing] {};
	\end{tikzpicture}
	\end{center}

	\caption{Relevante Dateien des Architekturmoduls}
	\label{fig:arch-further}
\end{figure}
