\subsection{Architektur}

\todo[inline]{
	* Syntax Tree \\
	* Factory \\
	* Architecture \\
	* Teilung common/riscv \\
	* ValidationResult \\
	* InstructionNode \\
}

Die Architektur (kurz: \textit{Arch}) kümmert sich um die Ausführung der
Assembler Programme und wurde vor dem Hintergrund entworfen, einen möglichst
Architektur-unabhängigen Simulator zu entwickeln. So stellt sie Klassen zur
Verfügung, mit der eine konkrete Architektur (wie z.B. RISC-V) umgesetzt
werden kann. Eine Architektur wird in \texttt{YAML} Konfigurationsdateien
beschrieben, und anschließend in das Programm geladen. \\
Da wir uns gegen die Ausführung von Maschinencode entschieden haben, bietet
die Architektur eine abstraktere Darstellung von Assembler Programmen in Form
eines Syntax Baums. Des Weiteren ist die Architektur für die Validierung einer
Instruktion zuständig, und stellt bei Fehlern entsprechende Nachrichten zur
Verfügung, die dem Nutzer helfen sollen, das Problem zu beheben.

Eine Übersicht über die Schnittstellen des Architekturmoduls bietet Abbildung
\ref{fig:arch-overview} (Die verwendeten Klassennahmen stimmen nicht mit denen
im Quellcode überein, es geht hier nur um das Prinzip).

\begin{figure}[H]
	\begin{center}
		\begin{tikzpicture}[node distance=3.0cm]
		\tikzstyle{class} = [rectangle, rounded corners, draw=black, drop shadow, fill=white]
		\tikzstyle{myarrow} = [->, thick]

		\node (architecture) [class, rectangle split, rectangle split parts=2]
		{
			\textbf{Architecture}
			\nodepart{second}
			\begin{tabular}{c}
				getName() \\
				getEndianness() \\
				getWordSize() \\
				\ldots
			\end{tabular}
		};
		\node (factory) [class, right = of architecture]
		{
			\textbf{NodeFactory}
		};
		\node (syntaxtree) [class, rectangle split, rectangle split parts=2, right = of factory]
		{
			\textbf{SyntaxTreeNode}
			\nodepart{second}
			\begin{tabular}{c}
				validate() \\
				getValue() \\
				\ldots
			\end{tabular}
		};
		\draw[myarrow] (architecture) edge node [yshift=2mm] {getFactory()} (factory);
		\draw[myarrow] (factory) edge node [yshift=2mm] {create()} (syntaxtree);
		\end{tikzpicture}
	\end{center}
	\caption{Übersicht der Architekturschnittstelle}
	\label{fig:arch-overview}
\end{figure}

Die Architektur ist in einen allgemeinen Teil (im Folgenden \textbf{common})
genannt) und in einen Architektur-spezifischen Teil (benannt nach der entsprechenden
Architektur, z.B. \textbf{riscv} aufgeteilt. Die Schnittstelle ist so konzeptioniert,
dass andere Module nichts von der konkreten Architektur mitbekommen, und lässt
sich folgendermaßen charakterisieren: \\
Das \textit{Architecture} Objekt stellt allgemeine Informationen über die
geladene Architektur zur Verfügung (wie z.B. der Name, die Byte-Reihenfolge oder
die Wortgröße) und bietet zusätzlich Zugang zur \textit{NodeFactory}.
Mit der NodeFactory werden die Syntax Bäume in Form von \textit{SyntaxTreeNode}
Objekten erzeugt, welche anschließend validiert und ausgeführt werden können. \\
Im Folgenden wird auf diese Klassen genauer eingegangen.
\subsubsection{Architecture}

\todo[inline]{
	* Builder Pattern \\
	* Formula/Brewery \\
	* Information Interfaces
}

Ein \textit{Architecture} Objekt repräsentiert eine geladene Architektur im
Simulator. Der Grundgedanke bei der Entwicklung war das modulare Design von RISC-V
zu unterstützen, dieses Konzept lässt sich aber leicht auf andere Prozessorarchitekturen
übertragen. So wird angenommen, dass eine Architektur immer aus einem Basismodul
besteht (bei RISC-V: \textit{RV32I}), welches wieder durch andere Module erweitert
werden kann (bei RISC-V z.B. Multiplikation/Division, Floating Point, etc.). \\
Um also ein Architecture Objekt zu erzeugen, wird im ersten Schritt eine
\textit{ArchitectureFormula} erstellt, die das Basismodul und alle Erweiterungen
spezifiziert. Ein Aufruf von \textit{Brew()} löst die Abhängigkeiten auf und liest
anschließend die Informationen aus den Konfigurationsdateien ein.

Das Architecture Objekt beschreibt lediglich die Eigenschaften einer Architektur,
d.h. es stellt Informationen wie die Byte-Reihenfolge, Wortgröße, Speicherausrichtung
oder die Eigenschaften der Register bereit. Es beschreibt auch, welche Instruktionen
zur Verfügung stehen, definiert aber nicht, wie diese konkret implementiert sind,
da dies für eine Konfigurationsdatei zu komplex wäre. Die Implementation der
einzelnen Instruktionen wird deshalb in C++ Code übernommen. Wie die Verbindung
zwischen der Beschreibung der Architektur und der Implementation abläuft, wird
im Folgenden erläutert.

\subsubsection{Syntax Tree}

\todo[inline]{
	* Immediate \\
	* Instruction \\
	* MemoryAccess \\
	* RegisterAccess \\
	* Arithmetic \\
	* DataNode \\
	* getValue() assemble() etc.
}

Der Syntax Baum mit durch Vererbung implementiert und in \textit{common} und
\textit{riscv} (oder jede beliebige andere Architektur) gegliedert. Eine Übersicht
bietet Abbildung \ref{fig:arch-syntax-tree}

\begin{figure}[H]
	\begin{center}
		\begin{tikzpicture}[node distance=0.5cm and 1.5cm]
		\pgfdeclarelayer{background}
		\pgfdeclarelayer{foreground}
		\pgfsetlayers{background,main,foreground}
		\tikzstyle{class} = [rectangle, rounded corners, draw=black, fill=white, drop shadow]
		\tikzstyle{inheritance-arrow} = [->, thick,>=open triangle 90,every node/.style={font=\sffamily\small}]
		\tikzstyle{empty} = [draw=none,fill=none]

		\node (super) [class] {AbstractSyntaxTreeNode};
		\node (invisible) [right = of super] {};
		\node (imm) [class, below = of invisible] {ImmediateNode};
		\node (bin) [class, below = of imm] {BinaryDataNode};
		\node (reg) [class, below = of bin] {AbstractRegisterNode};
		\node (instr) [class, below = of reg] {AbstractInstructionNode};
		\node (common) [below = of instr] {\textbf{common}};
		\node (reg-riscv) [class, right = of reg] {riscv::RegisterNode};
		\node (instr-riscv) [class, right = of instr] {riscv::AbstractInstructionNode};
		\node (riscv) [below = of instr-riscv] {\textbf{riscv}};

		\draw[inheritance-arrow] (imm) -- (imm -| super) -- (super);
		\draw[inheritance-arrow] (bin) -- (bin -| super) -- (super);
		\draw[inheritance-arrow] (instr) -- (instr -| super) -- (super);
		\draw[inheritance-arrow] (reg) -- (reg -| super) -- (super);

		\draw[inheritance-arrow] (instr-riscv) -- (instr);
		\draw[inheritance-arrow] (reg-riscv) -- (reg);

		\begin{pgfonlayer}{background}
		\path (super.west |- super.north)+(-0.5,0.5) node (a1) {};
		\path (instr.east |- instr.south)+(+0.5,-1.5) node (a2) {};
		\path[rounded corners, draw=black!50, dashed] (a1) rectangle (a2);
		\end{pgfonlayer}

		\begin{pgfonlayer}{background}
		\path (reg-riscv.west |- reg-riscv.north)+(-0.5,0.5) node (a1) {};
		\path (instr-riscv.east |- instr-riscv.south)+(+0.5,-1.5) node (a2) {};
		\path[rounded corners, draw=black!50, dashed] (a1) rectangle (a2);
		\end{pgfonlayer}
		\end{tikzpicture}
	\end{center}
	\caption{Klassendiagramm Syntax Baum}
	\label{fig:arch-syntax-tree}
\end{figure}

Die aufgeführten Klassen haben folgende Funktion:
\begin{itemize}
	\item \textbf{AbstractSyntaxTreeNode} ist die Oberklasse jedes Syntax Knotens
	und definiert, welche Methoden die Unterklassen implementieren müssen.
	\item \textbf{ImmediateNode} repräsentiert einen \textit{Immediate}-Wert,
	also einen Wert, der direkt im Assembler Quelltext angegeben ist. Architekturen
	stellen im Allgemeinen keine Spezialisierung eines Immediate Wertes zur
	Verfügung, weshalb diese Klasse vollständig im common Teil implementiert werden
	kann.
	\item \textbf{BinaryDataNode} enthält binäre Daten, wie z.B. Text Nachrichten.
	Konkret wird er für die Implementierung der Crash Instruktion verwendet.
	\item \textbf{AbstractRegisterNode} repräsentiert ein Register in der
	Instruktion. In RISC-V muss die Assemblierung speziell behandelt werden,
	weshalb es einen speziellen RegisterNode für RISC-V gibt.
	\item \textbf{AbstractInstructionNode} ist die oberste Ebene eines jeden Syntax
	Baums und repräsentiert die auszuführende Instruktion.
\end{itemize}
Mit diesen Knotentypen lassen sich alle RISC-V Instruktionen modellieren. Nicht
RISC Architekturen stellen aber häufig die Möglichkeit bereit, einen Speicherzugriff
während einer anderen Instruktion durchzuführen. Ein Beispiel in x86:
\begin{lstlisting}[language={[x86masm]Assembler}]
add eax, [ebx*2+2]
\end{lstlisting}
Um diese Instruktionen modellieren zu können, wurden folgende Knotentypen
konzeptioniert, die jedoch nicht im Quellcode definiert sind:
\begin{itemize}
	\item \textbf{MemoryAccessNode} repräsentiert einen Inline Speicherzugriff,
	im oben aufgeführten Beispiel also der Inhalt der eckigen Klammern.
	\item \textbf{ArithmeticNode} stellt eine arithmetische Operation dar, im
	obigen Beispiel also sowohl die Multiplikation, als auch die Addition.
\end{itemize}

\subsubsection{Node Factory}

\todo[inline]{
	* NodeFactoryCollection \\
	* NodeFactoryCollectionMaker \\
	* FactoryTypes
}

Die Node Factory basiert auf dem Abstract Factory Pattern und wird als Diagramm
in Abbildung \ref{fig:arch-node-factory} dargestellt.

\begin{figure}[H]
	\begin{center}
		\begin{tikzpicture}[node distance=0.5cm and 3cm]
		\pgfdeclarelayer{background}
		\pgfdeclarelayer{foreground}
		\pgfsetlayers{background,main,foreground}
		\tikzstyle{class} = [rectangle, rounded corners, draw=black, fill=white, drop shadow]
		\tikzstyle{inheritance-arrow} = [->, thick,>=open triangle 90]
	
		\node (instr-abstr) [class, anchor=west] {AbstractInstructionNodeFactory};
		\node (imm-abstr) [class, below = of instr-abstr] {AbstractImmediateNodeFactory};
		\node (reg-abstr) [class, below = of imm-abstr]	{AbstractRegisterNodeFactory};
		\node (data-abstr) [class, below = of reg-abstr] {AbstractDataNodeFactory};
		\node (mem-abstr) [class, below = of data-abstr] {AbstractMemoryAccessNodeFactory};
		\node (arithmetic-abstr) [class, below = of mem-abstr] {AbstractArithmeticNodeFactory};
		\node (common) [below = of arithmetic-abstr] {\textbf{common}};

		\node (instr-riscv) [class, right = of instr-abstr] {riscv::InstructionNodeFactory};
		\node (imm-riscv) [class, below = of instr-riscv] {riscv::ImmediateNodeFactory};
		\node (reg-riscv) [class, below = of imm-riscv] {riscv::RegisterNodeFactory};
		\node (data-riscv) [class, below = of reg-riscv] {riscv::DataNodeFactory};
		\node (riscv) [below = of data-riscv] {\textbf{riscv}};

		\draw[inheritance-arrow] (instr-riscv) edge (instr-abstr);
		\draw[inheritance-arrow] (imm-riscv) edge (imm-abstr);
		\draw[inheritance-arrow] (reg-riscv) edge (reg-abstr);
		\draw[inheritance-arrow] (data-riscv) edge (data-abstr);

		\begin{pgfonlayer}{background}
		\path (instr-abstr.west |- instr-abstr.north)+(-1,0.5) node (a1) {};
		\path (arithmetic-abstr.east |- arithmetic-abstr.south)+(+1,-1.5) node (a2) {};
		\path[rounded corners, draw=black!50, dashed] (a1) rectangle (a2);
		\end{pgfonlayer}

		\begin{pgfonlayer}{background}
		\path (instr-riscv.west |- instr-riscv.north)+(-0.5,0.5) node (a1) {};
		\path (data-riscv.east |- data-riscv.south)+(+1,-1.5) node (a2) {};
		\path[rounded corners, draw=black!50, dashed] (a1) rectangle (a2);
		\end{pgfonlayer}
		\end{tikzpicture}
	\end{center}
	\caption{Klassendiagramm Node Factory}
	\label{fig:arch-node-factory}
\end{figure}

\subsubsection{Architecture Description Language}

Eine Architektur kann in \texttt{YAML} Dateien beschrieben werden.

\todo[inline]{@Peter}