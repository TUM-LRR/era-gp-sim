% !TEX root = ../team-report.tex
% ERA-Großpraktikum: Team Bericht -- Organisatorisches (Vision)

\subsubsection{Die Vision}
\label{team:orga-plan-vision}

In den ersten sechs bis acht Wochen des Großpraktikumszeitraums beschäftigte
sich die Gruppe mit der Ausarbeitung eines groben Grundrisses für die
Architektur des Simulators, mit der Angewöhnung an die notwendigen Technologien
für die Implementierung sowie mit der Definition einer Vision und eines
Zeitplanes. Die Vision war primär von zwei Quellen beeinflusst. Zum einen waren
dies unsere eigenen Erfahrungen als Studierende in der Vorlesung
\emph{Einführung in die Rechnerarchitektur} (ERA). Da jeder von uns gerade erst
in derselben Situation wie unsere zukünftigen Kunden gesessen hatte, konnten wir
natürlich leicht spezifizieren, was uns und unseren Kommilitonen beim Erlernen
der maschinennahen Programmierung und Assemblersprachen geholfen hätte. Zum
anderen hatten wir durch den bereits existierenden und von uns in ERA genutzten
Simulator einen Referenzpunkt für die Entwicklung von \erasim{}. Wir konnten
analysieren, welche Aspekte von Jasmin uns geholfen hatten, konnten aber
insbesondere auch jene Features nennen, welche uns an Jasmin nicht gefielen und
diese als Ziele für \erasim{} festlegen. Die Überlegungen dieser ersten Phase
verfestigten sich schließlich in den folgenden langfristigen Zielen:

\begin{enumerate}
  \item Die Funktionen von \erasim{} sollten eine Übermenge jener von Jasmin
  sein. Das bedeutet, dass sämtliche Aufgaben aus früheren Zentralübungen,
  Klausuren und Tutorübungen in unserem Simulator ausführbar sein sollten,
  soweit diese in RISC-V übersetzbar sind.
  \item Der Simulator soll zwar primär für die Lehre gedacht, jedoch nicht durch
  diese beschränkt sein. Auch weitere Einsatzgebiete, wie die Forschung,
  sollten, zumindest durch Erweiterungen der Grundversion, offen bleiben. Eine
  Grundanforderung hierfür war es, echte RISC-V Programme ausführen zu können.
  \item Als Voraussetzung für (2) sollte der Simulator nicht \emph{zeilen}-,
  sondern \emph{address}basiert sein. Das bedeutet, dass Marken Adressen im
  Speicher und nicht Zeilen im Text referenzieren sollten. Hierfür müssten
  Instruktionen auch entsprechend in den Speicher assembliert werden.
  \item Als wohl wichtigste Eigenschaft sollte \erasim{}, im Vergleich zu
  \emph{Jasmin}, nicht nur für einen bestimmten Befehlssatz wie x86 oder ARM
  implementiert sein, sondern für beliebige Architekturen erweiterbar bleiben.
\end{enumerate}
