% !TEX root = ../team-report.tex
% ERA-Großpraktikum: Team Bericht -- Organisatorisches (Roadmap)

\subsubsection{Der Zeitplan}
\label{team:orga-plan-time}
\vspace{-0.2cm}

Neben der Definition einer Vision füllten wir die erste Phase des
Entwicklungszeitraums mit der Ausarbeitung einer langfristigen \emph{Roadmap}
sowie einer kurzfristigen, kontinuierlichen Strategie für die Entwicklungsarbeit
über die nächsten sechs Monate. Wir befassten uns also mit allen Aspekten der
Zeitplanung unseres Projekt. Hierbei ließen wir uns insbesondere durch die in
der Industrie allgegenwärtigen \emph{agilen Methoden} leiten. Wir wollten also
auf Wasserfalldiagramme verzichten, die die Entwicklung bis ins kleinste Detail,
Schritt für Schritt, vorschreiben. Stattdessen fokussierten wir uns auf die
Bestimmung von monatlichen \emph{Meilensteinen} für jede Untergruppe und
einigten uns darauf, diese Meilensteine in zwei-wöchigen \emph{Sprints}
bewältigen zu wollen. Die vollständige Roadmap kann in der initialen
Spezifikation des Projekts gefunden werden. Eine komprimierte Version lautet:
\begin{sitemize}{-0.35cm}
  \bolditem{Juni}: Definition und Realisierung des Architekturinterfaces, parsen einfacher Befehle und Dummy Versionen von Speichermodell und GUI.
  \bolditem{Juli}: Implementierung und Parsen sämtlicher arithmetischer Befehle, Fertigstellung des Speichermodells und Herstellung erster Verbindungswege zwischen Modulen.
  \bolditem{August \& September}: Assemblierung von Befehlen zur Darstellung im Speicher, Parsen von Direktiven, Marken und Konstanten sowie vollständige Verbindung zwischen GUI und Core.
  \bolditem{Oktober}: Entwicklung von Pseudoinstruktionen, Direktiven, Makros und Serialisierungslogik.
  \bolditem{November}: Fertigstellung der visuellen I/O Module und projektweite Fehlerbehebung.
  \bolditem{Dezember \& Januar} Dokumentation und Bericht.
  \vspace{-0.3cm}
\end{sitemize}
