% !TEX root = ../team-report.tex
% ERA-Großpraktikum: Team Bericht -- Organisatorisches (Languages)

\subsubsection{Sprachen}
\label{team:orga-workflow-lang}

Wie bei vielen großen Softwareprojekten üblich, haben wir für die Entwicklung
von \erasim{} eine Programmiersprache als Hauptwerkzeug genutzt, gegebenenfalls
aber nach dem Prinzip "Use the Right Tool for the Job" auch in anderen
Sprachen entwickelt. Unser Hauptwerkzeug war hierbei \emph{C++}\footnote{Wir
haben insbesondere darauf Wert gelegt, die neuen C++11 und C++14 Standards zu
nutzen.}, womit das gesamte "Backend" sowie auch bestimmte Teile des
"Frontends", beispielsweise Datenmodelle, implementiert sind. Für die
Entwicklung der GUI wurde das \emph{Qt}\footnote{\url{https://www.qt.io}} C++
Framework verwendet, wobei wir die \emph{QML} und \emph{Javascript}
Skriptingsprachen zur Definition der Benutzeroberfläche nutzten. Zu den anderen
Sprachen zählt beispielsweise \emph{Python}, womit einige Tools zur
Produktivitätssteigerung sowie ein $\text{SASS} \rightarrow \text{JSON}$
Konvertierer als Teil unserer UI-Theming Implementierung geschrieben wurden,
\LaTeX{} zur Ausarbeitung von Spezifikationen sowie
\emph{SASS}\footnote{\url{http://sass-lang.com}} zur Definition von UI Themes.
Insgesamt wurden in diesen Sprachen ca. 40,000 Zeilen Code geschrieben. \autoref{fig:lang} zeigt hierzu den prozentuellen Anteil jeder Sprache an dieser
Zahl, wobei \autoref{tbl:lang} eine genauere Aufschlüsselung präsentiert.

\begin{figure}[t!]
  \centering
  \begin{tikzpicture}
    \newlength{\height}\setlength{\height}{0.4cm}\relax

    % C++
    \fill [Red, rounded corners=2pt]
          (0, 0) rectangle ++(10.9, \height);
    \draw [Red] (5, {\height + 0.5cm})
          node {\small\texttt{C++ (78.7\%)}};

    % QML
    \fill [Green, rounded corners=2pt]
          (10.8, 0) rectangle ++(1.9, \height);
    \draw [Green] (11.5, -0.5cm)
          node {\small\texttt{QML (13.7\%)}};

    % LaTeX
    \fill [orange, rounded corners=2pt]
          (12.6, 0) rectangle ++(1, \height);
    \draw [orange] (13.75, -0.5cm)
          node {\small\texttt{\LaTeX (4.4\%)}};

    % Python
    \fill [ProcessBlue, rounded corners=2pt]
          (13.5, 0) rectangle ++(0.5, \height);
    \draw [ProcessBlue] (14, {\height + 0.5cm})
          node {\small\texttt{Python (1.5\%)}};

    % Other
    \fill [Gray, rounded corners=2pt]
          (13.9, 0) rectangle ++(0.5, \height);
    \draw [Gray] (16, {\height/2})
          node {\small\texttt{Andere (1.7\%)}};
  \end{tikzpicture}
  \caption{Der prozentuelle Anteil der von uns genutzten Programmiersprachen  an der Gesamtzahl an Codezeilen.}
  \label{fig:lang}
\end{figure}

\begin{table}[h!]
  \centering
  \begin{tabular}{lrrr}
    \textbf{Programmiersprache} & \textbf{Kommentarzeilen} & \textbf{Codezeilen} & \textbf{Gesamt} \\
    \midrule
    C\texttt{++} & 17981 & 24785 & 42766 \\
    QML & 2168 & 5949 & 8117 \\
    TeX & 588 & 5772 & 6360 \\
    YAML & 36 & 1959 & 1995 \\
    SASS & 11 & 666 & 677 \\
    Python & 242 & 543 & 785 \\
    CMake & 118 & 313 & 431 \\
    Bash & 27 & 153 & 180 \\
    JavaScript & 27 & 71 & 98 \\
    make & 1 & 4 & 5 \\
    DOS Batch & 0 & 2 & 2 \\
    \bottomrule
    Alle & 21199 & 40217 & 61416
  \end{tabular}
  \caption{Eine Aufschlüsselung der für die Entwicklung von \erasim{} geschriebenen Codezeilen in den von uns genutzten Programmiersprachen. Die erste Spalte nennt die Programmiersprache, die zweite Spalte listet die Anzahl an Kommentaren und die dritte Spalte die Anzahl an echten Codezeilen in dieser Sprache.}
  \label{tbl:lang}
  \vspace{0.5cm}
\end{table}

\pagebreak

\textbf{Lernphase}

Eine besondere Herausforderung bei der Entwicklung von \erasim{} war, dass nur
zwei der (anfangs) zwölf Mitglieder zur effektiven Entwicklung ausreichendes
Wissen in C++ besaßen. Natürlich waren alle in der Programmierung bereits
erfahren, jedoch in anderen Sprachen wie Java oder C\#. Somit musste der
Großteil der Teammitglieder zu Beginn des Projekts Zeit darin investieren, C++
zu erlernen. Da C++ für seine steile Lernkurve berüchtigt ist, war dies keine
kleine Aufgabe. Dennoch fanden alle Entwickler erfreulich schnell den Einstieg
in die Sprache und konnten bald darin produktiv werden. Hilfreich hierfür waren
unter anderem die Lektüre bestimmter Bücher wie \emph{C++
Primer}\footnote{\emph{C++ Primer}; Lipmann, Lajoie \& Moo (2012)} oder \emph{A
Tour of C++}\footnote{\emph{A Tour of C++}; Bjarne Stroustrup (2013)}, ein
eigener Kanal für C++-bezogene Fragen in Slack, gründliche Code Reviews von
Kameraden mit mehr C++ Erfahrung sowie auch die Einhaltung eines strikten Style
Guides, der die Feature-Vielfalt von C++ etwas reduziert.

\textbf{Evaluation}

In Retrospektive können wir als Team folgenden Schluss zur Wahl von C++ als
wichtigste Programmiersprache für \erasim{} ziehen: C++ ist \emph{schwer},
aber \emph{effektiv}. Wir können einige Nachteile der Sprache identifizieren,
können aber insbesondere auch ihre vielen Vorteile wertschätzen. Folgende
Eigenschaften von C++ würden wir als positiv für die Effektivität und Effizienz
unserer Entwicklung einschätzen:
\begin{sitemize}{-0.45cm}
  \item C++ ist umfangreich und bietet viele Möglichkeiten, zum Beispiel im
  Bereich Threading, generischem Programmieren via Templates und Operator
  Überladungen,
  \item Die Hardwarenähe und Speicherverwaltungsmöglichkeiten der Sprache können
  in vielen Fällen Leistungsverbesserungen im Vergleich zu anderen Sprachen
  bringen,
  \item Gleichzeitig schützt das Prinzip von \emph{Resource Acquisition Is
  Initialization} (RAII) und Standardklassen wie \texttt{shared\_ptr} vor Memory
  Leaks und anderen Gefahren,
  \item Tools wie {\ttfamily\small clang-format}, \texttt{clang-tidy} oder
  \texttt{cpplint} beschleunigen die (fehlerfreie) Entwicklung.
  \vspace{-0.35cm}
\end{sitemize}

Gleichzeitig sind uns folgende Aspekte negativ aufgefallen:
\begin{sitemize}{-0.4cm}
  \item C++ ist umfangreich und bietet viele Möglichkeiten, zum Beispiel im
  Bereich Threading, generischem Programmieren via Templates und Operator
  Überladungen,
  \item Fehlermeldungen können sehr komplex sein und deren Behebung oftmals mehr
  Zeit in Anspruch nehmen, als die eigentlich Entwicklung,
  \item Die Lernkurve ist sehr steil und erschwert den Einstieg in die Sprache,
  \item Compiler, Libraries und Tools sind oftmals sehr stark plattformabhängig,
  was zu erhöhter Komplexität und erhöhtem Aufwand führen kann (im Vergleich
  zu JVM-basierten Sprachen),
  \item Das Fehlen eines Modulsystems (und die damit verbundene Notwendigkeit
  von \emph{Forward-Deklarationen}) schnell zu Problemen, wie
  \texttt{\#include}-Zyklen, führen.
  \vspace{0.3cm}
\end{sitemize}
