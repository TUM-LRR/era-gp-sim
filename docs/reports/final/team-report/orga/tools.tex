% !TEX root = ../team-report.tex
% ERA-Großpraktikum: Team Bericht -- Organisatorisches (Tools)

\subsubsection{Hilfsmittel}
\label{team:orga-workflow-tools}
\vspace{-0.3cm}

Neben den von uns genutzten Programmiersprachen und Entwicklungsmethoden war
unser Workflow auch stark von unseren \emph{Hilfsmitteln} bzw. \emph{Tools}
geprägt. Als \emph{Hilfsmittel} bezeichnen wir jene Programme oder Services, die
offline oder online unsere Entwicklung sowie die übrigen Aspekte unseres
Workflows unterstützten. Hierbei lassen sich unsere Hilfsmittel grob in zwei
Gruppen teilen: technische Hilfsmittel und organisatorische Hilfsmittel.

In Retrospektive war der Einsatz der weiter unten erläuterten technischen
Hilfsmittel ein enorm wichtiger Teil unserer Entwicklung. Insbesondere halfen
uns diese Hilfsmittel bei der Bewältigung der Feature-Vielfalt und Komplexität
der C++-Sprache. Auch in der Industrie wird die Meinung sehr stark vertreten,
dass die Verfügbarkeit korrekter und effektiver Tools oft ebenso wichtig ist,
wie die Funktionalität der Sprache selbst. So ist es insbesondere erfreulich,
dass in den letzten Jahren eine Vielzahl von Tools aus dem LLVM/Clang
Compiler-Projekt hervorgetreten sind, die C++ Entwicklern, beispielsweise durch
statische Analyse, helfen. Unter den von uns genutzten Tools möchten wir drei
insbesondere hervorheben: \texttt{clang-format}, \texttt{cpplint} und Travis CI.
Diese lassen sich detaillierter wie folgt beschreiben:
\begin{sitemize}{-0.6cm}

  \item \texttt{clang-format} ist ein Hilfsmittel aus der Familie von
  \texttt{clang} Tools, das zur Einhaltung eines Style Guides\footnote{Als
  \emph{Style Guide} bezeichnen wir ein Dokument, das im Falle mehrerer
  Möglichkeiten für die Formatierung oder Implementierung von Code eine dieser
  Möglichkeiten bevorzugt und diese Bevorzugung schriftlich festlegt.} dient.
  Wir betrachten die Instandhaltung eines rigorosen Style Guides als einen
  notwendigen und essentiellen Aspekt der Entwicklung von C++ Code. Die
  Begründung hierfür liegt darin, dass C++ im Bezug auf die Formatierung eine
  enorm flexible Sprache ist. Ebenso kann es manchmal nicht nur bei der
  Formatierung, sondern auch bei der Implementierung eines Stück Codes, mehrere
  Möglichkeiten geben. So ist beispielsweise überall dort, wo eine Referenz
  möglich ist, ebenso auch ein Zeiger, ohne Verlust von Semantik, nutzbar. Ein
  Style Guide hilft erfahrungsgemäß auch bei der Integration neuer Entwickler,
  da man diesen die Entscheidungsfreiheit (bzw. -notwendigkeit) gewissermaßen
  entnimmt. Das \texttt{clang-format} Tool hilft dabei, die Formatierung von
  Programmcode anhand vordefinierter Regeln zu automatisieren. Dies erleichtert
  zum einen die Standardisierung von Formatierungsregeln, entnimmt dem
  Programmierer aber ebenso das manuelle Einrücken von Code oder ähnliche
  mühsame Aufgaben.
  \vspace{0.3cm}

  \item \texttt{cpplint} ist ein \emph{Linter}\footnote{Mit einem \emph{Linter}
  ist ein Programm gemeint, dass Code statisch untersucht und Hinweise auf
  mögliche Defekte liefert.}, der von Google, Inc. zur Einhaltung des Google C++
  Style Guide entwickelt wurde. Im Unterschied zu \texttt{clang-format} warnt
  \texttt{cpplint} nicht nur vor syntaktischen, sondern auch vor semantischen
  Fehlern. Vergisst man beispielsweise, Konstruktiven, die nur einen Parameter
  nehmen, mit \texttt{explicit} zu markieren, so wird man von \texttt{cpplint}
  darauf aufmerksam gemacht. Dies verhindert implizite Konvertierungen, die
  unter Umständen zu Problemen führen können.

  \item \emph{Travis CI} (kurz \emph{Travis}) ist ein Online-Test-Server,
  welcher Unit Tests in isolierten Umgebungen auf Linux Maschinen ausführt. Wir
  haben Travis hierbei so konfiguriert, dass nach jedem Commit auf unser GitHub
  Repository die Kompilierung unseres Codes und die Ausführung von Tests
  gestartet werden. Travis hat uns insbesondere dabei geholfen, den Erfolg
  unserer Tests auf einer "unabhängigen" Platform laufend zu verifizieren. So
  kann stets von jedem die Gesundheit des Projekts überprüft werden, ohne selbst
  den Simulator kompilieren und ausführen zu müssen.

\end{sitemize}

Neben den obigen technischen Werkzeugen haben wir bei der Entwicklung von
\erasim{} auch von einer Reihe von Hilfsmitteln Gebrauch gemacht, die uns bei
organisatorischen Aspekten halfen. Als Beispiel möchten wir hier
\emph{Waffle}\footnote{\url{http://waffle.io}} nennen, welches wir als Online
Kanban Board verwendet haben.
