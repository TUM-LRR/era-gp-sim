% !TEX root = ../team-report.tex
% ERA-Großpraktikum: Team Bericht -- Organisatorisches

\section{Organisatorisches}
\label{team:orga}

Dieses Kapitel behandelt die "Logistik" hinter der Entstehung von \erasim{} und
beschreibt, welche Strategie wir als Gruppe bei der Entwicklung des Simulators
befolgt haben. \autoref{team:orga-structure} skizziert die Struktur und
Hierarchie der Gruppe, \autoref{team:orga-plan} diskutiert den initialen und
letztendlichen Zeitplan für die Entwicklung und
\autoref{team:orga-workflow} schildert schlussendlich unseren Workflow.


\tikzset{hide on/.code={\only<#1>{\color{white}}}}
\tikzset{
  guitree/.style=
    {rectangle,rounded corners,
    minimum height=7mm,
    fill=white,
    draw=black!50,
    align=center,
    text depth = 0pt
    },
  edge from parent/.style=
    {,
    draw=black!50,
    thick
    }}
%
%\begin{slide}{Aufbau der GUI}
%\begin{tikzpicture}
%  \Tree [.\node[guitree, color=white,draw=white]{ApplicationWindow\\(main.qml)};
%            \edge[draw=white];[.\node[guitree, color=white,draw=white]{Menubar}; ]
%            \edge[draw=white];[.\node[guitree, color=white,draw=white]{Toolbar}; ]
%            \edge[draw=white];[.\node[guitree, color=white,draw=white]{ProjectTabView};
%                \edge[draw=white];[.\node[guitree, color=white,draw=white]{Splitview};
%                     \edge[draw=white];[.\node[guitree, color=white,draw=white]{InnerSplitviews};
%                         \edge[draw=white];[.\node[guitree, color=white,draw=white]{SplitViewItem};
%                             \edge[draw=white];[.\node[guitree]{Snapshots}; ]
%                             \edge[draw=white];[.\node[guitree]{Help}; ]
%                             \edge[draw=white];[.\node[guitree]{Register}; ]
%                             \edge[draw=white];[.\node[guitree]{Memory}; ]
%                             \edge[draw=white];[.\node[guitree]{Input/Output}; ] ] ]
%                     \edge[draw=white];[.\node[guitree, color=white,draw=white]{InnerSplitviews-\\Editor};
%                         \edge[draw=white];[.\node[guitree, color=white,draw=white]{SplitViewItem}; ]
%                         \edge[draw=white];[.\node[guitree, color=white,draw=white]{Editor}; ] ] ] ] ]
%\end{tikzpicture}
%\end{slide}
%
%% Mit ApplicationWindow
%\begin{slide}{Aufbau der GUI}
%\begin{tikzpicture}
%  \Tree [.\node[guitree]{ApplicationWindow\\(main.qml)};
%            \edge[draw=white];[.\node[guitree, color=white,draw=white]{Menubar}; ]
%            \edge[draw=white];[.\node[guitree, color=white,draw=white]{Toolbar}; ]
%            \edge[draw=white];[.\node[guitree, color=white,draw=white]{ProjectTabView};
%                \edge[draw=white];[.\node[guitree, color=white,draw=white]{Splitview};
%                     \edge[draw=white];[.\node[guitree, color=white,draw=white]{InnerSplitviews};
%                         \edge[draw=white];[.\node[guitree, color=white,draw=white]{SplitViewItem};
%                             \edge[draw=white];[.\node[guitree]{Snapshots}; ]
%                             \edge[draw=white];[.\node[guitree]{Help}; ]
%                             \edge[draw=white];[.\node[guitree]{Registers}; ]
%                             \edge[draw=white];[.\node[guitree]{Memory}; ]
%                             \edge[draw=white];[.\node[guitree]{Input/Output}; ] ] ]
%                     \edge[draw=white];[.\node[guitree, color=white,draw=white]{InnerSplitviews-\\Editor};
%                         \edge[draw=white];[.\node[guitree, color=white,draw=white]{SplitViewItem}; ]
%                         \edge[draw=white];[.\node[guitree, color=white,draw=white]{Editor}; ] ] ] ] ]
%\end{tikzpicture}
%\end{slide}
%
%% Mit MenuBar, Toolbar, ProjectTabView
%\begin{slide}{Aufbau der GUI}
%\begin{tikzpicture}
%  \Tree [.\node[guitree]{ApplicationWindow\\(main.qml)};
%            [.\node[guitree]{Menubar}; ]
%            [.\node[guitree]{Toolbar}; ]
%            [.\node[guitree]{ProjectTabView};
%                \edge[draw=white];[.\node[guitree, color=white,draw=white]{Splitview};
%                     \edge[draw=white];[.\node[guitree, color=white,draw=white]{InnerSplitviews};
%                         \edge[draw=white];[.\node[guitree, color=white,draw=white]{SplitViewItem};
%                             \edge[draw=white];[.\node[guitree]{Snapshots}; ]
%                             \edge[draw=white];[.\node[guitree]{Help}; ]
%                             \edge[draw=white];[.\node[guitree]{Registers}; ]
%                             \edge[draw=white];[.\node[guitree]{Memory}; ]
%                             \edge[draw=white];[.\node[guitree]{Input/Output}; ] ] ]
%                     \edge[draw=white];[.\node[guitree, color=white,draw=white]{InnerSplitviews-\\Editor};
%                         \edge[draw=white];[.\node[guitree, color=white,draw=white]{SplitViewItem}; ]
%                         \edge[draw=white];[.\node[guitree, color=white,draw=white]{Editor}; ] ] ] ] ]
%\end{tikzpicture}
%\end{slide}
%
%% Mit SplitView
%\begin{slide}{Aufbau der GUI}
%\begin{tikzpicture}
%  \Tree [.\node[guitree]{ApplicationWindow\\(main.qml)};
%            [.\node[guitree]{Menubar}; ]
%            [.\node[guitree]{Toolbar}; ]
%            [.\node[guitree]{ProjectTabView};
%                [.\node[guitree]{Splitview};
%                     \edge[draw=white];[.\node[guitree, color=white,draw=white]{InnerSplitviews};
%                         \edge[draw=white];[.\node[guitree, color=white,draw=white]{SplitViewItem};
%                             \edge[draw=white];[.\node[guitree]{Snapshots}; ]
%                             \edge[draw=white];[.\node[guitree]{Help}; ]
%                             \edge[draw=white];[.\node[guitree]{Registers}; ]
%                             \edge[draw=white];[.\node[guitree]{Memory}; ]
%                             \edge[draw=white];[.\node[guitree]{Input/Output}; ] ] ]
%                     \edge[draw=white];[.\node[guitree, color=white,draw=white]{InnerSplitviews-\\Editor};
%                         \edge[draw=white];[.\node[guitree, color=white,draw=white]{SplitViewItem}; ]
%                         \edge[draw=white];[.\node[guitree, color=white,draw=white]{Editor}; ] ] ] ] ]
%\end{tikzpicture}
%\end{slide}
%
%% Mit SplitViewItems
%\begin{slide}{Aufbau der GUI}
%\begin{tikzpicture}
%  \Tree [.\node[guitree]{ApplicationWindow\\(main.qml)};
%            [.\node[guitree]{Menubar}; ]
%            [.\node[guitree]{Toolbar}; ]
%            [.\node[guitree]{ProjectTabView};
%                [.\node[guitree]{Splitview};
%                     [.\node[guitree]{InnerSplitviews};
%                         \edge[draw=white];[.\node[guitree, color=white,draw=white]{SplitViewItem};
%                             \edge[draw=white];[.\node[guitree]{Snapshots}; ]
%                             \edge[draw=white];[.\node[guitree]{Help}; ]
%                             \edge[draw=white];[.\node[guitree]{Registers}; ]
%                             \edge[draw=white];[.\node[guitree]{Memory}; ]
%                             \edge[draw=white];[.\node[guitree]{Input/Output}; ] ] ]
%                     [.\node[guitree]{InnerSplitviews-\\Editor};
%                         \edge[draw=white];[.\node[guitree, color=white,draw=white]{SplitViewItem}; ]
%                         \edge[draw=white];[.\node[guitree, color=white,draw=white]{Editor}; ] ] ] ] ]
%\end{tikzpicture}
%\end{slide}
%
%% Mit SplitViewItems, Editor
%\begin{slide}{Aufbau der GUI}
%\begin{tikzpicture}
%  \Tree [.\node[guitree]{ApplicationWindow\\(main.qml)};
%            [.\node[guitree]{Menubar}; ]
%            [.\node[guitree]{Toolbar}; ]
%            [.\node[guitree]{ProjectTabView};
%                [.\node[guitree]{Splitview};
%                     [.\node[guitree]{InnerSplitviews};
%                         [.\node[guitree]{SplitViewItem};
%                             \edge[draw=white];[.\node[guitree]{Snapshots}; ]
%                             \edge[draw=white];[.\node[guitree]{Help}; ]
%                             \edge[draw=white];[.\node[guitree]{Registers}; ]
%                             \edge[draw=white];[.\node[guitree]{Memory}; ]
%                             \edge[draw=white];[.\node[guitree]{Input/Output}; ] ] ]
%                     [.\node[guitree]{InnerSplitviews-\\Editor};
%                         [.\node[guitree]{SplitViewItem}; ]
%                         [.\node[guitree]{Editor}; ] ] ] ] ]
%\end{tikzpicture}
%\end{slide}

% Mit SplitViewItems, Editor
\begin{slide}{Aufbau der GUI}
\begin{tikzpicture}[edge from parent fork down, level distance=1.3cm, font=\tiny]
  \Tree [.\node[guitree]{ApplicationWindow\\(main.qml)};
            [.\node[guitree]{Menubar}; ]
            [.\node[guitree]{Toolbar}; ]
            [.\node[guitree]{ProjectTabView};
                [.\node[guitree]{Splitview};
                     [.\node[guitree]{InnerSplitviews};
                         [.\node[guitree]{SplitViewItem};
                             [.\node[guitree]{Snapshots}; ]
                             [.\node[guitree]{Help}; ]
                             [.\node[guitree]{Registers}; ]
                             [.\node[guitree]{Memory}; ]
                             [.\node[guitree]{Input/Output}; ] ] ]
                     [.\node[guitree]{InnerSplitviews-\\Editor};
                         [.\node[guitree]{SplitViewItem}; ]
                         [.\node[guitree]{Editor}; ] ] ] ] ]
\end{tikzpicture}
\end{slide}


\subsection{Planung}
\label{team:orga-plan}

Die meisten Unternehmungen in der Arbeitswelt und darüber hinaus, für welche man
sich Erfolg wünscht, sind von drei essentiellen Grundbausteinen geprägt: eine
Vision für das Endergebnis; ein handfester Plan, wie man diese Vision
verwirklichen möchte sowie letztendlich die notwendige Tugend und Investition an
Zeit, um den Plan zu befolgen und die Vision zu realisieren. Diese Sektion
beschreibt, welche Ausprägung wir den ersten beiden Bausteinen gegeben haben.

% !TEX root = ../team-report.tex
% ERA-Großpraktikum: Team Bericht -- Organisatorisches (Vision)

\subsubsection{Die Vision}
\label{team:orga-plan-vision}

In den ersten sechs bis acht Wochen des Großpraktikumszeitraums beschäftigte
sich die Gruppe mit der Ausarbeitung eines groben Grundrisses für die
Architektur des Simulators, mit der Angewöhnung an die notwendigen Technologien
für die Implementierung sowie mit der Definition einer Vision und eines
Zeitplanes. Die Vision war primär von zwei Quellen beeinflusst. Zum einen waren
dies unsere eigenen Erfahrungen als Studierende in der Vorlesung
\emph{Einführung in die Rechnerarchitektur} (ERA). Da jeder von uns gerade erst
in derselben Situation wie unsere zukünftigen Kunden gesessen hatte, konnten wir
natürlich leicht spezifizieren, was uns und unseren Kommilitonen beim Erlernen
der maschinennahen Programmierung und Assemblersprachen geholfen hätte. Zum
anderen hatten wir durch den bereits existierenden und von uns in ERA genutzten
Simulator einen Referenzpunkt für die Entwicklung von \erasim{}. Wir konnten
analysieren, welche Aspekte von Jasmin uns geholfen hatten, konnten aber
insbesondere auch jene Features nennen, welche uns an Jasmin nicht gefielen und
deren Verbesserung als Ziel für \erasim{} festlegen.

Die Überlegungen dieser ersten Phase verfestigten sich schließlich in den
folgenden langfristigen Zielen:
\begin{senumerate}{-0.45cm}
  \item Die Funktionen von \erasim{} sollten eine Übermenge jener von Jasmin
  sein. Das bedeutet, dass sämtliche Aufgaben aus früheren Zentralübungen,
  Klausuren und Tutorübungen (soweit in RISC-V übersetzbar) in unserem Simulator ausführbar sein sollten.
  \item Der Simulator soll zwar primär für die Lehre gedacht, jedoch nicht durch
  diese beschränkt sein. Auch weitere Einsatzgebiete, wie die Forschung,
  sollten, zumindest durch Erweiterungen der Grundversion, offen bleiben. Eine
  Grundanforderung hierfür war es, echte RISC-V Programme ausführen zu können.
  \item Als Voraussetzung für (2) sollte der Simulator nicht \emph{zeilen}-,
  sondern \emph{address}basiert sein. Das bedeutet, dass Marken Adressen im
  Speicher und nicht Zeilen im Text referenzieren sollten. Hierfür müssten
  Instruktionen auch entsprechend in den Speicher assembliert werden.
  \item Als wohl wichtigste Eigenschaft sollte \erasim{}, im Vergleich zu
  \emph{Jasmin}, nicht nur für einen bestimmten Befehlssatz wie x86 oder ARM
  implementiert sein, sondern für beliebige Architekturen erweiterbar bleiben.
  \vspace{-0.4cm}
\end{senumerate}

% !TEX root = ../team-report.tex
% ERA-Großpraktikum: Team Bericht -- Organisatorisches (Roadmap)

\subsubsection{Der Zeitplan}
\label{team:orga-plan-time}
\vspace{-0.2cm}

Neben der Definition einer Vision füllten wir die erste Phase des
Entwicklungszeitraums mit der Ausarbeitung einer langfristigen \emph{Roadmap}
sowie einer kurzfristigen, kontinuierlichen Strategie für die Entwicklungsarbeit
über die nächsten sechs Monate. Wir befassten uns also mit allen Aspekten der
Zeitplanung unseres Projekt. Hierbei ließen wir uns insbesondere durch die in
der Industrie allgegenwärtigen \emph{agilen Methoden} leiten. Wir wollten also
auf Wasserfalldiagramme verzichten, die die Entwicklung bis ins kleinste Detail,
Schritt für Schritt, vorschreiben. Stattdessen fokussierten wir uns auf die
Bestimmung von monatlichen \emph{Meilensteinen} für jede Untergruppe und
einigten uns darauf, diese Meilensteine in zwei-wöchigen \emph{Sprints}
bewältigen zu wollen. Die vollständige Roadmap kann in der initialen
Spezifikation des Projekts gefunden werden. Eine komprimierte Version lautet:
\begin{sitemize}{-0.35cm}
  \bolditem{Juni}: Definition und Realisierung des Architekturinterfaces, parsen einfacher Befehle und Dummy Versionen von Speichermodell sowie GUI.
  \bolditem{Juli}: Implementierung und Parsen sämtlicher arithmetischer Befehle, Fertigstellung des Speichermodells und Herstellung erster Verbindungswege zwischen Modulen.
  \bolditem{August \& September}: Assemblierung von Befehlen zur Darstellung im Speicher, Parsen von Direktiven, Marken und Konstanten sowie vollständige Verbindung zwischen GUI und Core.
  \bolditem{Oktober}: Entwicklung von Pseudoinstruktionen, Direktiven, Makros und Serialisierungslogik.
  \bolditem{November}: Fertigstellung der visuellen I/O Module und projektweite Fehlerbehebung.
  \bolditem{Dezember \& Januar} Dokumentation und Bericht.
  \vspace{-0.3cm}
\end{sitemize}

% !TEX root = ../team-report.tex
% ERA-Großpraktikum: Team Bericht -- Gruppendynamik (Hangouts)

\subsection{Analyse und Diskussion}
\label{team:group-anal}

Im Rückblick war die Kommunikation auf Slack zweifellos der wichtigste Baustein
der Entwicklung von \erasim{}, da dieser erst sämtliche weitere Aspekte,
insbesondere die technischen, ermöglichte. Würde man die Gruppendynamik
bezüglich der Kommunikation auf Slack evaluieren, so könnte man sagen, dass die
aktive, regelmäßige Beteiligung an Diskussionen aller Mitglieder ein höchst
erfreuliches Merkmal unseres Entwicklungsalltags war. Unter den Mitgliedern, die
bis zum Schluss im \erasim{} Team waren (also nicht ausgetreten sind), gab
es niemanden, der nicht mehr oder minder täglich an Diskussionen teilnahm,
Fragen beantwortete, Fragen stellte und sein oder ihr Wissen mit anderen teilte.

Die Tatsache, dass die Kommunikation auf Slack nur schriftlich und nicht verbal
vonstatten ging, hat unsere Effektivität nicht beschränkt. Vielmehr finden wir,
dass Slack gegenüber verbaler Kommunikation den Vorteil aufweist, dass sich auch
schüchternere Mitglieder eher zu Wort trauten. Auch war der Nachteil verbaler
Kommunikation in einer Gruppe, dass nur eine Person zu einem Zeitpunkt sprechen
sollte und die Auswahl dieser Person bekanntlich oftmals chaotisch und
ineffizient ist, bei Chats nicht gegeben. Nichtsdestotrotz sind wir uns auch
einig, dass die Entscheidung, wichtigere oder komplexere Diskussionen auf
Hangouts auszulagern, sowie auch sporadisch persönliche Treffen zu veranstalten,
richtig war. Summa summarum würden wir unsere "Kommunikationshierarchie", also
$\text{Slack} \succ \text{Hangouts} \succ \text{Treffen}$, als äußerst effektiv
beurteilen.


\subsection{Workflow}
\label{team:orga-workflow}
\vspace{-0.3cm}

Diese Sektion gibt einen Einblick in den gesamten Workflow der Entwicklung von
\erasim{}. Mit \emph{Worfklow} sind die genutzten Programmiersprachen, Prinzipen
und Methoden sowie auch Hilfsmittel gemeint. Hierbei möchten wir insbesondere
darauf eingehen, wie leicht oder schwer es für das Team war, den Workflow zu
erlernen sowie auch evaluieren, wie effektiv unsere Entwicklung aufgrund des von
uns gewählten Workflows war. Wir wollen hierbei jedoch nicht in Detail auf die
\emph{praktischen} Aspekte des Workflows eingehen, da diese im Entwicklerbericht
genauestens ausgeführt sind.

% !TEX root = ../team-report.tex
% ERA-Großpraktikum: Team Bericht -- Organisatorisches (Languages)

\subsubsection{Sprachen}
\label{team:orga-workflow-lang}

Wie bei vielen großen Softwareprojekten üblich, haben wir für die Entwicklung
von \erasim{} eine Programmiersprache als Hauptwerkzeug genutzt, gegebenenfalls
aber nach dem Prinzip ``Use the Right Tool for the Job'' auch in anderen
Sprachen entwickelt. Unser Hauptwerkzeug war hierbei \emph{C++}\footnote{Wir
haben insbesondere darauf Wert gelegt, die neuen C++11 und C++14 Standards zu
nutzen.}, womit das gesamte ``Backend'' sowie auch bestimmte Teile des
``Frontends'', beispielsweise Datenmodelle, implementiert sind. Für die
Entwicklung der GUI wurde das \emph{Qt}\footnote{\url{https://www.qt.io}} C++
Framework verwendet, wobei wir die \emph{QML} und \emph{Javascript}
Skriptingsprachen zur Definition der Benutzeroberfläche nutzten. Zu den anderen
Sprachen zählt beispielsweise \emph{Python}, womit einige Tools zur
Produktivitätssteigerung sowie ein $\text{SASS} \rightarrow \text{JSON}$
Konvertierer als Teil unserer UI-Theming Implementierung geschrieben wurden,
\LaTeX{} zur Ausarbeitung von Spezifikationen sowie
\emph{SASS}\footnote{\url{http://sass-lang.com}} zur Definition von UI Themes.
Insgesamt wurden in diesen Sprachen ca. 35,000 Zeilen Code geschrieben. Grafik
\ref{fig:lang} zeigt hierzu den prozentuellen Anteil jeder Sprache an dieser
Zahl, wobei Tabelle \ref{tbl:lang} eine genauere Aufschlüsselung präsentiert.

\begin{figure}[h!]
  \centering
  \vspace{-0.2cm}
  \begin{tikzpicture}
    \newlength{\height}\setlength{\height}{0.4cm}\relax

    % C++
    \fill [Red, rounded corners=2pt]
          (0, 0) rectangle ++(10.9, \height);
    \draw [Red] (5, {\height + 0.5cm})
          node {\small\texttt{C++ (78.7\%)}};

    % QML
    \fill [Green, rounded corners=2pt]
          (10.8, 0) rectangle ++(1.9, \height);
    \draw [Green] (11.5, -0.5cm)
          node {\small\texttt{QML (13.7\%)}};

    % LaTeX
    \fill [orange, rounded corners=2pt]
          (12.6, 0) rectangle ++(1, \height);
    \draw [orange] (13.75, -0.5cm)
          node {\small\texttt{\LaTeX (4.4\%)}};

    % Python
    \fill [ProcessBlue, rounded corners=2pt]
          (13.5, 0) rectangle ++(0.5, \height);
    \draw [ProcessBlue] (14, {\height + 0.5cm})
          node {\small\texttt{Python (1.5\%)}};

    % Other
    \fill [Gray, rounded corners=2pt]
          (13.9, 0) rectangle ++(0.5, \height);
    \draw [Gray] (16, {\height/2})
          node {\small\texttt{Andere (1.7\%)}};
  \end{tikzpicture}
  \caption{Der prozentuelle Anteil der von uns genutzten Programmiersprachen  an der Gesamtzahl an Codezeilen.}
  \label{fig:lang}
\end{figure}

\pagebreak

\begin{table}
  \centering
  \begin{tabular}{lrrr}
    \textbf{Programmiersprache} & \textbf{Kommentarzeilen} & \textbf{Codezeilen} & \textbf{Gesamt} \\
    \midrule
    C\texttt{++} & 17981 & 24785 & 42766 \\
    QML & 2168 & 5949 & 8117 \\
    TeX & 588 & 5772 & 6360 \\
    YAML & 36 & 1959 & 1995 \\
    SASS & 11 & 666 & 677 \\
    Python & 242 & 543 & 785 \\
    CMake & 118 & 313 & 431 \\
    Bourne Shell & 27 & 153 & 180 \\
    JavaScript & 27 & 71 & 98 \\
    make & 1 & 4 & 5 \\
    DOS Batch & 0 & 2 & 2 \\
    \bottomrule
    Alle & 21199 & 40217 & 61416
  \end{tabular}
  \caption{Eine Aufschlüsselung der für die Entwicklung von \erasim{} geschriebenen Codezeilen in den von uns genutzten Programmiersprachen. Die erste Spalte nennt die Programmiersprache, die zweite Spalte listet die Anzahl an Kommentaren und die dritte Spalte die Anzahl an \emph{echten} Codezeilen in dieser Sprache.}
  \label{tbl:lang}
\end{table}

\textbf{Lernphase}

Eine besondere Herausforderung bei der Entwicklung von \erasim{} war, dass nur
zwei der (anfangs) zwölf Mitglieder zur effektiven Entwicklung ausreichendes
Wissen in C++ besaßen. Natürlich waren alle in der Programmierung bereits
erfahren, jedoch in anderen Sprachen wie Java oder C\#. Somit musste
der Großteil der Teammitglieder zu Beginn des Projekts darin investieren, C++ zu
erlernen. Da C++ für seine steile Lernkurve berüchtigt ist, war dies keine
kleine Aufgabe. Dennoch fanden alle Entwickler erfreulich schnell den Einstieg
in die Sprache und konnten bald darin produktiv werden. Hilfreich hierfür waren
unter anderem die Lektüre bestimmter Bücher wie \emph{C++
Primer}\footnote{\emph{C++ Primer}; Lipmann, Lajoie \& Moo (2012)} oder \emph{A
Tour of C++}\footnote{\emph{A Tour of C++}; Bjarne Stroustrup (2013)}, ein
eigener Kanal für C++-bezogene Fragen in Slack, gründliche Code Reviews
von Kameraden mit mehr C++ Erfahrung sowie auch die Einhaltung eines
strikten Style Guides, der die Feature-Vielfalt von C++ etwas reduziert.

\textbf{Evaluation}

In Retrospektive können wir als Team folgenden Schluss zur Wahl von C++ als hauptsächliche Programmiersprache für \erasim{} ziehen: C++ ist \emph{schwer}, aber \emph{effektiv}. Wir können einige Nachteile der Sprache identifizieren, können aber insbesondere auch ihre vielen Vorteile wertschätzen. Folgende Eigenschaften von C++ würden wir als positiv für die Effektivität und Effizienz unserer Entwicklung einschätzen:
\begin{sitemize}{-0.2cm}
  \item C++ ist umfangreich und bereitet viele Möglichkeiten, zum Beispiel im
  Bereich Threading, generischem Programmieren via Templates und Operator
  Überladungen,
  \item Die Hardwarenähe und Speicherverwaltungsmöglichkeiten der Sprache können
  in vielen Fällen Leistungsverbesserungen im Vergleich zu anderen Sprachen
  bringen,
  \item Gleichzeitig schützt das Prinzip von \emph{Resource Acquisition Is
  Initialization} (RAII) und Standardklassen wie \texttt{shared\_ptr} vor Memory
  Leaks und anderen Gefahren,
  \item Tools wie \texttt{clang-format}, \texttt{clang-tidy} oder
  \texttt{cpplint} beschleunigen die (fehlerfreie) Entwicklung.
\end{sitemize}

Gleichzeitig sind uns folgende Aspekte negativ aufgefallen:
\begin{sitemize}{-0.3cm}
  \item C++ ist umfangreich und bereitet viele Möglichkeiten, zum Beispiel im
  Bereich Threading, generischem Programmieren via Templates und Operator
  Überladungen,
  \item Fehlermeldungen können sehr komplex sein und deren Behebung oftmals mehr
  Zeit in Anspruch nehmen, als die eigentlich Entwicklung,
  \item Die Lernkurve ist sehr hoch und erschwert den Einstieg in die Sprache,
  \item Compiler, Libraries und Tools sind oftmals sehr stark plattformabhängig,
  was zu erhöhter Komplexität und erhöhtem Aufwand führen kann (im Vergleich
  zu JVM-basierten Sprachen),
  \item Das Fehlen eines Modulsystems (und die damit verbundene Notwendigkeit
  von \emph{Forward-Deklarationen}) kann bei fehlender Vorsicht schnell zu
  Problemen, wie \texttt{\#include}-Zyklen, führen.
\end{sitemize}

% !TEX root = ../team-report.tex
% ERA-Großpraktikum: Team Bericht -- Organisatorisches (Method)

\subsubsection{Methoden}
\label{team:orga-workflow-methods}
\vspace{-0.4cm}

Mit unseren \emph{Methoden} sind sämtliche Aspekte der Entwicklung von \erasim{}
gemeint, die nicht notwendigerweise von technischer oder praktischer Natur sind,
sondern die Gruppenorganisation oder Zeiteinteilung betreffen. Hierbei haben wir
uns von Anfang an insbesondere an den \emph{agilen Methoden} orientiert, welche
in der Industrie allgegenwärtig sind. Ganz allgemein schreiben die agilen
Methoden einen iterativen Entwicklungszyklus vor, wobei möglichst klein und
kompakt gehaltene \emph{Features} eines Projekts in kurzen, mehrwöchigen
\emph{Sprints} bearbeitet werden. Hierbei wird es vorgezogen, häufig kleine
Features zu herauszugeben, um schnell von Endbenutzern Feedback zu erhalten,
anstatt nur einige Male im Jahr (beispielsweise einmal pro Quartal) einen
großen, gewichtigen Release zu veröffentlichen.

Insbesondere zu Beginn des Entwicklungszeitraums, wo noch viele wichtige
(beispielsweise infrastrukturelle) Aufgaben zu bewältigen waren, war es für uns
enorm wichtig, einen rigorosen Sprintzyklus einzuhalten. Wir entschieden uns
hierbei für eine Periode von zwei Wochen. Das bedeutet, dass sich jede
Untergruppe alle zwei Wochen in Person oder via VoIP traf, um von jenen
Features, die für den nächsten Meilenstein auf unserer Roadmap vorgesehen waren,
eine realistische Anzahl auf jedes Mitglied zu verteilen. Die
Allokationsstrategie war hierbei weitestgehend jeder Gruppe selbst überlassen
--- die Projektleitung nahm nur ab und an zur Unterstützung bei
Designentscheidungen oder zum Überprüfen des Fortschritts an den Treffen teil.

Während der eigentlichen Entwicklung nutzten wir zwei weitere Elemente aus der
Welt der agilen Entwicklung: ein Kanban Board sowie unsere \emph{Definitions
of Done}. Ersteres ist ganz einfach eine (virtuelle) Tafel mit mehreren Spalten,
wobei Features während der Entwicklung von einer Spalte zur nächsten wandern.
Diese Spalten waren bei uns die folgenden vier:
\begin{senumerate}{-0.5cm}
  \bolditem{Open}: Das Feature steht noch offen und wird zurzeit von niemandem
  bearbeitet,
  \bolditem{In Progress}: Das Feature ist einem Mitglied zugeteilt und wird von
  diesem aktiv bearbeitet,
  \bolditem{Review}: Das Feature ist vorerst fertig bearbeitet und wird momentan
  von Teammitgliedern reviewed,
  \bolditem{Done}: Das Feature hat den Bearbeitung \cyclearrow Review Zyklus verlassen und liegt auf dem Master Branch.
\end{senumerate}
\vspace{-0.5cm}

Diese vier Spalten folgen auch grob unseren Definitions of Done. Allgemein
definieren Definitions of Done bei der agilen Softwareentwicklung jene Schritte,
die ein Feature bzw. ein Stück Code durchschreiten muss, um von A bis Z zu
gelangen, also von der Konzeption bis zum Master Branch. Insgesamt hatten wir
sieben solcher Schritte:
\begin{senumerate}{-0.5cm}
  \item Code wurde geschrieben,
  \item \emph{Unit Tests} stehen und passen,
  \item Der Code ist dokumentiert,
  \item Der Code ist formatiert,
  \item Ein \emph{Pull Request} wurde auf GitHub geöffnet,
  \item Code-Reviewer geben ihr OK,
  \item Merge des \emph{Feature Branch} in den \emph{Master Branch}.
  \vspace{-0.5cm}
\end{senumerate}

Hierbei möchten wir insbesondere Punkt (2), das Schreiben von (Unit-)Tests als
besonders wichtig hervorheben. Wir haben von der ersten Zeile Code an großen
Wert darauf gelegt, eine robuste Test Suite für sämtliche Module zu entwickeln.
Ausgenommen hiervon ist lediglich das GUI Modul, da keine praktikablen Werkzeuge
zum Testen von Frontend QML Code existieren. Für \emph{Arch}, \emph{Core} und
\emph{Parser} halfen uns die über 280 individuellen Unit Tests jedoch dabei, die
Korrektheit unserer Interfaces und Implementierungen zu verifizieren und
Resultate auf verschiedenen Systemen und Plattformen reproduzierbar zu machen.
Es sei angemerkt, dass unsere Entwicklung zwar Tests ernst nahm und davon
profitierte, jedoch nicht zwingend \emph{test-driven} war. Hiermit ist gemeint,
dass Tests meist nach dem Code  geschrieben wurden und nicht umgekehrt, wie es
das \emph{Test Driven Development} (TDD) vorschreibt.
\vspace{-0.7cm}

% !TEX root = ../team-report.tex
% ERA-Großpraktikum: Team Bericht -- Organisatorisches (Tools)

\subsubsection{Hilfsmittel}
\label{team:orga-workflow-tools}

Neben den von uns genutzten Programmiersprachen und Entwicklungsmethoden war
unser Workflow auch stark von unseren \emph{Hilfsmitteln} bzw. \emph{Tools}
geprägt. Als \emph{Hilfsmittel} bezeichnen wir jene Programme oder Services, die
offline oder online unsere Entwicklung bzw. die übrigen Aspekte unseres
Workflows unterstützten. Hierbei lassen sich unsere Hilfsmittel grob in zwei
Gruppen teilen: technische Hilfsmittel und organisatorische Hilfsmittel.

Technische Hilfsmittel halfen uns insbesondere bei der Bewältigung der
Feature-Vielfalt und Komplexizität der C++-Sprache. In Retrospektive war der
Einsatz der weiter unten erläuterten technischen Hilfsmittel ein enorm wichtiger
Teil unserer Entwicklung. Ebenso wird auch in der Industrie die Meinung sehr
stark vertreten, dass die Verfügbarkeit korrekter und effektiver Tools oft
ebenso wichtig ist, wie die Funktionalität der Sprache selbst. So ist es
insbesondere erfreulich, dass in den letzten Jahren eine Vielzahl von
Hilfsmitteln aus dem LLVM/\texttt{clang} Compiler-Projekt hervorgetreten sind,
die C++ Entwicklern, beispielsweise durch statische Analyse, helfen. Unter den
von uns genutzten technischen Tools möchten wir drei insbesondere hervorheben:
\texttt{clang-format}, \texttt{cpplint} und Travis CI. Diese lassen sich
detaillierter wie folgt beschreiben:

\begin{itemize}

  \item \texttt{clang-format} ist ein Hilfsmittel aus der Familie von
  \texttt{clang} Tools, das zur Einhaltung eines Style Guides\footnote{Als
  \emph{Style Guide} bezeichnen wir ein Dokument, dass im Falle mehrerer
  Möglichkeiten für die Formatierung oder Implementierung von Code eine dieser
  Möglichkeiten bevorzugt und diese Bevorzugung schriftlich festlegt.} dient.
  Wir betrachten die Instandhaltung eines rigorosen Style Guides als einen
  notwendigen und essentiellen Aspekt der Entwicklung von C++ Code. Die
  Begründung hierfür liegt darin, dass C++ im Bezug auf die Formatierung eine
  enorm laxe Sprache ist. Ebenso kann es manchmal nicht nur bei der
  Formatierung, sondern auch bei der Implementierung eines Stück Codes, mehrere
  Möglichkeiten geben. So ist beispielsweise überall dort, wo eine Referenz
  möglich ist, ebenso auch ein Zeiger, ohne Verlust von Semantik, nutzbar. Ein
  Style Guide hilft erfahrungsgemäß auch bei der Integration neuer Entwickler,
  da man diesen die Entscheidungsfreiheit (bzw. -notwendigkeit) gewissermaßen
  entnimmt. Das \texttt{clang-format} Tool hilft dabei, die Formatierung von
  Programmcode anhand vordefinierter Regeln zu automatisieren. Dies erleichtert
  zum einen die Standardisierung von Formatierungsregeln, entnimmt dem
  Programmierer aber ebenso das manuelle Einrücken von Code oder ähnliche
  mühsame Aufgaben.

  \item \texttt{cpplint} ist ein \emph{Linter}\footnote{Mit einem \emph{Linter}
  ist ein Programm gemeint, dass Code statisch untersucht und Hinweise auf
  mögliche Defekte liefert.}, der von Google Inc. zur Einhaltung des Google C++
  Style Guide entwickelt wurde. Im Unterschied zu \texttt{clang-format} warnt
  \texttt{cpplint} nicht nur vor syntaktischen, sondern auch vor semantischen
  Fehlern. Vergisst man beispielsweise, Konstruktiven, die nur einen Parameter
  nehmen, mit \texttt{explicit} zu markieren, so wird man von \texttt{cpplint}
  darauf aufmerksam gemacht. Dies verhindert implizite Konvertierungen, die
  unter Umständen zu ungeahnten Problemen führen können.

  \item \emph{Travis CI} (kurz \emph{Travis}) ist ein Online-Test-Server,
  welcher Unit Tests in isolierten Umgebungen auf Linux Maschinen ausführt. Wir
  haben Travis hierbei so konfiguriert, dass nach jedem Commit auf das GitHub
  Repository die Kompilierung unseres Codes und Ausführung der Tests gestartet
  werden. Travis hat uns insbesondere dabei geholfen, den Erfolg des
  Kompilierens und unserer Tests auf einer ``unabhängigen'' Platform laufend zu
  verifizieren. So kann stets von jedem die Gesundheit des Projekts überprüft
  werden, ohne selbst den Simulator kompilieren und ausführen zu müssen.

\end{itemize}

Neben den obigen technischen Werkzeugen haben wir bei der Entwicklung von
\erasim{} auch von einer Reihe von Hilfsmitteln Gebrauch gemacht, die uns bei
organisatorischen Aspekten halfen. Zum einen möchten wir hier
\emph{Waffle}\footnote{\url{http://waffle.io}} nennen, welches wir als Online
Kanban Board nutzten. Zum anderen war wohl das wichtigste Hilfsmittel während
der gesamten Entwicklung von \erasim{} unser Kommunikationstool
\emph{Slack}\footnote{\url{http://slack.com/}}, welches uns erlaubt hat, rund um
die Uhr miteinander die Realisierung des Simulators zu koordinieren.

