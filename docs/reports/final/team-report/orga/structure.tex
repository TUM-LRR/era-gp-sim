% !TEX root = ../team-report.tex
% ERA-Großpraktikum: Team Bericht -- Organisatorisches (Struktur)

\subsection{Struktur}
\label{team:orga-structure}

Die Mannschaft hinter der Entwicklung des \erasim{} Simulators bestand anfangs
aus 12 und schlussendlich aus 10 Studierenden. Das Team hat sich schon früh in
verschiedene, thematisch abgeschlossene Untergruppen aufgeteilt, um die
Bearbeitung der zahlreichen und vielfältigen Aufgaben effektiver zu gestalten.
Des Weiteren wurde aus der Gruppe ein Mitglied als Projektleiter bestimmt, um
die Entwicklung des Simulators zu koordinieren, als Ansprechperson für
Projektbetreuer zu dienen sowie auch bestimmte infrastrukturelle Entscheidungen,
wie die Nutzung des Online-Test-Servers \emph{Travis CI}, zu treffen.

Die Bestimmung der einzelnen Untergruppen und Aufteilung des Teams in diese
wurde während einem der ersten Treffen vollzogen und hat sich letztendlich als
eine der wichtigsten und besten "Design-Entscheidungen" für die effektive
Verwirklichung des Simulators herausgestellt. Konkret waren diese Untergruppen
\emph{Arch, Core, Parser} und \emph{GUI} genannt. Jede Gruppe bestand aus zwei
bis vier Mitgliedern und behandelte einen grundlegenden Aspekt der Entwicklung.
Genauer befasste sich
\begin{sitemize}{-0.5cm}
  \item \textbf{Arch}, mit der Abstraktion und Beschreibung von allgemeinen Befehlssätzen (ISAs) sowie der konkreten Implementierung der RISC-V ISA;
  \item \textbf{Core}, mit architekturunabhängigen Implementierung eines Speichermodells sowie der Verbindung der Schnittstellen sämtlicher Module;
  \item \textbf{Parser}, mit der syntaktischen Beschreibung eines RISC-V Assemblerdialekts sowie Implementierung eines Codeinterpreters und
  \item \textbf{GUI}, mit der Entwicklung der grafischen Benutzeroberfläche des Simulators.
\end{sitemize}
\vspace{-0.4cm}

Die Aufteilung in diese Untergruppen hat es uns erlaubt, uns jeweils thematisch
auf einen Aspekt der Simulatorentwicklung zu fokussieren, Expertenwissen
innerhalb jeden Gebietes anzuhäufen und auszutauschen sowie Aufgaben effektiv zu
verteilen. Während der initialen Strukturierung des Teams hatten wir uns auch
überlegt, einen Repräsentanten aus jeder Untergruppe zu wählen. Da die Gruppe zu
Beginn mit 12 Personen eine nicht unbeachtliche Größe hatte, war diese
Überlegung durchaus sinnvoll. Um die Hierarchie jedoch nicht unnötig zu erhöhen,
haben wir uns letztendlich dagegen entschieden und nur einen Leiter für die
gesamte Gruppe, nicht aber jede Untergruppe gewählt. Dies hat sich als
vollkommen plausible Entscheidung herausgestellt.

Der Gruppenleiter wurde noch im Mai 2016 demokratisch bestimmt. Hierbei standen
zwei selbsternannte Kandidaten, Daniel Riedel und Peter Goldsborough, zur Wahl.
Letzterer wurde schließlich zum Leiter gewählt.
