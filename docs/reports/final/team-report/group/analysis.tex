% !TEX root = ../team-report.tex
% ERA-Großpraktikum: Team Bericht -- Gruppendynamik (Hangouts)

\subsection{Analyse und Diskussion}
\label{team:group-anal}

Im Rückblick war die Kommunikation auf Slack zweifellos der wichtigste Baustein
der Entwicklung von \erasim{}, da dieser erst sämtliche weitere Aspekte,
insbesondere die technischen, ermöglichte. Würde man die Gruppendynamik
bezüglich der Kommunikation auf Slack evaluieren, so könnte man sagen, dass die
aktive, regelmäßige Beteiligung an Diskussionen aller Mitglieder ein höchst
erfreuliches Merkmal unseres Entwicklungsalltags war. Unter den Mitgliedern, die
bis zum Schluss im \erasim{} Team waren (also nicht ausgetreten sind), gab
es niemanden, der nicht mehr oder minder täglich an Diskussionen teilnahm,
Fragen beantwortete, Fragen stellte und sein oder ihr Wissen mit anderen teilte.

Die Tatsache, dass die Kommunikation auf Slack nur schriftlich und nicht verbal
vonstatten ging, hat unsere Effektivität nicht beschränkt. Vielmehr finden wir,
dass Slack gegenüber verbaler Kommunikation den Vorteil aufweist, dass sich auch
schüchternere Mitglieder eher zu Wort trauten. Auch war der Nachteil verbaler
Kommunikation in einer Gruppe, dass nur eine Person zu einem Zeitpunkt sprechen
sollte und die Auswahl dieser Person bekanntlich oftmals chaotisch und
ineffizient ist, bei Chats nicht gegeben. Nichtsdestotrotz sind wir uns auch
einig, dass die Entscheidung, wichtigere oder komplexere Diskussionen auf
Hangouts auszulagern, sowie auch sporadisch persönliche Treffen zu veranstalten,
richtig war. Summa summarum würden wir unsere "Kommunikationshierarchie", also
$\text{Slack} \succ \text{Hangouts} \succ \text{Treffen}$, als äußerst effektiv
beurteilen.
