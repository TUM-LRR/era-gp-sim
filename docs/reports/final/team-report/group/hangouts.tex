% !TEX root = ../team-report.tex
% ERA-Großpraktikum: Team Bericht -- Gruppendynamik (Hangouts)

\vspace{-0.2cm}
\subsection{Hangouts}
\label{team:group-slack}
\vspace{-0.2cm}

Mag die Kommunikation über Slack in unserem Team auch schon sehr effektiv
gewesen sein, so ist natürlich dennoch ein gewisser Grad an persönlichem,
verbalem Austausch für ein gutes Klima im Team unerlässlich. Besonders
Brainstorming, das Besprechen von Designs für Softwarekomponenten oder weitere
Diskussionen ließen sich besser mündlich statt schriftlich erledigen. In unserem
Team konkretisierte sich die verbale Kommunikation insbesondere virtuell, über
\emph{Hangouts}\footnote{\emph{Google Hangouts}
(\url{http://hangouts.google.com}) ist ein VoIP-Service von Google, Inc.}. Es
traf sich meist mehrmals im Monat jede Untergruppe per Hangout, um den nächsten
Sprint bzw. die nächsten Schritte zu planen. Bei diesen Hangouts stand es auch
Mitgliedern anderer Gruppen frei, teilzunehmen, um beispielsweise über die
Integration zweier Module zu diskutieren. Teilweise nahm auch der
Gesamtgruppenleiter an den Hangouts anderer Gruppen bei, um die Entwicklung zu
überprüfen und womöglich technische Ratschläge zu geben. Letztlich gab es zu
seltenen (aber wichtigen) Anlässen auch Hangouts, bei denen alle Mitglieder
präsent waren. Diese waren aufgrund der Größe der Gruppe jedoch organisatorisch
schwer zu planen und zu koordinieren.
