% !TEX root = ../team-report.tex
% ERA-Großpraktikum: Team Bericht -- Gruppendynamik

\section{Gruppendynamik und Kommunikation}
\label{team:group}

Die wohl wichtigste Säule bei gemeinsamer Arbeit in einem Team, ob für die
Entwicklung eines Assembler-Simulators oder jedes sonstige Bestreben, ist die
Kommunikation innerhalb der Gruppe. Insbesondere wird diese Säule essentiell,
wenn sich das Team nicht jeden Tag im Büro begegnet, sondern verstreut und
verteilt, womöglich sogar in verschiedenen Teilen der Welt, am gemeinsamen Ziel
arbeitet. Da die meisten Mitglieder des \erasim{}-Teams aufgrund des
Studentenlebens, wie bereits besprochen, nur unregelmäßig verfügbar waren, und
der Gruppenleiter auch den Großteil des Entwicklungszeitraums in London Praktika
absolvierte, war auch für \erasim{} die Instandhaltung effektiver
Kommunikationskanäle und regelmäßiger, virtueller oder persönlicher, Treffen von
höchster Wichtigkeit. Die folgenden Abschnitte wollen einen Einblick geben, auf
welche Weise uns diese Instandhaltung via \emph{Slack}, \emph{Hangouts} und
persönlichen Treffen gelungen ist. Hierbei wollen wir auch die empfundene
Dynamik im Team erörtern.

% !TEX root = ../team-report.tex
% ERA-Großpraktikum: Team Bericht -- Gruppendynamik (Slack)

\subsection{Slack}
\label{team:group-slack}

Unser wichtigster Kommunikationskanal war die Teamchat-Software \emph{Slack}.
Slack erlaubt sowohl persönliche, "vier-Augen" Chats, als auch die Aufspaltung
der Diskussion in themenspezifische \emph{Kanäle} (engl. \emph{Channels}). Bei
der Entwicklung von \erasim{} verteilten wir unsere Kommunikation hierbei auf
zwei Arten von Kanälen: \emph{Teamkanäle} und \emph{Topickanäle}.

Teamkanäle waren jene Kanäle, wo jede Untergruppe von \erasim{}, also
\emph{Arch, Core, Parser} und \emph{GUI}, Diskussionen führen konnten, die sich
rein mit den Aspekten der Entwicklung von \erasim{} befassten, die jeder Gruppe
entsprachen. In diesen Chats wurde über den Fortschritt sowie aufgetretene
Probleme eines einzelnen Mitglieds der Untergruppe berichtet, wichtige
Informationen zur Entwicklungsumgebung diskutiert (z.B. eine neue Qt Version
oder die Veröffentlichung einer neuen RISC-V Spezifikation) oder auf scheiternde
Tests aufmerksam gemacht. Wichtig hierbei war insbesondere, dass diese Teamchats
nicht für die Untergruppe privat, sondern für das ganze \erasim{} Team
öffentlich war. Somit konnten auch von Mitgliedern anderer Untergruppen Fragen
gestellt bzw. wichtige Entwicklungen mitgeteilt werden. Wollte beispielsweise
ein Mitglied des Core wissen, wie man am besten über die Registergröße der
momentan geladenen Architektur erfährt, so konnte er oder sie diese Frage im
Arch-Chat stellen und ihm oder ihr wurde meist auch schnell geholfen.

Neben Teamkanälen wurde auch kräftig in den Topickanälen diskutiert. Ein
Topickanal beschränkte sich auf ein spezifisches Gebiet der Entwicklung oder
anderen Themen. Ein wichtiger und vielgenutzter Kanal war beispielsweise der
C\texttt{++} Channel. In diesem konnten Fragen zu C++ gestellt, stilistische
Aspekte des Google Style Guides diskutiert oder auch bezüglich kryptischen
Kompilierfehlern nachgefragt werden. Mitglieder mit mehr C++ Erfahrung konnten
diese Fragen dann beantworten oder auf entsprechende Ressourcen hinweisen. Ein
anderer Topickanal war \emph{report} genannt, in welchem die Planung und
Fertigstellung des ersten und finalen Berichts besprochen wurde. Ein weiteres
nützliches Feature von Slack ist die Integration mit externen Dienstleistungen.
So gab es einen weiteren Topickanal, genannt \emph{notifications}, in welchen
wir unseren Test-Server \emph{Travis} integriert hatten. Immer wenn unsere Tests
auf Travis scheiterten oder erfolgreich waren, erhielten wir in diesem Kanal
eine Benachrichtigung. Dadurch musste man auch nicht mehr die Website von Travis
besuchen, um über den Status der Tests zu erfahren.

In Summe waren gegen Ende der Entwicklung auf Slack 12 Kanäle offen --- 4
Teamkanäle und 8 Topickanäle. In diesen Kanälen und über private Chats wurden
über den gesamten Entwicklungszeitraum hinweg gesammelt 14,800 Nachrichten
ausgetauscht. \autoref{fig:slack} schlüsselt diese Zahl weiter auf.

\begin{figure}[h!]
  \centering
  \begin{tikzpicture}[thick]

    % Public Channels
    \fill [SkyBlue]
          (0, -2) -- (0, 0) -- (65:2cm)
          arc [radius=2cm, start angle=65, end angle=270];

    % Direct Messages
    \fill [orange]
          (0, -2) -- (0, 0) -- (65:2cm)
          arc [radius=2cm, start angle=65, end angle=-90];

    % Pie border
    \draw (0, 0) circle [radius=2cm];

    % Sector border
    \draw (0, -2) -- (0, 0) -- (65:2cm);

    % Labels
    \node at (-1, 0) [align=center] {57\% \\ (8,436)};
    \node at (+1, 0) [align=center] {43\% \\ (6,364)};

    % Legend
    \path (5, 0.5) coordinate [fill, orange, circle, inner sep=3pt] (public)
          node [right] {\hspace{0.1cm} Öffentliche Kanäle};
    \fill [SkyBlue] (public)+(0, -0.65cm) circle [radius=4pt]
          node [right] {\hspace{0.2cm}\color{black} Private Nachrichten};
    \draw [thick, rounded corners=1pt]
          (public)+(-0.4, +0.5) rectangle ++(4, -1.25);

  \end{tikzpicture}
  \caption{Eine Aufschlüsselung der über Slack versendeten
  Nachrichten. Der relative Anteil und die absolute Anzahl an Nachrichten in
  öffentlichen Kanälen sind in Orange eingezeichnet, private Nachrichten in
  Türkis. Insgesamt wurden über 10 Monate hinweg 14,800 Nachrichten verschickt.}
  \label{fig:slack}
\end{figure}

\vspace{-0.5cm}

% !TEX root = ../team-report.tex
% ERA-Großpraktikum: Team Bericht -- Gruppendynamik (Hangouts)

\subsection{Hangouts}
\label{team:group-slack}

Mag die Kommunikation über Slack in unserem Team auch schon sehr effektiv
gewesen sein, so ist natürlich dennoch ein gewisser Grad an persönlichem,
verbalem Austausch für ein gutes Teamklima unerlässlich. Dies war bei uns
insbesondere für Brainstorming, dem Besprechen von Designs für
Softwarekomponenten oder die Diskussion über andere, komplexere Themen zu
diskutieren. In unserem Team konkretisierte sich die verbale Kommunikation
insbesondere virtuell, über \emph{Hangouts}\footnote{\emph{Google Hangouts}
(\url{http://hangouts.google.com}) ist ein VoIP-Service von Google, Inc.}. Es
traf sich meist mehrmals im Monat jede Untergruppe per Hangout,

\vspace{0.2cm}
\pagebreak

um den nächsten Sprint bzw. die nächsten Schritte zu planen. Bei diesen Hangouts
war es auch Mitgliedern anderer Gruppen frei, teilzunehmen, um beispielsweise
über die Integration zweier Module zu diskutieren. Teilweise nahm auch der
Gesamtgruppenleiter an den Hangouts anderer Gruppen bei, um die Entwicklung zu
überprüfen und womöglich technische Ratschläge zu geben. Letztlich gab es zu
seltenen (aber wichtigen) Anlässen auch Hangouts, bei welchen alle Mitglieder
präsent waren. Diese waren aufgrund der Größe der Gruppe jedoch organisatorisch
schwer zu planen und zu koordinieren.

% !TEX root = ../team-report.tex
% ERA-Großpraktikum: Team Bericht -- Gruppendynamik (Hangouts)


\subsection{Persönliche Treffen}
\label{team:group-pers}

Neben Diskussionen auf Slack oder Hangouts gab es auch einige Male persönliche
Treffen, bei denen sich die ganze Gruppe und manchmal auch die
Projektbetreuung persönlich trafen. Diese Treffen fanden ganz am Anfang des
Großpraktikums, im April und Mai, häufiger, im späteren Teil des
Entwicklungszeitraums weniger häufig statt. Es gibt zwei hauptsächliche Gründe,
wieso wir die Kommunikation über Slack und Hangouts bevorzugten. Zum einen ist
die Planung eines Treffens von 10 bis 12 Personen platz- und zeittechnisch eine
große Herausforderung. Zum anderen ergab sich schlussendlich aus der physischen
Präsenz der Mitglieder auch kein nennenswerter Vorteil gegenüber einem viel
praktikableren virtuellen Treffen via Hangouts.

% !TEX root = ../team-report.tex
% ERA-Großpraktikum: Team Bericht -- Organisatorisches (Analysis)

\subsubsection{Analyse und Diskussion}
\label{team:orga-plan-anal}

Im Vergleich zwischen dem von uns festgelegten Plan und der Manier, in welcher wir das Projekt tatsächlich bewältigt haben, lassen sich zwei grundlegende Beobachtungen treffen:
\vspace{-0.2cm}
\begin{enumerate}
  \item Aufgrund mehrerer Faktoren ist die Fertigstellung einzelner Meilensteine, vor allem gegen Ende der Arbeitszeit, von obigem Zeitplan merkbar abgewichen,
  \item Dennoch haben wir unsere Vision bis auf wenige Aspekte realisiert und ein vollständiges Produkt entwickeln und abliefern können.
\end{enumerate}

Zusammenfassend kann man sagen, dass wir zwar alle initial festgelegten
Meilensteine erreicht haben, jedoch nicht immer in den geplanten Zeiträumen.
Beispielsweise wurde das Speichermodell nicht wie vorgesehen bereits im Juli,
sondern erst im September fertiggestellt. Ebenso wurde die (korrekte)
Assemblierungslogik zur realitätsnahen Darstellung von Instruktionen im Speicher
im Dezember und nicht bereits im August vollendet. Auch verschob sich ein nicht
unbeachtlicher Teil der Entwicklung der GUI auf Dezember und Januar. Grafik
\ref{fig:commit-history} zeigt ein Histogramm der Commitzahlen
auf das GitHub Repository des Projekts, während Grafik \ref{fig:time-frame}
einen Eindruck über die Diskrepanz zwischen geplanter und tatsächlicher
Fertigstellung von Meilensteinen gibt.

\begin{figure}[b!]
  \centering
  \includegraphics[scale=0.45]{figures/commit-history}
  \caption{Ein Histogramm der gesammelten Commitzahlen auf das GitHub Repository von \erasim{} während des Entwicklungszeitraumes. Quelle: {\small\url{https://github.com/TUM-LRR/era-gp-sim/graphs/contributors}}, letzter Zugriff \today.}
  \label{fig:commit-history}
\end{figure}

\pagebreak
Wir behaupten, dass sich unsere Verplanung bei manchen der Meilensteinen auf drei Ursachen zurückzuführen lässt:

\begin{enumerate}

  \emphitem{Mangel an Erfahrung und Einschätzungsvermögen}. Da zum Zeitpunkt
  unserer Planungsphase kein Mitglied des Teams Erfahrung mit der ``A-Z''
  Entwicklung eines solch anspruchsvollen Projekts hatte, behaupten wir, dass es
  uns schwer fiel, den Arbeitsaufwand für bestimmte Features realistisch
  einzuschätzen. Dies ist insbesondere bei der Entwicklung der GUI merkbar, wo
  gegen Ende der Entwicklungszeit noch viele weitere Hände, als ursprünglich
  geplant, mithelfen mussten, um ein korrektes und intuitives User Interface zu
  gestalten. Könnten wir heute, mit unserer neu gewonnenen Erfahrung, die
  Entwicklungsschritte neu priorisieren, so würden wir bestimmte Aufgaben gewiss
  anders gewichten. \emphitem{Irreguläre Verfügbarkeit}. Während (1) zu Fehlern
  in der \emph{Vorarbeit} geführt hat, war es insbesondere die unregelmäßige
  Verfügbarkeit von Mitgliedern während der Entwicklungsphase, die uns bei der
  \emph{Exekution} der geplanten Schritte Zeit gekostet hat. Bestimmte
  Ereignisse, die im studentischen Alltag auftreten können, seien es
  Prüfungszeiten, Sommerferien oder Praktika, haben es schwer gemacht,
  regelmäßige Sprints zu planen und aufrechtzuerhalten. Beispielsweise erkennt
  man an Grafik \ref{fig:commit-history}, dass gegen Ende Juli und Anfang August
  das Vorbereiten für Klausuren höhere Priorität als die Entwicklung des
  Simulators hatte. Ebenso hinkte im November die Produktivität aufgrund von
  Lern- oder Praktikumsstress sichtlich nach. Diese Unregelmäßigkeit ist
  schlicht ein Resultat davon, dass wir als Studierende an dem Projekt nie
  \emph{vollzeit} arbeiten konnten. Es ist natürlich viel einfacher in einem
  Team von Vollzeitangestellten (bspw. in einer Firma) einen rigorosen
  Entwicklungszyklus einzuplanen und einzuhalten.

  \emphitem{Austritt einiger Mitglieder}. Letztlich identifizieren wir als eine
  weitere Produktivitätsschranke den Austritt zweier Teamkameraden im dritten
  Viertel der Entwicklungszeit. Diesen Mitgliedern waren Aufgaben zugeteilt,
  dessen Abschluss sie uns (verspätet) zusicherten, an welchen sie aber nur
  sporadisch arbeiteten und niemals abschlossen. Als diese Personen dann doch,
  relativ spät, aus dem Projekt vollkommen ausschieden, blieben deren Features
  übrig. Da die bestehende Arbeit dieser Entwickler zum Teil undokumentiert oder
  inkorrekt war, musste noch zusätzliche Arbeitszeit von anderen Teammitgliedern
  investiert werden, um diese Features verspätet aber doch abzuschließen. Im Rückblick hätten wir wohl schneller klarstellen sollen, ob diese Mitglieder sich wirklich weiter für \erasim{} engagieren wollen. Wäre das ``Nein'' als Antwort früher klargeworden, hätten wir deren Aufgaben schneller neu verteilen und fertigstellen können.

\end{enumerate}

\begin{figure}[h!]
  \centering
  \begin{tikzpicture}[thick]
    \draw [|->] (0, 0) -- ++(16, 0);

    \newcount\x\relax
    \x=1\relax
    \foreach \month in {%
      Juni, Juli, August, September, Oktober, November, Dezember, Januar} {
      \draw (\x, 0.125) -- (\x, -0.125) node [below] {\small\month};
      \global\advance\x by 2\relax
    }

    \foreach \name/\planned/\actual/\y in {%
      Speichermodell/3/7/1.6,%
      Assemblierung/5/13/2.4,%
      RISC-V Instruktionen/6/3/3,%
      Direktiven und Makros/5/9/0.8,%
      Ausgabeanzeigen/11/15/3,%
      Verbindung aller Module/3/9/4%
    } {
      \path (\planned, \y) coordinate [fill, circle, inner sep=1.2pt] (p\name);
      \path (\actual, \y) coordinate [fill, circle, inner sep=1.2pt] (a\name);
      \ifnum\actual<\planned
        \newcommand{\arrowcolor}{NavyBlue}
      \else
      \newcommand{\arrowcolor}{Red}
      \fi
      \draw [->, very thick, dotted, \arrowcolor]
            (p\name) -- (a\name) node [above, midway]
            {\color{black}\small\name};
    }

  \end{tikzpicture}
  \caption{Diese Grafik beschreibt die Fertigstellung einer Auswahl an Meilensteinen. Hierbei werden für jeden Meilenstein der geplante und tatsächliche Zeitpunkt der Fertigstellung dargestellt. Pfeile verbinden diese Zeitpunkte, wobei Pfeilspitzen auf das tatsächliche Abschlussdatum zeigen.}
  \label{fig:time-frame}
\end{figure}

