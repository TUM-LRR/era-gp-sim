% ERA-Großpraktikums: Benutzeranleitung -- Hilfe

\section{Fehlermeldungen, Warnungen und Hilfetexte}

Im folgenden werden Fehlermeldungen aufgelistet, die bei der Benutzung des Simulators auftreten können.
Eine Übersicht über die Übersetzungsfehler von RISC-V gibt es in \autoref{user-manual-riscv-errors}.

\begin{itemize}
    \item \textbf{Fehler beim Starten von \erasim} Falls einer dieser Fehler
    auftritt, deutet dies auf eine kaputte oder fehlende Installation des
    \texttt{\$Home/.erasim} Ordners hin. Daher wird in diesem Fall Empfohlen,
    eine Neuinstallation mit Hilfe des Installers vorzunehmen.
        \begin{itemize}
            \item Could not find settings directory
            \item Could not open settings file for loading
            \item Contents of settings are empty. Why?
            \item {[\dots]} when creating settings file
            \item Could not open settings file for writing
            \item Could not write to settings file
        \end{itemize}
    \item \textbf{Fehler beim Laden von Themes} Falls beim Laden einer Theme
    einer der folgenden Fehler auftritt, deutet dies auf eine fehlerhafte
    Installation der Themes oder auf generell fehlerhafte Themes hin. Themes
    werden aus dem \texttt{\$HOME/.erasim/themes} Ordner geladen. Eine
    Neuinstallation sollte diese Fehler beheben. Zu beachten ist, dass Fehler
    dieser Art auch beim Programmstart auftreten können, da zu diesem Zeitpunkt
    auch eine Theme geladen wird.
        \begin{itemize}
            \item Could not find theme directory
            \item Could not find theme {[\dots]}
            \item Could not open theme {[\dots]}
            \item Contents of theme {[\dots]}
        \end{itemize}
    \item \textbf{Fehler beim Laden/Speichern von Snapshots oder Textdateien}
    Falls in dieser Situation ein Fehler auftritt, sollte überprüft werden, ob
    die entsprechenden Dateien existieren und passende Zugriffsrechte haben.
    Snapshots werden unter dem in den Einstellungen festgelegten Pfad
    gespeichert.
    \item \textbf{Fehler bei der Ausführung des Programms} Während der
    Ausführung können \emph{Runtime Errors} auftreten. Dies deutet auf einen
    Fehler im Assembler-Code hin, beispielsweise eine falsche Sprungadresse.
\end{itemize}
