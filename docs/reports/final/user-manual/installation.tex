% ERA-Großpraktikums: Benutzeranleitung -- Installation

\section{Installation}

\erasim{} läuft unter Linux, OS X und Windows und wird primär über einen Installer ausgeliefert. Dieser kann unter \url{...}\todo{Link zu Website oder Github-Release} heruntergeladen werden. Momentan ist \erasim{} unter Windows nur als 32 Bit Applikation verfügbar.\\
Das Suffix \texttt{linux} bezeichnet den Installer für Linux, der für jede Distribution funktionieren sollte; \texttt{macos} den für MacOS (OS X) und \texttt{winx86} den Installer für Windows.\\
Die Installer liefern passende Abhängigkeiten, wie z.B. die verwendeten Qt-Grafikbibliotheken, mit. Daher benötigt \erasim{} am Installationsort bis zu ~200MiB freien Speicher. Wir verzichten bewusst auf eine Installer-Version, die z.B. unter Linux bereits vorhandene Qt Installationen verwendet und diese dann nicht mitliefert. Einige Features in Qt werden von höheren Qt-Versionen nicht mehr unterstützt, daher liefern wir die passende Qt-Version mit dem Simulator mit.

Im Laufe des Installationsprozesses bietet der Installer die Möglichkeit, einen Startmenüeintrag unter Windows bzw. einen Desktop-Eintrag unter Unix zu erstellen.\\
Möchte man \erasim{} über die Kommandozeile starten, kann man den Installationsordner des Simulators zur \texttt{PATH} Umgebungsvariable hinzufügen.