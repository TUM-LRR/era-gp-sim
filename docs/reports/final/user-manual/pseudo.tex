% !TEX root = ../combined/final.tex
% ERA-Großpraktikum: Benutzeranleitung -- Pseudoinstruktionen

\newcommand{\pseudoinst}[3]{
  \centering
  \begin{tikzpicture}[thick]
    \node [align=center, text width=4.5cm] (inst) at (0, 0) {\texttt{#1}};
    \node [align=left, text width=12cm] (op) at (8.5, 0) {#2};

    \node (desc) at (6, -1.4) [align=center, text width=15cm] {#3};

    \draw [rounded corners=1pt]
          (inst.north west)+(-0.3, 0.6)
          rectangle
          (desc.south east)+(0.4, 0.4);
  \end{tikzpicture}
  \vspace{-0.5cm}
}

\newcommand{\pseudoinsttwo}[3]{
  \centering
  \begin{tikzpicture}[thick, node distance=0.1cm]
    \node (inst) at (0, 0) {\texttt{#1}};
    \node [below = of inst] (desc) [align=center, text width=\textwidth] {#2};
    \draw [rounded corners=1pt]
          ({-\textwidth / 2}, 0.6) rectangle ({\textwidth / 2}, -#3);
  \end{tikzpicture}
  \vspace{-0.5cm}
}

\pseudoinst
{
  \texttt{la <rd>, <symbol>}
}
{
  \texttt{rd}: Register, in das die Adresse reingeladen wird \\
  \texttt{symbol}: 32-Bit Adresse (relativ zum Programmzähler
}
{
  Lädt die zum Programmzähler relative Adresse \texttt{symbol} in das Register \texttt{rd}. Im Gegensatz zu den Kontrollflussinstruktionen kann hier eine bis zu 32-Bit große Konstante verwendet werden.
}

\pseudoinsttwo
{
  lbg <rd>, <mem\_addr>
}
{
  Gleiche Funktionalität wie \texttt{lb}, nur dass \texttt{mem\_addr} bis zu 32-Bit lang sein darf.
}
{1.5cm}

\pseudoinsttwo
{
  lhg <rd>, <mem\_addr>
}
{
  Gleiche Funktionalität wie \texttt{lh}, nur dass \texttt{mem\_addr} bis zu 32-Bit lang sein darf.
}
{1.5cm}

\pseudoinsttwo
{
  lwg <rd>, <mem\_addr>
}
{
  Gleiche Funktionalität wie \texttt{lw}, nur dass \texttt{mem\_addr} bis zu 32-Bit lang sein darf.
}
{1.5cm}

\pseudoinsttwo
{
  sbg <rd>, <mem\_addr>, <rt>
}
{
  Gleiche Funktionalität wie \texttt{sb}, nur dass \texttt{mem\_addr} bis zu 32-Bit lang sein darf.\\\texttt{rt} ist hier ein Transferregister, in das die Adresse geladen wird.
}
{2cm}

\pseudoinsttwo
{
  shg <rd>, <mem\_addr>, <rt>
}
{
  Gleiche Funktionalität wie \texttt{sh}, nur dass \texttt{mem\_addr} bis zu 32-Bit lang sein darf.\\\texttt{rt} ist hier ein Transferregister, in das die Adresse geladen wird.
}
{2cm}

\pseudoinsttwo
{
  swg <rd>, <mem\_addr>, <rt>
}
{
  Gleiche Funktionalität wie \texttt{sw}, nur dass \texttt{mem\_addr} bis zu 32-Bit lang sein darf.\\\texttt{rt} ist hier ein Transferregister, in das die Adresse geladen wird.
}
{2cm}

\pseudoinsttwo
{
  nop
}
{
  Eine No-Operation. Es wird kein Register und keine Speicherzelle verändert.
}
{1.5cm}

\pseudoinst
{
  li <rd>, <immediate>
}
{
  \texttt{rd}: Zielregister, in das die Konstante geladen wird \\
  \texttt{immediate}: 32-Bit Konstante
}
{
  Lädt die gegebene 32-Bit Konstante in das Register \texttt{rd}.
}

\footnotetext{\label{user-manual-riscv-overview-pseudos-foot1}Da jede Marke eine
Adresse darstellt führt die Verwendung einer Marke hier nicht zu einem
Übersetzungsfehler. Anders als bei den relativen Kontrollflussinstruktionen wird
hier eine Marke \textbf{nicht} automatisch in die passende relative Adresse
umgewandelt}

\pseudoinst
{
  mv <rd>, <rs>
}
{
  \texttt{rd}: Zielregister \\
  \texttt{rs}: Register, dessen Wert kopiert wird
}
{
  Kopiert den Inhalt des Registers \texttt{rs} in Register \texttt{rd}.
}

\pseudoinst
{
  not <rd>, <rs>
}
{
  \texttt{rd}: Zielregister \\
  \texttt{rs}: Register, dessen Wert bitweise negiert wird
}
{
  Negiert den Wert von \texttt{rs} bitweise und schreibt das Ergebnis nach
  \texttt{rd} ($\equiv$ Negation im 1er-Komplement)
}

\pseudoinst
{
  neg <rd>, <rs>
}
{
  \texttt{rd}: Zielregister \\
  \texttt{rs}: Register, dessen Wert negiert wird
}
{
  Ändert das Vorzeichen des Registers \texttt{rs} und schreibt das Ergebnis
  nach \texttt{rd} ($\equiv$ Negation im 2er-Komplement).
}

\pseudoinst
{
  seqz <rd>, <rs>
}
{
  \texttt{rd}: Zielregister \\
  \texttt{rs}: Register, das überprüft wird
}
{
  Legt in \texttt{rd} eine 1 ab, wenn \texttt{rs} $=$ \texttt{0} gilt, und
  sonst eine 0.
}

\pseudoinst
{
  snez <rd>, <rs>
}
{
  \texttt{rd}: Zielregister \\
  \texttt{rs}: Register, das überprüft wird
}
{
  Legt in \texttt{rd} eine 1 ab, wenn \texttt{rs} $\neq$ \texttt{0} gilt, und
  sonst eine 0.
}

\pseudoinst
{
  sltz <rd>, <rs>
}
{
  \texttt{rd}: Zielregister \\
  \texttt{rs}: Register, das überprüft wird
}
{
  Legt in \texttt{rd} eine 1 ab, wenn \texttt{rs} $<$ \texttt{0} gilt, und
  sonst eine 0.
}

\pseudoinst
{
  sgtz <rd>, <rs>
}
{
  \texttt{rd}: Zielregister \\
  \texttt{rs}: Register, das überprüft wird
}
{
  Legt in \texttt{rd} eine 1 ab, wenn \texttt{rs} $>$ \texttt{0} gilt, und
  sonst eine 0.
}

\pseudoinst
{
  beqz <rs>, <marke>
}
{
  \texttt{rs}: Register für die Rücksprungadresse und Vergleich mit 0 \\
  \texttt{marke}: Sprungmarke, die angesprungen wird
}
{
  Springt zur Marke, wenn \texttt{rs} $=$ \texttt{0} gilt.
}

\pseudoinst
{
  bnez <rs>, <marke>
}
{
  \texttt{rs}: Register für die Rücksprungadresse und Vergleich mit 0 \\
  \texttt{marke}: Sprungmarke, die angesprungen wird
}
{
  Springt zur Marke, wenn \texttt{rs} $\neq$ \texttt{0} gilt.
}

\pseudoinst
{
  blez <rs>, <marke>
}
{
  \texttt{rs}: Register für die Rücksprungadresse und Vergleich mit 0 \\
  \texttt{marke}: Sprungmarke, die angesprungen wird
}
{
  Springt zur Marke, wenn \texttt{rs} $\leq$ \texttt{0} gilt.
}

\pseudoinst
{
  bgez <rs>, <marke>
}
{
  \texttt{rs}: Register für die Rücksprungadresse und Vergleich mit 0 \\
  \texttt{marke}: Sprungmarke, die angesprungen wird
}
{
  Springt zur Marke, wenn \texttt{rs} $\geq$ \texttt{0} gilt.
}

\pseudoinst
{
  bltz <rs>, <marke>
}
{
  \texttt{rs}: Register für die Rücksprungadresse und Vergleich mit 0 \\
  \texttt{marke}: Sprungmarke, die angesprungen wird
}
{
  Springt zur Marke, wenn \texttt{rs} $<$ \texttt{0} gilt.
}

\pseudoinst
{
  bgtz <rs>, <marke>
}
{
  \texttt{rs}: Register für die Rücksprungadresse und Vergleich mit 0 \\
  \texttt{marke}: Sprungmarke, die angesprungen wird
}
{
  Springt zur Marke, wenn \texttt{rs} $>$ \texttt{0} gilt.
}

\pseudoinst
{
  j <marke>
}
{
  \texttt{marke}: Sprungmarke
}
{
  Springt zu \texttt{marke} und verwirft die Rücksprungadresse
}

\pseudoinst
{
  jal <marke>
}
{
  \texttt{marke}: Sprungmarke
}
{
  Springt zu \texttt{marke} und speichert die Rücksprungandresse in Register \texttt{x1}.
}

\pseudoinst
{
  jr <reg>
}
{
  \texttt{reg}: Enthält die Adresse, an die gesprungen wird
}
{
  Springt an die Adresse in Register \texttt{reg} und verwirft die Rücksprungadresse.
}

\pseudoinst
{
  jalr <reg>
}
{
  \texttt{reg}: Enthält die Adresse, an die gesprungen wird
}
{
  Springt an die Adresse in Register \texttt{reg} und speichert die
  Rücksprungadresse in Register \texttt{x1}
}

\pseudoinsttwo
{
  ret
}
{
  Springt an die Adresse in Register x1.\footnotemark
}
{1.5cm}

\pseudoinst
{
  tail <marke>
}
{
  \texttt{marke}: Sprungmarke
}
{
  Springt zu \texttt{marke} und verwirft die Rücksprungadresse. Für die Adressberechnung wird Register x6 verwendet. Es kann beliebige Adresse im 32-Bit Adressbereich angesprungen werden.
}

\pseudoinst
{
  call <marke>
}
{
  \texttt{marke}: Sprungmarke
}
{
  Springt zu \texttt{marke} und legt die Rücksprungadresse in Register x1 ab\textsuperscript{\ref{call-stack}}.  Für die Adressberechnung wird Register x6 verwendet. Es kann beliebige Adresse im 32-Bit Adressbereich angesprungen werden.
}

\footnotetext{\label{call-stack}Im Gegensatz zu x86 arbeiten die \texttt{call} bzw. \texttt{ret} Instruktionen nicht mit einem Callstack. \texttt{ret} kehrt also nur zum letzten Aufruf von \texttt{call} zurück, da x1 überschrieben wird.}

\pagebreak
Das 64Bit Basismodul von RISC-V fügt noch weitere Pseudoinstruktionen hinzu:\\
\vspace{0.5cm}


\pseudoinsttwo
{
  ldg <rd>, <mem\_addr>
}
{
  Gleiche Funktionalität wie \textbf{ld}, nur dass \texttt{mem\_addr} bis zu 32-Bit lang sein darf.
}
{2cm}

\pseudoinsttwo
{
  sdg <rd>, <mem\_addr>, <rt>
}
{
  Gleiche Funktionalität wie \textbf{sd}, nur dass \texttt{mem\_addr} bis zu 32-Bit lang sein darf. \texttt{rt} ist hier ein Transferregister, in das die Adresse geladen wird
}
{2cm}

\pseudoinst
{
  negw <rd>, <rs>
}
{
  \texttt{rd}: Zielregister \\
  \texttt{rs}: Register, dessen untere 32Bit negiert werden
}
{
  Ändert das Vorzeichen der unteren 32 Bit des Registers \texttt{rs} und schreibt das Ergebnis nach \texttt{rd}.
}

\pseudoinst
{
  sext.w <rd>, <rs>
}
{
  \texttt{rd}: Zielregister \\
  \texttt{rs}: Register, dessen untere 32 Bit mit dem\\ richtigen Vorzeichen erweitert werden
}
{
  Erweitert die unteren 32Bit des Registers \texttt{rs} und schreibt das Ergebnis nach \texttt{rd}.
}
