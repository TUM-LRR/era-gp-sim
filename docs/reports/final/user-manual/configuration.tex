% ERA-Großpraktikums: Benutzeranleitung -- Konfiguration

\section{Konfiguration}
\label{user-manual-configuration}

Im Menu unter \texttt{Editor > Settings} oder mit dem Tastaturkürzel
\texttt{Strg+,} kann das Einstellungsfenster für den Simulator geöffnet werden.
Hier kann der Standardpfad für die Snapshots angegeben werden und ein Theme
ausgewählt werden.\\
Außerdem finden sich hier die Einstellungen für das
Standardverhalten bei Warnungen bezüglich Snapshots. Normalerweise erscheint
eine Warnung, wenn man Snapshots überschreiben oder löschen will.  Innerhalb
dieser Warnung kann eine Checkbox gesetzt werden, wenn keine Warnung erwünscht
ist. Diese Einstellung kann im Konfigurationsfenster geändert werden.


\begin{figure}[ht]
	\centering
  \includegraphics[scale=0.9]{Images/Settings}
	\caption{Das Einstellungsfenster}
	\label{Settings}
\end{figure}


Sobald eine Änderung getätigt wurde, erscheint ein \texttt{Save} Button am unteren Rand.


\subsection{Konfiguration der einzelnen Komponenten}

Die meisten Ein- und Ausgabekomponenten haben besondere Einstellungen. Die
Einstellungen können aufgerufen werden, indem man mit der Maus an den oberen
Rand einer Komponente fährt, bis sich die Leiste für die Auswahl der Komponenten
öffnet.  Daraufhin muss man an der rechten Seite auf das Zahnradbild klicken
(\includegraphics[scale=0.22]{Images/SettingsIcon}). Daraufhin öffnet sich in
den meisten Fällen ein Fenster, in dem die entsprechenden Einstellungen für die
Komponenten gesetzt werden können. \\
Für eine detaillierte Beschreibung der
Einstellungen siehe die Kapitel zu den Komponenten.
