% ERA-Großpraktikums: Benutzeranleitung -- Konfiguration

\section{Konfiguration}

Im Menu unter \texttt{Editor > Settings} oder mit dem Tastaturkürzel \texttt{Strg+,} kann das Einstellungsfenster für den Simulator geöffnet werden.
Hier kann der Standardpfad für die Snapshots angegeben werden und ein Theme ausgewählt werden.\\
Außerdem finden sich hier die Einstellungen für das Standardverhalten bei Warnungen bezüglich Snapshots. Normalerweise erscheint eine Warnung, wenn man Snapshots überschreiben oder löschen will. 
Innerhalb dieser Warnung kann eine Checkbox gesetzt werden, wenn keine Warnung erwünscht ist. Diese Einstellung kann im Konfigurationsfenster geändert werden.


\begin{figure}[ht]
	\centering
  \includegraphics[scale=0.9]{Images/Settings}
	\caption{Das Einstellungsfenster}
	\label{Settings}
\end{figure}


Sobald eine Änderung getätigt wurde, erscheint ein \texttt{Save} Button am unteren Rand.