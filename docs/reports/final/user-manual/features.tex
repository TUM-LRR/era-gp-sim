% ERA-Großpraktikums: Benutzeranleitung -- Features

\section{Features}

\subsection*{Überblick}
\todo[inline]{Am besten mit Bild und nummerierten Pfeilen}

% Ich schlage eine Teilung der einzelnen Features nach Projektverwaltung(Laden, Speichern, Snapshots), Erstellen von Programmen (Editor, Hilfetexte), Ausführung von Programmen(Ausfühmodi, Breakpoints), Interaktion mit laufenden Programmen (Input/Output-Components) und sonstiges vor

\subsection{Projekt}

\subsubsection{Projekt erstellen}
\todo[inline]{V.a. Projekt erstellen: Auswahl Architektur, Bitlänge, Modlue, Parser, etc.}

\subsubsection{Speichern/Laden}
\todo[inline]{Speichern/Laden von Projekt und Code}

\subsubsection{Snapshots}
\todo[inline]{Speichern/Laden + Was ist das?; Welche Eigenschaften muss mein erstelltes Projekt haben, damit ich SnapshotXY laden kann (passende Architektur/passende Module)}


\subsection{Erstellen von Programmen}

\subsubsection{Editor}
\todo[inline]{Alles zum Editor (zoomen, Makros aufklappen, Fehlermeldungen anzeigen, Tooltips etc)}

\subsubsection{Hilfetextkomponente}
\label{help-component}
\todo[inline]{Was wird angezeigt, zu welchem Befehl (Zeile!), wann (nach dem Parsen)}


\subsection{Ausführung von Programmen}

\subsubsection{Ausführungsmodi}
\todo[inline]{komplett, schritt-für-schritt, zum nächsten breakpoint; hier am besten mmit bildern der einzelnen Knöpfe}

\subsubsection{Registerkomponente}
\todo[inline]{Alles zu den Registerkomponenten, wie verstelle ich die Zahlendarstellung, Highlight bei Änderung etc}

\subsubsection{Speicherkomponente}
\todo[inline]{Alles zur Speicherkomponente: Wie verstelle ich die Zahlendarstellung, Hinzufügen/Entfernen von Spalten}


\subsection{Interaktion mit laufenden Programmen}
\todo[inline]{Hier sollte (für alle Komponenten ja gleich) die Einstellungsmöglichkeiten (Zahnrad-knopf) und welches Symbol welche Komponente bedeutet}

\subsubsection{Eingabe -- Pfeiltasten}
\todo[inline]{v.a. welche Werte werden für welchen Pfeil geschrieben}

\subsubsection{Eingabe -- Mausklick}
\todo[inline]{welche werte werden geschrieben, Reihenfolge?}

\subsubsection{Eingabe -- Text}
\todo[inline]{wann wird geschriebener Text in den Speicher geschrieben?, Wie lang kann mein Text sein?}

\subsubsection{Ausgabe -- Lichtbandanzeige}
\todo[inline]{besondere Einstellungen! Farbwahl! Manuelles setzen der Strips möglich!}

\subsubsection{Ausgabe -- 7-Segmentanzeige}
\todo[inline]{besondere Einstellungen! Manuelles setzen der Segmente möglich, Anzeige des Index bei Maus-hover}

\subsubsection{Ausgabe -- Konsole}
\todo[inline]{pipe-like, array-based; }


\subsection{Sonstige Features}