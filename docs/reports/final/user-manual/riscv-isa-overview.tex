% ERA-Groß?praktikum: Benutzeranleitung -- Übersicht Befehlssatz RISC-V

\section{Übersicht Befehlssatz RISC-V}
\todo[inline]{Hier Erläuterung: Unterschied Parser-direktiven, Instruktionen und Pseudo-instruktionen}
Die Menge an verwendbaren Befehlen für den Simulator lässt sich grob in zwei Teile aufteilen. Die erste Menge an Befehlen sind die sogenannten Parser-Direktiven. Das sind diejenigen Befehle bzw. Direktiven, die schon zum Zeitpunkt der Übersetzung (also Assemblierung) aufgelöst und verarbeitet werden. Direktiven tauchen also nicht direkt im assemblierten Code auf, dennoch nehmen sie u.a. durch Anordnung des Speichers (Sektionen, Speicherreservierungen) oder durch Codegenerierung (Makros) wesentlich Einfluss auf das assemblierte Programm. Zur Unterscheidung haben Direktiven im Simulator ein .-Präfix. Direktiven werden im ersten Teil behandelt.\\
Die zweite Menge sind die eigentlichen RISC-V Instruktionen. Diese werden durch den Assemblierer in maschinenlesbaren Code umgewandelt und können dann auch ausgeführt werden. Für alle unterstützten RISC-V Instruktionen gibt es im zweiten Teil eine Übersicht.

{\subsection{Parser-Direktiven}
\todo[inline]{.equ und .section nicht vergessen!}

\subsubsection{Speicherreservierung}
Speicherreservierungen sind eine Möglichkeit, eine oder mehrere Speicherzellen (hier: byte) im Speicher freizuhalten. Freihalten bedeutet in dem Fall, dass der Assemblierer keine assemblierten Instruktionen an diese Speicherzelle schreibt.
So kann man sicherstellen, dass man bei Verwendung dieser Speicherzelle(n) nicht den gerade auszuführenden Code korrumpiert.
Da die Adresse der Speicherzelle(n) vom Assemblierer abhängt, sollte man die Adresse mit einem vorgeschobenen Label ''speichern''. Das Label vor einer Speicherreservierungs-Direktive nimmt die Anfangsadresse des reservierten Speicherblocks an.\\

\begin{centering}
	\begin{tabular}{rcl}
		reserve byte & \textbf{.resb <anz>} & reserviert \texttt{anz} Byte\\
		reserve half word & \textbf{.resh <anz>} & reserviert \texttt{anz} Halbwörter (also jeweils 16 Bit)\\
		reserve word & \textbf{.resw <anz>} & reserviert \texttt{anz} Wörter (also jeweils 32 Bit)\\
		reserve double word & \textbf{.resd <anz>} & reserviert \texttt{anz} Doppelwörter (also jeweils 64 Bit)\\
	\end{tabular}
\end{centering}

\begin{warningblock}
	\texttt{anz} muss eine positive Ganzzahl sein.\\
	Ist \texttt{anz<=0} wird kein Speicher reserviert und ein Übersetzungsfehler angezeigt
\end{warningblock}

\begin{exampleblock}{Beispiel zur Speicherreservierung}
	Es sollen 32 Bit an Speicherplatz unter dem Label \texttt{memory} reserviert werden:\\
	\begin{lstlisting}{style=risc-v_Assembler}
	memory: .resb 4
	\end{lstlisting}
	oder
	\begin{lstlisting}{style=risc-v_Assembler}
	memory: .resh 2
	\end{lstlisting}
	oder
	\begin{lstlisting}{style=risc-v_Assembler}
	memory: .resw 1
	\end{lstlisting}
	Die obigen Befehle sind gleichwertig, es empfiehlt sich jedoch eine Variante zu wählen, die der Verwendung am nächsten kommt, um die Lesbarkeit zu erhöhen. Wird beispielsweise byte für byte gearbeitet, ist auch eine byte-weise Reservierung des Speichers anschaulich. 
\end{exampleblock}

\subsubsection{Speicherdefinition}

Speicherdefinitionen sind die andere Möglichkeit Speicherzellen von assemblierten Instruktionen freizuhalten. Im Gegensatz zu Speicherreservierungen wird hier der Speicher nicht nur reserviert, sondern gleichzeitig auch mit dem übergebenen Datum gefüllt. Mit Speicherdefinitionen kann man also den Speicher noch vor der Ausführung des Programms mit Daten befüllen. Da auch hier die genaue Adresse des Speicherblocks von der Assemblierung abhängt, sollte zur Verwendung ein Label vorangestellt werden.\\

\begin{centering}
	\begin{tabular}{rcl}
		define byte & \textbf{.byte <data>} & legt \texttt{data} als bytes in den Speicher\\
		define half word & \textbf{.half <data>} & legt \texttt{data} als Halbwörter (16 Bit) in den Speicher\\
		define word & \textbf{.word <data>} & legt \texttt{data} als Wort (32 Bit) in den Speicher\\
		define double word & \textbf{.dword <data>} & legt \texttt{data} als Doppelwort (64 Bit) in den Speicher\\
	\end{tabular}
\end{centering}
\\
\texttt{data} kann hier eine oder mehrere Zahlen sein (durch Kommas getrennt), die dann hintereinander als bytes, Halbwörter, usw. abgelegt werden.
\begin{exampleblock}{Beispiel Speicherdefinition}
	Es soll die Abfolge \texttt{0x12, 0x34, 0x56, 0x78} in den Speicher unter dem Label \texttt{sequence} abgelegt werden:
	\begin{lstlisting}{style=risc-v_Assembler}
		sequence: .byte 0x12, 0x34, 0x56, 0x78
	\end{lstlisting}
\end{exampleblock}
\begin{infoblock}{Zeichenketten in den Speicher legen}
	Es ist auch möglich eine komplette Zeichenkette ganz bequem in den Speicher zu legen.
	\begin{lstlisting}{style=risc-v_Assembler}
		label: .byte "foo? bar!"
	\end{lstlisting}
	Die Zeichen werden im ASCII-Format abgelegt.
\end{infoblock}

\subsection{Makros}
\todo[inline]{Wie definiere ich ein Makro, Grenzen (Zyklische Aufrufe, Rekursion, Anzahl Parameter)}

\subsection{RISC-V Instruktionen}
\todo[inline]{Erklärung Aufteilung in Module}
Die RISC-V Instruktionen sind in verschiedene Module aufgeteilt. Durch eine Modularisierung des Dialekts kann ganz bewusst auf Arten von Instruktionen verzichtet werden. Es braucht z.B. nicht jeder Mikrocontroller Multiplikation oder atomare Operationen. Der Simulator unterstützt folgende Module:\\
\begin{itemize}
	\item \textbf{RV32I} Das ''Basispaket'' mit 32 Bit Arithmetik-, Sprung- und Verzweigungsinstruktionen
	\item\textbf{ RV32M} Fügt dem Basispaket Befehle zur Multiplikation und Division für 32 Bit hinzu
	\item \textbf{RV64I} Das ''Basispaket'' mit 64 Bit Arithmetik-, Sprung- und Verzweigungsinstruktionen
	\item \textbf{RV64M} Fügt dem Basispaket Befehle zur Multiplikation und Division für 64 Bit hinzu
\end{itemize}
Es kann generell für ein Projekt im Simulator entweder 32 oder 64 Bit als Registerlänge gewählt werden. Dementsprechend können nur Module mit der selben Länge verwendet werden. Die 64 Bit Erweiterungen enthalten alle Instruktionen ihrer 32 Bit Gegenstücke, allerdings funktionieren diese auf den gesamten 64 Bit.\\
Im Folgenden wird jeder unterstützte Befehl kurz erläutert. Eine detailliertere Beschreibung ist den Hilfetextanzeigen (\autoref{help-component}) im Simulator zu entnehmen.\\

Alle regulären Instruktionen verwenden in der einen oder anderen Art Register also Operanden. Die im Simulator unterstützten Module stellen dem Programmierer 32 Allzweckregister zur Verfügung. Diese Register sind je nach gewähltem Basismodul 32 oder 64 bit lang. Das Register 0 (''zero''-Register) hat unveränderbar den Wert 0. Für eine gute Verwendungsmöglichkeit siehe \autoref{manual-riscv-jump-instructions}. Die Register werden über \texttt{x<id>} angesprochen, wobei \texttt{id} die Nummer des Registers ist. So steht \texttt{x0} für das Register 0 und \texttt{x17} für das Register 17.\\
Zur Angabe von Zahl-Konstanten (sog. ''Immediates'') unterstützt der Simulator folgende 3 Möglichkeiten:\\
Mit dem Präfix \texttt{0b} lässt sich die Konstante in Binärdarstellung ausdrücken. \texttt{0b1101} steht z.B. für die Zahl 13.\\
Mit dem Präfix \texttt{0x} werden Konstanten in Hexadezimal geschrieben. \texttt{0xaffe} $\widehat{=}$ 45 054.\\
Wird die Zahl ohne Präfix geschrieben, interpretiert sie der Simulator als Zahl in Dezimaldarstellung.

\begin{infoblock}{Arithmetische Ausdrücke zur Übersetzungszeit}
	Der Simulator unterstützt zudem arithmetische Ausdrücke für Konstanten. So ist \texttt{andi x1,x1, 2*3} gleichwertig zu \texttt{andi x1, x1, 6}. Das ist besonders hilfreich, da auch mit Labels und \texttt{.equ} definierte Konstanten gerechnet werden kann.\\
	\begin{lstlisting}{style=risc-v_Assembler}
		.equ mem_loc, 0xbeef+2
		.equ addr, mem_loc + offset
		.equ offset, 7
		...
		lb x1, x0, addr
	\end{lstlisting}
	Die Definition von Konstanten und Labels kann auch unterhalb der Verwendung im arithmetischen Ausdruck oder in einem anderen Segment sein, solange sich keine zyklische Abhängigkeit ergibt. Es sind allerdings nur Ausdrücke erlaubt, die zur Übersetzungszeit in eine Zahl auflösbar sind. \texttt{addi x1, x0, 2*x4} ist nicht erlaubt, da der Wert des Registers 4 zur Übersetzungszeit noch nicht feststeht\\
	Die Operatoren für die Ausdrücke sind an Operatoren aus C angelehnt:\\
		\begin{tabular}{|c|c|}
			+ & Addition \\
			- & Subtraktion \\
			* & Multiplikation \\
			/ & Division \\
			\% & Modulo \\
			<< & shift links\\
			>> & logischer shift rechts\\
			<= & Vergleich: kleiner gleich\\
			< & Vergleich: kleiner\\
			>= & Vergleich: größer gleich\\
			> & Vergleich: größer\\
			! & boolsche Negation\\
			== & Vergleich: Gleichheit\\
			!= & Vergleich: Ungleichheit\\
		\end{tabular}\\
	Hinweis zu boolschen Operationen: eine Zahl $=0$ repräsentiert gleichzeitig einen falschen, eine Zahl $\neq 0$ einen wahren Ausdruck. Vergleiche ergeben jeweils $0$ bzw. $1$.
\end{infoblock}

\begin{warningblock}
	Die Interpretation von Zahldarstellungen in binärer oder hexadizimaler Form folgt nicht immer dem Zweierkomplement.
	Die Zahldarstellungen werden stets als eine 32 Bit lange Zahl interpretiert\footnote{Dies ermöglicht arithmetisches Rechnen auch (leicht) außerhalb des Wertebereichs der erwarteten Konstante. Dies ist möglich, wenn nur Zwischenergebnisse den Wertebereich verlassen, das Ergebnis aber im Wertebereich liegt. Bsp. \texttt{addi x1, x0, (2000*3)/4; Legt 3/4 * 2000 in x1}. Hier wird ein Zwischenergebnis (3*2000) größer als 2048 produziert. Würde die Zahldarstellung auf die Operandengröße beschränkt werden, wäre das (unerwartete) Ergebnis $(3 \cdot 2000 \mod 2048)/4 = 476$ anstatt $1500$}. Das hat zur Folge, dass \texttt{addi x1,x0, 0xfff} mit einem Übersetzungsfehler als ''zu groß'' abgelehnt wird. Grund ist hier, dass der Befehl eine Konstante im Wertebereich von vorzeichenbehafteten 12 Bit erwartet. Die größte darstellbare Zahl ist $2^{11}-1 = 2047$ während 0xfff $ \widehat{=} 4095$, also zu groß ist.\\
	Es ist ein Fehler zu denken, dass die obige Instruktion -1 in das Register 1 schreibt. Um das zu erreichen, sollte \texttt{addi x1, x0, -1} oder \texttt{addi x1, x0, -0x1} o.ä verwendet werden.
\end{warningblock}

\subsubsection{Arithmetische Befehle}
Jeder arithmetische Befehl hat eine Register-Register und eine Register-Immediate Variante. Letztere hat das Suffix ''i''.\\
\begin{warningblock}
	Die Register-Register Instruktionen arbeiten auf den vollen 32/64 Bit.\\
	Der Immediate-Operand ist aber auf den Wertebreich von vorzeichenbehafteten 12 Bit beschränkt (-2048 -- 2047)
\end{warningblock}
Das Format der arithmetischen Operationen ist gleich aufgebaut:\\
 \begin{centering}
 	\texttt{instr <dest>, <reg1>, <reg2|imm>}\\
 \end{centering}
 Die beiden Operanden \texttt{reg1} und \texttt{reg2} bei Register-Register bzw. \texttt{reg1} und \texttt{imm} bei Register-Immediate dienen als Operanden für die arithmetische Funktion. Das Ergebnis wird immer im Register \texttt{dest} abgelegt. Die Operandenregister bleiben also u.U. unverändert.\\


\begin{tabular}{|c|p{8cm}|p{4cm}|}
	\hline
	\textbf{add(i)} & Addiert die beiden Operanden & \\
	\hline
	\textbf{sub} & Subtrahiert \texttt{reg2} von \texttt{reg1}& Es gibt kein \texttt{subi}, aber \texttt{addi rd, rs1, -imm} \\
	\hline
	\textbf{and(i)} & Führt eine logische, bitweise UND-Operation durch & \\
	\hline
	\textbf{or(i)} & Führt eine logische, bitweise ODER-Operation durch & \\
	\hline
	\textbf{xor(i)} & Führt eine logische, bitweise exklusiv-ODER-Operation durch & \\
	\hline
	\textbf{sll(i)} & Shiftet \texttt{reg1} um \texttt{reg2|imm} Stellen logisch nach links (Am LSBit werden \texttt{reg2|imm} Nullen eingeschoben) & \multirow{3}{4cm}{Als Stellenanzahl werden nur die unteren 5 Bit betrachtet} \\
	\cline{1-2}
	\textbf{srl(i)} & Shiftet \texttt{reg1} um \texttt{reg2|imm} Stellen logisch nach rechts (Am MSBit werden \texttt{reg2|imm} Nullen eingeschoben) & \\
	\cline{1-2}
	\textbf{sra(i)} & Shiftet \texttt{reg1} um \texttt{reg2|imm} Stellen arithmetisch nach rechts (Am MSBit wird \texttt{reg2|imm}-mal das Sign-Bit eingeschoben) & \\
	\hline
\end{tabular}

\begin{warningblock}
	Alle Shiftinstruktionen der Variante Register-Immediate erlauben eine Konstante im Wertebereich zwischen 5 und 12 Bit. Obwohl nur die unteren 5 Bit verwendet werden, führt eine Zahl, die größer als 5 Bit ist, nicht zu einem Übersetzungsfehler.
\end{warningblock}
Das 64 Bit Modul \textbf{RV64I} fügt zusätzliche spezielle arithmetische Befehle hinzu. Mit RV64I werden die ''normalen'' Instruktionen (siehe Tabelle oben) auf 64 Bit erweitert. Die nun folgenden ''word''-Instruktionen (erkennbar am Suffix ''w'') betrachten jeweils nur die unteren 32 Bit der 64 Bit breiten Operandenregister und berechnen ein 32 Bit breites Ergebnis, das auf 64bit erweitert wird.
\begin{tabular}{|c|p{8cm}|p{4cm}|}
	\hline
	\textbf{add(i)w} & Addiert die beiden Operanden & \\
	\hline
	\textbf{subw} & Subtrahiert \texttt{reg2} von \texttt{reg1}& Es gibt kein \texttt{subiw}, aber \texttt{addiw rd, rs1, -imm} \\
	\hline
	\textbf{sll(i)} & Shiftet \texttt{reg1} um \texttt{reg2|imm} Stellen logisch nach links (Am LSBit werden \texttt{reg2|imm} Nullen eingeschoben) & \multirow{3}{4cm}{Arbeitet auf den vollen 64Bit. Als Stellenanzahl werden nur die unteren 6 Bit betrachtet} \\
	\cline{1-2}
	\textbf{srl(i)} & Shiftet \texttt{reg1} um \texttt{reg2|imm} Stellen logisch nach rechts (Am MSBit werden \texttt{reg2|imm} Nullen eingeschoben) & \\
	\cline{1-2}
	\textbf{sra(i)} & Shiftet \texttt{reg1} um \texttt{reg2|imm} Stellen arithmetisch nach rechts (Am MSBit wird \texttt{reg2|imm}-mal das Sign-Bit eingeschoben) & \\
	\hline
	\textbf{sll(i)w} & Shiftet \texttt{reg1} um \texttt{reg2|imm} Stellen logisch nach links (Am LSBit werden \texttt{reg2|imm} Nullen eingeschoben) & \multirow{3}{4cm}{Als Stellenanzahl werden nur die unteren 5 Bit betrachtet} \\
	\cline{1-2}
	\textbf{srl(i)w} & Shiftet \texttt{reg1} um \texttt{reg2|imm} Stellen logisch nach rechts (Am MSBit werden \texttt{reg2|imm} Nullen eingeschoben) & \\
	\cline{1-2}
	\textbf{sra(i)w} & Shiftet \texttt{reg1} um \texttt{reg2|imm} Stellen arithmetisch nach rechts (Am MSBit wird \texttt{reg2|imm}-mal das Sign-Bit eingeschoben) & \\
	\hline
\end{tabular}

\subsubsection{LUI, AUIPC, SLT}
\textbf{lui} und \textbf{auipc} sind spezielle Instruktionen, um 32-Bit große Konstanten in ein Register zu bringen. Da die Instruktionslänge bei RISC-V nur 32 Bit beträgt, kann also keine 32 Bit große Konstante und der Ladebefehl gleichzeitig in einer Instruktion zusammengefasst werden. Somit muss das Laden der Konstante in mehreren Instruktionen erfolgen.\\

Mit \textbf{lui} kann eine bis zu 20 Bit große Konstante \texttt{imm} direkt in die oberen 20 Bit des Registers \texttt{reg} geladen werden:
\texttt{lui <reg>, <imm>}
\begin{exampleblock}{32 Bit Konstante laden}
Die Konstante 0xabcd1234 in x2 laden:\\
\begin{lstlisting}{style=risc-v_Assembler}
  lui x2, 0xabcd1 	;Lade obere 20 Bit
  addi x2, x2, 0x234	;Lade untere 12 Bit
\end{lstlisting}
Oder eleganter:
\begin{lstlisting}{style=risc-v_Assembler}	
  .equ myConstant, 0xabcd1234
  lui x2, myConstant>>12        ;Lade obere 20 Bit
  addi x2, x2, myConstant&0xfff ;Lade untere 12 Bit
\end{lstlisting}
Vorsicht, die obige Implementierung funktioniert nicht bei Zahlen, in deren Repräsentation das 11. Bit gesetzt ist. Ist dies der Fall muss für den lui-Teil zusätzlich noch eins addiert werden. Wem das zu kompliziert ist, verwendet am besten eine passende Pseudoinstruktion (\autoref{user-manual-riscv-overview-pseudos})
\end{exampleblock}
\textbf{auipc} ermöglicht es, 32 Bit große Sprungadressen relativ zum Programmzähler zu bauen. Wie auch schon bei \texttt{lui} werden hier mehrere Instruktionen benötigt. \texttt{auipc} lädt eine bis zu 20 Bit große Konstante \texttt{imm} in die oberen 20 Bit des Registers \texttt{reg} und addiert danach den Programmzähler auf das Register:
\texttt{auipc <reg>, <imm>}
\begin{exampleblock}{Ganz weiter Sprung}
Sprung um 0xbadeaffe Byte nach unten:
\begin{lstlisting}{style=risc-v_Assembler}
  auipc x2, 0xbadea	;Lade obere 20 Bit
  jalr x0, 0xffe	;Addiere untere 12Bit & Sprung
\end{lstlisting}
Auch hier ist eine elegantere Variante mit Konstante und Übersetzungszeit Arithmetik möglich.
\end{exampleblock}

\textbf{slt} \texttt{<dest>, <reg1>, <reg2>} vergleicht reg1 und reg2 und schreibt eine 1 in Register dest, wenn reg1 kleiner reg2. Ist dies nicht der Fall, wird eine 0 in Register dest geschrieben. \textbf{sltu} vergleicht die Register vorzeichenlos. \textbf{slti} bzw. \textbf{sltiu} akzeptieren anstatt eines zweiten Registers eine 12-Bit vorzeichenbehaftete bzw. vorzeichenlose Zahl.

\subsubsection{Kontrollflussbefehle (Sprung und Verzweigung)}
\label{manual-riscv-jump-instructions}
In RISC-V gibt es zwei Arten von Sprung- bzw. Verzweigungsbefehle: unbedingte und bedingte Sprünge.

\textbf{Unbedigte Sprünge}\\

Die folgenden Instruktionen springen an die gegebenen Adressen, sobald sie ausgeführt werden.\\
\textbf{jal} \texttt{<rd>, <marke>} springt zu \texttt{marke} und legt die Rücksprungadresse in \texttt{rd} ab.\\
\textbf{jalr} \texttt{<rd>, <basis>, <offset>} springt zur Adresse \texttt{<basis> + <offset>} und legt die Rücksprungadresse in \texttt{rd} ab.\\

Im Gegensatz zu Intel-386 besitzt RISC-V keine Instruktionen zur Verwaltung des Stacks. Benötigt man einen Stack (z.B. für rekursive Aufrufe), muss entsprechende push, pop, call und ret Funktionalitäten selbst schreiben.

\begin{exampleblock}{Sprünge}
Das folgende Programm springt zu sub\_routine und kehrt wieder zurück:
\begin{lstlisting}{style=risc-v_Assembler}
  jal x4, sub_routine
  ; hierhin wird zurückgekehrt
  
  ; ...
sub_routine:
  ; ... Ausführung Unterprogramm
  jalr x0, x4, 0
\end{lstlisting}
Der \texttt{jal}-Ausdruck speichert die Adresse des nachfolgenden Befehls (durch Kommentar angedeutet) in Register x4 und spring das Unterprogramm an. Der \texttt{jalr}-Ausdruck springt daraufhin am Ende des Unterprogramms zur Adresse in Register 4. Die Adresse dessen nachfolgenden Befehls wird verworfen (also in x0 gespeichert)
\end{exampleblock}

\textbf{Bedingte Sprünge}\\

Die Syntax der bedingten Sprünge ist gleich: \textbf{op} \texttt{<cmp1>, <cmp2>, <marke>}\\
Bei Ausführung prüft die Instruktion eine Bedingung (spezifiziert durch \texttt{op}) zwischen den beiden Registern \texttt{cmp1} und \texttt{cmp2}. Ist diese Bedingung wahr, springt die Instruktion zu \texttt{marke}. Ist die Bedingung falsch, wird kein Sprung ausgeführt, d.h. die Ausführung läuft ganz normal mit dem nächsten Befehl weiter.\\
\begin{tabular}{|c|c|l|}
	\hline
	op & Bedingung & Besonderheit\\
	\hline
	beq & $\texttt{cmp1} = \texttt{cmp2}$ & \\
	\hline
	bne & $\texttt{cmp1} \ne \texttt{cmp2}$ & \\
	\hline
	blt & $\texttt{cmp1} < \texttt{cmp2}$ & Vorzeichenbehafteter Vergleich\\
	\hline
	bltu & $\texttt{cmp1} < \texttt{cmp2}$ & Vorzeichenloser Vergleich\\
	\hline
	bge & $\texttt{cmp1} \ge \texttt{cmp2}$ & Vorzeichenbehafteter Vergleich\\
	\hline
	bgeu & $\texttt{cmp1} \ge \texttt{cmp2}$ & Vorzeichenloser Vergleich\\
	\hline
\end{tabular}

\begin{infoblock}{Umgang mit Labels in Sprungbefehlen}
	Ein Label enthält die absolute Adresse an der es definiert ist. Einige der Kontrollflussinstruktionen funktionieren jedoch mit Programmzähler-relativen Adressen, d.h. eigentlich ist eine Umrechnung von absoluter Labeladresse zu relativer Adresse nötig. Diese Umrechnung führt der Simulator aber automatisch für die Befehle durch, die relativ adressiert werden. So können auch relative Kontrollflussinstruktionen wie gewohnt an ein angegebenes Label springen.\\
	Vorsicht ist lediglich geboten, wenn ein Label in einer relativ adressierten Instruktion noch durch Arithmetik verändert wird. Hier wird das Label zuerst in relative Form gebracht und dann durch die Arithmetik verändert. Es können also unerwartete Ergebnisse entstehen.
\end{infoblock}

\begin{warningblock}
	Es ist möglich, statt einer Marke auch einfach nur eine Zahl anzugeben (z.B. \texttt{jal x0, 8}).
	Alle Kontrollflussinstruktionen außer \texttt{jalr} sind Programmzähler-relativ und springen relativ zum Programmzähler um das doppelte des eigentlichen Werts.\\
\begin{lstlisting}{style=risc-v_Assembler}
jal x0, 4 ; Springt 2 Instruktionen (8 Byte) nach unten
beq x0, x0, -2 ; Springt eine Instruktion (4 Byte) nach oben
jalr x0, x0, 0x44 ; Springt an Adresse 0x44
\end{lstlisting}
\end{warningblock}


\subsubsection{Speicherinteraktion}
Für Interaktion mit dem Speicher (Laden und Speichern) stellt RISC-V folgende Instruktionen bereit:\\

\textbf{Store-Instruktionen}\\

Alle Instruktionen zum Speichern eines Wertes aus einem Register sind gleich aufgebaut:\\
 \textbf{op} \texttt{<source>, <base>, <offset>}. Der Inhalt aus \texttt{source} (\texttt{op} entscheidet welche Byte) wird an Speicheradresse \texttt{base+offset} und folgende geschrieben.\\

\begin{tabular}{|c|c|l|}
	\hline
	Befehl & Größe & Beispiel (source= 0x1234 5678)\\
	\hline
	sb & unteres Byte & 0x78\\
	\hline
	sh & unteres Halbwort (2 Byte) & 0x5678\\
	\hline
	sw & (unteres) Wort (4 Byte) & 0x1234 5678\\
	\hline
\end{tabular}\\

Für RV64 kommt eine weitere Instruktion dazu, die oberen Instruktionen funktionieren wie angegeben.\\

\begin{tabular}{|c|c|l|}
	\hline
	Befehl & Größe & Beispiel (source= 0x1234 5678 9abc def0)\\
	\hline
	sd & Doppelwort (8 Byte) & 0x1234 5678 9abc def0\\
	\hline
\end{tabular}

\textbf{Load-Instruktionen}\\

Auch die Lade-Instruktionen sind gleich aufgebaut: \textbf{op} \texttt{<dest>, <base>, <offset>}. Der Inhalt aus Speicherzelle(n) ab \texttt{base+offset} wird ins Register \texttt{dest} geschrieben. Analog zu den Store-Instruktionen spezifiert auch hier \texttt{op} die genaue Anzahl an Bytes. Übrige Bytes im Register \texttt{dest} werden auf 0 gesetzt.\\

\begin{tabular}{|c|c|l|}
	\hline
	Befehl & Größe & Beispiel (Speicher: 0x01|0x23|0x45|0x67|0x89)\\
	\hline
	lb & 1 Byte in unterstes Byte & dest = 0x0000 0001\\
	\hline
	lh & 2 Byte in unteres Halbwort & dest = 0x0000 000123\\
	\hline
	lw & 4 Byte in (unteres) Wort & dest =  0x0123 4567\\
	\hline
\end{tabular}\\

Für RV64 kommt eine weitere Instruktion dazu:\\

\begin{tabular}{|c|c|l|}
	\hline
	Befehl & Größe & Beispiel (Speicher: 0x01|0x23|0x45|0x67|0x89|0xab|0xcd|0xef)\\
	\hline
	sd & 8 Byte in Doppelwort& dest = 0x0123 4567 89ab cdef\\
	\hline
\end{tabular}

\begin{warningblock}
	Ein schreibender Speicherzugriff in den Text-Bereich, wo die assemblierten Instruktionen des Programms liegen, führt zu einem Laufzeitfehler. Ein lesender Zugriff ist jedoch erlaubt.
\end{warningblock}

\begin{infoblock}{Reihenfolge der Load-Operanden}
	Die RISC-V Spezifikation definiert die Reihenfolge der Operanden aller Load-Instruktionen anders: Das Zielregister liegt zwischen Basisregister und Offset. Dies ist unintuitiv, daher sind die Load-Instruktionen im Simulator analog zu den Store-Instruktionen definiert.
\end{infoblock}

\subsubsection{Simulatorbefehle}
Um ein einfaches Debuggen zu ermöglichen, unterstützt der Simulator noch Instruktionen, die nicht im RISC-V Standard enthalten sind und vom Verhalten auch eigentlich nicht mehr einer Maschineninstruktion entsprechen. Die Funktionalität ist jedoch, zur einfachen Benutzung, auch als Instruktion im selben Stil implementiert.\\

\textbf{simucrash }\texttt{<msg>} akzeptiert als einzigen Operanden ein String-Literal. Wird diese Instruktion ausgeführt, terminiert das Programm und die eingegebene Nachricht wird angezeigt. Diese Simulatorinstruktion eignet sich also zur Absicherung des Codes, ähnlich wie Exceptions in höheren Programmiersprachen.\\
\todo[inline]{Evtl. Bild des Dialogs der angzeigt wird}

\textbf{simusleep }\texttt{<ms>} akzeptiert als einzigen Operanden eine ganzzahlige Konstante, die die Zeit in Millisekunden repräsentiert. Wird der Befehl ausgeführt, wird die Ausführung der folgenden Instruktion um mindestens\footnote{Aufgrund spezieller Umstände kann das Betriebssystem die Wartezeit auch verlängern. In den allermeisten Fällen handelt es sich dabei aber um einen nicht spürbare Verzögerung.} \texttt{ms} Millisekunden verzögert.


\subsection{Pseudoinstruktionen}
\label{user-manual-riscv-overview-pseudos}
\todo[inline]{Was sind Pseudoinstruktionen + Liste mit Verhalten}
Pseudoinstruktionen sind keine wirklichen Instruktionen, die es im RISC-V Befehlssatz so gibt. Ihre Funktionalität lässt sich mit einer Kombination von RISC-V Instruktionen erreichen. Diese Kombination kann aber unter Umständen recht kompliziert oder mühsam sein, daher stellen viele Assembler-Dialekte Pseudo-Instruktionen zur Verfügung. Diese können wie eine normale Instruktion verwendet werden, werden aber beim Übersetzen/Assemblieren in die eigentliche Kombination zerlegt.\\
Diese Funktionalität bietet auch der Simulator. Die Pseudoinstruktionen sind hier als Makros ''vordefiniert'', können also überall genutzt werden. Durch Aufklappen der Makros kann man diejenigen Instruktionen sehen, in die die Pseudo-Instruktion zerlegt wird.\\

\begin{tabular}{|c|p{5cm}|p{5cm}|}
	\hline
	Pseudoinstruktion & Operanden & Beschreibung\\
	\hline
	\textbf{la} \texttt{<rd>, <symbol>} & 
	\begin{itemize}
		\item \texttt{rd}: Register, in das die Adresse reingeladen wird\
		\item \texttt{symbol}: 32-Bit Adresse (relativ zum Programmzähler)\footnote{\label{user-manual-riscv-overview-pseudos-foot1}Da jede Marke eine Adresse darstellt führt die Verwendung einer Marke hier nicht zu einem Übersetzungsfehler. Anders als bei den relativen Kontrollflussinstruktionen wird hier eine Marke \textbf{nicht} automatisch in die passende relative Adresse umgewandelt}
	\end{itemize}
	& Lädt die zum Programmzähler relative Adresse \texttt{symbol} in das Register \texttt{rd}. Im Gegensatz zu den Kontrollflussinstruktionen kann hier eine bis zu 32-Bit große Konstante verwendet werden.\\
	\hline
	\textbf{lbg} \texttt{<rd>, <mem\_addr>} &
	\multicolumn{2}{|p{10cm}|}{Gleiche Funktionalität wie \textbf{lb}, nur dass \texttt{mem\_addr} bis zu 32-Bit lang sein darf}\\
	\hline
	\textbf{lhg} \texttt{<rd>, <mem\_addr>} &
	\multicolumn{2}{|p{10cm}|}{Gleiche Funktionalität wie \textbf{lh}, nur dass \texttt{mem\_addr} bis zu 32-Bit lang sein darf}\\
	\hline
	\textbf{lwg} \texttt{<rd>, <mem\_addr>} &
	\multicolumn{2}{|p{10cm}|}{Gleiche Funktionalität wie \textbf{lw}, nur dass \texttt{mem\_addr} bis zu 32-Bit lang sein darf}\\
	\hline
	\textbf{sbg} \texttt{<rd>, <mem\_addr>, <rt>} &
	\multicolumn{2}{|p{10cm}|}{Gleiche Funktionalität wie \textbf{sb}, nur dass \texttt{mem\_addr} bis zu 32-Bit lang sein darf. \texttt{rt} ist hier ein Transferregister, in das die Adresse geladen wird}\\
	\hline
	\textbf{shg} \texttt{<rd>, <mem\_addr>, <rt>} &
	\multicolumn{2}{|p{10cm}|}{Gleiche Funktionalität wie \textbf{sh}, nur dass \texttt{mem\_addr} bis zu 32-Bit lang sein darf. \texttt{rt} ist hier ein Transferregister, in das die Adresse geladen wird}\\
	\hline
	\textbf{swg} \texttt{<rd>, <mem\_addr>, <rt>} &
	\multicolumn{2}{|p{10cm}|}{Gleiche Funktionalität wie \textbf{sw}, nur dass \texttt{mem\_addr} bis zu 32-Bit lang sein darf. \texttt{rt} ist hier ein Transferregister, in das die Adresse geladen wird}\\
	\hline
	\textbf{nop} & & Eine No-Operation. Es wird kein Register und keine Speicherzelle verändert.\\
	\hline
	\textbf{li} \texttt{<rd>, <immediate>} &
	\begin{itemize}
		\item \texttt{rd}: Zielregister, in das die Konstante geladen wird
		\item \texttt{immediate}: 32-Bit Konstante
	\end{itemize}
	& Lädt die gegebene 32-Bit Konstante in das Register \texttt{rd}.\\
	\hline
\end{tabular}\\

\begin{tabular}{|c|p{5cm}|p{5cm}|}
	\hline
	Pseudoinstruktion & Operanden & Beschreibung\\
	\hline
	\textbf{mv} \texttt{<rd>, <rs>} &
	\begin{itemize}
		\item \texttt{rd}: Zielregister
		\item \texttt{rs}: Register, dessen Wert kopiert wird
	\end{itemize}
	& Kopiert den Inhalt des Registers \texttt{rs} in Register \texttt{rd}\\
	\hline
	\textbf{not} \texttt{<rd>, <rs>} & 
	\begin{itemize}
		\item \texttt{rd}: Zielregister
		\item \texttt{rs}: Register, dessen Wert bitweise negiert wird
	\end{itemize}
	& Negiert den Wert von \texttt{rs} bitweise und schreibt das Ergebnis nach \texttt{rd} ($\equiv$ Negation im 1er-Komplement).\\
	\hline
	\textbf{neg} \texttt{<rd>, <rs>} &
	\begin{itemize}
		\item \texttt{rd}: Zielregister
		\item \texttt{rs}: Register, dessen Wert negiert wird
	\end{itemize}
	& Ändert das Vorzeichen des Registers \texttt{rs} und schreibt das Ergebnis nach \texttt{rd} ($\equiv$ Negation im 2er-Komplement).\\
	\hline
	\textbf{seqz} \texttt{<rd>, <rs>} &
	\multirow{4}{5cm}
		{\begin{itemize}
			\item \texttt{rd}: Zielregister
			\item \texttt{rs}: Register, das überprüft wird
		\end{itemize}}
	& 
	\multirow{4}{5cm}{
		\begin{tabular}{ccc}
			\multirow{4}{2cm}{Legt 1 in Register \texttt{rd} ab, wenn}& $=0$ & \multirow{4}{0.5cm}{Sonst 0}\\
			& $\ne0$ &\\
			& $<0$ &\\
			& $>0$ &\\
		\end{tabular}
		}\\
	\textbf{snez} \texttt{<rd>, <rs>} &  &\\
	\textbf{sltz} \texttt{<rd>, <rs>} &  &\\
	\textbf{sgtz} \texttt{<rd>, <rs>} &  &\\
	\hline
	\textbf{beqz} \texttt{<rs>, <marke>} & 
	\multirow{6}{5cm}{
		\begin{itemize}
			\item \texttt{rs}: Register für die Rücksprungadresse und Register für den Vergleich mit 0
			\item \texttt{marke}: Sprungmarke, die angesprungen wird, wenn die Bedingung stimmt
		\end{itemize}}
	& \begin{tabular}{cc}
		\multirow{6}{2cm}{Springt, wenn} & $rs = 0$\\
		$rs
	\end{tabular}
\end{tabular}