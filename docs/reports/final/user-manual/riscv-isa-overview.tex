% !TEX root = user-manual.tex
% ERA-Großpraktikum: Benutzeranleitung -- Übersicht Befehlssatz RISC-V

\section{Übersicht Befehlssatz RISC-V}

Die Menge an verwendbaren Befehlen für den Simulator lässt sich grob in zwei
Teile aufteilen. Die erste Menge an Befehlen sind die sogenannten
Parser-Direktiven. Das sind diejenigen Befehle bzw. Direktiven, die schon
Zeitpunkt der Übersetzung (also Assemblierung) aufgelöst und verarbeitet werden.
Direktiven tauchen also nicht direkt im assemblierten Code auf, dennoch nehmen
sie u.a. durch Anordnung des Speichers (Sektionen, Speicherreservierungen) oder
durch Codegenerierung (Makros) wesentlich Einfluss auf das assemblierte
Programm. Zur Unterscheidung haben Direktiven im Simulator ein .-Präfix.
Direktiven werden im ersten Teil behandelt.\\
Die zweite Menge sind die eigentlichen RISC-V Instruktionen. Diese werden durch
den Assemblierer in maschinenlesbaren Code umgewandelt und können dann auch
ausgeführt werden. Für alle unterstützten RISC-V Instruktionen gibt es im
zweiten Teil eine Übersicht.

\subsection{Parser-Direktiven}

\subsubsection{Konstanten}
Um die Übersichtlichkeit eines Programmes zu verbessern, können häufig benötigte
Konstanten eigens definiert werden. Auf diese kann dann über einen Namen
zugegriffen werden.

\texttt{.equ <name>, <wert>} erzeugt eine Konstante \texttt{name} mit Wert
\texttt{wert}

\begin{exampleblock}{Beispiel: Konstanten}
	\begin{riscv}
	.equ my_number, 10

	; Konstanten können die Werte
	; anderer Konstanten benutzen
	.equ my_number_double, my_number * 2

	; Addiere 10 zu x1
	addi x1, x1, my_number

	; Addiere 20 zu x2
	addi x2, x2, my_number_double

	; Addiere 15 zu x3
	addi x3, x3, my_number + 5
	\end{riscv}
\end{exampleblock}

\subsubsection{Sektionswechsel}

Beim Assemblieren wird der Speicher in zwei Sektionen unterteilt: eine
\emph{Datensektion} sowie eine \emph{Textsektion}. Dies dient der strikten
Trennung von benutzerdefinierten Daten und assemblierten Programmcode.\\

Instruktionen müssen dabei in der Textsektion, Speicherreservierungen und
-initialisierung in der Datensektion stehen. Eine \texttt{.section} Direktive
definiert den Beginn einer solchen Sektion:

\begin{itemize}
  \item \texttt{.section data}: wechselt zur Datensektion,
  \item \texttt{.section text}: wechselt zur Textsektion.
\end{itemize}

\begin{exampleblock}{Beispiel zum Sektionswechsel}
	\begin{riscv}
	\dots
	.section data ; Wechsel zur Datensektion
	.word 0xff ; Speicherdefinitionen
	.half 123

	.section text ; Wechsel zurück zum Programmcode
	addi x1, x1, 10
	\dots
  \end{riscv}
\end{exampleblock}

\subsubsection{Speicherreservierung}
Speicherreservierungen sind eine Möglichkeit, eine oder mehrere Speicherzellen
(hier: Byte) im Speicher freizuhalten. Freihalten bedeutet in diesem Fall, dass der
Assemblierer keine assemblierten Instruktionen in diese Speicherzelle schreibt.
So kann man sicherstellen, dass man bei Verwendung dieser Speicherzelle(n) nicht
den gerade auszuführenden Code korrumpiert. Da die Adresse der Speicherzelle(n)
vom Assemblierer abhängt, sollte man die Adresse mit einem vorgeschobenen Label
"speichern". Das Label vor einer Speicherreservierungs-Direktive nimmt die
Anfangsadresse des reservierten Speicherblocks an.

\begin{table}[H]
\centering
	\begin{tabular}{>{\ttfamily}c l}
    \textbf{Direktive} & \textbf{Beschreibung}\\
    \toprule
		.resb <n> & reserviert \texttt{n} Byte\\
		.resh <n> & reserviert \texttt{n} Halbwörter (also jeweils 16 Bit)\\
		.resw <n> & reserviert \texttt{n} Wörter (also jeweils 32 Bit)\\
		.resd <n> & reserviert \texttt{n} Doppelwörter (also jeweils 64 Bit)\\
	\end{tabular}
\end{table}

\begin{warningblock}
	Speicherreservierungen sollten sich nur in der Datensektion befinden!
\end{warningblock}

\begin{exampleblock}{Beispiel zur Speicherreservierung}
	Es sollen 32 Bit an Speicherplatz unter dem Label \texttt{memory} reserviert werden:
	\begin{riscv}
	memory: .resb 4
	\end{riscv}
	oder
	\begin{riscv}
	memory: .resh 2
	\end{riscv}
	oder
	\begin{riscv}
	memory: .resw 1
	\end{riscv}
	Die obigen Befehle sind gleichwertig, es empfiehlt sich jedoch eine Variante zu wählen, die der Verwendung am nächsten kommt, um die Lesbarkeit zu erhöhen. Wird beispielsweise Byte für Byte gearbeitet, ist auch eine Byte-weise Reservierung des Speichers anschaulich.
\end{exampleblock}

\subsubsection{Speicherdefinition}

Speicherdefinitionen sind die andere Möglichkeit Speicherzellen von
assemblierten Instruktionen freizuhalten. Im Gegensatz zu Speicherreservierungen
wird hier der Speicher nicht nur reserviert, sondern gleichzeitig auch mit dem
übergebenen Datum gefüllt. Mit Speicherdefinitionen kann man also den Speicher
noch vor der Ausführung des Programms mit Daten befüllen. Da auch hier die
genaue Adresse des Speicherblocks von der Assemblierung abhängt, sollte zur
Verwendung ein Label vorangestellt werden.\\

\begin{table}[H]
  \centering
	\begin{tabular}{>{\ttfamily}cl}
    \textbf{Direktive} & \textbf{Beschreibung}\\
    \toprule
		.byte <data> & legt \texttt{data} als bytes in den Speicher\\
		.half <data> & legt \texttt{data} als Halbwörter (16 Bit) in den Speicher\\
		.word <data> & legt \texttt{data} als Wort (32 Bit) in den Speicher\\
		.dword <data> & legt \texttt{data} als Doppelwort (64 Bit) in den Speicher\\
	\end{tabular}
\end{table}

\texttt{data} kann hier eine oder mehrere Zahlen sein (durch Kommas getrennt), die dann hintereinander als Bytes, Halbwörter, usw. abgelegt werden.

\begin{exampleblock}{Beispiel Speicherdefinition}
	Es soll die Abfolge \texttt{0x12, 0x34, 0x56, 0x78} in den Speicher unter dem Label \texttt{sequence} abgelegt werden:
	\begin{riscv}
		sequence: .byte 0x12, 0x34, 0x56, 0x78
	\end{riscv}
\end{exampleblock}

\begin{infoblock}{Zeichenketten in den Speicher legen}
	Es ist auch möglich eine komplette Zeichenkette ganz bequem in den Speicher zu legen.
	\begin{riscv}
		label: .byte "foo? bar!"
	\end{riscv}
	Die Zeichen werden im ASCII-Format abgelegt.
\end{infoblock}

\begin{warningblock}
	Speicherdefinitionen sollten sich nur in der Datensektion befinden!
\end{warningblock}

\subsubsection{Makros}

Häufig benötigte Folgen von Befehlen können als Makro definiert werden. Hierzu
existieren folgende Direktiven:

\begin{itemize}
	\item \texttt{.macro <name> [, <parameter>]} beginnt die Definition eines
	Makros mit dem Namen \texttt{name}. Optional kann eine durch Kommata getrennte Liste
	von Parametern angegeben werden.
	\item \texttt{.endm} schließt die Definition eines Makros ab.
\end{itemize}

Alle Befehle zwischen \texttt{.macro} und \texttt{.endm} sind Teil des Makros.
Im Programm kann das Makro wie jeder andere Befehl mit seinem Namen aufgerufen
werden.

\begin{exampleblock}{Beispiel: Einfaches Makro}
	Es soll ein Makro definiert werden, dass die Register \texttt{x1}, \texttt{x2}
	und \texttt{x3} verdoppelt:
	\begin{riscv}
		.macro verdopple
		add x1, x1, x1
		add x2, x2, x2
		add x3, x3, x3
		.endm

		; Aufruf
		verdopple
	\end{riscv}
\end{exampleblock}

\begin{exampleblock}{Beispiel: Parameterisiertes Makro}
	Es soll ein Makro definiert werden, dass das angegebene Register verdoppelt:
	\begin{riscv}
		.macro verdopple, r
		add \r, \r, \r
		.endm

		; Aufrufe
		verdopple x1
		verdopple x2
	\end{riscv}
\end{exampleblock}

\begin{exampleblock}{Beispiel: Parameterisiertes Makro mit Standardwerten}
	Es soll ein Makro definiert werden, dass bis zu 3 angegebene Register
	verdoppelt:
	\begin{riscv}
		.macro verdopple, r1, r2=x0, r3=x0
		add \r1, \r1, \r1
		add \r2, \r2, \r2
		add \r3, \r3, \r3
		.endm

		; Aufrufe
		verdopple x1
		verdopple x2, x3
		verdopple x1, x2, x3
	\end{riscv}
\end{exampleblock}

\begin{infoblock}{Überladung von Makros}
	Es ist möglich, mehrere Makros mit dem gleichen Namen zu definieren, solange
	sich deren Parameteranzahl unterscheidet.
\end{infoblock}

\subsection{RISC-V Instruktionen}

Die RISC-V Instruktionen sind in verschiedene Module aufgeteilt. Durch eine
Modularisierung des Dialekts kann ganz bewusst auf Arten von Instruktionen
verzichtet werden. Es braucht z.B. nicht jeder Mikrocontroller Multiplikation
oder atomare Operationen. Der Simulator unterstützt folgende Module:

\begin{itemize}
	\item \textbf{RV32I} Das "Basispaket" mit 32 Bit Arithmetik-, Sprung- und Verzweigungsinstruktionen
	\item\textbf{ RV32M} Fügt dem Basispaket Befehle zur Multiplikation und Division für 32 Bit hinzu
	\item \textbf{RV64I} Das "Basispaket" mit 64 Bit Arithmetik-, Sprung- und Verzweigungsinstruktionen
	\item \textbf{RV64M} Fügt dem Basispaket Befehle zur Multiplikation und Division für 64 Bit hinzu
\end{itemize}
Es kann generell für ein Projekt im Simulator entweder 32 oder 64 Bit als Registerlänge gewählt werden. Dementsprechend können nur Module mit der gleichen Länge verwendet werden. Die 64 Bit Erweiterungen enthalten alle Instruktionen ihrer 32 Bit Gegenstücke, allerdings funktionieren diese auf den gesamten 64 Bit.\\
Im Folgenden wird jeder unterstützte Befehl kurz erläutert. Eine detailliertere Beschreibung ist den Hilfetextanzeigen (\autoref{help-component}) im Simulator zu entnehmen.\\

Alle regulären Instruktionen verwenden in der einen oder anderen Art Register
also Operanden. Die im Simulator unterstützten Module stellen dem Programmierer
32 Allzweckregister zur Verfügung. Diese Register sind je nach gewähltem
Basismodul 32 oder 64 bit lang. Das Register 0 ("zero"-Register) hat
unveränderbar den Wert 0. Für eine gute Verwendungsmöglichkeit siehe
\autoref{manual-riscv-jump-instructions}. Die Register werden über
\texttt{x<id>} angesprochen, wobei \texttt{id} die Nummer des Registers ist. So
steht \texttt{x0} für das Register 0 und \texttt{x17} für das Register 17.\\
Zur Angabe von Zahl-Konstanten (sog. "Immediates") unterstützt der Simulator
folgende 3 Möglichkeiten:
\begin{itemize}
	\item Mit dem Präfix \texttt{0b} lässt sich die Konstante in Binärdarstellung
	ausdrücken. \texttt{0b1101} steht z.B. für die Zahl 13.
	\item Mit dem Präfix \texttt{0x} werden Konstanten in Hexadezimal geschrieben.
	\texttt{0xaffe} $\widehat{=}$ 45 054.

	\item Wird die Zahl ohne Präfix geschrieben, interpretiert sie der Simulator als Zahl
	in Dezimaldarstellung.
\end{itemize}

\begin{infoblock}{Arithmetische Ausdrücke zur Übersetzungszeit}
	Der Simulator unterstützt zudem arithmetische Ausdrücke für Konstanten. So ist \texttt{andi x1,x1, 2*3} gleichwertig zu \texttt{andi x1, x1, 6}. Das ist besonders hilfreich, da auch mit Labels und \texttt{.equ} definierten Konstanten gerechnet werden kann.\\
	\begin{riscv}
		.equ mem_loc, 0xbeef+2
		.equ addr, mem_loc + offset
		.equ offset, 7
		\dots
		lb x1, x0, addr
	\end{riscv}

  Die Definition von Konstanten und Labels kann auch unterhalb der Verwendung im
  arithmetischen Ausdruck oder in einem anderen Segment sein, solange sich keine
  zyklische Abhängigkeit ergibt. Es sind allerdings nur Ausdrücke erlaubt, die
  zur Übersetzungszeit in eine Zahl aufgelöst werden können. \texttt{addi x1,
  x0, 2*x4} ist nicht erlaubt, da der Wert des Registers 4 zur Übersetzungszeit
  noch nicht feststeht. Die Operatoren für die Ausdrücke sind an Operatoren aus
  C angelehnt:

	\begin{center}
    \begin{tabular}{ll}
      \textbf{Symbol} & \textbf{Beschreibung}\\
      \toprule
      + & Addition \\
      - & Subtraktion \\
      * & Multiplikation \\
      / & Division \\
      \% & Modulo \\
      << & Links-Shift\\
      >> & Logischer Rechts-Shift\\
      <= & Vergleich: kleiner gleich\\
      < & Vergleich: kleiner\\
      >= & Vergleich: größer gleich\\
      > & Vergleich: größer\\
      ! & Boolsche Negation\\
      == & Vergleich: Gleichheit\\
      != & Vergleich: Ungleichheit\\
    \end{tabular}
	\end{center}

	Hinweis zu boolschen Operationen: eine Zahl $=0$ repräsentiert gleichzeitig
	einen falschen, eine Zahl $\neq 0$ einen wahren Ausdruck. Vergleiche ergeben
	jeweils $0$ bzw. $1$.

\end{infoblock}

\begin{warningblock}

  Die Interpretation von Zahldarstellungen in binärer oder hexadizimaler Form
  folgt nicht immer dem Zweierkomplement. Die Zahldarstellungen werden stets als
  eine 32 Bit lange Zahl interpretiert\footnote{Dies ermöglicht arithmetisches
  Rechnen auch (leicht) außerhalb des Wertebereichs der erwarteten Konstante.
  Dies ist möglich, wenn nur Zwischenergebnisse den Wertebereich verlassen, das
  Ergebnis aber im Wertebereich liegt. Bsp. \texttt{addi x1, x0, (2000*3)/4;
  Legt 3/4 * 2000 in x1}. Hier wird ein Zwischenergebnis (3*2000) größer als
  2048 produziert. Würde die Zahldarstellung auf die Operandengröße beschränkt
  werden, wäre das (unerwartete) Ergebnis $(3 \cdot 2000 \mod 2048)/4 = 476$
  anstatt $1500$}. Das hat zur Folge, dass \texttt{addi x1,x0, 0xfff} mit einem
  Übersetzungsfehler als "zu groß" abgelehnt wird. Grund ist hier, dass der
  Befehl eine Konstante im Wertebereich von vorzeichenbehafteten 12 Bit
  erwartet. Die größte darstellbare Zahl ist $2^{11}-1 = 2047$ während $\mathtt{0xfff}  \widehat{=} 4095$, also zu groß ist. Die obige Instruktion schreibt also nicht
  -1 in das Register 1. Um das zu erreichen, sollte \texttt{addi x1, x0, -1}
  oder \texttt{addi x1, x0, -0x1} o.ä verwendet werden.

\end{warningblock}

\subsubsection{Arithmetische Befehle}
Jeder arithmetische Befehl hat eine Register-Register und eine Register-Immediate Variante. Letztere hat das Suffix \texttt{i}.
\begin{warningblock}

	Die Register-Register Instruktionen arbeiten auf den vollen 32/64 Bit. Der
	Immediate-Operand ist aber auf den Wertebreich von vorzeichenbehafteten 12 Bit
	beschränkt (-2048 -- 2047)

\end{warningblock}

Das Format der arithmetischen Operationen ist gleich aufgebaut:
\begin{center}
	\texttt{instr <dest>, <reg1>, <reg2|imm>}
\end{center}

Die beiden Operanden \texttt{reg1} und \texttt{reg2} bei Register-Register bzw.
\texttt{reg1} und \texttt{imm} bei Register-Immediate dienen als Operanden für
die arithmetische Funktion. Das Ergebnis wird immer im Register \texttt{dest}
abgelegt. Die Operandenregister bleiben also u.U. unverändert.

\vspace{-0.2cm}
\begin{table}[H]
  \begin{tabular}{ c p{8cm} p{4cm}}
    \textbf{Instruktion} & \textbf{Beschreibung} & \textbf{Anmerkung}\\
    \toprule
  	\texttt{add(i)} & Addiert die beiden Operanden & \\

    \midrule

  	\texttt{sub} & Subtrahiert \texttt{reg2} von \texttt{reg1} & Es gibt kein
  	\texttt{subi}, aber \texttt{addi rd, rs1, -imm} \\ \midrule

  	\texttt{and(i)} & Führt eine logische, bitweise UND-Operation durch & \\
  	\midrule

  	\texttt{or(i)} & Führt eine logische, bitweise ODER-Operation durch & \\
  	\midrule

  	\texttt{xor(i)} & Führt eine logische, bitweise exklusiv-ODER-Operation
  	durch & \\ \midrule

  	\texttt{sll(i)} & Shiftet \texttt{reg1} um \texttt{reg2|imm} Stellen logisch
  	nach links (Am LSBit werden \texttt{reg2|imm} Nullen eingeschoben) &
  	\multirow{3}{4cm}{Es werden nur die unteren 5 Bit betrachtet} \\

    \midrule

  	\texttt{srl(i)} & Shiftet \texttt{reg1} um \texttt{reg2|imm} Stellen logisch
  	nach rechts (Am MSBit werden \texttt{reg2|imm} Nullen eingeschoben) & \\

    \midrule

  	\texttt{sra(i)} & Shiftet \texttt{reg1} um \texttt{reg2|imm} Stellen
  	arithmetisch nach rechts (Am MSBit wird \texttt{reg2|imm}-mal das Sign-Bit
  	eingeschoben) & \\

  \end{tabular}
\end{table}

\begin{warningblock}
Alle Shift-Instruktionen der Variante Register-Immediate erlauben eine
Konstante im Wertebereich zwischen 5 und 12 Bit. Obwohl nur die unteren 5 Bit
verwendet werden, führt eine Zahl, die größer als 5 Bit ist, nicht zu einem
Übersetzungsfehler.
\end{warningblock}

Das 64 Bit Modul \textbf{RV64I} fügt zusätzliche spezielle arithmetische Befehle
hinzu. Mit RV64I werden die "normalen" Instruktionen (siehe Tabelle oben) auf 64
Bit erweitert. Die nun folgenden "word"-Instruktionen (erkennbar am Suffix "w")
betrachten jeweils nur die unteren 32 Bit der 64 Bit breiten Operandenregister
und berechnen ein 32 Bit breites Ergebnis, das auf 64 Bit erweitert wird.

\begin{table}[H]
  \begin{tabular}{c p{8cm} p{4cm}}
    \textbf{Instruktion} & \textbf{Beschreibung} & \textbf{Anmerkung}\\
    \toprule

  	\texttt{add(i)w} & Addiert die beiden Operanden & \\

  	\texttt{subw} & Subtrahiert \texttt{reg2} von \texttt{reg1}& Es gibt kein
  	\texttt{subiw}, aber \texttt{addiw rd, rs1, -imm} \\

  	\texttt{sll(i)} & Shiftet \texttt{reg1} um \texttt{reg2|imm} Stellen logisch
  	nach links (Am LSBit werden \texttt{reg2|imm} Nullen eingeschoben) &
  	\multirow{3}{4cm}{Arbeitet auf den vollen 64Bit. Als Stellenanzahl werden
  	nur die unteren 6 Bit betrachtet} \\ \midrule

  	\texttt{srl(i)} & Shiftet \texttt{reg1} um \texttt{reg2|imm} Stellen logisch
  	nach rechts (Am MSBit werden \texttt{reg2|imm} Nullen eingeschoben) & \\
  	\midrule

  	\texttt{sra(i)} & Shiftet \texttt{reg1} um \texttt{reg2|imm} Stellen
  	arithmetisch nach rechts (Am MSBit wird \texttt{reg2|imm}-mal das Sign-Bit
  	eingeschoben) & \\

  	\texttt{sll(i)w} & Shiftet \texttt{reg1} um \texttt{reg2|imm} Stellen
  	logisch nach links (Am LSBit werden \texttt{reg2|imm} Nullen eingeschoben) &
  	\multirow{3}{4cm}{Als Stellenanzahl werden nur die unteren 5 Bit betrachtet} \\
  	\midrule

  	\textbf{srl(i)w} & Shiftet \texttt{reg1} um \texttt{reg2|imm} Stellen
  	logisch nach rechts (Am MSBit werden \texttt{reg2|imm} Nullen eingeschoben) & \\
  	\midrule

  	\texttt{sra(i)w} & Shiftet \texttt{reg1} um \texttt{reg2|imm} Stellen
  	arithmetisch nach rechts (Am MSBit wird \texttt{reg2|imm}-mal das Sign-Bit
  	eingeschoben) & \\

  \end{tabular}
\end{table}

\subsubsection{\texttt{LUI}, \texttt{AUIPC} und \texttt{SLT}}

Die Befehle \textbf{lui} und \textbf{auipc} sind spezielle Instruktionen, um
32-Bit große Konstanten in ein Register zu übertragen. Da die Instruktionslänge
bei RISC-V nur 32 Bit beträgt, kann also keine 32 Bit große Konstante und der
Ladebefehl gleichzeitig in einer Instruktion zusammengefasst werden. Somit muss
das Laden der Konstante in mehreren Instruktionen erfolgen.

Mit \textbf{lui} kann eine bis zu 20 Bit große Konstante \texttt{imm} direkt in die oberen 20 Bit des Registers \texttt{reg} geladen werden:
\texttt{lui <reg>, <imm>}.

\begin{exampleblock}{32 Bit Konstante laden}
Die Konstante 0xabcd1234 in x2 laden:\\
\begin{riscv}
  lui x2, 0xabcd1 	;Lade obere 20 Bit
  addi x2, x2, 0x234	;Lade untere 12 Bit
\end{riscv}
Oder eleganter:
\begin{riscv}
  .equ myConstant, 0xabcd1234
  lui x2, myConstant>>12        ;Lade obere 20 Bit
  addi x2, x2, myConstant&0xfff ;Lade untere 12 Bit
\end{riscv}

Vorsicht, die obige Implementierung funktioniert nicht bei Zahlen, in deren
Repräsentation das 11. Bit gesetzt ist. Ist dies der Fall muss für den lui-Teil
zusätzlich noch eins addiert werden. Wem das zu kompliziert ist, verwendet am
besten eine passende Pseudoinstruktion
(\autoref{user-manual-riscv-overview-pseudos})

\end{exampleblock}

\texttt{auipc} ermöglicht es, 32 Bit große Sprungadressen relativ zum
Programmzähler zu bauen. Wie auch schon bei \texttt{lui} werden hier mehrere
Instruktionen benötigt. \texttt{auipc} lädt eine bis zu 20 Bit große Konstante
\texttt{imm} in die oberen 20 Bit des Registers \texttt{reg} und addiert danach
den Programmzähler auf das Register: \texttt{auipc <reg>, <imm>}.

\begin{exampleblock}{Ganz weiter Sprung}
Sprung um 0xbadeaffe Byte nach unten:
\begin{riscv}
  auipc x2, 0xbadea	;Lade obere 20 Bit
  jalr x0, 0xffe	;Addiere untere 12Bit & Sprung
\end{riscv}
Auch hier ist eine elegantere Variante mit Konstante und Übersetzungszeit
Arithmetik möglich.
\end{exampleblock}

\texttt{slt} \texttt{<dest>, <reg1>, <reg2>} vergleicht reg1 und reg2 und
schreibt eine 1 in Register dest, wenn reg1 kleiner reg2. Ist dies nicht der
Fall, wird eine 0 in Register dest geschrieben. \textbf{sltu} vergleicht die
Register vorzeichenlos. \textbf{slti} bzw. \textbf{sltiu} akzeptieren anstatt
eines zweiten Registers eine 12-Bit vorzeichenbehaftete bzw. vorzeichenlose
Zahl.

\subsubsection{Kontrollflussbefehle} \label{manual-riscv-jump-instructions} In
RISC-V gibt es zwei Arten von Sprung- bzw. Verzweigungsbefehle: unbedingte und
bedingte Sprünge. Die folgenden Instruktionen sind unbedingte Sprünge, springen
also an die gegebenen Adressen, sobald sie ausgeführt werden:

\texttt{jal} \texttt{<rd>, <marke>} springt zu \texttt{marke} und legt die
Rücksprungadresse in \texttt{rd} ab.

\texttt{jalr} \texttt{<rd>, <basis>, <offset>} springt zur Adresse
\texttt{<basis> + <offset>} und legt die Rücksprungadresse in \texttt{rd} ab.

Im Gegensatz zu Intel-386 besitzt RISC-V keine Instruktionen zur Verwaltung des
Stacks. Benötigt man einen Stack (z.B. für rekursive Aufrufe), muss man
entsprechende push, pop, call und ret Funktionalitäten selbst schreiben.

\begin{exampleblock}{Sprünge}
Das folgende Programm springt zu sub\_routine und kehrt wieder zurück:
\begin{riscv}
  jal x4, sub_routine
  ; hierhin wird zurückgekehrt

  ; \dots
sub_routine:
  ; \dots Ausführung Unterprogramm
  jalr x0, x4, 0
\end{riscv}

Der \texttt{jal}-Ausdruck speichert die Adresse des nachfolgenden Befehls (durch
Kommentar angedeutet) in Register x4 und spring das Unterprogramm an. Der
\texttt{jalr}-Ausdruck springt daraufhin am Ende des Unterprogramms zur Adresse
in Register 4. Die Adresse des nachfolgenden Befehls wird verworfen (also in x0
gespeichert)
\end{exampleblock}

Die Syntax der bedingten Sprünge ist gleich: \textbf{op} \texttt{<cmp1>, <cmp2>,
<marke>} Bei Ausführung prüft die Instruktion eine Bedingung (spezifiziert durch
\texttt{op}) zwischen den beiden Registern \texttt{cmp1} und \texttt{cmp2}. Ist
diese Bedingung wahr, springt die Instruktion zu \texttt{marke}. Ist die
Bedingung falsch, wird kein Sprung ausgeführt, d.h. die Ausführung läuft ganz
normal mit dem nächsten Befehl weiter.

\begin{table}[H]
  \centering
  \begin{tabular}{>{\ttfamily}l c l}

  	\textbf{Instruktion} & \textbf{Bedingung} & \textbf{Besonderheit}\\
    \toprule

  	beq & \texttt{x} $ = $ \texttt{y} & \\

  	bne & \texttt{x} $ \ne $ \texttt{y} & \\

  	blt & \texttt{x} $ < $ \texttt{y} & Vorzeichenbehafteter Vergleich\\

  	bltu & \texttt{x} $ < $ \texttt{y} & Vorzeichenloser Vergleich\\

  	bge & \texttt{x} $ \ge $ \texttt{y} & Vorzeichenbehafteter Vergleich\\

  	bgeu & \texttt{x} $ \ge $ \texttt{y} & Vorzeichenloser Vergleich\\

  \end{tabular}

\end{table}

\begin{infoblock}{Umgang mit Labels in Sprungbefehlen}
  Ein Label enthält die absolute Adresse an der es definiert ist. Einige der
  Kontrollflussinstruktionen funktionieren jedoch mit Programmzähler-relativen
  Adressen, d.h. eigentlich ist eine Umrechnung von absoluter Labeladresse zu
  relativer Adresse nötig. Diese Umrechnung führt der Simulator aber automatisch
  für die Befehle durch, die relativ adressiert werden. So können auch relative
  Kontrollflussinstruktionen wie gewohnt an ein angegebenes Label springen.\\
	Vorsicht ist lediglich geboten, wenn ein Label in einer relativ adressierten
	Instruktion noch durch Arithmetik verändert wird. Hier wird das Label zuerst
	in relative Form gebracht und dann durch die Arithmetik verändert. Es können
	also unerwartete Ergebnisse entstehen.
\end{infoblock}

\begin{warningblock}
	Es ist möglich, statt einer Marke auch einfach nur eine Zahl anzugeben (z.B. \texttt{jal x0, 8}).
	Alle Kontrollflussinstruktionen außer \texttt{jalr} sind Programmzähler-relativ und springen relativ zum Programmzähler um das doppelte des eigentlichen Werts.\\
\begin{riscv}
jal x0, 4 ; Springt 2 Instruktionen (8 Byte) nach unten
beq x0, x0, -2 ; Springt eine Instruktion (4 Byte) nach oben
jalr x0, x0, 0x44 ; Springt an Adresse 0x44
\end{riscv}
\end{warningblock}


\subsubsection{Speicherinteraktion}

Für Interaktion mit dem Speicher (Laden und Speichern) stellt RISC-V weiter
Instruktionen bereit, die im Folgenden beschrieben werden.

\paragraph{Store-Instruktionen}

Alle Instruktionen zum Speichern eines Wertes aus einem Register sind gleich
aufgebaut:

\begin{center}
  \texttt{op <source>, <base>, <offset>}
\end{center}

Der Inhalt aus \texttt{source} (\texttt{op} entscheidet welche Byte) wird an
Speicheradresse \texttt{base+offset} und folgende geschrieben.

\begin{table}[H]
  \centering
  \begin{tabular}{>{\ttfamily}lcl}

  	\normalfont\textbf{Befehl} & \textbf{Größe} & \textbf{Beispiel} (Quelle = \texttt{0x1234 5678})\\
    \toprule

  	sb & unteres Byte & 0x78\\

  	sh & unteres Halbwort (2 Byte) & 0x5678\\

  	sw & (unteres) Wort (4 Byte) & 0x1234 5678\\

  \end{tabular}
\end{table}

Für RV64 kommt eine weitere Instruktion dazu, die oberen Instruktionen
funktionieren wie angegeben.

\begin{table}[H]
  \centering
  \begin{tabular}{>{\ttfamily}lcl}
    \normalfont\textbf{Befehl} & \textbf{Größe} & \textbf{Beispiel} (Quelle = \texttt{0x1234 5678 9abc def0})\\
    \toprule

  	sd & Doppelwort (8 Byte) & 0x1234 5678 9abc def0\\

  \end{tabular}
\end{table}

\pagebreak
\textbf{Load-Instruktionen}

Auch die Lade-Instruktionen sind gleich aufgebaut: \textbf{op} \texttt{<dest>,
<base>, <offset>}. Der Inhalt aus Speicherzelle(n) ab \texttt{base+offset} wird
ins Register \texttt{dest} geschrieben. Analog zu den Store-Instruktionen
spezifiert auch hier \texttt{op} die genaue Anzahl an Bytes. Übrige Bytes im
Register \texttt{dest} werden auf 0 gesetzt.
\begin{table}[H]
  \centering
  \begin{tabular}{>{\ttfamily}lcl}

    \normalfont\textbf{Befehl} & \textbf{Größe} & \textbf{Beispiel} (Speicher = \texttt{0x01|0x23|0x45|0x67|0x89})\\
    \toprule

  	lb & 1 Byte in unterstes Byte & dest = 0x0000 0001\\

  	lh & 2 Byte in unteres Halbwort & dest = 0x0000 000123\\

  	lw & 4 Byte in (unteres) Wort & dest =  0x0123 4567\\

  \end{tabular}
\end{table}
\vspace{-0.6cm}

Für RV64 kommt eine weitere Instruktion dazu:
\begin{table}[H]
  \centering
  \begin{tabular}{>{\ttfamily}{l}cl}

    \normalfont\textbf{Befehl} & \textbf{Größe} & \textbf{Beispiel} (Speicher = \footnotesize\texttt{0x01|0x23|0x45|0x67|0x89|0xab|0xcd|0xef})\\
    \toprule

  	sd & 8 Byte in Doppelwort& dest = 0x0123 4567 89ab cdef\\

  \end{tabular}
\end{table}
\vspace{-0.4cm}

\begin{warningblock}
  Ein schreibender Speicherzugriff in den Text-Bereich, in dem assemblierten
  Instruktionen des Programms liegen, führt zu einem Laufzeitfehler. Ein
  lesender Zugriff ist jedoch erlaubt.
\end{warningblock}

\begin{infoblock}{Reihenfolge der Load-Operanden}
Die RISC-V Spezifikation definiert die Reihenfolge der Operanden aller
Load-Instruktionen anders: Das Zielregister liegt zwischen Basisregister und
Offset. Dies ist unintuitiv, daher sind die Load-Instruktionen im Simulator
analog zu den Store-Instruktionen definiert.
\end{infoblock}
\vspace{-0.2cm}
\subsubsection{Simulatorbefehle}

Um ein einfaches Debuggen zu ermöglichen, unterstützt der Simulator noch
Instruktionen, die nicht im RISC-V Standard enthalten sind und vom Verhalten
auch eigentlich nicht mehr einer Maschineninstruktion entsprechen. Die
Funktionalität ist jedoch, zur einfachen Benutzung, auch als Instruktion im
selben Stil implementiert.

\texttt{simucrash } \texttt{<msg>} akzeptiert als einzigen Operanden ein
String-Literal. Wird diese Instruktion ausgeführt, terminiert das Programm und
die eingegebene Nachricht wird angezeigt. Diese Simulatorinstruktion eignet sich
also zur Absicherung des Codes, ähnlich wie Exceptions in höheren
Programmiersprachen.

\texttt{simusleep }\texttt{<ms>} akzeptiert als einzigen Operanden eine
ganzzahlige Konstante, die die Zeit in Millisekunden repräsentiert. Wird der
Befehl ausgeführt, wird die Ausführung der folgenden Instruktion um
mindestens\footnote{Aufgrund spezieller Umstände kann das Betriebssystem die
Wartezeit auch verlängern. In den meisten Fällen handelt es sich dabei aber
um eine nicht spürbare Verzögerung.} \texttt{ms} Millisekunden verzögert.

\subsection{Pseudoinstruktionen}
\label{user-manual-riscv-overview-pseudos}

Pseudoinstruktionen sind keine wirklichen Instruktionen, die es im RISC-V
Befehlssatz so gibt. Ihre Funktionalität lässt sich mit einer Kombination von
RISC-V Instruktionen erreichen. Diese Kombination kann aber unter Umständen
recht kompliziert oder mühsam sein, daher stellen viele Assembler-Dialekte
Pseudo-Instruktionen zur Verfügung. Diese können wie eine normale Instruktion
verwendet werden, werden aber beim Übersetzen/Assemblieren in die eigentliche
Kombination zerlegt. Diese Funktionalität bietet auch der Simulator. Die
Pseudoinstruktionen sind hier als Makros "vordefiniert", können also überall
genutzt werden. Durch Aufklappen der Makros kann man diejenigen Instruktionen
sehen, in die die Pseudo-Instruktion zerlegt wird.

% !TEX root = ../combined/final.tex
% ERA-Großpraktikum: Benutzeranleitung -- Pseudoinstruktionen

\newcommand{\pseudoinst}[3]{
  \centering
  \begin{tikzpicture}[thick]
    \node [align=center, text width=4.5cm] (inst) at (0, 0) {\texttt{#1}};
    \node [align=left, text width=12cm] (op) at (8.5, 0) {#2};

    \node (desc) at (6, -1.4) [align=center, text width=15cm] {#3};

    \draw [rounded corners=1pt]
          (inst.north west)+(-0.3, 0.6)
          rectangle
          (desc.south east)+(0.4, 0.4);
  \end{tikzpicture}
  \vspace{-0.5cm}
}

\newcommand{\pseudoinsttwo}[3]{
  \centering
  \begin{tikzpicture}[thick, node distance=0.1cm]
    \node (inst) at (0, 0) {\texttt{#1}};
    \node [below = of inst] (desc) [align=center, text width=\textwidth] {#2};
    \draw [rounded corners=1pt]
          ({-\textwidth / 2}, 0.6) rectangle ({\textwidth / 2}, -#3);
  \end{tikzpicture}
  \vspace{-0.5cm}
}

\pseudoinst
{
  \texttt{la <rd>, <symbol>}
}
{
  \texttt{rd}: Register, in das die Adresse reingeladen wird \\
  \texttt{symbol}: 32-Bit Adresse (relativ zum Programmzähler
}
{
  Lädt die zum Programmzähler relative Adresse \texttt{symbol} in das Register \texttt{rd}. Im Gegensatz zu den Kontrollflussinstruktionen kann hier eine bis zu 32-Bit große Konstante verwendet werden.
}

\pseudoinsttwo
{
  lbg <rd>, <mem\_addr>
}
{
  Gleiche Funktionalität wie \texttt{lb}, nur dass \texttt{mem\_addr} bis zu 32-Bit lang sein darf.
}
{1.5cm}

\pseudoinsttwo
{
  lhg <rd>, <mem\_addr>
}
{
  Gleiche Funktionalität wie \texttt{lh}, nur dass \texttt{mem\_addr} bis zu 32-Bit lang sein darf.
}
{1.5cm}

\pseudoinsttwo
{
  lwg <rd>, <mem\_addr>
}
{
  Gleiche Funktionalität wie \texttt{lw}, nur dass \texttt{mem\_addr} bis zu 32-Bit lang sein darf.
}
{1.5cm}

\pseudoinsttwo
{
  sbg <rd>, <mem\_addr>, <rt>
}
{
  Gleiche Funktionalität wie \texttt{sb}, nur dass \texttt{mem\_addr} bis zu 32-Bit lang sein darf.\\\texttt{rt} ist hier ein Transferregister, in das die Adresse geladen wird.
}
{2cm}

\pseudoinsttwo
{
  shg <rd>, <mem\_addr>, <rt>
}
{
  Gleiche Funktionalität wie \texttt{sh}, nur dass \texttt{mem\_addr} bis zu 32-Bit lang sein darf.\\\texttt{rt} ist hier ein Transferregister, in das die Adresse geladen wird.
}
{2cm}

\pseudoinsttwo
{
  swg <rd>, <mem\_addr>, <rt>
}
{
  Gleiche Funktionalität wie \texttt{sw}, nur dass \texttt{mem\_addr} bis zu 32-Bit lang sein darf.\\\texttt{rt} ist hier ein Transferregister, in das die Adresse geladen wird.
}
{2cm}

\pseudoinsttwo
{
  nop
}
{
  Eine No-Operation. Es wird kein Register und keine Speicherzelle verändert.
}
{1.5cm}

\pseudoinst
{
  li <rd>, <immediate>
}
{
  \texttt{rd}: Zielregister, in das die Konstante geladen wird \\
  \texttt{immediate}: 32-Bit Konstante
}
{
  Lädt die gegebene 32-Bit Konstante in das Register \texttt{rd}.
}

\footnotetext{\label{user-manual-riscv-overview-pseudos-foot1}Da jede Marke eine
Adresse darstellt führt die Verwendung einer Marke hier nicht zu einem
Übersetzungsfehler. Anders als bei den relativen Kontrollflussinstruktionen wird
hier eine Marke \textbf{nicht} automatisch in die passende relative Adresse
umgewandelt}

\pseudoinst
{
  mv <rd>, <rs>
}
{
  \texttt{rd}: Zielregister \\
  \texttt{rs}: Register, dessen Wert kopiert wird
}
{
  Kopiert den Inhalt des Registers \texttt{rs} in Register \texttt{rd}.
}

\pseudoinst
{
  not <rd>, <rs>
}
{
  \texttt{rd}: Zielregister \\
  \texttt{rs}: Register, dessen Wert bitweise negiert wird
}
{
  Negiert den Wert von \texttt{rs} bitweise und schreibt das Ergebnis nach
  \texttt{rd} ($\equiv$ Negation im 1er-Komplement)
}

\pseudoinst
{
  neg <rd>, <rs>
}
{
  \texttt{rd}: Zielregister \\
  \texttt{rs}: Register, dessen Wert negiert wird
}
{
  Ändert das Vorzeichen des Registers \texttt{rs} und schreibt das Ergebnis
  nach \texttt{rd} ($\equiv$ Negation im 2er-Komplement).
}

\pseudoinst
{
  seqz <rd>, <rs>
}
{
  \texttt{rd}: Zielregister \\
  \texttt{rs}: Register, das überprüft wird
}
{
  Legt in \texttt{rd} eine 1 ab, wenn \texttt{rs} $=$ \texttt{0} gilt, und
  sonst eine 0.
}

\pseudoinst
{
  snez <rd>, <rs>
}
{
  \texttt{rd}: Zielregister \\
  \texttt{rs}: Register, das überprüft wird
}
{
  Legt in \texttt{rd} eine 1 ab, wenn \texttt{rs} $\neq$ \texttt{0} gilt, und
  sonst eine 0.
}

\pseudoinst
{
  sltz <rd>, <rs>
}
{
  \texttt{rd}: Zielregister \\
  \texttt{rs}: Register, das überprüft wird
}
{
  Legt in \texttt{rd} eine 1 ab, wenn \texttt{rs} $<$ \texttt{0} gilt, und
  sonst eine 0.
}

\pseudoinst
{
  sgtz <rd>, <rs>
}
{
  \texttt{rd}: Zielregister \\
  \texttt{rs}: Register, das überprüft wird
}
{
  Legt in \texttt{rd} eine 1 ab, wenn \texttt{rs} $>$ \texttt{0} gilt, und
  sonst eine 0.
}

\pseudoinst
{
  beqz <rs>, <marke>
}
{
  \texttt{rs}: Register für die Rücksprungadresse und Vergleich mit 0 \\
  \texttt{marke}: Sprungmarke, die angesprungen wird
}
{
  Springt zur Marke, wenn \texttt{rs} $=$ \texttt{0} gilt.
}

\pseudoinst
{
  bnez <rs>, <marke>
}
{
  \texttt{rs}: Register für die Rücksprungadresse und Vergleich mit 0 \\
  \texttt{marke}: Sprungmarke, die angesprungen wird
}
{
  Springt zur Marke, wenn \texttt{rs} $\neq$ \texttt{0} gilt.
}

\pseudoinst
{
  blez <rs>, <marke>
}
{
  \texttt{rs}: Register für die Rücksprungadresse und Vergleich mit 0 \\
  \texttt{marke}: Sprungmarke, die angesprungen wird
}
{
  Springt zur Marke, wenn \texttt{rs} $\leq$ \texttt{0} gilt.
}

\pseudoinst
{
  bgez <rs>, <marke>
}
{
  \texttt{rs}: Register für die Rücksprungadresse und Vergleich mit 0 \\
  \texttt{marke}: Sprungmarke, die angesprungen wird
}
{
  Springt zur Marke, wenn \texttt{rs} $\geq$ \texttt{0} gilt.
}

\pseudoinst
{
  bltz <rs>, <marke>
}
{
  \texttt{rs}: Register für die Rücksprungadresse und Vergleich mit 0 \\
  \texttt{marke}: Sprungmarke, die angesprungen wird
}
{
  Springt zur Marke, wenn \texttt{rs} $<$ \texttt{0} gilt.
}

\pseudoinst
{
  bgtz <rs>, <marke>
}
{
  \texttt{rs}: Register für die Rücksprungadresse und Vergleich mit 0 \\
  \texttt{marke}: Sprungmarke, die angesprungen wird
}
{
  Springt zur Marke, wenn \texttt{rs} $>$ \texttt{0} gilt.
}

\pseudoinst
{
  j <marke>
}
{
  \texttt{marke}: Sprungmarke
}
{
  Springt zu \texttt{marke} und verwirft die Rücksprungadresse
}

\pseudoinst
{
  jal <marke>
}
{
  \texttt{marke}: Sprungmarke
}
{
  Springt zu \texttt{marke} und speichert die Rücksprungandresse in Register \texttt{x1}.
}

\pseudoinst
{
  jr <reg>
}
{
  \texttt{reg}: Enthält die Adresse, an die gesprungen wird
}
{
  Springt an die Adresse in Register \texttt{reg} und verwirft die Rücksprungadresse.
}

\pseudoinst
{
  jalr <reg>
}
{
  \texttt{reg}: Enthält die Adresse, an die gesprungen wird
}
{
  Springt an die Adresse in Register \texttt{reg} und speichert die
  Rücksprungadresse in Register \texttt{x1}
}

\pseudoinsttwo
{
  ret
}
{
  Springt an die Adresse in Register x1.\footnotemark
}
{1.5cm}

\pseudoinst
{
  tail <marke>
}
{
  \texttt{marke}: Sprungmarke
}
{
  Springt zu \texttt{marke} und verwirft die Rücksprungadresse. Für die Adressberechnung wird Register x6 verwendet. Es kann beliebige Adresse im 32-Bit Adressbereich angesprungen werden.
}

\pseudoinst
{
  call <marke>
}
{
  \texttt{marke}: Sprungmarke
}
{
  Springt zu \texttt{marke} und legt die Rücksprungadresse in Register x1 ab\textsuperscript{\ref{call-stack}}.  Für die Adressberechnung wird Register x6 verwendet. Es kann beliebige Adresse im 32-Bit Adressbereich angesprungen werden.
}

\footnotetext{\label{call-stack}Im Gegensatz zu x86 arbeiten die \texttt{call} bzw. \texttt{ret} Instruktionen nicht mit einem Callstack. \texttt{ret} kehrt also nur zum letzten Aufruf von \texttt{call} zurück, da x1 überschrieben wird.}

\pagebreak
Das 64Bit Basismodul von RISC-V fügt noch weitere Pseudoinstruktionen hinzu:\\
\vspace{0.5cm}


\pseudoinsttwo
{
  ldg <rd>, <mem\_addr>
}
{
  Gleiche Funktionalität wie \textbf{ld}, nur dass \texttt{mem\_addr} bis zu 32-Bit lang sein darf.
}
{2cm}

\pseudoinsttwo
{
  sdg <rd>, <mem\_addr>, <rt>
}
{
  Gleiche Funktionalität wie \textbf{sd}, nur dass \texttt{mem\_addr} bis zu 32-Bit lang sein darf. \texttt{rt} ist hier ein Transferregister, in das die Adresse geladen wird
}
{2cm}

\pseudoinst
{
  negw <rd>, <rs>
}
{
  \texttt{rd}: Zielregister \\
  \texttt{rs}: Register, dessen untere 32Bit negiert werden
}
{
  Ändert das Vorzeichen der unteren 32 Bit des Registers \texttt{rs} und schreibt das Ergebnis nach \texttt{rd}.
}

\pseudoinst
{
  sext.w <rd>, <rs>
}
{
  \texttt{rd}: Zielregister \\
  \texttt{rs}: Register, dessen untere 32 Bit mit dem\\ richtigen Vorzeichen erweitert werden
}
{
  Erweitert die unteren 32Bit des Registers \texttt{rs} und schreibt das Ergebnis nach \texttt{rd}.
}


\begin{infoblock}{Pseudoinstruktionen}
	Der RISC-V Standard geht beim den Pseudoinstruktionen von erheblich größeren Programmen aus, als im Simulator verwendet werden. Einige Instruktionen sind daher nur der Vollständigkeit halber definiert. Beispielsweise lassen sich die \texttt{call} und \texttt{tail} Instruktionen auch immer durch ein \texttt{jal} ersetzen, da eine Sprungadresse außerhalb des 12-Bit Bereichs im Simulator nicht vorkommen wird (Es müssten mehr als 512 Instruktionen übersprungen werden!).
\end{infoblock}

\pagebreak
\subsection{Übersetzungsfehler}
\label{user-manual-riscv-errors}
Hier werden einige Übersetzungsfehler aufgelistet und erklärt, die für einen Anfänger vielleicht nicht ganz zu verstehen und dann zu beheben sind.

\begin{tabular}{p{5cm} p{5cm} p{5cm}}

  \textbf{Fehlermeldung} & \textbf{Häufiger Grund} & \textbf{Mögliche Lösung}\\
  \toprule

	Immediate operand must be 12(20) bit or less & Eine zu große Konstante als Operand übergeben & Die erwartete Konstante muss inklusive Vorzeichenbit in 12(20) Bit darstellbar sein.\\

	The given offset \emph{x} will result in an invalid jump adress (pc+\emph{y}) &
	Die errechnete Adresse des Sprungs wird einen Laufzeitfehler generieren, da sie keinen Befehl anspringt oder zwischen zwei Befehle springt. Meist ist die Offset-Konstante ungerade oder zu groß/klein. & Alle Instruktionen des Programms liegen hintereinander im Speicher, die Anfangsadresse kann durch Speicherreservierungen davor nach unten verschoben werden. Die Instruktionen sind stets auf 4 Byte angeordnet.\\

	Jump(Branch) offset would invalidate program counter to \emph{x} & Die errechnete Sprungadresse ist ungültig. Es wird entweder außerhalb des Textsegments gesprungen oder zwischen zwei Befehle &
	Der Wert des Basisregisters ist möglicherweise falsch.\\

	The immediate-register instructions must have 2 registers and 1 immediate as operands & Der Befehl erwartet erst 2 Register und dann eine Konstante. & Möglicherweise wird statt z.B. \texttt{addi} der Register-Register Befehl \texttt{add} genutzt\\

	The register-register instructions must have 3 registers as operands & Der Befehl erwartet 3 Register als Operanden& Möglicherweise wird hier ein Immediate-Register-Befehl, z.B. \texttt{addi} verwendet. Für 3 Register muss jedoch die Register-Register Variante \texttt{add} verwendet werden.\\

	The memory area you are trying to access is protected (area: [\emph{a},\emph{b}]) & Ein Speicherzugriff greift auf eine ungültige Speicheradresse zu. Entweder der angesprochene Speicherbereich beginnt/endet außerhalb des Speichers oder es wird ein schreibender Zugriff auf Teile des assemblierten Programms durchgeführt. & Die Grenzen \emph{a} und \emph{b} (beide inklusive) geben einen Anhaltspunkt, wo eine fehlerhafte Berechnung der Adresse erfolgt sein könnte. Die aus Basisregister und Offset berechnete Adresse einer Load-/Store-Instruktion gibt immer die Anfangsadresse des Speicherbereichs an. Die Länge ist implizit durch den Befehl (\texttt{lb}: 1 Byte, \texttt{lw}: 4 Byte, \dots) gegeben\\

\end{tabular}
