% ERA-Groß?praktikum: Benutzeranleitung -- Übersicht Befehlssatz RISC-V

\section{Übersicht Befehlssatz RISC-V}
\todo[inline]{Hier Erläuterung: Unterschied Parser-direktiven, Instruktionen und Pseudo-instruktionen}

{\subsection{Parser-Direktiven}
\todo[inline]{.equ und .section nicht vergessen!}

\subsubsection{Speicherreservierung}
\todo[inline]{Alle Reservierungs-Instruktionen (also nicht schreibend): z.B. .resb 5}

\subsubsection{Speicherdefinition}
\todo[inline]{Alle Definitions-Instruktionen (also die die schreibenden): z.B .byte 0xff; auch auf Möglichkeit .byte ''string'' hinweisen}

\subsection{Makros}
\todo[inline]{Wie definiere ich ein Makro, Grenzen (Zyklische Aufrufe, Rekursion, Anzahl Parameter)}

\subsection{RISC-V Instruktionen}
\todo[inline]{Erklärung Aufteilung in Module}

\subsubsection{Arithmetische Befehle}
\todo[inline]{add(i), sub, and(i), or(i), xor(i), sll(i), srl(i), sra(i)}

\subsubsection{LUI, AUIPC, SLT}
\todo[inline]{lui, auipc, slt, sltu, slti, sltiu}

\subsubsection{Kontrollflussbefehle (Sprung und Verzweigung)}
\todo[inline]{jal, jalr, beq, bne, blt(u), bge(u)}

\subsubsection{Speicherinteraktion}
\todo[inline]{sb,sw,sd,lb,lw,ld; Hier Infobox zu vertauschten Operanden bei Store}

\subsubsection{Simulatorbefehle}
\todo[inline]{simucrash, simusleep}

\subsection{Pseudoinstruktionen}
\todo[inline]{Was sind Pseudoinstruktionen + Liste mit Verhalten}