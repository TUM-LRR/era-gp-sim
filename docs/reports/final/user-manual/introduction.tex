% !TEX root = user-manual.tex
% ERA-Großpraktikum: Benutzeranleitung -- Einleitung

\section{Einleitung}

Diese Benutzeranleitung enthält eine umfassende Beschreibung des ERA Simulators
aus Sicht des Endnutzers. Wir beschreiben, wie man grundlegend die Ausführung
von Assembler Programmen simulieren kann und welche Optionen dem Nutzer hierbei
zur Verfügung stehen.

\subsection{Beschreibung}

\erasim ist ein Assembler-Simulator, entwickelt für die Begleitung von
Lehrveranstaltungen des Lehrstuhls für Rechnertechnik und Rechnerorganisation
der Technischen Universität München. Der Simulator soll eine Umgebung schaffen,
in welcher der Nutzer, ohne komplexe Systemkonfiguration, praktisch die
Grundlagen der maschinennahen Programmierung in einer Assemblersprache erlernen
kann.

Mit \erasim kann Assembler-Code (insbesondere für den RISC-V Befehlssatz)
geschrieben und dessen Ausführung simuliert werden. Die Erstellung von
Programmen wird durch farbliche Hervorhebung und Trennung sprachlicher
Syntaxkonstrukte, eingeblendeten Informationen zu Befehlen,
Übersetzungsfehlermeldungen und einer Vielzahl weiterer Features erleichtert.
Die Ausführung des geschriebenen Codes kann komplett, schrittweise oder bis zum
nächsten \emph{Breakpoint} erfolgen. Dabei werden die simulierten Register und
der simulierte Speicher mit Inhalt realitätsnah dargestellt. In diesen Punkten
ähnelt der Simulator einer integrierten Entwicklungsumgebung (IDE).\\
Desweiteren bietet \erasim sogenannte Ein- und Ausgabekomponenten an, mit denen
der Benutzer bzw. das laufende Programm interagieren kann. Solche Komponenten,
wie z.B. erleichtertes Abfangen von Eingaben von Maus und Tastatur, das Anzeigen
von Text und Bildern oder das erleichterte Ansteuern von Lichtband- und
7-Segment-Anzeigen, sind in den Simulator integriert. In der Realität
funktioniert die Ausführung einiger Komponenten ähnlich (z.B. das Ansteuern
angeschlossener Anzeigen an einen Mikrocontroller), ist aber meist
komplizierter, durch Bibliotheksfunktionen bereits implementiert sowie
plattformbhängig. Die im Simulator zur Verfügung gestellten Komponenten sollen
keineswegs existierende Anzeigen oder Displays simulieren, sondern lediglich ein
Grundverständnis vermitteln, wie die Steuerung einer Anzeige funktionieren kann.
Zum Anderen erlauben solche Bauteile ein größeres Feld an Herausforderungen zum
Ausprobieren.\\ Der in diesem Simulator verwendete RISC-V Dialekt unterscheidet
sich geringfügig von der offiziellen Spezifikation des
Befehlssatzes\footnote{\url{https://riscv.org/}}: Einige Instruktionen wurden
angepasst (siehe \texttt{STORE}) und bestimmte Klassen von Befehlen wurden
ausgelassen. Das RISC-V Instruction Set Manual spezifiziert nämlich eine ganze
Reihe weiterer Befehle (z.B. System-Calls, Traps und Interrupts) sowie Timer-
und Counter-Register. Zum einen geht der Umgang mit solchen Funktionalitäten
über das Grundverständnis, dessen Erwerb unterstützt werden soll, hinaus. Zum
anderen würde die Simulation von Systemaufrufen o.ä. den Umfang der Grundversion
überschreiten.\\

Die weiteren Abschnitte dieser Anleitung beschreiben den Umgang mit der
Benutzeroberfläche des Simulators. Folgende Kästchen dienen dabei im Text zur
grafischen Hervorhebung von Informationen oder Hinweisen:
\begin{exampleblock}{Beispiel}
	In so einem Kästchen steht ein Beispiel zur Verdeutlichung
\end{exampleblock}

\begin{infoblock}{Info}
	Hier werden nice-to-know Informationen angezeigt, bspw. was im Simulator anders funktioniert als in der Realität
\end{infoblock}

\begin{warningblock}
	Hier wird auf häufige Fehlerquellen und unintuitive Vorgänge hingewiesen
\end{warningblock}

\todo[inline] {Diese Kästen bitte auch nutzen! Sie sollen wichtige Informationen schneller ersichtlich machen}
