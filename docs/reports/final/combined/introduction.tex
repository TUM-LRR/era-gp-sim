% !TEX root = final.tex
% ERA-Großpraktikums: Endbericht -- Einleitung

\chapter*{Einleitung}
\addcontentsline{toc}{chapter}{Einleitung}

Dieses Dokument gibt einen Einblick in die Entwicklung und Nutzung von
\erasim{}, unserem architektur-unabhängigen Assembler-Simulator. Wir wollen in
den folgenden Seiten den momentanen Stand des Simulators beschreiben. Wir wollen
erläutern, welche Design-Entscheidungen unsere Entwicklung geprägt und welche
Ziele unsere Entwicklung geleitet haben. Wir wollen erklären, wie ein Nutzer die
Feature-Vielfalt des Simulators entdecken und effektiv nutzen kann. Wir wollen
aber auch insbesondere festlegen, wie der Simulator in der nahen oder fernen
Zukunft erweitert, verbessert oder repariert werden kann.

Hierzu ist dieser Bericht im Weiteren in drei Kapitel gegliedert. \autoref{dev}
ist der \emph{Entwickler-Bericht} und beschreibt sämtliche Aspekte der
bisherigen und zukünftigen Entwicklung von \erasim{}. Im Anschluss daran befasst
sich \autoref{user}, der \emph{Benutzer-Bericht}, ausschließlich damit, die
Features, die Komponenten und die Nutzung des Simulators zu erläutern. Letztlich
dokumentiert der \emph{Team-Bericht} in \autoref{team} unsere ursprüngliche
Vision für \erasim{}, unsere Planung, die Struktur des Teams sowie die
Erfahrungen, die wir als Gruppe in 10 Monaten Entwicklungszeit gemacht haben.

In jeder Zeile dieses Berichts und jeder Zeile Code des Simulators steckt nicht
nur das mechanische Tippen einer Tastatur. Es fließen in sie viele Monate an
Anstrengung, an Einsatz, an Zweifel, an Neugier, an Erfahrung, an Entwicklung,
an Fragen, an Antworten, an Zusammenarbeit, an individuellem Beitrag, an
Planung, an Nachbesserung, an Kritik, an Verbesserung, an langen Nächten und
vielen gescheiterten Versuchen. In Summe ist so Versuch über Versuch und
Erfahrung über Erfahrung ein Stück Software entstanden, das wir heute als
einsatzbereit bezeichnen können. Wir nennen unseren Simulator funktionsfähig,
jedoch weder vollständig noch abgeschlossen. Wir präsentieren kein Endprodukt,
sondern eine Basis. Wir sehen dies nicht als das Ende der Entwicklung von
\erasim{}, sondern als das Ende des Anfangs und hoffen, dass uns die schwerste
aller Aufgaben --- das Design eines flexiblen, erweiterbaren Grundgerüsts ---
gelungen ist.
