\section{Die RISC-V ISA}

RISC-V ist ein neuer quelloffener Befehlssatz, der
ursprünglich zu Lern- und Forschungszielen entwickelt wurde.

Es gibt zwei Versionen mit 32-Bit- und 64-Bit-Architektur. Wir werden
vorerst ausschließlich erstere Variante betrachten.

Die Architektur ist in Module auf geteilt, die nach zu einer Konfiguration nach Belieben hinzugefügt werden können.
Lediglich das Basismodul für Ganzzahloperationen ``I'' (Basic Integer Module) muss dabei in allen Implementierungen vorhanden sein.
Es enthält Rechen-, Lade-, Speicher- und Kontrollflussbefehle auf ganzzahlige Werte und kann durch folgende
Standardmodule erweitert werden:

\begin{itemize}
\item ``M'' Multiplikation und Division für ganze Zahlen
\item ``A'' atomare Operationen für prozessorübergreifende Synchronisation
\item ``F'' und ``D'' für Gleitkommazahlen einfacher und doppelter Genauigkeit (floats und doubles)
\end{itemize}

Die Instruktionen des Standardbefehlssatzes sind 32 Bit lang und auf
32-Bit-Grenzen ausgerichtet. Jedoch unterstützt RISC-V auch Befehle variabler Länge, die
aus einer beliebigen Anzahl von 16-Bit-Blöcken (\emph{Parcels}) bestehen können.

Die Speicherausrichtung ist bei RISC-V standardmäßig Little-Endian.

Standardmäßig wird Little-Endian als Speicherausrichtung verwendet.

\subsection{Grundlegende Ganzzahl-Befehle (Integer Basic Module)}

Rechenbefehle auf ganzen Zahlen verursachen keine arithmetischen Ausnahmen.

\subsubsection{Register-Unmittelbar}

\begin{lstlisting}[style=risc-v_Assembler]
ADDI rd, rs1, 0  ; (add immediate) addiert die Konstante und den Wert von r1, das ganze wird dann nach rd gespeichert
SLTI rd, rs1, 0  ; (set less than immediate) setzt den Wert von rd=1, falls rs1 < 0, sonst rd=0
SLTIU rd, rs1, 1 ; (set less than immediate unsigned) wie SLTI, nur vorzeichenloser Vergleich
\end{lstlisting}

\lstinline[style=risc-v_Assembler]!ANDI!, \lstinline[style=risc-v_Assembler]!ORI!, \lstinline[style=risc-v_Assembler]!XORI! sind logische Operationen, die dem bitweisen Und, Oder bzw. Exklusiv-Oder entsprechen (Register-Register-Äquivalent: \lstinline[style=risc-v_Assembler]!AND!, \lstinline[style=risc-v_Assembler]!OR! bzw. \lstinline[style=risc-v_Assembler]!XOR!). Ihr
Befehlsformat entspricht dem obigen.

Zudem existieren logische und arithmetische Shifts (\lstinline[style=risc-v_Assembler]!SLLI!, \lstinline[style=risc-v_Assembler]!SRLI!, \lstinline[style=risc-v_Assembler]!SLAI!, \lstinline[style=risc-v_Assembler]!SRAI!). Die Anzahl
der geshifteten Bits ist gleich der Zahl, die aus den ersten fünf Bits der angegebenen Konstante entsteht.

In den nächsten beiden Befehlen ist die unmittelbar angegebene Konstante 20 Bit lang, anstatt 12 Bit.

\begin{lstlisting}[style=risc-v_Assembler]
LUI dest, immediate  ; (load upper immediate)
AUIPC dest, immediate  ; (add upper immediate to program counter)
\end{lstlisting}

\lstinline[style=risc-v_Assembler]!LUI! platziert die Konstante in die höheren 20 Bits des Zielregisters, die unteren 12 Bits werden genullt.

\lstinline[style=risc-v_Assembler]!AUIPC! addiert die Konstante mit den höheren 20 Bits des Befehlszählers und speichert das Ergebnis in das Zielregister.

\subsubsection{Register-Register}

\begin{lstlisting}[style=risc-v_Assembler]
ADD/SLT/SLTU dest, src1, src2
AND/OR/XOR   dest, src1, src2
SLL/SRL      dest, src1, src2
SLA/SRA      dest, src1, src2

SLTU rd, x0, rs2 ; (set less than unsigned) setzt rd zu 0, nur wenn rs2 = 0 (da x0 auf 0 fest verdrahtet ist)
\end{lstlisting}

\subsubsection{NOP-Instruktion}

\begin{lstlisting}[style=risc-v_Assembler]
ADDI x0, x0, 0
\end{lstlisting}

\subsubsection{Unbedingte Sprünge}

\begin{lstlisting}[style=risc-v_Assembler]
JAL rd, immediate  ; rd = (pc + 4), Konstante (immediate) ist 20 Bit lang, Bereich: +-1MiB
\end{lstlisting}

\lstinline[style=risc-v_Assembler]!JAL! (jump and link) speichert \lstinline[style=risc-v_Assembler]!pc + 4! (Befehlszähler um 4 vorgeschaltet) in das Register rd und setzt anschließend den Befehlszähler auf \lstinline[style=risc-v_Assembler]!pc+imm! (Befehlszähler + Konstante). Somit kann der Wert in rd als Rücksprungadresse dienen. Die Konvention ist, dass man diese normalerweise nach x1 speichert. Wenn aber nicht zurückgesprungen werden soll, so kann die Rücksprungadresse auch auf \lstinline[style=risc-v_Assembler]!x0! (die fest verdrahtete \lstinline[style=risc-v_Assembler]!0!) geschrieben werden und geht verloren.

\begin{lstlisting}[style=risc-v_Assembler]
JALR rd, rs1, immediate  ; rd = (pc + 4), die Konstante (immediate) 12 Bit lang
\end{lstlisting}

\lstinline[style=risc-v_Assembler]!JALR! setzt den Befehlszähler auf \lstinline[style=risc-v_Assembler]!rs1 + imm! (Quellregister 1 + Konstante). Zusammen mit dem Befehl \lstinline[style=risc-v_Assembler]!LUI! kann
man damit auf eine beliebige Adresse im Adressraum springen.

\subsubsection{Bedingte Verzweigungen}

Ein wichtiger Aspekt des RISC-V-Befehlssatzes ist, dass es weder Flags noch ein Statusregister gibt. Stattdessen
werden Verzweigungsbefehle verwendet:

\begin{lstlisting}[style=risc-v_Assembler]
BEQ/BNE src1, src2, offset ; (branch equal/not equal)
BLT[U]/BGE[U]  src1, src2, offset  ; (branch less/greater than)
\end{lstlisting}

Falls die Bedingung erfüllt ist, wird \lstinline[style=risc-v_Assembler]!offset! zu dem Befehlszähler addiert.

\subsubsection{Lade-/Speicher-Instruktionen}

\begin{lstlisting}[style=risc-v_Assembler]
LW/LH/LB rd, rs1, offset ; (load word/half/byte) kopiert 32/16/8 Bits vom Speicher von der Adresse (rs1+offset)
SW/SH/SB rd, rs1, offset ; (store word/half/byte) kopiert den Wert von rd in den Speicher
\end{lstlisting}

Neben \lstinline[style=risc-v_Assembler]!LH! und \lstinline[style=risc-v_Assembler]!LB! existieren auch \lstinline[style=risc-v_Assembler]!LHU! und \lstinline[style=risc-v_Assembler]!LBU! (analog \lstinline[style=risc-v_Assembler]!SHU!, \lstinline[style=risc-v_Assembler]!SBU!). Der Unterschied ist, dass der
kopierte 16-/8-Bit-Wert mit Nullen, hingegen bei \lstinline[style=risc-v_Assembler]!LH! und \lstinline[style=risc-v_Assembler]!LB! mit dem Vorzeichenbit auf 32 Bits aufgefüllt wird.

\subsection{Befehlsformate von RISC-V}

Wir stellen fest, dass es bei RISC-V nur wenige Befehlsformate gibt, in der Tat sind es genau vier:

\begin{itemize}
\item R-Typ --- Operation Ziel, Quelle 1, Quelle 2 (\lstinline[style=risc-v_Assembler]!opcode rd, rs1, rs2!)
\item I-Typ --- Operation Ziel, Quelle 1, Konstante (\lstinline[style=risc-v_Assembler]!opcode rd, rs1, immediate!)
\item S-Typ --- Operation Quelle 1, Quelle 2, Konstante (\lstinline[style=risc-v_Assembler]!opcode rs1, rs2, immediate!)
\item U-Typ --- Operation Ziel, Konstante (\lstinline[style=risc-v_Assembler]!opcode rd, immediate!)
\end{itemize}

Beim U-Typ ist die Konstante 20 Bit lang, sonst 12 Bit, da bereits zwei Register kodiert werden müssen.
